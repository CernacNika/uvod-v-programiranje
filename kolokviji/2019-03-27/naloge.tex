
        \documentclass[arhiv, 10pt]{../izpit}
        \usepackage{fouriernc}
        \usepackage{minted}
        \usepackage{booktabs}
        \usepackage{multicol}
        \usepackage{paralist}
        \usepackage{inconsolata}
        \usepackage[T1]{fontenc}
        \usepackage{xcolor}
        \usepackage[most]{tcolorbox}

        \definecolor{light-red}{rgb}{1,0.5,0.5}
        \usemintedstyle{colorful}
        \setlength{\columnsep}{30pt}
        \setlength{\textheight}{1.1\textheight}
        \tcbset{enhanced jigsaw,size=tight,colback=light-red,boxrule=0pt,extrude by=1pt,rounded corners,interior style={opacity=0.3}}
        \newcommand{\inlinepy}[1]{\mintinline{python}{#1}}
        \newcommand{\answerbox}[1]{\framebox{\vphantom{\large M}\hspace{#1cm}}}
        \newcommand{\wrongbox}[1]{\tcbox{\vphantom{M}#1}}
    \begin{document}
            \izpit[ucilnica=205, naloge=-1]{Uvod v programiranje: Kolokvij \#001}{27.\ marec 2019}{
                Pri vsaki nalogi obkrožite črko pred pravilnim odgovorom ali vpišite pravilno vrednost v ustrezen prostor. \\
                Čas reševanja je 30 minut. Veliko uspeha!
            }
            
        \naloga*
        
        Kateri izmed programov pri začetnem stanju
            \inlinepy{k = 7} in
            \inlinepy{m = 1}
        nastavi vrednosti
            \inlinepy{k = 1},
            \inlinepy{m = 7} in
            \inlinepy{n = 8}?
    
        \begin{multicols}{4}
        \begin{enumerate}[(a)]
\item 
            \begin{minted}[autogobble]{python}
            n = k
            k = m
            m = n
            n = k + m
            \end{minted}
        
\item 
                \begin{minted}[autogobble]{python}
                k = m
                n = m
                m = k
                n = k + m
                \end{minted}
            
\item 
                \begin{minted}[autogobble]{python}
                n = k
                m = n
                k = m
                n = k + m
                \end{minted}
            
\item 
                \begin{minted}[autogobble]{python}
                k = m
                m = n
                n = k
                n = k + m
                \end{minted}
            
\end{enumerate}

        \end{multicols}
    
        \naloga*
        \begin{multicols}{2}
        \noindent
        Kakšne vrste napak vsebuje program na desni?

        \begin{enumerate}[(a)]
\item oblikovne napake, ki ne vplivajo na pravilnost rezultata
\item sintaktične napake, zaradi katerih Python programa noče izvesti.
\item napake, zaradi katerih Python prekine z izvajanjem programa
\item vsebinske napake, zaradi katerih Python izračuna napačen rezultat
\end{enumerate}

        \columnbreak

        \begin{minted}[baselinestretch=1.2,escapeinside=||, autogobble]{python}
        
            def fibonacci(n):
                if n <= 0:
                    return 0
                elif n == 1:
                    return 1
                else:
                    a = fibonacci(n - '1')
                    b = fib(n - 2)
                    return a + b
            
        \end{minted}

        \end{multicols}

    
        \naloga*
        Katere vrstice izpiše klic \inlinepy{print(f(g(9)))}, če sta funkciji \inlinepy{f} in \inlinepy{g} definirani kot spodaj?

        \begin{multicols}{2}
        \begin{minted}[autogobble]{python}
        
            def f(x):
                return x + 1
                print(x)

            def g(y):
                print(y)
                return 6 * y
        
        \end{minted}

        \begin{enumerate}[(a)]
\item \inlinepy{[9, 55]}
\item \inlinepy{[55]}
\item \inlinepy{[9, 54, 55]}
\item \inlinepy{[54, 55]}
\end{enumerate}

        \end{multicols}
    
        \naloga*

        \begin{multicols}{2}
        \noindent
        Kateri pogoj preverja spodnja funkcija?
        \begin{minted}[autogobble]{python}
        
            def f(besedilo):
                for znak in besedilo:
                    if znak in 'aeiouAEIOU':
                        return False
                return True
            
        \end{minted}

        \begin{enumerate}[(a)]
\item ali niz vsebuje znak, ki ni samoglasnik
\item ali niz ne vsebuje nobenega samoglasnika
\item ali niz vsebuje kakšen samoglasnik
\item ali niz vsebuje samo samoglasnike
\end{enumerate}

        \end{multicols}
    
        \naloga*
        \begin{multicols}{2}
        \noindent
        V vsak prostor vpišite \textbf{natanko en znak} tako, da bo dobljeni program v spremenljivko \inlinepy{zm} shranil zmnožek števil \inlinepy{x} in \inlinepy{y}:
        
        \columnbreak
        \begin{minted}[baselinestretch=1.2,escapeinside=||]{python}
        zm = |\answerbox{0.5}|
        while x > |\answerbox{0.5}|:
            zm += |\answerbox{0.5}|
            |\answerbox{0.5}| -= 1
        \end{minted}
        \end{multicols}
    
        \clearpage
        \naloga
        
        Katera izmed spodnjih funkcij izračuna ostanek pri deljenju naravnega števila \inlinepy{u} z naravnim številom \inlinepy{v}?
    
        \begin{multicols}{2}
        \begin{enumerate}[(a)]
\item 
                \begin{minted}[autogobble]{python}
                def ostanek(u, v):
                    if u < v:
                        return 0
                    else:
                        return ostanek(u - v, v)
                \end{minted}
            
\item 
                \begin{minted}[autogobble]{python}
                def ostanek(u, v):
                    if v == 0:
                        return u
                    else:
                        return ostanek(v, u % v)
                \end{minted}
            
\item 
                \begin{minted}[autogobble]{python}
                def ostanek(u, v):
                    if u < v:
                        return u
                    else:
                        return ostanek(u % v)
                \end{minted}
            
\item 
            \begin{minted}[autogobble]{python}
            def ostanek(u, v):
                if u < v:
                    return u
                else:
                    return ostanek(u - v, v)
            \end{minted}
        
\end{enumerate}

        \end{multicols}
    
        \naloga*
        
        Katera izmed funkcij vrača drugačne rezultate kot ostale?
    
        \begin{multicols}{2}
        \begin{enumerate}[(a)]
\item 
            \begin{minted}[autogobble]{python}
            def h(p, q, r):
                if not r:
                    return p and q
                else:
                    return False
            \end{minted}
        
\item 
                \begin{minted}[autogobble]{python}
                def h(p, q, r):
                    if not q:
                        return False
                    else:
                        return p and r
                \end{minted}
            
\item 
                \begin{minted}[autogobble]{python}
                def h(p, q, r):
                    if p:
                        return q and r
                    else:
                        return False
                \end{minted}
            
\item 
                \begin{minted}[autogobble]{python}
                def h(p, q, r):
                    if p and q:
                        return r
                    else:
                        return False
                \end{minted}
            
\end{enumerate}

        \end{multicols}
    
        \naloga*
        
        Kateri izmed spodnjih programov ima drugačen izpis kot ostali?
    
        \begin{multicols}{2}
        \begin{enumerate}[(a)]
\item 
                \begin{minted}[autogobble]{python}
                    for i in range(1, 10):
                        if i % 2 == 1:
                            print(i)
                        continue
                \end{minted}
            
\item 
                \begin{minted}[autogobble]{python}
                    for i in range(1, 20, 2):
                        if i < 10:
                            print(i)
                \end{minted}
            
\item 
            \begin{minted}[autogobble]{python}
                for i in range(1, 10):
                    if i % 2 != 0:
                        continue
                    print(i)
            \end{minted}
        
\item 
                \begin{minted}[autogobble]{python}
                    for i in range(1, 20, 2):
                        if i > 10:
                            break
                        print(i)
                \end{minted}
            
\end{enumerate}

        \end{multicols}
    
        \naloga*
        \begin{multicols}{2}
        \noindent
        Napišite primer vrednosti spremenljivk \inlinepy{niz1} in \inlinepy{niz2}, za kateri klic \inlinepy{g(niz1, niz2)} vrne \inlinepy{True}.
        \begin{minted}[baselinestretch=1.2, escapeinside=||]{python}
        niz1 = |\answerbox{3}|
        niz2 = |\answerbox{3}|
        \end{minted}
        \vfil
        \columnbreak
        \begin{minted}[autogobble]{python}
        def g(niz1, niz2):
            if len(niz1) != len(niz2) or len(niz1) < 3:
                return False
            for j in range(len(niz1)):
                if j % 2 == 1 and niz1[j] != niz2[j]:
                    return False
            return True
        \end{minted}
        \end{multicols}
    
        \naloga*
        \begin{multicols}{2}
        \noindent 
        S številkami od $0$ do $4$ označite vrstni red, v katerem moramo izvesti ukaze na desni, da bo na koncu spremenljivka \inlinepy{sez1} kazala na seznam \inlinepy{[2, 4, 6]}?
    
        \columnbreak
        \noindent
        \begin{minted}[baselinestretch=1.2, escapeinside=||]{python}
|\answerbox{0.5}| sez2 = [2]
|\answerbox{0.5}| sez2 = [4]
|\answerbox{0.5}| sez1 = sez2
|\answerbox{0.5}| sez1.append(6)
|\answerbox{0.5}| sez1 = sez2 + sez1

        \end{minted}
        \end{multicols}
    
            \izpit[ucilnica=205, naloge=-1]{Uvod v programiranje: Kolokvij \#002}{27.\ marec 2019}{
                Pri vsaki nalogi obkrožite črko pred pravilnim odgovorom ali vpišite pravilno vrednost v ustrezen prostor. \\
                Čas reševanja je 30 minut. Veliko uspeha!
            }
            
        \naloga*
        
        Kateri izmed programov pri začetnem stanju
            \inlinepy{a = 1} in
            \inlinepy{b = 2}
        nastavi vrednosti
            \inlinepy{a = 2},
            \inlinepy{b = 1} in
            \inlinepy{c = 3}?
    
        \begin{multicols}{4}
        \begin{enumerate}[(a)]
\item 
                \begin{minted}[autogobble]{python}
                a = b
                b = c
                c = a
                c = a + b
                \end{minted}
            
\item 
                \begin{minted}[autogobble]{python}
                a = b
                c = b
                b = a
                c = a + b
                \end{minted}
            
\item 
                \begin{minted}[autogobble]{python}
                c = a
                b = c
                a = b
                c = a + b
                \end{minted}
            
\item 
            \begin{minted}[autogobble]{python}
            c = a
            a = b
            b = c
            c = a + b
            \end{minted}
        
\end{enumerate}

        \end{multicols}
    
        \naloga*
        \begin{multicols}{2}
        \noindent
        Kakšne vrste napak vsebuje program na desni?

        \begin{enumerate}[(a)]
\item sintaktične napake, zaradi katerih Python programa noče izvesti.
\item vsebinske napake, zaradi katerih Python izračuna napačen rezultat
\item napake, zaradi katerih Python prekine z izvajanjem programa
\item oblikovne napake, ki ne vplivajo na pravilnost rezultata
\end{enumerate}

        \columnbreak

        \begin{minted}[baselinestretch=1.2,escapeinside=||, autogobble]{python}
        
            def fibonacci(n):
                if n <= 0:
                    return 0
                elif n == 1:
                    return 0
                else:
                    a = fibonacci(n - 1)
                    b = fibonacci(n - 3)
                    return a * b
            
        \end{minted}

        \end{multicols}

    
        \naloga*
        Katere vrstice izpiše klic \inlinepy{print(p(q(5)))}, če sta funkciji \inlinepy{p} in \inlinepy{q} definirani kot spodaj?

        \begin{multicols}{2}
        \begin{minted}[autogobble]{python}
        
            def p(a):
                return a + 4
                print(a)

            def q(b):
                return 7 * b
                print(b)
        
        \end{minted}

        \begin{enumerate}[(a)]
\item \inlinepy{[5, 35, 39]}
\item \inlinepy{[5, 39]}
\item \inlinepy{[39]}
\item \inlinepy{[35, 39]}
\end{enumerate}

        \end{multicols}
    
        \naloga*

        \begin{multicols}{2}
        \noindent
        Kateri pogoj preverja spodnja funkcija?
        \begin{minted}[autogobble]{python}
        
            def f(niz):
                for x in niz:
                    if x not in 'aeiouAEIOU':
                        return False
                return True
            
        \end{minted}

        \begin{enumerate}[(a)]
\item ali niz vsebuje samo samoglasnike
\item ali niz vsebuje kakšen samoglasnik
\item ali niz vsebuje znak, ki ni samoglasnik
\item ali niz ne vsebuje nobenega samoglasnika
\end{enumerate}

        \end{multicols}
    
        \naloga*
        \begin{multicols}{2}
        \noindent
        V vsak prostor vpišite \textbf{natanko en znak} tako, da bo dobljeni program v spremenljivko \inlinepy{zm} shranil zmnožek števil \inlinepy{b} in \inlinepy{a}:
        
        \columnbreak
        \begin{minted}[baselinestretch=1.2,escapeinside=||]{python}
        zm = |\answerbox{0.5}|
        while b > |\answerbox{0.5}|:
            zm += |\answerbox{0.5}|
            |\answerbox{0.5}| -= 1
        \end{minted}
        \end{multicols}
    
        \clearpage
        \naloga
        
        Katera izmed spodnjih funkcij izračuna ostanek pri deljenju naravnega števila \inlinepy{a} z naravnim številom \inlinepy{b}?
    
        \begin{multicols}{2}
        \begin{enumerate}[(a)]
\item 
                \begin{minted}[autogobble]{python}
                def ostanek(a, b):
                    if b == 0:
                        return a
                    else:
                        return ostanek(b, a % b)
                \end{minted}
            
\item 
            \begin{minted}[autogobble]{python}
            def ostanek(a, b):
                if a < b:
                    return a
                else:
                    return ostanek(a - b, b)
            \end{minted}
        
\item 
                \begin{minted}[autogobble]{python}
                def ostanek(a, b):
                    if a < b:
                        return 0
                    else:
                        return ostanek(a - b, b)
                \end{minted}
            
\item 
                \begin{minted}[autogobble]{python}
                def ostanek(a, b):
                    if a < b:
                        return a
                    else:
                        return ostanek(a % b)
                \end{minted}
            
\end{enumerate}

        \end{multicols}
    
        \naloga*
        
        Katera izmed funkcij vrača drugačne rezultate kot ostale?
    
        \begin{multicols}{2}
        \begin{enumerate}[(a)]
\item 
            \begin{minted}[autogobble]{python}
            def f(y, x, z):
                if not z:
                    return y and x
                else:
                    return False
            \end{minted}
        
\item 
                \begin{minted}[autogobble]{python}
                def f(y, x, z):
                    if y and x:
                        return z
                    else:
                        return False
                \end{minted}
            
\item 
                \begin{minted}[autogobble]{python}
                def f(y, x, z):
                    if y:
                        return x and z
                    else:
                        return False
                \end{minted}
            
\item 
                \begin{minted}[autogobble]{python}
                def f(y, x, z):
                    if not x:
                        return False
                    else:
                        return y and z
                \end{minted}
            
\end{enumerate}

        \end{multicols}
    
        \naloga*
        
        Kateri izmed spodnjih programov ima drugačen izpis kot ostali?
    
        \begin{multicols}{2}
        \begin{enumerate}[(a)]
\item 
            \begin{minted}[autogobble]{python}
                for i in range(1, 50):
                    if i % 5 != 0:
                        continue
                    print(i)
            \end{minted}
        
\item 
                \begin{minted}[autogobble]{python}
                    for i in range(1, 250, 5):
                        if i > 50:
                            break
                        print(i)
                \end{minted}
            
\item 
                \begin{minted}[autogobble]{python}
                    for i in range(1, 250, 5):
                        if i < 50:
                            print(i)
                \end{minted}
            
\item 
                \begin{minted}[autogobble]{python}
                    for i in range(1, 50):
                        if i % 5 == 1:
                            print(i)
                        continue
                \end{minted}
            
\end{enumerate}

        \end{multicols}
    
        \naloga*
        \begin{multicols}{2}
        \noindent
        Napišite primer vrednosti spremenljivk \inlinepy{str1} in \inlinepy{str2}, za kateri klic \inlinepy{f(str1, str2)} vrne \inlinepy{True}.
        \begin{minted}[baselinestretch=1.2, escapeinside=||]{python}
        str1 = |\answerbox{3}|
        str2 = |\answerbox{3}|
        \end{minted}
        \vfil
        \columnbreak
        \begin{minted}[autogobble]{python}
        def f(str1, str2):
            if len(str1) != len(str2) or len(str1) < 3:
                return False
            for i in range(len(str1)):
                if i % 2 == 0 and str1[i] != str2[i]:
                    return False
            return True
        \end{minted}
        \end{multicols}
    
        \naloga*
        \begin{multicols}{2}
        \noindent 
        S številkami od $0$ do $4$ označite vrstni red, v katerem moramo izvesti ukaze na desni, da bo na koncu spremenljivka \inlinepy{sez2} kazala na seznam \inlinepy{[9, 3, 8]}?
    
        \columnbreak
        \noindent
        \begin{minted}[baselinestretch=1.2, escapeinside=||]{python}
|\answerbox{0.5}| sez2.append(8)
|\answerbox{0.5}| sez2 = sez1
|\answerbox{0.5}| sez2 = sez1 + sez2
|\answerbox{0.5}| sez1 = [3]
|\answerbox{0.5}| sez1 = [9]

        \end{minted}
        \end{multicols}
    
            \izpit[ucilnica=205, naloge=-1]{Uvod v programiranje: Kolokvij \#003}{27.\ marec 2019}{
                Pri vsaki nalogi obkrožite črko pred pravilnim odgovorom ali vpišite pravilno vrednost v ustrezen prostor. \\
                Čas reševanja je 30 minut. Veliko uspeha!
            }
            
        \naloga*
        
        Kateri izmed programov pri začetnem stanju
            \inlinepy{k = 6} in
            \inlinepy{m = 8}
        nastavi vrednosti
            \inlinepy{k = 8},
            \inlinepy{m = 6} in
            \inlinepy{n = 14}?
    
        \begin{multicols}{4}
        \begin{enumerate}[(a)]
\item 
            \begin{minted}[autogobble]{python}
            n = k
            k = m
            m = n
            n = k + m
            \end{minted}
        
\item 
                \begin{minted}[autogobble]{python}
                n = k
                m = n
                k = m
                n = k + m
                \end{minted}
            
\item 
                \begin{minted}[autogobble]{python}
                k = m
                m = n
                n = k
                n = k + m
                \end{minted}
            
\item 
                \begin{minted}[autogobble]{python}
                k = m
                n = m
                m = k
                n = k + m
                \end{minted}
            
\end{enumerate}

        \end{multicols}
    
        \naloga*
        \begin{multicols}{2}
        \noindent
        Kakšne vrste napak vsebuje program na desni?

        \begin{enumerate}[(a)]
\item oblikovne napake, ki ne vplivajo na pravilnost rezultata
\item napake, zaradi katerih Python prekine z izvajanjem programa
\item vsebinske napake, zaradi katerih Python izračuna napačen rezultat
\item sintaktične napake, zaradi katerih Python programa noče izvesti.
\end{enumerate}

        \columnbreak

        \begin{minted}[baselinestretch=1.2,escapeinside=||, autogobble]{python}
        
            define fibonacci(n):
                if n <= 0:
                    return 0
                elif n == 1
                    return 1
                else:
                    a = fibonacci(n - 1)
                    b = fibonacci(n - 2)
                  return a + b
            
        \end{minted}

        \end{multicols}

    
        \naloga*
        Katere vrstice izpiše klic \inlinepy{print(a(b(4)))}, če sta funkciji \inlinepy{a} in \inlinepy{b} definirani kot spodaj?

        \begin{multicols}{2}
        \begin{minted}[autogobble]{python}
        
            def a(x):
                return x + 3
                print(x)

            def b(y):
                return 6 * y
                print(y)
        
        \end{minted}

        \begin{enumerate}[(a)]
\item \inlinepy{[4, 27]}
\item \inlinepy{[24, 27]}
\item \inlinepy{[4, 24, 27]}
\item \inlinepy{[27]}
\end{enumerate}

        \end{multicols}
    
        \naloga*

        \begin{multicols}{2}
        \noindent
        Kateri pogoj preverja spodnja funkcija?
        \begin{minted}[autogobble]{python}
        
            def f(besedilo):
                for x in besedilo:
                    if x in 'aeiouAEIOU':
                        return False
                return True
            
        \end{minted}

        \begin{enumerate}[(a)]
\item ali niz vsebuje samo samoglasnike
\item ali niz ne vsebuje nobenega samoglasnika
\item ali niz vsebuje znak, ki ni samoglasnik
\item ali niz vsebuje kakšen samoglasnik
\end{enumerate}

        \end{multicols}
    
        \naloga*
        \begin{multicols}{2}
        \noindent
        V vsak prostor vpišite \textbf{natanko en znak} tako, da bo dobljeni program v spremenljivko \inlinepy{vs} shranil vsoto števil \inlinepy{p} in \inlinepy{q}:
        
        \columnbreak
        \begin{minted}[baselinestretch=1.2,escapeinside=||]{python}
        vs = |\answerbox{0.5}|
        while p > |\answerbox{0.5}|:
            vs += |\answerbox{0.5}|
            |\answerbox{0.5}| -= 1
        \end{minted}
        \end{multicols}
    
        \clearpage
        \naloga
        
        Katera izmed spodnjih funkcij izračuna ostanek pri deljenju naravnega števila \inlinepy{m} z naravnim številom \inlinepy{n}?
    
        \begin{multicols}{2}
        \begin{enumerate}[(a)]
\item 
            \begin{minted}[autogobble]{python}
            def ostanek(m, n):
                if m < n:
                    return m
                else:
                    return ostanek(m - n, n)
            \end{minted}
        
\item 
                \begin{minted}[autogobble]{python}
                def ostanek(m, n):
                    if m < n:
                        return 0
                    else:
                        return ostanek(m - n, n)
                \end{minted}
            
\item 
                \begin{minted}[autogobble]{python}
                def ostanek(m, n):
                    if n == 0:
                        return m
                    else:
                        return ostanek(n, m % n)
                \end{minted}
            
\item 
                \begin{minted}[autogobble]{python}
                def ostanek(m, n):
                    if m < n:
                        return m
                    else:
                        return ostanek(m % n)
                \end{minted}
            
\end{enumerate}

        \end{multicols}
    
        \naloga*
        
        Katera izmed funkcij vrača drugačne rezultate kot ostale?
    
        \begin{multicols}{2}
        \begin{enumerate}[(a)]
\item 
                \begin{minted}[autogobble]{python}
                def g(c, b, a):
                    if not b:
                        return False
                    else:
                        return c and a
                \end{minted}
            
\item 
            \begin{minted}[autogobble]{python}
            def g(c, b, a):
                if not a:
                    return c and b
                else:
                    return False
            \end{minted}
        
\item 
                \begin{minted}[autogobble]{python}
                def g(c, b, a):
                    if c:
                        return b and a
                    else:
                        return False
                \end{minted}
            
\item 
                \begin{minted}[autogobble]{python}
                def g(c, b, a):
                    if c and b:
                        return a
                    else:
                        return False
                \end{minted}
            
\end{enumerate}

        \end{multicols}
    
        \naloga*
        
        Kateri izmed spodnjih programov ima drugačen izpis kot ostali?
    
        \begin{multicols}{2}
        \begin{enumerate}[(a)]
\item 
                \begin{minted}[autogobble]{python}
                    for n in range(1, 250, 5):
                        if n < 50:
                            print(n)
                \end{minted}
            
\item 
                \begin{minted}[autogobble]{python}
                    for n in range(1, 50):
                        if n % 5 == 1:
                            print(n)
                        continue
                \end{minted}
            
\item 
                \begin{minted}[autogobble]{python}
                    for n in range(1, 250, 5):
                        if n > 50:
                            break
                        print(n)
                \end{minted}
            
\item 
            \begin{minted}[autogobble]{python}
                for n in range(1, 50):
                    if n % 5 != 0:
                        continue
                    print(n)
            \end{minted}
        
\end{enumerate}

        \end{multicols}
    
        \naloga*
        \begin{multicols}{2}
        \noindent
        Napišite primer vrednosti spremenljivk \inlinepy{sez1} in \inlinepy{sez2}, za kateri klic \inlinepy{h(sez1, sez2)} vrne \inlinepy{True}.
        \begin{minted}[baselinestretch=1.2, escapeinside=||]{python}
        sez1 = |\answerbox{3}|
        sez2 = |\answerbox{3}|
        \end{minted}
        \vfil
        \columnbreak
        \begin{minted}[autogobble]{python}
        def h(sez1, sez2):
            if len(sez1) != len(sez2) or len(sez1) < 3:
                return False
            for i in range(len(sez1)):
                if i % 2 == 1 and sez1[i] != sez2[i]:
                    return False
            return True
        \end{minted}
        \end{multicols}
    
        \naloga*
        \begin{multicols}{2}
        \noindent 
        S številkami od $0$ do $4$ označite vrstni red, v katerem moramo izvesti ukaze na desni, da bo na koncu spremenljivka \inlinepy{sez2} kazala na seznam \inlinepy{[5, 8, 9]}?
    
        \columnbreak
        \noindent
        \begin{minted}[baselinestretch=1.2, escapeinside=||]{python}
|\answerbox{0.5}| sez2 = sez1 + sez2
|\answerbox{0.5}| sez1 = [8]
|\answerbox{0.5}| sez2 = sez1
|\answerbox{0.5}| sez1 = [5]
|\answerbox{0.5}| sez2.append(9)

        \end{minted}
        \end{multicols}
    
            \izpit[ucilnica=205, naloge=-1]{Uvod v programiranje: Kolokvij \#004}{27.\ marec 2019}{
                Pri vsaki nalogi obkrožite črko pred pravilnim odgovorom ali vpišite pravilno vrednost v ustrezen prostor. \\
                Čas reševanja je 30 minut. Veliko uspeha!
            }
            
        \naloga*
        
        Kateri izmed programov pri začetnem stanju
            \inlinepy{k = 7} in
            \inlinepy{m = 8}
        nastavi vrednosti
            \inlinepy{k = 8},
            \inlinepy{m = 7} in
            \inlinepy{n = 15}?
    
        \begin{multicols}{4}
        \begin{enumerate}[(a)]
\item 
                \begin{minted}[autogobble]{python}
                k = m
                m = n
                n = k
                n = k + m
                \end{minted}
            
\item 
            \begin{minted}[autogobble]{python}
            n = k
            k = m
            m = n
            n = k + m
            \end{minted}
        
\item 
                \begin{minted}[autogobble]{python}
                n = k
                m = n
                k = m
                n = k + m
                \end{minted}
            
\item 
                \begin{minted}[autogobble]{python}
                k = m
                n = m
                m = k
                n = k + m
                \end{minted}
            
\end{enumerate}

        \end{multicols}
    
        \naloga*
        \begin{multicols}{2}
        \noindent
        Kakšne vrste napak vsebuje program na desni?

        \begin{enumerate}[(a)]
\item vsebinske napake, zaradi katerih Python izračuna napačen rezultat
\item napake, zaradi katerih Python prekine z izvajanjem programa
\item oblikovne napake, ki ne vplivajo na pravilnost rezultata
\item sintaktične napake, zaradi katerih Python programa noče izvesti.
\end{enumerate}

        \columnbreak

        \begin{minted}[baselinestretch=1.2,escapeinside=||, autogobble]{python}
        
            def fibonacci(n):
                if n <= 0:
                    return 0
                elif n == 1:
                    return 1
                else:
                    a = fibonacci(n - '1')
                    b = fib(n - 2)
                    return a + b
            
        \end{minted}

        \end{multicols}

    
        \naloga*
        Katere vrstice izpiše klic \inlinepy{print(p(q(6)))}, če sta funkciji \inlinepy{p} in \inlinepy{q} definirani kot spodaj?

        \begin{multicols}{2}
        \begin{minted}[autogobble]{python}
        
            def p(a):
                print(a)
                return a + 9

            def q(b):
                return 5 * b
                print(b)
        
        \end{minted}

        \begin{enumerate}[(a)]
\item \inlinepy{[6, 39]}
\item \inlinepy{[39]}
\item \inlinepy{[30, 39]}
\item \inlinepy{[6, 30, 39]}
\end{enumerate}

        \end{multicols}
    
        \naloga*

        \begin{multicols}{2}
        \noindent
        Kateri pogoj preverja spodnja funkcija?
        \begin{minted}[autogobble]{python}
        
            def f(stavek):
                for znak in stavek:
                    if znak not in 'aeiouAEIOU':
                        return True
                return False
            
        \end{minted}

        \begin{enumerate}[(a)]
\item ali niz ne vsebuje nobenega samoglasnika
\item ali niz vsebuje kakšen samoglasnik
\item ali niz vsebuje samo samoglasnike
\item ali niz vsebuje znak, ki ni samoglasnik
\end{enumerate}

        \end{multicols}
    
        \naloga*
        \begin{multicols}{2}
        \noindent
        V vsak prostor vpišite \textbf{natanko en znak} tako, da bo dobljeni program v spremenljivko \inlinepy{zm} shranil zmnožek števil \inlinepy{y} in \inlinepy{x}:
        
        \columnbreak
        \begin{minted}[baselinestretch=1.2,escapeinside=||]{python}
        zm = |\answerbox{0.5}|
        while y > |\answerbox{0.5}|:
            zm += |\answerbox{0.5}|
            |\answerbox{0.5}| -= 1
        \end{minted}
        \end{multicols}
    
        \clearpage
        \naloga
        
        Katera izmed spodnjih funkcij izračuna ostanek pri deljenju naravnega števila \inlinepy{u} z naravnim številom \inlinepy{v}?
    
        \begin{multicols}{2}
        \begin{enumerate}[(a)]
\item 
                \begin{minted}[autogobble]{python}
                def ostanek(u, v):
                    if u < v:
                        return u
                    else:
                        return ostanek(u % v)
                \end{minted}
            
\item 
                \begin{minted}[autogobble]{python}
                def ostanek(u, v):
                    if v == 0:
                        return u
                    else:
                        return ostanek(v, u % v)
                \end{minted}
            
\item 
                \begin{minted}[autogobble]{python}
                def ostanek(u, v):
                    if u < v:
                        return 0
                    else:
                        return ostanek(u - v, v)
                \end{minted}
            
\item 
            \begin{minted}[autogobble]{python}
            def ostanek(u, v):
                if u < v:
                    return u
                else:
                    return ostanek(u - v, v)
            \end{minted}
        
\end{enumerate}

        \end{multicols}
    
        \naloga*
        
        Katera izmed funkcij vrača drugačne rezultate kot ostale?
    
        \begin{multicols}{2}
        \begin{enumerate}[(a)]
\item 
                \begin{minted}[autogobble]{python}
                def h(y, x, z):
                    if not x:
                        return False
                    else:
                        return y and z
                \end{minted}
            
\item 
                \begin{minted}[autogobble]{python}
                def h(y, x, z):
                    if y:
                        return x and z
                    else:
                        return False
                \end{minted}
            
\item 
                \begin{minted}[autogobble]{python}
                def h(y, x, z):
                    if y and x:
                        return z
                    else:
                        return False
                \end{minted}
            
\item 
            \begin{minted}[autogobble]{python}
            def h(y, x, z):
                if not z:
                    return y and x
                else:
                    return False
            \end{minted}
        
\end{enumerate}

        \end{multicols}
    
        \naloga*
        
        Kateri izmed spodnjih programov ima drugačen izpis kot ostali?
    
        \begin{multicols}{2}
        \begin{enumerate}[(a)]
\item 
            \begin{minted}[autogobble]{python}
                for j in range(1, 10):
                    if j % 2 != 0:
                        continue
                    print(j)
            \end{minted}
        
\item 
                \begin{minted}[autogobble]{python}
                    for j in range(1, 20, 2):
                        if j < 10:
                            print(j)
                \end{minted}
            
\item 
                \begin{minted}[autogobble]{python}
                    for j in range(1, 10):
                        if j % 2 == 1:
                            print(j)
                        continue
                \end{minted}
            
\item 
                \begin{minted}[autogobble]{python}
                    for j in range(1, 20, 2):
                        if j > 10:
                            break
                        print(j)
                \end{minted}
            
\end{enumerate}

        \end{multicols}
    
        \naloga*
        \begin{multicols}{2}
        \noindent
        Napišite primer vrednosti spremenljivk \inlinepy{niz1} in \inlinepy{niz2}, za kateri klic \inlinepy{f(niz1, niz2)} vrne \inlinepy{True}.
        \begin{minted}[baselinestretch=1.2, escapeinside=||]{python}
        niz1 = |\answerbox{3}|
        niz2 = |\answerbox{3}|
        \end{minted}
        \vfil
        \columnbreak
        \begin{minted}[autogobble]{python}
        def f(niz1, niz2):
            if len(niz1) != len(niz2) or len(niz1) < 3:
                return False
            for j in range(len(niz1)):
                if j % 2 == 1 and niz1[j] != niz2[j]:
                    return False
            return True
        \end{minted}
        \end{multicols}
    
        \naloga*
        \begin{multicols}{2}
        \noindent 
        S številkami od $0$ do $4$ označite vrstni red, v katerem moramo izvesti ukaze na desni, da bo na koncu spremenljivka \inlinepy{lst2} kazala na seznam \inlinepy{[4, 7, 2]}?
    
        \columnbreak
        \noindent
        \begin{minted}[baselinestretch=1.2, escapeinside=||]{python}
|\answerbox{0.5}| lst1 = [4]
|\answerbox{0.5}| lst2 = lst1
|\answerbox{0.5}| lst2 = lst1 + lst2
|\answerbox{0.5}| lst2.append(2)
|\answerbox{0.5}| lst1 = [7]

        \end{minted}
        \end{multicols}
    
            \izpit[ucilnica=205, naloge=-1]{Uvod v programiranje: Kolokvij \#005}{27.\ marec 2019}{
                Pri vsaki nalogi obkrožite črko pred pravilnim odgovorom ali vpišite pravilno vrednost v ustrezen prostor. \\
                Čas reševanja je 30 minut. Veliko uspeha!
            }
            
        \naloga*
        
        Kateri izmed programov pri začetnem stanju
            \inlinepy{u = 4} in
            \inlinepy{v = 9}
        nastavi vrednosti
            \inlinepy{u = 9},
            \inlinepy{v = 4} in
            \inlinepy{w = 13}?
    
        \begin{multicols}{4}
        \begin{enumerate}[(a)]
\item 
            \begin{minted}[autogobble]{python}
            w = u
            u = v
            v = w
            w = u + v
            \end{minted}
        
\item 
                \begin{minted}[autogobble]{python}
                u = v
                v = w
                w = u
                w = u + v
                \end{minted}
            
\item 
                \begin{minted}[autogobble]{python}
                w = u
                v = w
                u = v
                w = u + v
                \end{minted}
            
\item 
                \begin{minted}[autogobble]{python}
                u = v
                w = v
                v = u
                w = u + v
                \end{minted}
            
\end{enumerate}

        \end{multicols}
    
        \naloga*
        \begin{multicols}{2}
        \noindent
        Kakšne vrste napak vsebuje program na desni?

        \begin{enumerate}[(a)]
\item sintaktične napake, zaradi katerih Python programa noče izvesti.
\item oblikovne napake, ki ne vplivajo na pravilnost rezultata
\item vsebinske napake, zaradi katerih Python izračuna napačen rezultat
\item napake, zaradi katerih Python prekine z izvajanjem programa
\end{enumerate}

        \columnbreak

        \begin{minted}[baselinestretch=1.2,escapeinside=||, autogobble]{python}
        
            def fibonacci(n):
                if n <= 0:
                    return 0
                elif n == 1:
                    return 0
                else:
                    a = fibonacci(n - 1)
                    b = fibonacci(n - 3)
                    return a * b
            
        \end{minted}

        \end{multicols}

    
        \naloga*
        Katere vrstice izpiše klic \inlinepy{print(a(b(1)))}, če sta funkciji \inlinepy{a} in \inlinepy{b} definirani kot spodaj?

        \begin{multicols}{2}
        \begin{minted}[autogobble]{python}
        
            def a(x):
                return x + 2
                print(x)

            def b(y):
                print(y)
                return 4 * y
        
        \end{minted}

        \begin{enumerate}[(a)]
\item \inlinepy{[1, 6]}
\item \inlinepy{[4, 6]}
\item \inlinepy{[1, 4, 6]}
\item \inlinepy{[6]}
\end{enumerate}

        \end{multicols}
    
        \naloga*

        \begin{multicols}{2}
        \noindent
        Kateri pogoj preverja spodnja funkcija?
        \begin{minted}[autogobble]{python}
        
            def f(besedilo):
                for znak in besedilo:
                    if znak not in 'aeiouAEIOU':
                        return False
                return True
            
        \end{minted}

        \begin{enumerate}[(a)]
\item ali niz vsebuje kakšen samoglasnik
\item ali niz vsebuje samo samoglasnike
\item ali niz ne vsebuje nobenega samoglasnika
\item ali niz vsebuje znak, ki ni samoglasnik
\end{enumerate}

        \end{multicols}
    
        \naloga*
        \begin{multicols}{2}
        \noindent
        V vsak prostor vpišite \textbf{natanko en znak} tako, da bo dobljeni program v spremenljivko \inlinepy{vs} shranil vsoto števil \inlinepy{p} in \inlinepy{q}:
        
        \columnbreak
        \begin{minted}[baselinestretch=1.2,escapeinside=||]{python}
        vs = |\answerbox{0.5}|
        while p > |\answerbox{0.5}|:
            vs += |\answerbox{0.5}|
            |\answerbox{0.5}| -= 1
        \end{minted}
        \end{multicols}
    
        \clearpage
        \naloga
        
        Katera izmed spodnjih funkcij izračuna ostanek pri deljenju naravnega števila \inlinepy{x} z naravnim številom \inlinepy{y}?
    
        \begin{multicols}{2}
        \begin{enumerate}[(a)]
\item 
            \begin{minted}[autogobble]{python}
            def ostanek(x, y):
                if x < y:
                    return x
                else:
                    return ostanek(x - y, y)
            \end{minted}
        
\item 
                \begin{minted}[autogobble]{python}
                def ostanek(x, y):
                    if x < y:
                        return 0
                    else:
                        return ostanek(x - y, y)
                \end{minted}
            
\item 
                \begin{minted}[autogobble]{python}
                def ostanek(x, y):
                    if x < y:
                        return x
                    else:
                        return ostanek(x % y)
                \end{minted}
            
\item 
                \begin{minted}[autogobble]{python}
                def ostanek(x, y):
                    if y == 0:
                        return x
                    else:
                        return ostanek(y, x % y)
                \end{minted}
            
\end{enumerate}

        \end{multicols}
    
        \naloga*
        
        Katera izmed funkcij vrača drugačne rezultate kot ostale?
    
        \begin{multicols}{2}
        \begin{enumerate}[(a)]
\item 
                \begin{minted}[autogobble]{python}
                def h(z, x, y):
                    if not x:
                        return False
                    else:
                        return z and y
                \end{minted}
            
\item 
                \begin{minted}[autogobble]{python}
                def h(z, x, y):
                    if z and x:
                        return y
                    else:
                        return False
                \end{minted}
            
\item 
                \begin{minted}[autogobble]{python}
                def h(z, x, y):
                    if z:
                        return x and y
                    else:
                        return False
                \end{minted}
            
\item 
            \begin{minted}[autogobble]{python}
            def h(z, x, y):
                if not y:
                    return z and x
                else:
                    return False
            \end{minted}
        
\end{enumerate}

        \end{multicols}
    
        \naloga*
        
        Kateri izmed spodnjih programov ima drugačen izpis kot ostali?
    
        \begin{multicols}{2}
        \begin{enumerate}[(a)]
\item 
            \begin{minted}[autogobble]{python}
                for i in range(1, 10):
                    if i % 3 != 0:
                        continue
                    print(i)
            \end{minted}
        
\item 
                \begin{minted}[autogobble]{python}
                    for i in range(1, 10):
                        if i % 3 == 1:
                            print(i)
                        continue
                \end{minted}
            
\item 
                \begin{minted}[autogobble]{python}
                    for i in range(1, 30, 3):
                        if i > 10:
                            break
                        print(i)
                \end{minted}
            
\item 
                \begin{minted}[autogobble]{python}
                    for i in range(1, 30, 3):
                        if i < 10:
                            print(i)
                \end{minted}
            
\end{enumerate}

        \end{multicols}
    
        \naloga*
        \begin{multicols}{2}
        \noindent
        Napišite primer vrednosti spremenljivk \inlinepy{niz1} in \inlinepy{niz2}, za kateri klic \inlinepy{g(niz1, niz2)} vrne \inlinepy{True}.
        \begin{minted}[baselinestretch=1.2, escapeinside=||]{python}
        niz1 = |\answerbox{3}|
        niz2 = |\answerbox{3}|
        \end{minted}
        \vfil
        \columnbreak
        \begin{minted}[autogobble]{python}
        def g(niz1, niz2):
            if len(niz1) != len(niz2) or len(niz1) < 3:
                return False
            for j in range(len(niz1)):
                if j % 2 == 0 and niz1[j] != niz2[j]:
                    return False
            return True
        \end{minted}
        \end{multicols}
    
        \naloga*
        \begin{multicols}{2}
        \noindent 
        S številkami od $0$ do $4$ označite vrstni red, v katerem moramo izvesti ukaze na desni, da bo na koncu spremenljivka \inlinepy{sez1} kazala na seznam \inlinepy{[8, 3, 2]}?
    
        \columnbreak
        \noindent
        \begin{minted}[baselinestretch=1.2, escapeinside=||]{python}
|\answerbox{0.5}| sez1 = sez2
|\answerbox{0.5}| sez2 = [3]
|\answerbox{0.5}| sez2 = [8]
|\answerbox{0.5}| sez1.append(2)
|\answerbox{0.5}| sez1 = sez2 + sez1

        \end{minted}
        \end{multicols}
    
            \izpit[ucilnica=205, naloge=-1]{Uvod v programiranje: Kolokvij \#006}{27.\ marec 2019}{
                Pri vsaki nalogi obkrožite črko pred pravilnim odgovorom ali vpišite pravilno vrednost v ustrezen prostor. \\
                Čas reševanja je 30 minut. Veliko uspeha!
            }
            
        \naloga*
        
        Kateri izmed programov pri začetnem stanju
            \inlinepy{a = 3} in
            \inlinepy{b = 4}
        nastavi vrednosti
            \inlinepy{a = 4},
            \inlinepy{b = 3} in
            \inlinepy{c = 7}?
    
        \begin{multicols}{4}
        \begin{enumerate}[(a)]
\item 
            \begin{minted}[autogobble]{python}
            c = a
            a = b
            b = c
            c = a + b
            \end{minted}
        
\item 
                \begin{minted}[autogobble]{python}
                a = b
                c = b
                b = a
                c = a + b
                \end{minted}
            
\item 
                \begin{minted}[autogobble]{python}
                a = b
                b = c
                c = a
                c = a + b
                \end{minted}
            
\item 
                \begin{minted}[autogobble]{python}
                c = a
                b = c
                a = b
                c = a + b
                \end{minted}
            
\end{enumerate}

        \end{multicols}
    
        \naloga*
        \begin{multicols}{2}
        \noindent
        Kakšne vrste napak vsebuje program na desni?

        \begin{enumerate}[(a)]
\item napake, zaradi katerih Python prekine z izvajanjem programa
\item oblikovne napake, ki ne vplivajo na pravilnost rezultata
\item vsebinske napake, zaradi katerih Python izračuna napačen rezultat
\item sintaktične napake, zaradi katerih Python programa noče izvesti.
\end{enumerate}

        \columnbreak

        \begin{minted}[baselinestretch=1.2,escapeinside=||, autogobble]{python}
        
            def fibonacci(n):
                if n <= 0:
                    return 0
                elif n == 1:
                    return 1
                else:
                    a = fibonacci(n - '1')
                    b = fib(n - 2)
                    return a + b
            
        \end{minted}

        \end{multicols}

    
        \naloga*
        Katere vrstice izpiše klic \inlinepy{print(a(b(1)))}, če sta funkciji \inlinepy{a} in \inlinepy{b} definirani kot spodaj?

        \begin{multicols}{2}
        \begin{minted}[autogobble]{python}
        
            def a(x):
                return x + 3
                print(x)

            def b(y):
                print(y)
                return 9 * y
        
        \end{minted}

        \begin{enumerate}[(a)]
\item \inlinepy{[9, 12]}
\item \inlinepy{[12]}
\item \inlinepy{[1, 12]}
\item \inlinepy{[1, 9, 12]}
\end{enumerate}

        \end{multicols}
    
        \naloga*

        \begin{multicols}{2}
        \noindent
        Kateri pogoj preverja spodnja funkcija?
        \begin{minted}[autogobble]{python}
        
            def f(besedilo):
                for x in besedilo:
                    if x in 'aeiouAEIOU':
                        return True
                return False
            
        \end{minted}

        \begin{enumerate}[(a)]
\item ali niz vsebuje kakšen samoglasnik
\item ali niz vsebuje znak, ki ni samoglasnik
\item ali niz ne vsebuje nobenega samoglasnika
\item ali niz vsebuje samo samoglasnike
\end{enumerate}

        \end{multicols}
    
        \naloga*
        \begin{multicols}{2}
        \noindent
        V vsak prostor vpišite \textbf{natanko en znak} tako, da bo dobljeni program v spremenljivko \inlinepy{zm} shranil zmnožek števil \inlinepy{y} in \inlinepy{x}:
        
        \columnbreak
        \begin{minted}[baselinestretch=1.2,escapeinside=||]{python}
        zm = |\answerbox{0.5}|
        while y > |\answerbox{0.5}|:
            zm += |\answerbox{0.5}|
            |\answerbox{0.5}| -= 1
        \end{minted}
        \end{multicols}
    
        \clearpage
        \naloga
        
        Katera izmed spodnjih funkcij izračuna ostanek pri deljenju naravnega števila \inlinepy{m} z naravnim številom \inlinepy{n}?
    
        \begin{multicols}{2}
        \begin{enumerate}[(a)]
\item 
            \begin{minted}[autogobble]{python}
            def ostanek(m, n):
                if m < n:
                    return m
                else:
                    return ostanek(m - n, n)
            \end{minted}
        
\item 
                \begin{minted}[autogobble]{python}
                def ostanek(m, n):
                    if m < n:
                        return m
                    else:
                        return ostanek(m % n)
                \end{minted}
            
\item 
                \begin{minted}[autogobble]{python}
                def ostanek(m, n):
                    if m < n:
                        return 0
                    else:
                        return ostanek(m - n, n)
                \end{minted}
            
\item 
                \begin{minted}[autogobble]{python}
                def ostanek(m, n):
                    if n == 0:
                        return m
                    else:
                        return ostanek(n, m % n)
                \end{minted}
            
\end{enumerate}

        \end{multicols}
    
        \naloga*
        
        Katera izmed funkcij vrača drugačne rezultate kot ostale?
    
        \begin{multicols}{2}
        \begin{enumerate}[(a)]
\item 
                \begin{minted}[autogobble]{python}
                def f(x, y, z):
                    if not y:
                        return False
                    else:
                        return x and z
                \end{minted}
            
\item 
                \begin{minted}[autogobble]{python}
                def f(x, y, z):
                    if x and y:
                        return z
                    else:
                        return False
                \end{minted}
            
\item 
                \begin{minted}[autogobble]{python}
                def f(x, y, z):
                    if x:
                        return y and z
                    else:
                        return False
                \end{minted}
            
\item 
            \begin{minted}[autogobble]{python}
            def f(x, y, z):
                if not z:
                    return x and y
                else:
                    return False
            \end{minted}
        
\end{enumerate}

        \end{multicols}
    
        \naloga*
        
        Kateri izmed spodnjih programov ima drugačen izpis kot ostali?
    
        \begin{multicols}{2}
        \begin{enumerate}[(a)]
\item 
                \begin{minted}[autogobble]{python}
                    for n in range(1, 150, 3):
                        if n > 50:
                            break
                        print(n)
                \end{minted}
            
\item 
                \begin{minted}[autogobble]{python}
                    for n in range(1, 150, 3):
                        if n < 50:
                            print(n)
                \end{minted}
            
\item 
            \begin{minted}[autogobble]{python}
                for n in range(1, 50):
                    if n % 3 != 0:
                        continue
                    print(n)
            \end{minted}
        
\item 
                \begin{minted}[autogobble]{python}
                    for n in range(1, 50):
                        if n % 3 == 1:
                            print(n)
                        continue
                \end{minted}
            
\end{enumerate}

        \end{multicols}
    
        \naloga*
        \begin{multicols}{2}
        \noindent
        Napišite primer vrednosti spremenljivk \inlinepy{niz1} in \inlinepy{niz2}, za kateri klic \inlinepy{h(niz1, niz2)} vrne \inlinepy{True}.
        \begin{minted}[baselinestretch=1.2, escapeinside=||]{python}
        niz1 = |\answerbox{3}|
        niz2 = |\answerbox{3}|
        \end{minted}
        \vfil
        \columnbreak
        \begin{minted}[autogobble]{python}
        def h(niz1, niz2):
            if len(niz1) != len(niz2) or len(niz1) < 3:
                return False
            for j in range(len(niz1)):
                if j % 2 == 0 and niz1[j] != niz2[j]:
                    return False
            return True
        \end{minted}
        \end{multicols}
    
        \naloga*
        \begin{multicols}{2}
        \noindent 
        S številkami od $0$ do $4$ označite vrstni red, v katerem moramo izvesti ukaze na desni, da bo na koncu spremenljivka \inlinepy{sez2} kazala na seznam \inlinepy{[2, 8, 4]}?
    
        \columnbreak
        \noindent
        \begin{minted}[baselinestretch=1.2, escapeinside=||]{python}
|\answerbox{0.5}| sez2.append(4)
|\answerbox{0.5}| sez2 = sez1 + sez2
|\answerbox{0.5}| sez1 = [2]
|\answerbox{0.5}| sez2 = sez1
|\answerbox{0.5}| sez1 = [8]

        \end{minted}
        \end{multicols}
    
            \izpit[ucilnica=205, naloge=-1]{Uvod v programiranje: Kolokvij \#007}{27.\ marec 2019}{
                Pri vsaki nalogi obkrožite črko pred pravilnim odgovorom ali vpišite pravilno vrednost v ustrezen prostor. \\
                Čas reševanja je 30 minut. Veliko uspeha!
            }
            
        \naloga*
        
        Kateri izmed programov pri začetnem stanju
            \inlinepy{k = 2} in
            \inlinepy{m = 4}
        nastavi vrednosti
            \inlinepy{k = 4},
            \inlinepy{m = 2} in
            \inlinepy{n = 6}?
    
        \begin{multicols}{4}
        \begin{enumerate}[(a)]
\item 
                \begin{minted}[autogobble]{python}
                k = m
                n = m
                m = k
                n = k + m
                \end{minted}
            
\item 
                \begin{minted}[autogobble]{python}
                k = m
                m = n
                n = k
                n = k + m
                \end{minted}
            
\item 
            \begin{minted}[autogobble]{python}
            n = k
            k = m
            m = n
            n = k + m
            \end{minted}
        
\item 
                \begin{minted}[autogobble]{python}
                n = k
                m = n
                k = m
                n = k + m
                \end{minted}
            
\end{enumerate}

        \end{multicols}
    
        \naloga*
        \begin{multicols}{2}
        \noindent
        Kakšne vrste napak vsebuje program na desni?

        \begin{enumerate}[(a)]
\item napake, zaradi katerih Python prekine z izvajanjem programa
\item sintaktične napake, zaradi katerih Python programa noče izvesti.
\item vsebinske napake, zaradi katerih Python izračuna napačen rezultat
\item oblikovne napake, ki ne vplivajo na pravilnost rezultata
\end{enumerate}

        \columnbreak

        \begin{minted}[baselinestretch=1.2,escapeinside=||, autogobble]{python}
        
            define fibonacci(n):
                if n <= 0:
                    return 0
                elif n == 1
                    return 1
                else:
                    a = fibonacci(n - 1)
                    b = fibonacci(n - 2)
                  return a + b
            
        \end{minted}

        \end{multicols}

    
        \naloga*
        Katere vrstice izpiše klic \inlinepy{print(f(g(6)))}, če sta funkciji \inlinepy{f} in \inlinepy{g} definirani kot spodaj?

        \begin{multicols}{2}
        \begin{minted}[autogobble]{python}
        
            def f(x):
                print(x)
                return x + 7

            def g(y):
                return 1 * y
                print(y)
        
        \end{minted}

        \begin{enumerate}[(a)]
\item \inlinepy{[6, 6, 13]}
\item \inlinepy{[6, 13]}
\item \inlinepy{[6, 13]}
\item \inlinepy{[13]}
\end{enumerate}

        \end{multicols}
    
        \naloga*

        \begin{multicols}{2}
        \noindent
        Kateri pogoj preverja spodnja funkcija?
        \begin{minted}[autogobble]{python}
        
            def f(stavek):
                for z in stavek:
                    if z in 'aeiouAEIOU':
                        return True
                return False
            
        \end{minted}

        \begin{enumerate}[(a)]
\item ali niz vsebuje znak, ki ni samoglasnik
\item ali niz vsebuje samo samoglasnike
\item ali niz vsebuje kakšen samoglasnik
\item ali niz ne vsebuje nobenega samoglasnika
\end{enumerate}

        \end{multicols}
    
        \naloga*
        \begin{multicols}{2}
        \noindent
        V vsak prostor vpišite \textbf{natanko en znak} tako, da bo dobljeni program v spremenljivko \inlinepy{zm} shranil zmnožek števil \inlinepy{x} in \inlinepy{y}:
        
        \columnbreak
        \begin{minted}[baselinestretch=1.2,escapeinside=||]{python}
        zm = |\answerbox{0.5}|
        while x > |\answerbox{0.5}|:
            zm += |\answerbox{0.5}|
            |\answerbox{0.5}| -= 1
        \end{minted}
        \end{multicols}
    
        \clearpage
        \naloga
        
        Katera izmed spodnjih funkcij izračuna ostanek pri deljenju naravnega števila \inlinepy{u} z naravnim številom \inlinepy{v}?
    
        \begin{multicols}{2}
        \begin{enumerate}[(a)]
\item 
                \begin{minted}[autogobble]{python}
                def ostanek(u, v):
                    if u < v:
                        return u
                    else:
                        return ostanek(u % v)
                \end{minted}
            
\item 
                \begin{minted}[autogobble]{python}
                def ostanek(u, v):
                    if v == 0:
                        return u
                    else:
                        return ostanek(v, u % v)
                \end{minted}
            
\item 
                \begin{minted}[autogobble]{python}
                def ostanek(u, v):
                    if u < v:
                        return 0
                    else:
                        return ostanek(u - v, v)
                \end{minted}
            
\item 
            \begin{minted}[autogobble]{python}
            def ostanek(u, v):
                if u < v:
                    return u
                else:
                    return ostanek(u - v, v)
            \end{minted}
        
\end{enumerate}

        \end{multicols}
    
        \naloga*
        
        Katera izmed funkcij vrača drugačne rezultate kot ostale?
    
        \begin{multicols}{2}
        \begin{enumerate}[(a)]
\item 
                \begin{minted}[autogobble]{python}
                def f(y, z, x):
                    if not z:
                        return False
                    else:
                        return y and x
                \end{minted}
            
\item 
                \begin{minted}[autogobble]{python}
                def f(y, z, x):
                    if y and z:
                        return x
                    else:
                        return False
                \end{minted}
            
\item 
            \begin{minted}[autogobble]{python}
            def f(y, z, x):
                if not x:
                    return y and z
                else:
                    return False
            \end{minted}
        
\item 
                \begin{minted}[autogobble]{python}
                def f(y, z, x):
                    if y:
                        return z and x
                    else:
                        return False
                \end{minted}
            
\end{enumerate}

        \end{multicols}
    
        \naloga*
        
        Kateri izmed spodnjih programov ima drugačen izpis kot ostali?
    
        \begin{multicols}{2}
        \begin{enumerate}[(a)]
\item 
            \begin{minted}[autogobble]{python}
                for x in range(1, 20):
                    if x % 2 != 0:
                        continue
                    print(x)
            \end{minted}
        
\item 
                \begin{minted}[autogobble]{python}
                    for x in range(1, 40, 2):
                        if x > 20:
                            break
                        print(x)
                \end{minted}
            
\item 
                \begin{minted}[autogobble]{python}
                    for x in range(1, 40, 2):
                        if x < 20:
                            print(x)
                \end{minted}
            
\item 
                \begin{minted}[autogobble]{python}
                    for x in range(1, 20):
                        if x % 2 == 1:
                            print(x)
                        continue
                \end{minted}
            
\end{enumerate}

        \end{multicols}
    
        \naloga*
        \begin{multicols}{2}
        \noindent
        Napišite primer vrednosti spremenljivk \inlinepy{lst1} in \inlinepy{lst2}, za kateri klic \inlinepy{g(lst1, lst2)} vrne \inlinepy{True}.
        \begin{minted}[baselinestretch=1.2, escapeinside=||]{python}
        lst1 = |\answerbox{3}|
        lst2 = |\answerbox{3}|
        \end{minted}
        \vfil
        \columnbreak
        \begin{minted}[autogobble]{python}
        def g(lst1, lst2):
            if len(lst1) != len(lst2) or len(lst1) < 3:
                return False
            for i in range(len(lst1)):
                if i % 2 == 0 and lst1[i] != lst2[i]:
                    return False
            return True
        \end{minted}
        \end{multicols}
    
        \naloga*
        \begin{multicols}{2}
        \noindent 
        S številkami od $0$ do $4$ označite vrstni red, v katerem moramo izvesti ukaze na desni, da bo na koncu spremenljivka \inlinepy{lst2} kazala na seznam \inlinepy{[9, 6, 7]}?
    
        \columnbreak
        \noindent
        \begin{minted}[baselinestretch=1.2, escapeinside=||]{python}
|\answerbox{0.5}| lst2.append(7)
|\answerbox{0.5}| lst1 = [6]
|\answerbox{0.5}| lst2 = lst1
|\answerbox{0.5}| lst1 = [9]
|\answerbox{0.5}| lst2 = lst1 + lst2

        \end{minted}
        \end{multicols}
    
            \izpit[ucilnica=205, naloge=-1]{Uvod v programiranje: Kolokvij \#008}{27.\ marec 2019}{
                Pri vsaki nalogi obkrožite črko pred pravilnim odgovorom ali vpišite pravilno vrednost v ustrezen prostor. \\
                Čas reševanja je 30 minut. Veliko uspeha!
            }
            
        \naloga*
        
        Kateri izmed programov pri začetnem stanju
            \inlinepy{a = 4} in
            \inlinepy{b = 7}
        nastavi vrednosti
            \inlinepy{a = 7},
            \inlinepy{b = 4} in
            \inlinepy{c = 11}?
    
        \begin{multicols}{4}
        \begin{enumerate}[(a)]
\item 
                \begin{minted}[autogobble]{python}
                a = b
                c = b
                b = a
                c = a + b
                \end{minted}
            
\item 
                \begin{minted}[autogobble]{python}
                c = a
                b = c
                a = b
                c = a + b
                \end{minted}
            
\item 
            \begin{minted}[autogobble]{python}
            c = a
            a = b
            b = c
            c = a + b
            \end{minted}
        
\item 
                \begin{minted}[autogobble]{python}
                a = b
                b = c
                c = a
                c = a + b
                \end{minted}
            
\end{enumerate}

        \end{multicols}
    
        \naloga*
        \begin{multicols}{2}
        \noindent
        Kakšne vrste napak vsebuje program na desni?

        \begin{enumerate}[(a)]
\item vsebinske napake, zaradi katerih Python izračuna napačen rezultat
\item oblikovne napake, ki ne vplivajo na pravilnost rezultata
\item napake, zaradi katerih Python prekine z izvajanjem programa
\item sintaktične napake, zaradi katerih Python programa noče izvesti.
\end{enumerate}

        \columnbreak

        \begin{minted}[baselinestretch=1.2,escapeinside=||, autogobble]{python}
        
            def fibonacci(n):
                if n<=0:
                    return (0)
                elif n==1:
                    return (1)
                else:
                    a = fibonacci(n - 1)
                    b = fibonacci(n - 2)
                    return   a + b
            
        \end{minted}

        \end{multicols}

    
        \naloga*
        Katere vrstice izpiše klic \inlinepy{print(f(g(9)))}, če sta funkciji \inlinepy{f} in \inlinepy{g} definirani kot spodaj?

        \begin{multicols}{2}
        \begin{minted}[autogobble]{python}
        
            def f(a):
                print(a)
                return a + 4

            def g(b):
                print(b)
                return 2 * b
        
        \end{minted}

        \begin{enumerate}[(a)]
\item \inlinepy{[9, 18, 22]}
\item \inlinepy{[18, 22]}
\item \inlinepy{[22]}
\item \inlinepy{[9, 22]}
\end{enumerate}

        \end{multicols}
    
        \naloga*

        \begin{multicols}{2}
        \noindent
        Kateri pogoj preverja spodnja funkcija?
        \begin{minted}[autogobble]{python}
        
            def f(stavek):
                for z in stavek:
                    if z in 'aeiouAEIOU':
                        return True
                return False
            
        \end{minted}

        \begin{enumerate}[(a)]
\item ali niz vsebuje samo samoglasnike
\item ali niz vsebuje kakšen samoglasnik
\item ali niz vsebuje znak, ki ni samoglasnik
\item ali niz ne vsebuje nobenega samoglasnika
\end{enumerate}

        \end{multicols}
    
        \naloga*
        \begin{multicols}{2}
        \noindent
        V vsak prostor vpišite \textbf{natanko en znak} tako, da bo dobljeni program v spremenljivko \inlinepy{vs} shranil vsoto števil \inlinepy{q} in \inlinepy{p}:
        
        \columnbreak
        \begin{minted}[baselinestretch=1.2,escapeinside=||]{python}
        vs = |\answerbox{0.5}|
        while q > |\answerbox{0.5}|:
            vs += |\answerbox{0.5}|
            |\answerbox{0.5}| -= 1
        \end{minted}
        \end{multicols}
    
        \clearpage
        \naloga
        
        Katera izmed spodnjih funkcij izračuna ostanek pri deljenju naravnega števila \inlinepy{m} z naravnim številom \inlinepy{n}?
    
        \begin{multicols}{2}
        \begin{enumerate}[(a)]
\item 
            \begin{minted}[autogobble]{python}
            def ostanek(m, n):
                if m < n:
                    return m
                else:
                    return ostanek(m - n, n)
            \end{minted}
        
\item 
                \begin{minted}[autogobble]{python}
                def ostanek(m, n):
                    if n == 0:
                        return m
                    else:
                        return ostanek(n, m % n)
                \end{minted}
            
\item 
                \begin{minted}[autogobble]{python}
                def ostanek(m, n):
                    if m < n:
                        return m
                    else:
                        return ostanek(m % n)
                \end{minted}
            
\item 
                \begin{minted}[autogobble]{python}
                def ostanek(m, n):
                    if m < n:
                        return 0
                    else:
                        return ostanek(m - n, n)
                \end{minted}
            
\end{enumerate}

        \end{multicols}
    
        \naloga*
        
        Katera izmed funkcij vrača drugačne rezultate kot ostale?
    
        \begin{multicols}{2}
        \begin{enumerate}[(a)]
\item 
                \begin{minted}[autogobble]{python}
                def h(z, x, y):
                    if not x:
                        return False
                    else:
                        return z and y
                \end{minted}
            
\item 
            \begin{minted}[autogobble]{python}
            def h(z, x, y):
                if not y:
                    return z and x
                else:
                    return False
            \end{minted}
        
\item 
                \begin{minted}[autogobble]{python}
                def h(z, x, y):
                    if z:
                        return x and y
                    else:
                        return False
                \end{minted}
            
\item 
                \begin{minted}[autogobble]{python}
                def h(z, x, y):
                    if z and x:
                        return y
                    else:
                        return False
                \end{minted}
            
\end{enumerate}

        \end{multicols}
    
        \naloga*
        
        Kateri izmed spodnjih programov ima drugačen izpis kot ostali?
    
        \begin{multicols}{2}
        \begin{enumerate}[(a)]
\item 
                \begin{minted}[autogobble]{python}
                    for x in range(1, 100, 5):
                        if x < 20:
                            print(x)
                \end{minted}
            
\item 
                \begin{minted}[autogobble]{python}
                    for x in range(1, 20):
                        if x % 5 == 1:
                            print(x)
                        continue
                \end{minted}
            
\item 
            \begin{minted}[autogobble]{python}
                for x in range(1, 20):
                    if x % 5 != 0:
                        continue
                    print(x)
            \end{minted}
        
\item 
                \begin{minted}[autogobble]{python}
                    for x in range(1, 100, 5):
                        if x > 20:
                            break
                        print(x)
                \end{minted}
            
\end{enumerate}

        \end{multicols}
    
        \naloga*
        \begin{multicols}{2}
        \noindent
        Napišite primer vrednosti spremenljivk \inlinepy{str1} in \inlinepy{str2}, za kateri klic \inlinepy{g(str1, str2)} vrne \inlinepy{True}.
        \begin{minted}[baselinestretch=1.2, escapeinside=||]{python}
        str1 = |\answerbox{3}|
        str2 = |\answerbox{3}|
        \end{minted}
        \vfil
        \columnbreak
        \begin{minted}[autogobble]{python}
        def g(str1, str2):
            if len(str1) != len(str2) or len(str1) < 3:
                return False
            for i in range(len(str1)):
                if i % 2 == 1 and str1[i] != str2[i]:
                    return False
            return True
        \end{minted}
        \end{multicols}
    
        \naloga*
        \begin{multicols}{2}
        \noindent 
        S številkami od $0$ do $4$ označite vrstni red, v katerem moramo izvesti ukaze na desni, da bo na koncu spremenljivka \inlinepy{sez1} kazala na seznam \inlinepy{[2, 9, 5]}?
    
        \columnbreak
        \noindent
        \begin{minted}[baselinestretch=1.2, escapeinside=||]{python}
|\answerbox{0.5}| sez2 = [2]
|\answerbox{0.5}| sez1 = sez2 + sez1
|\answerbox{0.5}| sez1.append(5)
|\answerbox{0.5}| sez1 = sez2
|\answerbox{0.5}| sez2 = [9]

        \end{minted}
        \end{multicols}
    
            \izpit[ucilnica=205, naloge=-1]{Uvod v programiranje: Kolokvij \#009}{27.\ marec 2019}{
                Pri vsaki nalogi obkrožite črko pred pravilnim odgovorom ali vpišite pravilno vrednost v ustrezen prostor. \\
                Čas reševanja je 30 minut. Veliko uspeha!
            }
            
        \naloga*
        
        Kateri izmed programov pri začetnem stanju
            \inlinepy{u = 8} in
            \inlinepy{v = 4}
        nastavi vrednosti
            \inlinepy{u = 4},
            \inlinepy{v = 8} in
            \inlinepy{w = 12}?
    
        \begin{multicols}{4}
        \begin{enumerate}[(a)]
\item 
            \begin{minted}[autogobble]{python}
            w = u
            u = v
            v = w
            w = u + v
            \end{minted}
        
\item 
                \begin{minted}[autogobble]{python}
                u = v
                v = w
                w = u
                w = u + v
                \end{minted}
            
\item 
                \begin{minted}[autogobble]{python}
                w = u
                v = w
                u = v
                w = u + v
                \end{minted}
            
\item 
                \begin{minted}[autogobble]{python}
                u = v
                w = v
                v = u
                w = u + v
                \end{minted}
            
\end{enumerate}

        \end{multicols}
    
        \naloga*
        \begin{multicols}{2}
        \noindent
        Kakšne vrste napak vsebuje program na desni?

        \begin{enumerate}[(a)]
\item oblikovne napake, ki ne vplivajo na pravilnost rezultata
\item napake, zaradi katerih Python prekine z izvajanjem programa
\item vsebinske napake, zaradi katerih Python izračuna napačen rezultat
\item sintaktične napake, zaradi katerih Python programa noče izvesti.
\end{enumerate}

        \columnbreak

        \begin{minted}[baselinestretch=1.2,escapeinside=||, autogobble]{python}
        
            def fibonacci(n):
                if n <= 0:
                    return 0
                elif n == 1:
                    return 0
                else:
                    a = fibonacci(n - 1)
                    b = fibonacci(n - 3)
                    return a * b
            
        \end{minted}

        \end{multicols}

    
        \naloga*
        Katere vrstice izpiše klic \inlinepy{print(f(g(6)))}, če sta funkciji \inlinepy{f} in \inlinepy{g} definirani kot spodaj?

        \begin{multicols}{2}
        \begin{minted}[autogobble]{python}
        
            def f(a):
                return a + 9
                print(a)

            def g(b):
                return 8 * b
                print(b)
        
        \end{minted}

        \begin{enumerate}[(a)]
\item \inlinepy{[57]}
\item \inlinepy{[6, 57]}
\item \inlinepy{[48, 57]}
\item \inlinepy{[6, 48, 57]}
\end{enumerate}

        \end{multicols}
    
        \naloga*

        \begin{multicols}{2}
        \noindent
        Kateri pogoj preverja spodnja funkcija?
        \begin{minted}[autogobble]{python}
        
            def f(besedilo):
                for x in besedilo:
                    if x not in 'aeiouAEIOU':
                        return False
                return True
            
        \end{minted}

        \begin{enumerate}[(a)]
\item ali niz ne vsebuje nobenega samoglasnika
\item ali niz vsebuje samo samoglasnike
\item ali niz vsebuje znak, ki ni samoglasnik
\item ali niz vsebuje kakšen samoglasnik
\end{enumerate}

        \end{multicols}
    
        \naloga*
        \begin{multicols}{2}
        \noindent
        V vsak prostor vpišite \textbf{natanko en znak} tako, da bo dobljeni program v spremenljivko \inlinepy{zm} shranil zmnožek števil \inlinepy{b} in \inlinepy{a}:
        
        \columnbreak
        \begin{minted}[baselinestretch=1.2,escapeinside=||]{python}
        zm = |\answerbox{0.5}|
        while b > |\answerbox{0.5}|:
            zm += |\answerbox{0.5}|
            |\answerbox{0.5}| -= 1
        \end{minted}
        \end{multicols}
    
        \clearpage
        \naloga
        
        Katera izmed spodnjih funkcij izračuna ostanek pri deljenju naravnega števila \inlinepy{u} z naravnim številom \inlinepy{v}?
    
        \begin{multicols}{2}
        \begin{enumerate}[(a)]
\item 
            \begin{minted}[autogobble]{python}
            def ostanek(u, v):
                if u < v:
                    return u
                else:
                    return ostanek(u - v, v)
            \end{minted}
        
\item 
                \begin{minted}[autogobble]{python}
                def ostanek(u, v):
                    if v == 0:
                        return u
                    else:
                        return ostanek(v, u % v)
                \end{minted}
            
\item 
                \begin{minted}[autogobble]{python}
                def ostanek(u, v):
                    if u < v:
                        return 0
                    else:
                        return ostanek(u - v, v)
                \end{minted}
            
\item 
                \begin{minted}[autogobble]{python}
                def ostanek(u, v):
                    if u < v:
                        return u
                    else:
                        return ostanek(u % v)
                \end{minted}
            
\end{enumerate}

        \end{multicols}
    
        \naloga*
        
        Katera izmed funkcij vrača drugačne rezultate kot ostale?
    
        \begin{multicols}{2}
        \begin{enumerate}[(a)]
\item 
                \begin{minted}[autogobble]{python}
                def h(p, q, r):
                    if p:
                        return q and r
                    else:
                        return False
                \end{minted}
            
\item 
                \begin{minted}[autogobble]{python}
                def h(p, q, r):
                    if p and q:
                        return r
                    else:
                        return False
                \end{minted}
            
\item 
                \begin{minted}[autogobble]{python}
                def h(p, q, r):
                    if not q:
                        return False
                    else:
                        return p and r
                \end{minted}
            
\item 
            \begin{minted}[autogobble]{python}
            def h(p, q, r):
                if not r:
                    return p and q
                else:
                    return False
            \end{minted}
        
\end{enumerate}

        \end{multicols}
    
        \naloga*
        
        Kateri izmed spodnjih programov ima drugačen izpis kot ostali?
    
        \begin{multicols}{2}
        \begin{enumerate}[(a)]
\item 
                \begin{minted}[autogobble]{python}
                    for x in range(1, 50):
                        if x % 2 == 1:
                            print(x)
                        continue
                \end{minted}
            
\item 
                \begin{minted}[autogobble]{python}
                    for x in range(1, 100, 2):
                        if x < 50:
                            print(x)
                \end{minted}
            
\item 
                \begin{minted}[autogobble]{python}
                    for x in range(1, 100, 2):
                        if x > 50:
                            break
                        print(x)
                \end{minted}
            
\item 
            \begin{minted}[autogobble]{python}
                for x in range(1, 50):
                    if x % 2 != 0:
                        continue
                    print(x)
            \end{minted}
        
\end{enumerate}

        \end{multicols}
    
        \naloga*
        \begin{multicols}{2}
        \noindent
        Napišite primer vrednosti spremenljivk \inlinepy{str1} in \inlinepy{str2}, za kateri klic \inlinepy{h(str1, str2)} vrne \inlinepy{True}.
        \begin{minted}[baselinestretch=1.2, escapeinside=||]{python}
        str1 = |\answerbox{3}|
        str2 = |\answerbox{3}|
        \end{minted}
        \vfil
        \columnbreak
        \begin{minted}[autogobble]{python}
        def h(str1, str2):
            if len(str1) != len(str2) or len(str1) < 3:
                return False
            for j in range(len(str1)):
                if j % 2 == 1 and str1[j] != str2[j]:
                    return False
            return True
        \end{minted}
        \end{multicols}
    
        \naloga*
        \begin{multicols}{2}
        \noindent 
        S številkami od $0$ do $4$ označite vrstni red, v katerem moramo izvesti ukaze na desni, da bo na koncu spremenljivka \inlinepy{sez2} kazala na seznam \inlinepy{[6, 8, 7]}?
    
        \columnbreak
        \noindent
        \begin{minted}[baselinestretch=1.2, escapeinside=||]{python}
|\answerbox{0.5}| sez2 = sez1
|\answerbox{0.5}| sez2 = sez1 + sez2
|\answerbox{0.5}| sez1 = [8]
|\answerbox{0.5}| sez1 = [6]
|\answerbox{0.5}| sez2.append(7)

        \end{minted}
        \end{multicols}
    
            \izpit[ucilnica=205, naloge=-1]{Uvod v programiranje: Kolokvij \#010}{27.\ marec 2019}{
                Pri vsaki nalogi obkrožite črko pred pravilnim odgovorom ali vpišite pravilno vrednost v ustrezen prostor. \\
                Čas reševanja je 30 minut. Veliko uspeha!
            }
            
        \naloga*
        
        Kateri izmed programov pri začetnem stanju
            \inlinepy{k = 3} in
            \inlinepy{m = 7}
        nastavi vrednosti
            \inlinepy{k = 7},
            \inlinepy{m = 3} in
            \inlinepy{n = 10}?
    
        \begin{multicols}{4}
        \begin{enumerate}[(a)]
\item 
            \begin{minted}[autogobble]{python}
            n = k
            k = m
            m = n
            n = k + m
            \end{minted}
        
\item 
                \begin{minted}[autogobble]{python}
                k = m
                m = n
                n = k
                n = k + m
                \end{minted}
            
\item 
                \begin{minted}[autogobble]{python}
                k = m
                n = m
                m = k
                n = k + m
                \end{minted}
            
\item 
                \begin{minted}[autogobble]{python}
                n = k
                m = n
                k = m
                n = k + m
                \end{minted}
            
\end{enumerate}

        \end{multicols}
    
        \naloga*
        \begin{multicols}{2}
        \noindent
        Kakšne vrste napak vsebuje program na desni?

        \begin{enumerate}[(a)]
\item oblikovne napake, ki ne vplivajo na pravilnost rezultata
\item napake, zaradi katerih Python prekine z izvajanjem programa
\item vsebinske napake, zaradi katerih Python izračuna napačen rezultat
\item sintaktične napake, zaradi katerih Python programa noče izvesti.
\end{enumerate}

        \columnbreak

        \begin{minted}[baselinestretch=1.2,escapeinside=||, autogobble]{python}
        
            def fibonacci(n):
                if n <= 0:
                    return 0
                elif n == 1:
                    return 0
                else:
                    a = fibonacci(n - 1)
                    b = fibonacci(n - 3)
                    return a * b
            
        \end{minted}

        \end{multicols}

    
        \naloga*
        Katere vrstice izpiše klic \inlinepy{print(a(b(3)))}, če sta funkciji \inlinepy{a} in \inlinepy{b} definirani kot spodaj?

        \begin{multicols}{2}
        \begin{minted}[autogobble]{python}
        
            def a(x):
                print(x)
                return x + 7

            def b(y):
                print(y)
                return 5 * y
        
        \end{minted}

        \begin{enumerate}[(a)]
\item \inlinepy{[3, 22]}
\item \inlinepy{[22]}
\item \inlinepy{[15, 22]}
\item \inlinepy{[3, 15, 22]}
\end{enumerate}

        \end{multicols}
    
        \naloga*

        \begin{multicols}{2}
        \noindent
        Kateri pogoj preverja spodnja funkcija?
        \begin{minted}[autogobble]{python}
        
            def f(besedilo):
                for z in besedilo:
                    if z not in 'aeiouAEIOU':
                        return True
                return False
            
        \end{minted}

        \begin{enumerate}[(a)]
\item ali niz vsebuje kakšen samoglasnik
\item ali niz vsebuje znak, ki ni samoglasnik
\item ali niz vsebuje samo samoglasnike
\item ali niz ne vsebuje nobenega samoglasnika
\end{enumerate}

        \end{multicols}
    
        \naloga*
        \begin{multicols}{2}
        \noindent
        V vsak prostor vpišite \textbf{natanko en znak} tako, da bo dobljeni program v spremenljivko \inlinepy{vs} shranil vsoto števil \inlinepy{b} in \inlinepy{a}:
        
        \columnbreak
        \begin{minted}[baselinestretch=1.2,escapeinside=||]{python}
        vs = |\answerbox{0.5}|
        while b > |\answerbox{0.5}|:
            vs += |\answerbox{0.5}|
            |\answerbox{0.5}| -= 1
        \end{minted}
        \end{multicols}
    
        \clearpage
        \naloga
        
        Katera izmed spodnjih funkcij izračuna ostanek pri deljenju naravnega števila \inlinepy{x} z naravnim številom \inlinepy{y}?
    
        \begin{multicols}{2}
        \begin{enumerate}[(a)]
\item 
                \begin{minted}[autogobble]{python}
                def ostanek(x, y):
                    if y == 0:
                        return x
                    else:
                        return ostanek(y, x % y)
                \end{minted}
            
\item 
                \begin{minted}[autogobble]{python}
                def ostanek(x, y):
                    if x < y:
                        return x
                    else:
                        return ostanek(x % y)
                \end{minted}
            
\item 
            \begin{minted}[autogobble]{python}
            def ostanek(x, y):
                if x < y:
                    return x
                else:
                    return ostanek(x - y, y)
            \end{minted}
        
\item 
                \begin{minted}[autogobble]{python}
                def ostanek(x, y):
                    if x < y:
                        return 0
                    else:
                        return ostanek(x - y, y)
                \end{minted}
            
\end{enumerate}

        \end{multicols}
    
        \naloga*
        
        Katera izmed funkcij vrača drugačne rezultate kot ostale?
    
        \begin{multicols}{2}
        \begin{enumerate}[(a)]
\item 
                \begin{minted}[autogobble]{python}
                def h(q, p, r):
                    if q:
                        return p and r
                    else:
                        return False
                \end{minted}
            
\item 
                \begin{minted}[autogobble]{python}
                def h(q, p, r):
                    if q and p:
                        return r
                    else:
                        return False
                \end{minted}
            
\item 
                \begin{minted}[autogobble]{python}
                def h(q, p, r):
                    if not p:
                        return False
                    else:
                        return q and r
                \end{minted}
            
\item 
            \begin{minted}[autogobble]{python}
            def h(q, p, r):
                if not r:
                    return q and p
                else:
                    return False
            \end{minted}
        
\end{enumerate}

        \end{multicols}
    
        \naloga*
        
        Kateri izmed spodnjih programov ima drugačen izpis kot ostali?
    
        \begin{multicols}{2}
        \begin{enumerate}[(a)]
\item 
                \begin{minted}[autogobble]{python}
                    for j in range(1, 20):
                        if j % 2 == 1:
                            print(j)
                        continue
                \end{minted}
            
\item 
                \begin{minted}[autogobble]{python}
                    for j in range(1, 40, 2):
                        if j < 20:
                            print(j)
                \end{minted}
            
\item 
            \begin{minted}[autogobble]{python}
                for j in range(1, 20):
                    if j % 2 != 0:
                        continue
                    print(j)
            \end{minted}
        
\item 
                \begin{minted}[autogobble]{python}
                    for j in range(1, 40, 2):
                        if j > 20:
                            break
                        print(j)
                \end{minted}
            
\end{enumerate}

        \end{multicols}
    
        \naloga*
        \begin{multicols}{2}
        \noindent
        Napišite primer vrednosti spremenljivk \inlinepy{niz1} in \inlinepy{niz2}, za kateri klic \inlinepy{f(niz1, niz2)} vrne \inlinepy{True}.
        \begin{minted}[baselinestretch=1.2, escapeinside=||]{python}
        niz1 = |\answerbox{3}|
        niz2 = |\answerbox{3}|
        \end{minted}
        \vfil
        \columnbreak
        \begin{minted}[autogobble]{python}
        def f(niz1, niz2):
            if len(niz1) != len(niz2) or len(niz1) < 3:
                return False
            for j in range(len(niz1)):
                if j % 2 == 1 and niz1[j] != niz2[j]:
                    return False
            return True
        \end{minted}
        \end{multicols}
    
        \naloga*
        \begin{multicols}{2}
        \noindent 
        S številkami od $0$ do $4$ označite vrstni red, v katerem moramo izvesti ukaze na desni, da bo na koncu spremenljivka \inlinepy{lst1} kazala na seznam \inlinepy{[8, 3, 4]}?
    
        \columnbreak
        \noindent
        \begin{minted}[baselinestretch=1.2, escapeinside=||]{python}
|\answerbox{0.5}| lst1 = lst2 + lst1
|\answerbox{0.5}| lst1 = lst2
|\answerbox{0.5}| lst1.append(4)
|\answerbox{0.5}| lst2 = [8]
|\answerbox{0.5}| lst2 = [3]

        \end{minted}
        \end{multicols}
    
            \izpit[ucilnica=205, naloge=-1]{Uvod v programiranje: Kolokvij \#011}{27.\ marec 2019}{
                Pri vsaki nalogi obkrožite črko pred pravilnim odgovorom ali vpišite pravilno vrednost v ustrezen prostor. \\
                Čas reševanja je 30 minut. Veliko uspeha!
            }
            
        \naloga*
        
        Kateri izmed programov pri začetnem stanju
            \inlinepy{u = 5} in
            \inlinepy{v = 3}
        nastavi vrednosti
            \inlinepy{u = 3},
            \inlinepy{v = 5} in
            \inlinepy{w = 8}?
    
        \begin{multicols}{4}
        \begin{enumerate}[(a)]
\item 
            \begin{minted}[autogobble]{python}
            w = u
            u = v
            v = w
            w = u + v
            \end{minted}
        
\item 
                \begin{minted}[autogobble]{python}
                w = u
                v = w
                u = v
                w = u + v
                \end{minted}
            
\item 
                \begin{minted}[autogobble]{python}
                u = v
                v = w
                w = u
                w = u + v
                \end{minted}
            
\item 
                \begin{minted}[autogobble]{python}
                u = v
                w = v
                v = u
                w = u + v
                \end{minted}
            
\end{enumerate}

        \end{multicols}
    
        \naloga*
        \begin{multicols}{2}
        \noindent
        Kakšne vrste napak vsebuje program na desni?

        \begin{enumerate}[(a)]
\item oblikovne napake, ki ne vplivajo na pravilnost rezultata
\item sintaktične napake, zaradi katerih Python programa noče izvesti.
\item vsebinske napake, zaradi katerih Python izračuna napačen rezultat
\item napake, zaradi katerih Python prekine z izvajanjem programa
\end{enumerate}

        \columnbreak

        \begin{minted}[baselinestretch=1.2,escapeinside=||, autogobble]{python}
        
            def fibonacci(n):
                if n <= 0:
                    return 0
                elif n == 1:
                    return 0
                else:
                    a = fibonacci(n - 1)
                    b = fibonacci(n - 3)
                    return a * b
            
        \end{minted}

        \end{multicols}

    
        \naloga*
        Katere vrstice izpiše klic \inlinepy{print(f(g(1)))}, če sta funkciji \inlinepy{f} in \inlinepy{g} definirani kot spodaj?

        \begin{multicols}{2}
        \begin{minted}[autogobble]{python}
        
            def f(a):
                print(a)
                return a + 3

            def g(b):
                return 6 * b
                print(b)
        
        \end{minted}

        \begin{enumerate}[(a)]
\item \inlinepy{[1, 9]}
\item \inlinepy{[6, 9]}
\item \inlinepy{[9]}
\item \inlinepy{[1, 6, 9]}
\end{enumerate}

        \end{multicols}
    
        \naloga*

        \begin{multicols}{2}
        \noindent
        Kateri pogoj preverja spodnja funkcija?
        \begin{minted}[autogobble]{python}
        
            def f(niz):
                for z in niz:
                    if z not in 'aeiouAEIOU':
                        return True
                return False
            
        \end{minted}

        \begin{enumerate}[(a)]
\item ali niz vsebuje znak, ki ni samoglasnik
\item ali niz vsebuje kakšen samoglasnik
\item ali niz vsebuje samo samoglasnike
\item ali niz ne vsebuje nobenega samoglasnika
\end{enumerate}

        \end{multicols}
    
        \naloga*
        \begin{multicols}{2}
        \noindent
        V vsak prostor vpišite \textbf{natanko en znak} tako, da bo dobljeni program v spremenljivko \inlinepy{zm} shranil zmnožek števil \inlinepy{b} in \inlinepy{a}:
        
        \columnbreak
        \begin{minted}[baselinestretch=1.2,escapeinside=||]{python}
        zm = |\answerbox{0.5}|
        while b > |\answerbox{0.5}|:
            zm += |\answerbox{0.5}|
            |\answerbox{0.5}| -= 1
        \end{minted}
        \end{multicols}
    
        \clearpage
        \naloga
        
        Katera izmed spodnjih funkcij izračuna ostanek pri deljenju naravnega števila \inlinepy{m} z naravnim številom \inlinepy{n}?
    
        \begin{multicols}{2}
        \begin{enumerate}[(a)]
\item 
            \begin{minted}[autogobble]{python}
            def ostanek(m, n):
                if m < n:
                    return m
                else:
                    return ostanek(m - n, n)
            \end{minted}
        
\item 
                \begin{minted}[autogobble]{python}
                def ostanek(m, n):
                    if m < n:
                        return m
                    else:
                        return ostanek(m % n)
                \end{minted}
            
\item 
                \begin{minted}[autogobble]{python}
                def ostanek(m, n):
                    if n == 0:
                        return m
                    else:
                        return ostanek(n, m % n)
                \end{minted}
            
\item 
                \begin{minted}[autogobble]{python}
                def ostanek(m, n):
                    if m < n:
                        return 0
                    else:
                        return ostanek(m - n, n)
                \end{minted}
            
\end{enumerate}

        \end{multicols}
    
        \naloga*
        
        Katera izmed funkcij vrača drugačne rezultate kot ostale?
    
        \begin{multicols}{2}
        \begin{enumerate}[(a)]
\item 
            \begin{minted}[autogobble]{python}
            def g(p, r, q):
                if not q:
                    return p and r
                else:
                    return False
            \end{minted}
        
\item 
                \begin{minted}[autogobble]{python}
                def g(p, r, q):
                    if p and r:
                        return q
                    else:
                        return False
                \end{minted}
            
\item 
                \begin{minted}[autogobble]{python}
                def g(p, r, q):
                    if p:
                        return r and q
                    else:
                        return False
                \end{minted}
            
\item 
                \begin{minted}[autogobble]{python}
                def g(p, r, q):
                    if not r:
                        return False
                    else:
                        return p and q
                \end{minted}
            
\end{enumerate}

        \end{multicols}
    
        \naloga*
        
        Kateri izmed spodnjih programov ima drugačen izpis kot ostali?
    
        \begin{multicols}{2}
        \begin{enumerate}[(a)]
\item 
                \begin{minted}[autogobble]{python}
                    for n in range(1, 50, 5):
                        if n < 10:
                            print(n)
                \end{minted}
            
\item 
                \begin{minted}[autogobble]{python}
                    for n in range(1, 50, 5):
                        if n > 10:
                            break
                        print(n)
                \end{minted}
            
\item 
            \begin{minted}[autogobble]{python}
                for n in range(1, 10):
                    if n % 5 != 0:
                        continue
                    print(n)
            \end{minted}
        
\item 
                \begin{minted}[autogobble]{python}
                    for n in range(1, 10):
                        if n % 5 == 1:
                            print(n)
                        continue
                \end{minted}
            
\end{enumerate}

        \end{multicols}
    
        \naloga*
        \begin{multicols}{2}
        \noindent
        Napišite primer vrednosti spremenljivk \inlinepy{str1} in \inlinepy{str2}, za kateri klic \inlinepy{h(str1, str2)} vrne \inlinepy{True}.
        \begin{minted}[baselinestretch=1.2, escapeinside=||]{python}
        str1 = |\answerbox{3}|
        str2 = |\answerbox{3}|
        \end{minted}
        \vfil
        \columnbreak
        \begin{minted}[autogobble]{python}
        def h(str1, str2):
            if len(str1) != len(str2) or len(str1) < 3:
                return False
            for j in range(len(str1)):
                if j % 2 == 0 and str1[j] != str2[j]:
                    return False
            return True
        \end{minted}
        \end{multicols}
    
        \naloga*
        \begin{multicols}{2}
        \noindent 
        S številkami od $0$ do $4$ označite vrstni red, v katerem moramo izvesti ukaze na desni, da bo na koncu spremenljivka \inlinepy{lst2} kazala na seznam \inlinepy{[7, 6, 2]}?
    
        \columnbreak
        \noindent
        \begin{minted}[baselinestretch=1.2, escapeinside=||]{python}
|\answerbox{0.5}| lst1 = [6]
|\answerbox{0.5}| lst2 = lst1 + lst2
|\answerbox{0.5}| lst2.append(2)
|\answerbox{0.5}| lst1 = [7]
|\answerbox{0.5}| lst2 = lst1

        \end{minted}
        \end{multicols}
    
            \izpit[ucilnica=205, naloge=-1]{Uvod v programiranje: Kolokvij \#012}{27.\ marec 2019}{
                Pri vsaki nalogi obkrožite črko pred pravilnim odgovorom ali vpišite pravilno vrednost v ustrezen prostor. \\
                Čas reševanja je 30 minut. Veliko uspeha!
            }
            
        \naloga*
        
        Kateri izmed programov pri začetnem stanju
            \inlinepy{k = 9} in
            \inlinepy{m = 8}
        nastavi vrednosti
            \inlinepy{k = 8},
            \inlinepy{m = 9} in
            \inlinepy{n = 17}?
    
        \begin{multicols}{4}
        \begin{enumerate}[(a)]
\item 
                \begin{minted}[autogobble]{python}
                n = k
                m = n
                k = m
                n = k + m
                \end{minted}
            
\item 
            \begin{minted}[autogobble]{python}
            n = k
            k = m
            m = n
            n = k + m
            \end{minted}
        
\item 
                \begin{minted}[autogobble]{python}
                k = m
                m = n
                n = k
                n = k + m
                \end{minted}
            
\item 
                \begin{minted}[autogobble]{python}
                k = m
                n = m
                m = k
                n = k + m
                \end{minted}
            
\end{enumerate}

        \end{multicols}
    
        \naloga*
        \begin{multicols}{2}
        \noindent
        Kakšne vrste napak vsebuje program na desni?

        \begin{enumerate}[(a)]
\item sintaktične napake, zaradi katerih Python programa noče izvesti.
\item napake, zaradi katerih Python prekine z izvajanjem programa
\item oblikovne napake, ki ne vplivajo na pravilnost rezultata
\item vsebinske napake, zaradi katerih Python izračuna napačen rezultat
\end{enumerate}

        \columnbreak

        \begin{minted}[baselinestretch=1.2,escapeinside=||, autogobble]{python}
        
            def fibonacci(n):
                if n <= 0:
                    return 0
                elif n == 1:
                    return 0
                else:
                    a = fibonacci(n - 1)
                    b = fibonacci(n - 3)
                    return a * b
            
        \end{minted}

        \end{multicols}

    
        \naloga*
        Katere vrstice izpiše klic \inlinepy{print(a(b(4)))}, če sta funkciji \inlinepy{a} in \inlinepy{b} definirani kot spodaj?

        \begin{multicols}{2}
        \begin{minted}[autogobble]{python}
        
            def a(x):
                print(x)
                return x + 2

            def b(y):
                print(y)
                return 5 * y
        
        \end{minted}

        \begin{enumerate}[(a)]
\item \inlinepy{[20, 22]}
\item \inlinepy{[4, 20, 22]}
\item \inlinepy{[22]}
\item \inlinepy{[4, 22]}
\end{enumerate}

        \end{multicols}
    
        \naloga*

        \begin{multicols}{2}
        \noindent
        Kateri pogoj preverja spodnja funkcija?
        \begin{minted}[autogobble]{python}
        
            def f(niz):
                for x in niz:
                    if x in 'aeiouAEIOU':
                        return True
                return False
            
        \end{minted}

        \begin{enumerate}[(a)]
\item ali niz vsebuje kakšen samoglasnik
\item ali niz vsebuje samo samoglasnike
\item ali niz ne vsebuje nobenega samoglasnika
\item ali niz vsebuje znak, ki ni samoglasnik
\end{enumerate}

        \end{multicols}
    
        \naloga*
        \begin{multicols}{2}
        \noindent
        V vsak prostor vpišite \textbf{natanko en znak} tako, da bo dobljeni program v spremenljivko \inlinepy{vs} shranil vsoto števil \inlinepy{y} in \inlinepy{x}:
        
        \columnbreak
        \begin{minted}[baselinestretch=1.2,escapeinside=||]{python}
        vs = |\answerbox{0.5}|
        while y > |\answerbox{0.5}|:
            vs += |\answerbox{0.5}|
            |\answerbox{0.5}| -= 1
        \end{minted}
        \end{multicols}
    
        \clearpage
        \naloga
        
        Katera izmed spodnjih funkcij izračuna ostanek pri deljenju naravnega števila \inlinepy{u} z naravnim številom \inlinepy{v}?
    
        \begin{multicols}{2}
        \begin{enumerate}[(a)]
\item 
            \begin{minted}[autogobble]{python}
            def ostanek(u, v):
                if u < v:
                    return u
                else:
                    return ostanek(u - v, v)
            \end{minted}
        
\item 
                \begin{minted}[autogobble]{python}
                def ostanek(u, v):
                    if u < v:
                        return u
                    else:
                        return ostanek(u % v)
                \end{minted}
            
\item 
                \begin{minted}[autogobble]{python}
                def ostanek(u, v):
                    if v == 0:
                        return u
                    else:
                        return ostanek(v, u % v)
                \end{minted}
            
\item 
                \begin{minted}[autogobble]{python}
                def ostanek(u, v):
                    if u < v:
                        return 0
                    else:
                        return ostanek(u - v, v)
                \end{minted}
            
\end{enumerate}

        \end{multicols}
    
        \naloga*
        
        Katera izmed funkcij vrača drugačne rezultate kot ostale?
    
        \begin{multicols}{2}
        \begin{enumerate}[(a)]
\item 
                \begin{minted}[autogobble]{python}
                def f(b, c, a):
                    if b and c:
                        return a
                    else:
                        return False
                \end{minted}
            
\item 
            \begin{minted}[autogobble]{python}
            def f(b, c, a):
                if not a:
                    return b and c
                else:
                    return False
            \end{minted}
        
\item 
                \begin{minted}[autogobble]{python}
                def f(b, c, a):
                    if b:
                        return c and a
                    else:
                        return False
                \end{minted}
            
\item 
                \begin{minted}[autogobble]{python}
                def f(b, c, a):
                    if not c:
                        return False
                    else:
                        return b and a
                \end{minted}
            
\end{enumerate}

        \end{multicols}
    
        \naloga*
        
        Kateri izmed spodnjih programov ima drugačen izpis kot ostali?
    
        \begin{multicols}{2}
        \begin{enumerate}[(a)]
\item 
                \begin{minted}[autogobble]{python}
                    for x in range(1, 100, 2):
                        if x > 50:
                            break
                        print(x)
                \end{minted}
            
\item 
            \begin{minted}[autogobble]{python}
                for x in range(1, 50):
                    if x % 2 != 0:
                        continue
                    print(x)
            \end{minted}
        
\item 
                \begin{minted}[autogobble]{python}
                    for x in range(1, 100, 2):
                        if x < 50:
                            print(x)
                \end{minted}
            
\item 
                \begin{minted}[autogobble]{python}
                    for x in range(1, 50):
                        if x % 2 == 1:
                            print(x)
                        continue
                \end{minted}
            
\end{enumerate}

        \end{multicols}
    
        \naloga*
        \begin{multicols}{2}
        \noindent
        Napišite primer vrednosti spremenljivk \inlinepy{niz1} in \inlinepy{niz2}, za kateri klic \inlinepy{f(niz1, niz2)} vrne \inlinepy{True}.
        \begin{minted}[baselinestretch=1.2, escapeinside=||]{python}
        niz1 = |\answerbox{3}|
        niz2 = |\answerbox{3}|
        \end{minted}
        \vfil
        \columnbreak
        \begin{minted}[autogobble]{python}
        def f(niz1, niz2):
            if len(niz1) != len(niz2) or len(niz1) < 3:
                return False
            for i in range(len(niz1)):
                if i % 2 == 0 and niz1[i] != niz2[i]:
                    return False
            return True
        \end{minted}
        \end{multicols}
    
        \naloga*
        \begin{multicols}{2}
        \noindent 
        S številkami od $0$ do $4$ označite vrstni red, v katerem moramo izvesti ukaze na desni, da bo na koncu spremenljivka \inlinepy{sez2} kazala na seznam \inlinepy{[3, 1, 2]}?
    
        \columnbreak
        \noindent
        \begin{minted}[baselinestretch=1.2, escapeinside=||]{python}
|\answerbox{0.5}| sez2 = sez1
|\answerbox{0.5}| sez2 = sez1 + sez2
|\answerbox{0.5}| sez1 = [3]
|\answerbox{0.5}| sez2.append(2)
|\answerbox{0.5}| sez1 = [1]

        \end{minted}
        \end{multicols}
    
            \izpit[ucilnica=205, naloge=-1]{Uvod v programiranje: Kolokvij \#013}{27.\ marec 2019}{
                Pri vsaki nalogi obkrožite črko pred pravilnim odgovorom ali vpišite pravilno vrednost v ustrezen prostor. \\
                Čas reševanja je 30 minut. Veliko uspeha!
            }
            
        \naloga*
        
        Kateri izmed programov pri začetnem stanju
            \inlinepy{k = 4} in
            \inlinepy{m = 3}
        nastavi vrednosti
            \inlinepy{k = 3},
            \inlinepy{m = 4} in
            \inlinepy{n = 7}?
    
        \begin{multicols}{4}
        \begin{enumerate}[(a)]
\item 
                \begin{minted}[autogobble]{python}
                k = m
                n = m
                m = k
                n = k + m
                \end{minted}
            
\item 
                \begin{minted}[autogobble]{python}
                n = k
                m = n
                k = m
                n = k + m
                \end{minted}
            
\item 
                \begin{minted}[autogobble]{python}
                k = m
                m = n
                n = k
                n = k + m
                \end{minted}
            
\item 
            \begin{minted}[autogobble]{python}
            n = k
            k = m
            m = n
            n = k + m
            \end{minted}
        
\end{enumerate}

        \end{multicols}
    
        \naloga*
        \begin{multicols}{2}
        \noindent
        Kakšne vrste napak vsebuje program na desni?

        \begin{enumerate}[(a)]
\item napake, zaradi katerih Python prekine z izvajanjem programa
\item sintaktične napake, zaradi katerih Python programa noče izvesti.
\item oblikovne napake, ki ne vplivajo na pravilnost rezultata
\item vsebinske napake, zaradi katerih Python izračuna napačen rezultat
\end{enumerate}

        \columnbreak

        \begin{minted}[baselinestretch=1.2,escapeinside=||, autogobble]{python}
        
            def fibonacci(n):
                if n<=0:
                    return (0)
                elif n==1:
                    return (1)
                else:
                    a = fibonacci(n - 1)
                    b = fibonacci(n - 2)
                    return   a + b
            
        \end{minted}

        \end{multicols}

    
        \naloga*
        Katere vrstice izpiše klic \inlinepy{print(f(g(7)))}, če sta funkciji \inlinepy{f} in \inlinepy{g} definirani kot spodaj?

        \begin{multicols}{2}
        \begin{minted}[autogobble]{python}
        
            def f(x):
                print(x)
                return x + 4

            def g(y):
                return 8 * y
                print(y)
        
        \end{minted}

        \begin{enumerate}[(a)]
\item \inlinepy{[7, 56, 60]}
\item \inlinepy{[7, 60]}
\item \inlinepy{[56, 60]}
\item \inlinepy{[60]}
\end{enumerate}

        \end{multicols}
    
        \naloga*

        \begin{multicols}{2}
        \noindent
        Kateri pogoj preverja spodnja funkcija?
        \begin{minted}[autogobble]{python}
        
            def f(stavek):
                for z in stavek:
                    if z in 'aeiouAEIOU':
                        return True
                return False
            
        \end{minted}

        \begin{enumerate}[(a)]
\item ali niz ne vsebuje nobenega samoglasnika
\item ali niz vsebuje kakšen samoglasnik
\item ali niz vsebuje znak, ki ni samoglasnik
\item ali niz vsebuje samo samoglasnike
\end{enumerate}

        \end{multicols}
    
        \naloga*
        \begin{multicols}{2}
        \noindent
        V vsak prostor vpišite \textbf{natanko en znak} tako, da bo dobljeni program v spremenljivko \inlinepy{vs} shranil vsoto števil \inlinepy{x} in \inlinepy{y}:
        
        \columnbreak
        \begin{minted}[baselinestretch=1.2,escapeinside=||]{python}
        vs = |\answerbox{0.5}|
        while x > |\answerbox{0.5}|:
            vs += |\answerbox{0.5}|
            |\answerbox{0.5}| -= 1
        \end{minted}
        \end{multicols}
    
        \clearpage
        \naloga
        
        Katera izmed spodnjih funkcij izračuna ostanek pri deljenju naravnega števila \inlinepy{m} z naravnim številom \inlinepy{n}?
    
        \begin{multicols}{2}
        \begin{enumerate}[(a)]
\item 
            \begin{minted}[autogobble]{python}
            def ostanek(m, n):
                if m < n:
                    return m
                else:
                    return ostanek(m - n, n)
            \end{minted}
        
\item 
                \begin{minted}[autogobble]{python}
                def ostanek(m, n):
                    if m < n:
                        return m
                    else:
                        return ostanek(m % n)
                \end{minted}
            
\item 
                \begin{minted}[autogobble]{python}
                def ostanek(m, n):
                    if m < n:
                        return 0
                    else:
                        return ostanek(m - n, n)
                \end{minted}
            
\item 
                \begin{minted}[autogobble]{python}
                def ostanek(m, n):
                    if n == 0:
                        return m
                    else:
                        return ostanek(n, m % n)
                \end{minted}
            
\end{enumerate}

        \end{multicols}
    
        \naloga*
        
        Katera izmed funkcij vrača drugačne rezultate kot ostale?
    
        \begin{multicols}{2}
        \begin{enumerate}[(a)]
\item 
                \begin{minted}[autogobble]{python}
                def g(a, b, c):
                    if a and b:
                        return c
                    else:
                        return False
                \end{minted}
            
\item 
                \begin{minted}[autogobble]{python}
                def g(a, b, c):
                    if a:
                        return b and c
                    else:
                        return False
                \end{minted}
            
\item 
            \begin{minted}[autogobble]{python}
            def g(a, b, c):
                if not c:
                    return a and b
                else:
                    return False
            \end{minted}
        
\item 
                \begin{minted}[autogobble]{python}
                def g(a, b, c):
                    if not b:
                        return False
                    else:
                        return a and c
                \end{minted}
            
\end{enumerate}

        \end{multicols}
    
        \naloga*
        
        Kateri izmed spodnjih programov ima drugačen izpis kot ostali?
    
        \begin{multicols}{2}
        \begin{enumerate}[(a)]
\item 
                \begin{minted}[autogobble]{python}
                    for n in range(1, 60, 3):
                        if n > 20:
                            break
                        print(n)
                \end{minted}
            
\item 
            \begin{minted}[autogobble]{python}
                for n in range(1, 20):
                    if n % 3 != 0:
                        continue
                    print(n)
            \end{minted}
        
\item 
                \begin{minted}[autogobble]{python}
                    for n in range(1, 60, 3):
                        if n < 20:
                            print(n)
                \end{minted}
            
\item 
                \begin{minted}[autogobble]{python}
                    for n in range(1, 20):
                        if n % 3 == 1:
                            print(n)
                        continue
                \end{minted}
            
\end{enumerate}

        \end{multicols}
    
        \naloga*
        \begin{multicols}{2}
        \noindent
        Napišite primer vrednosti spremenljivk \inlinepy{str1} in \inlinepy{str2}, za kateri klic \inlinepy{g(str1, str2)} vrne \inlinepy{True}.
        \begin{minted}[baselinestretch=1.2, escapeinside=||]{python}
        str1 = |\answerbox{3}|
        str2 = |\answerbox{3}|
        \end{minted}
        \vfil
        \columnbreak
        \begin{minted}[autogobble]{python}
        def g(str1, str2):
            if len(str1) != len(str2) or len(str1) < 3:
                return False
            for j in range(len(str1)):
                if j % 2 == 1 and str1[j] != str2[j]:
                    return False
            return True
        \end{minted}
        \end{multicols}
    
        \naloga*
        \begin{multicols}{2}
        \noindent 
        S številkami od $0$ do $4$ označite vrstni red, v katerem moramo izvesti ukaze na desni, da bo na koncu spremenljivka \inlinepy{sez1} kazala na seznam \inlinepy{[6, 9, 1]}?
    
        \columnbreak
        \noindent
        \begin{minted}[baselinestretch=1.2, escapeinside=||]{python}
|\answerbox{0.5}| sez1 = sez2
|\answerbox{0.5}| sez2 = [6]
|\answerbox{0.5}| sez1.append(1)
|\answerbox{0.5}| sez1 = sez2 + sez1
|\answerbox{0.5}| sez2 = [9]

        \end{minted}
        \end{multicols}
    
            \izpit[ucilnica=205, naloge=-1]{Uvod v programiranje: Kolokvij \#014}{27.\ marec 2019}{
                Pri vsaki nalogi obkrožite črko pred pravilnim odgovorom ali vpišite pravilno vrednost v ustrezen prostor. \\
                Čas reševanja je 30 minut. Veliko uspeha!
            }
            
        \naloga*
        
        Kateri izmed programov pri začetnem stanju
            \inlinepy{k = 6} in
            \inlinepy{m = 5}
        nastavi vrednosti
            \inlinepy{k = 5},
            \inlinepy{m = 6} in
            \inlinepy{n = 11}?
    
        \begin{multicols}{4}
        \begin{enumerate}[(a)]
\item 
            \begin{minted}[autogobble]{python}
            n = k
            k = m
            m = n
            n = k + m
            \end{minted}
        
\item 
                \begin{minted}[autogobble]{python}
                n = k
                m = n
                k = m
                n = k + m
                \end{minted}
            
\item 
                \begin{minted}[autogobble]{python}
                k = m
                m = n
                n = k
                n = k + m
                \end{minted}
            
\item 
                \begin{minted}[autogobble]{python}
                k = m
                n = m
                m = k
                n = k + m
                \end{minted}
            
\end{enumerate}

        \end{multicols}
    
        \naloga*
        \begin{multicols}{2}
        \noindent
        Kakšne vrste napak vsebuje program na desni?

        \begin{enumerate}[(a)]
\item vsebinske napake, zaradi katerih Python izračuna napačen rezultat
\item napake, zaradi katerih Python prekine z izvajanjem programa
\item oblikovne napake, ki ne vplivajo na pravilnost rezultata
\item sintaktične napake, zaradi katerih Python programa noče izvesti.
\end{enumerate}

        \columnbreak

        \begin{minted}[baselinestretch=1.2,escapeinside=||, autogobble]{python}
        
            def fibonacci(n):
                if n<=0:
                    return (0)
                elif n==1:
                    return (1)
                else:
                    a = fibonacci(n - 1)
                    b = fibonacci(n - 2)
                    return   a + b
            
        \end{minted}

        \end{multicols}

    
        \naloga*
        Katere vrstice izpiše klic \inlinepy{print(p(q(3)))}, če sta funkciji \inlinepy{p} in \inlinepy{q} definirani kot spodaj?

        \begin{multicols}{2}
        \begin{minted}[autogobble]{python}
        
            def p(a):
                print(a)
                return a + 2

            def q(b):
                return 8 * b
                print(b)
        
        \end{minted}

        \begin{enumerate}[(a)]
\item \inlinepy{[24, 26]}
\item \inlinepy{[26]}
\item \inlinepy{[3, 24, 26]}
\item \inlinepy{[3, 26]}
\end{enumerate}

        \end{multicols}
    
        \naloga*

        \begin{multicols}{2}
        \noindent
        Kateri pogoj preverja spodnja funkcija?
        \begin{minted}[autogobble]{python}
        
            def f(stavek):
                for z in stavek:
                    if z not in 'aeiouAEIOU':
                        return False
                return True
            
        \end{minted}

        \begin{enumerate}[(a)]
\item ali niz vsebuje kakšen samoglasnik
\item ali niz vsebuje samo samoglasnike
\item ali niz vsebuje znak, ki ni samoglasnik
\item ali niz ne vsebuje nobenega samoglasnika
\end{enumerate}

        \end{multicols}
    
        \naloga*
        \begin{multicols}{2}
        \noindent
        V vsak prostor vpišite \textbf{natanko en znak} tako, da bo dobljeni program v spremenljivko \inlinepy{vs} shranil vsoto števil \inlinepy{p} in \inlinepy{q}:
        
        \columnbreak
        \begin{minted}[baselinestretch=1.2,escapeinside=||]{python}
        vs = |\answerbox{0.5}|
        while p > |\answerbox{0.5}|:
            vs += |\answerbox{0.5}|
            |\answerbox{0.5}| -= 1
        \end{minted}
        \end{multicols}
    
        \clearpage
        \naloga
        
        Katera izmed spodnjih funkcij izračuna ostanek pri deljenju naravnega števila \inlinepy{u} z naravnim številom \inlinepy{v}?
    
        \begin{multicols}{2}
        \begin{enumerate}[(a)]
\item 
                \begin{minted}[autogobble]{python}
                def ostanek(u, v):
                    if u < v:
                        return 0
                    else:
                        return ostanek(u - v, v)
                \end{minted}
            
\item 
                \begin{minted}[autogobble]{python}
                def ostanek(u, v):
                    if v == 0:
                        return u
                    else:
                        return ostanek(v, u % v)
                \end{minted}
            
\item 
            \begin{minted}[autogobble]{python}
            def ostanek(u, v):
                if u < v:
                    return u
                else:
                    return ostanek(u - v, v)
            \end{minted}
        
\item 
                \begin{minted}[autogobble]{python}
                def ostanek(u, v):
                    if u < v:
                        return u
                    else:
                        return ostanek(u % v)
                \end{minted}
            
\end{enumerate}

        \end{multicols}
    
        \naloga*
        
        Katera izmed funkcij vrača drugačne rezultate kot ostale?
    
        \begin{multicols}{2}
        \begin{enumerate}[(a)]
\item 
                \begin{minted}[autogobble]{python}
                def h(b, c, a):
                    if b and c:
                        return a
                    else:
                        return False
                \end{minted}
            
\item 
            \begin{minted}[autogobble]{python}
            def h(b, c, a):
                if not a:
                    return b and c
                else:
                    return False
            \end{minted}
        
\item 
                \begin{minted}[autogobble]{python}
                def h(b, c, a):
                    if b:
                        return c and a
                    else:
                        return False
                \end{minted}
            
\item 
                \begin{minted}[autogobble]{python}
                def h(b, c, a):
                    if not c:
                        return False
                    else:
                        return b and a
                \end{minted}
            
\end{enumerate}

        \end{multicols}
    
        \naloga*
        
        Kateri izmed spodnjih programov ima drugačen izpis kot ostali?
    
        \begin{multicols}{2}
        \begin{enumerate}[(a)]
\item 
                \begin{minted}[autogobble]{python}
                    for x in range(1, 40, 2):
                        if x < 20:
                            print(x)
                \end{minted}
            
\item 
                \begin{minted}[autogobble]{python}
                    for x in range(1, 20):
                        if x % 2 == 1:
                            print(x)
                        continue
                \end{minted}
            
\item 
                \begin{minted}[autogobble]{python}
                    for x in range(1, 40, 2):
                        if x > 20:
                            break
                        print(x)
                \end{minted}
            
\item 
            \begin{minted}[autogobble]{python}
                for x in range(1, 20):
                    if x % 2 != 0:
                        continue
                    print(x)
            \end{minted}
        
\end{enumerate}

        \end{multicols}
    
        \naloga*
        \begin{multicols}{2}
        \noindent
        Napišite primer vrednosti spremenljivk \inlinepy{str1} in \inlinepy{str2}, za kateri klic \inlinepy{f(str1, str2)} vrne \inlinepy{True}.
        \begin{minted}[baselinestretch=1.2, escapeinside=||]{python}
        str1 = |\answerbox{3}|
        str2 = |\answerbox{3}|
        \end{minted}
        \vfil
        \columnbreak
        \begin{minted}[autogobble]{python}
        def f(str1, str2):
            if len(str1) != len(str2) or len(str1) < 3:
                return False
            for i in range(len(str1)):
                if i % 2 == 1 and str1[i] != str2[i]:
                    return False
            return True
        \end{minted}
        \end{multicols}
    
        \naloga*
        \begin{multicols}{2}
        \noindent 
        S številkami od $0$ do $4$ označite vrstni red, v katerem moramo izvesti ukaze na desni, da bo na koncu spremenljivka \inlinepy{sez2} kazala na seznam \inlinepy{[7, 8, 2]}?
    
        \columnbreak
        \noindent
        \begin{minted}[baselinestretch=1.2, escapeinside=||]{python}
|\answerbox{0.5}| sez2.append(2)
|\answerbox{0.5}| sez1 = [8]
|\answerbox{0.5}| sez1 = [7]
|\answerbox{0.5}| sez2 = sez1
|\answerbox{0.5}| sez2 = sez1 + sez2

        \end{minted}
        \end{multicols}
    
            \izpit[ucilnica=205, naloge=-1]{Uvod v programiranje: Kolokvij \#015}{27.\ marec 2019}{
                Pri vsaki nalogi obkrožite črko pred pravilnim odgovorom ali vpišite pravilno vrednost v ustrezen prostor. \\
                Čas reševanja je 30 minut. Veliko uspeha!
            }
            
        \naloga*
        
        Kateri izmed programov pri začetnem stanju
            \inlinepy{a = 7} in
            \inlinepy{b = 9}
        nastavi vrednosti
            \inlinepy{a = 9},
            \inlinepy{b = 7} in
            \inlinepy{c = 16}?
    
        \begin{multicols}{4}
        \begin{enumerate}[(a)]
\item 
                \begin{minted}[autogobble]{python}
                c = a
                b = c
                a = b
                c = a + b
                \end{minted}
            
\item 
            \begin{minted}[autogobble]{python}
            c = a
            a = b
            b = c
            c = a + b
            \end{minted}
        
\item 
                \begin{minted}[autogobble]{python}
                a = b
                b = c
                c = a
                c = a + b
                \end{minted}
            
\item 
                \begin{minted}[autogobble]{python}
                a = b
                c = b
                b = a
                c = a + b
                \end{minted}
            
\end{enumerate}

        \end{multicols}
    
        \naloga*
        \begin{multicols}{2}
        \noindent
        Kakšne vrste napak vsebuje program na desni?

        \begin{enumerate}[(a)]
\item oblikovne napake, ki ne vplivajo na pravilnost rezultata
\item vsebinske napake, zaradi katerih Python izračuna napačen rezultat
\item napake, zaradi katerih Python prekine z izvajanjem programa
\item sintaktične napake, zaradi katerih Python programa noče izvesti.
\end{enumerate}

        \columnbreak

        \begin{minted}[baselinestretch=1.2,escapeinside=||, autogobble]{python}
        
            define fibonacci(n):
                if n <= 0:
                    return 0
                elif n == 1
                    return 1
                else:
                    a = fibonacci(n - 1)
                    b = fibonacci(n - 2)
                  return a + b
            
        \end{minted}

        \end{multicols}

    
        \naloga*
        Katere vrstice izpiše klic \inlinepy{print(f(g(8)))}, če sta funkciji \inlinepy{f} in \inlinepy{g} definirani kot spodaj?

        \begin{multicols}{2}
        \begin{minted}[autogobble]{python}
        
            def f(a):
                print(a)
                return a + 9

            def g(b):
                return 7 * b
                print(b)
        
        \end{minted}

        \begin{enumerate}[(a)]
\item \inlinepy{[8, 65]}
\item \inlinepy{[8, 56, 65]}
\item \inlinepy{[56, 65]}
\item \inlinepy{[65]}
\end{enumerate}

        \end{multicols}
    
        \naloga*

        \begin{multicols}{2}
        \noindent
        Kateri pogoj preverja spodnja funkcija?
        \begin{minted}[autogobble]{python}
        
            def f(niz):
                for z in niz:
                    if z not in 'aeiouAEIOU':
                        return False
                return True
            
        \end{minted}

        \begin{enumerate}[(a)]
\item ali niz vsebuje znak, ki ni samoglasnik
\item ali niz vsebuje samo samoglasnike
\item ali niz vsebuje kakšen samoglasnik
\item ali niz ne vsebuje nobenega samoglasnika
\end{enumerate}

        \end{multicols}
    
        \naloga*
        \begin{multicols}{2}
        \noindent
        V vsak prostor vpišite \textbf{natanko en znak} tako, da bo dobljeni program v spremenljivko \inlinepy{vs} shranil vsoto števil \inlinepy{x} in \inlinepy{y}:
        
        \columnbreak
        \begin{minted}[baselinestretch=1.2,escapeinside=||]{python}
        vs = |\answerbox{0.5}|
        while x > |\answerbox{0.5}|:
            vs += |\answerbox{0.5}|
            |\answerbox{0.5}| -= 1
        \end{minted}
        \end{multicols}
    
        \clearpage
        \naloga
        
        Katera izmed spodnjih funkcij izračuna ostanek pri deljenju naravnega števila \inlinepy{m} z naravnim številom \inlinepy{n}?
    
        \begin{multicols}{2}
        \begin{enumerate}[(a)]
\item 
                \begin{minted}[autogobble]{python}
                def ostanek(m, n):
                    if m < n:
                        return 0
                    else:
                        return ostanek(m - n, n)
                \end{minted}
            
\item 
                \begin{minted}[autogobble]{python}
                def ostanek(m, n):
                    if n == 0:
                        return m
                    else:
                        return ostanek(n, m % n)
                \end{minted}
            
\item 
                \begin{minted}[autogobble]{python}
                def ostanek(m, n):
                    if m < n:
                        return m
                    else:
                        return ostanek(m % n)
                \end{minted}
            
\item 
            \begin{minted}[autogobble]{python}
            def ostanek(m, n):
                if m < n:
                    return m
                else:
                    return ostanek(m - n, n)
            \end{minted}
        
\end{enumerate}

        \end{multicols}
    
        \naloga*
        
        Katera izmed funkcij vrača drugačne rezultate kot ostale?
    
        \begin{multicols}{2}
        \begin{enumerate}[(a)]
\item 
                \begin{minted}[autogobble]{python}
                def h(a, c, b):
                    if not c:
                        return False
                    else:
                        return a and b
                \end{minted}
            
\item 
                \begin{minted}[autogobble]{python}
                def h(a, c, b):
                    if a and c:
                        return b
                    else:
                        return False
                \end{minted}
            
\item 
                \begin{minted}[autogobble]{python}
                def h(a, c, b):
                    if a:
                        return c and b
                    else:
                        return False
                \end{minted}
            
\item 
            \begin{minted}[autogobble]{python}
            def h(a, c, b):
                if not b:
                    return a and c
                else:
                    return False
            \end{minted}
        
\end{enumerate}

        \end{multicols}
    
        \naloga*
        
        Kateri izmed spodnjih programov ima drugačen izpis kot ostali?
    
        \begin{multicols}{2}
        \begin{enumerate}[(a)]
\item 
                \begin{minted}[autogobble]{python}
                    for x in range(1, 30, 3):
                        if x > 10:
                            break
                        print(x)
                \end{minted}
            
\item 
            \begin{minted}[autogobble]{python}
                for x in range(1, 10):
                    if x % 3 != 0:
                        continue
                    print(x)
            \end{minted}
        
\item 
                \begin{minted}[autogobble]{python}
                    for x in range(1, 10):
                        if x % 3 == 1:
                            print(x)
                        continue
                \end{minted}
            
\item 
                \begin{minted}[autogobble]{python}
                    for x in range(1, 30, 3):
                        if x < 10:
                            print(x)
                \end{minted}
            
\end{enumerate}

        \end{multicols}
    
        \naloga*
        \begin{multicols}{2}
        \noindent
        Napišite primer vrednosti spremenljivk \inlinepy{niz1} in \inlinepy{niz2}, za kateri klic \inlinepy{f(niz1, niz2)} vrne \inlinepy{True}.
        \begin{minted}[baselinestretch=1.2, escapeinside=||]{python}
        niz1 = |\answerbox{3}|
        niz2 = |\answerbox{3}|
        \end{minted}
        \vfil
        \columnbreak
        \begin{minted}[autogobble]{python}
        def f(niz1, niz2):
            if len(niz1) != len(niz2) or len(niz1) < 3:
                return False
            for i in range(len(niz1)):
                if i % 2 == 0 and niz1[i] != niz2[i]:
                    return False
            return True
        \end{minted}
        \end{multicols}
    
        \naloga*
        \begin{multicols}{2}
        \noindent 
        S številkami od $0$ do $4$ označite vrstni red, v katerem moramo izvesti ukaze na desni, da bo na koncu spremenljivka \inlinepy{sez1} kazala na seznam \inlinepy{[2, 3, 6]}?
    
        \columnbreak
        \noindent
        \begin{minted}[baselinestretch=1.2, escapeinside=||]{python}
|\answerbox{0.5}| sez2 = [3]
|\answerbox{0.5}| sez1 = sez2
|\answerbox{0.5}| sez2 = [2]
|\answerbox{0.5}| sez1.append(6)
|\answerbox{0.5}| sez1 = sez2 + sez1

        \end{minted}
        \end{multicols}
    
            \izpit[ucilnica=205, naloge=-1]{Uvod v programiranje: Kolokvij \#016}{27.\ marec 2019}{
                Pri vsaki nalogi obkrožite črko pred pravilnim odgovorom ali vpišite pravilno vrednost v ustrezen prostor. \\
                Čas reševanja je 30 minut. Veliko uspeha!
            }
            
        \naloga*
        
        Kateri izmed programov pri začetnem stanju
            \inlinepy{x = 5} in
            \inlinepy{y = 2}
        nastavi vrednosti
            \inlinepy{x = 2},
            \inlinepy{y = 5} in
            \inlinepy{z = 7}?
    
        \begin{multicols}{4}
        \begin{enumerate}[(a)]
\item 
            \begin{minted}[autogobble]{python}
            z = x
            x = y
            y = z
            z = x + y
            \end{minted}
        
\item 
                \begin{minted}[autogobble]{python}
                x = y
                y = z
                z = x
                z = x + y
                \end{minted}
            
\item 
                \begin{minted}[autogobble]{python}
                z = x
                y = z
                x = y
                z = x + y
                \end{minted}
            
\item 
                \begin{minted}[autogobble]{python}
                x = y
                z = y
                y = x
                z = x + y
                \end{minted}
            
\end{enumerate}

        \end{multicols}
    
        \naloga*
        \begin{multicols}{2}
        \noindent
        Kakšne vrste napak vsebuje program na desni?

        \begin{enumerate}[(a)]
\item vsebinske napake, zaradi katerih Python izračuna napačen rezultat
\item napake, zaradi katerih Python prekine z izvajanjem programa
\item sintaktične napake, zaradi katerih Python programa noče izvesti.
\item oblikovne napake, ki ne vplivajo na pravilnost rezultata
\end{enumerate}

        \columnbreak

        \begin{minted}[baselinestretch=1.2,escapeinside=||, autogobble]{python}
        
            def fibonacci(n):
                if n <= 0:
                    return 0
                elif n == 1:
                    return 1
                else:
                    a = fibonacci(n - '1')
                    b = fib(n - 2)
                    return a + b
            
        \end{minted}

        \end{multicols}

    
        \naloga*
        Katere vrstice izpiše klic \inlinepy{print(p(q(6)))}, če sta funkciji \inlinepy{p} in \inlinepy{q} definirani kot spodaj?

        \begin{multicols}{2}
        \begin{minted}[autogobble]{python}
        
            def p(a):
                print(a)
                return a + 4

            def q(b):
                print(b)
                return 8 * b
        
        \end{minted}

        \begin{enumerate}[(a)]
\item \inlinepy{[6, 48, 52]}
\item \inlinepy{[6, 52]}
\item \inlinepy{[52]}
\item \inlinepy{[48, 52]}
\end{enumerate}

        \end{multicols}
    
        \naloga*

        \begin{multicols}{2}
        \noindent
        Kateri pogoj preverja spodnja funkcija?
        \begin{minted}[autogobble]{python}
        
            def f(besedilo):
                for z in besedilo:
                    if z in 'aeiouAEIOU':
                        return True
                return False
            
        \end{minted}

        \begin{enumerate}[(a)]
\item ali niz vsebuje kakšen samoglasnik
\item ali niz vsebuje samo samoglasnike
\item ali niz ne vsebuje nobenega samoglasnika
\item ali niz vsebuje znak, ki ni samoglasnik
\end{enumerate}

        \end{multicols}
    
        \naloga*
        \begin{multicols}{2}
        \noindent
        V vsak prostor vpišite \textbf{natanko en znak} tako, da bo dobljeni program v spremenljivko \inlinepy{vs} shranil vsoto števil \inlinepy{y} in \inlinepy{x}:
        
        \columnbreak
        \begin{minted}[baselinestretch=1.2,escapeinside=||]{python}
        vs = |\answerbox{0.5}|
        while y > |\answerbox{0.5}|:
            vs += |\answerbox{0.5}|
            |\answerbox{0.5}| -= 1
        \end{minted}
        \end{multicols}
    
        \clearpage
        \naloga
        
        Katera izmed spodnjih funkcij izračuna ostanek pri deljenju naravnega števila \inlinepy{m} z naravnim številom \inlinepy{n}?
    
        \begin{multicols}{2}
        \begin{enumerate}[(a)]
\item 
                \begin{minted}[autogobble]{python}
                def ostanek(m, n):
                    if m < n:
                        return m
                    else:
                        return ostanek(m % n)
                \end{minted}
            
\item 
                \begin{minted}[autogobble]{python}
                def ostanek(m, n):
                    if m < n:
                        return 0
                    else:
                        return ostanek(m - n, n)
                \end{minted}
            
\item 
                \begin{minted}[autogobble]{python}
                def ostanek(m, n):
                    if n == 0:
                        return m
                    else:
                        return ostanek(n, m % n)
                \end{minted}
            
\item 
            \begin{minted}[autogobble]{python}
            def ostanek(m, n):
                if m < n:
                    return m
                else:
                    return ostanek(m - n, n)
            \end{minted}
        
\end{enumerate}

        \end{multicols}
    
        \naloga*
        
        Katera izmed funkcij vrača drugačne rezultate kot ostale?
    
        \begin{multicols}{2}
        \begin{enumerate}[(a)]
\item 
                \begin{minted}[autogobble]{python}
                def f(r, p, q):
                    if r:
                        return p and q
                    else:
                        return False
                \end{minted}
            
\item 
                \begin{minted}[autogobble]{python}
                def f(r, p, q):
                    if r and p:
                        return q
                    else:
                        return False
                \end{minted}
            
\item 
                \begin{minted}[autogobble]{python}
                def f(r, p, q):
                    if not p:
                        return False
                    else:
                        return r and q
                \end{minted}
            
\item 
            \begin{minted}[autogobble]{python}
            def f(r, p, q):
                if not q:
                    return r and p
                else:
                    return False
            \end{minted}
        
\end{enumerate}

        \end{multicols}
    
        \naloga*
        
        Kateri izmed spodnjih programov ima drugačen izpis kot ostali?
    
        \begin{multicols}{2}
        \begin{enumerate}[(a)]
\item 
            \begin{minted}[autogobble]{python}
                for j in range(1, 20):
                    if j % 3 != 0:
                        continue
                    print(j)
            \end{minted}
        
\item 
                \begin{minted}[autogobble]{python}
                    for j in range(1, 60, 3):
                        if j < 20:
                            print(j)
                \end{minted}
            
\item 
                \begin{minted}[autogobble]{python}
                    for j in range(1, 20):
                        if j % 3 == 1:
                            print(j)
                        continue
                \end{minted}
            
\item 
                \begin{minted}[autogobble]{python}
                    for j in range(1, 60, 3):
                        if j > 20:
                            break
                        print(j)
                \end{minted}
            
\end{enumerate}

        \end{multicols}
    
        \naloga*
        \begin{multicols}{2}
        \noindent
        Napišite primer vrednosti spremenljivk \inlinepy{lst1} in \inlinepy{lst2}, za kateri klic \inlinepy{h(lst1, lst2)} vrne \inlinepy{True}.
        \begin{minted}[baselinestretch=1.2, escapeinside=||]{python}
        lst1 = |\answerbox{3}|
        lst2 = |\answerbox{3}|
        \end{minted}
        \vfil
        \columnbreak
        \begin{minted}[autogobble]{python}
        def h(lst1, lst2):
            if len(lst1) != len(lst2) or len(lst1) < 3:
                return False
            for i in range(len(lst1)):
                if i % 2 == 1 and lst1[i] != lst2[i]:
                    return False
            return True
        \end{minted}
        \end{multicols}
    
        \naloga*
        \begin{multicols}{2}
        \noindent 
        S številkami od $0$ do $4$ označite vrstni red, v katerem moramo izvesti ukaze na desni, da bo na koncu spremenljivka \inlinepy{lst2} kazala na seznam \inlinepy{[3, 1, 9]}?
    
        \columnbreak
        \noindent
        \begin{minted}[baselinestretch=1.2, escapeinside=||]{python}
|\answerbox{0.5}| lst2 = lst1
|\answerbox{0.5}| lst1 = [1]
|\answerbox{0.5}| lst2.append(9)
|\answerbox{0.5}| lst1 = [3]
|\answerbox{0.5}| lst2 = lst1 + lst2

        \end{minted}
        \end{multicols}
    
            \izpit[ucilnica=205, naloge=-1]{Uvod v programiranje: Kolokvij \#017}{27.\ marec 2019}{
                Pri vsaki nalogi obkrožite črko pred pravilnim odgovorom ali vpišite pravilno vrednost v ustrezen prostor. \\
                Čas reševanja je 30 minut. Veliko uspeha!
            }
            
        \naloga*
        
        Kateri izmed programov pri začetnem stanju
            \inlinepy{u = 4} in
            \inlinepy{v = 8}
        nastavi vrednosti
            \inlinepy{u = 8},
            \inlinepy{v = 4} in
            \inlinepy{w = 12}?
    
        \begin{multicols}{4}
        \begin{enumerate}[(a)]
\item 
                \begin{minted}[autogobble]{python}
                w = u
                v = w
                u = v
                w = u + v
                \end{minted}
            
\item 
                \begin{minted}[autogobble]{python}
                u = v
                v = w
                w = u
                w = u + v
                \end{minted}
            
\item 
                \begin{minted}[autogobble]{python}
                u = v
                w = v
                v = u
                w = u + v
                \end{minted}
            
\item 
            \begin{minted}[autogobble]{python}
            w = u
            u = v
            v = w
            w = u + v
            \end{minted}
        
\end{enumerate}

        \end{multicols}
    
        \naloga*
        \begin{multicols}{2}
        \noindent
        Kakšne vrste napak vsebuje program na desni?

        \begin{enumerate}[(a)]
\item sintaktične napake, zaradi katerih Python programa noče izvesti.
\item vsebinske napake, zaradi katerih Python izračuna napačen rezultat
\item napake, zaradi katerih Python prekine z izvajanjem programa
\item oblikovne napake, ki ne vplivajo na pravilnost rezultata
\end{enumerate}

        \columnbreak

        \begin{minted}[baselinestretch=1.2,escapeinside=||, autogobble]{python}
        
            define fibonacci(n):
                if n <= 0:
                    return 0
                elif n == 1
                    return 1
                else:
                    a = fibonacci(n - 1)
                    b = fibonacci(n - 2)
                  return a + b
            
        \end{minted}

        \end{multicols}

    
        \naloga*
        Katere vrstice izpiše klic \inlinepy{print(a(b(1)))}, če sta funkciji \inlinepy{a} in \inlinepy{b} definirani kot spodaj?

        \begin{multicols}{2}
        \begin{minted}[autogobble]{python}
        
            def a(x):
                print(x)
                return x + 6

            def b(y):
                return 4 * y
                print(y)
        
        \end{minted}

        \begin{enumerate}[(a)]
\item \inlinepy{[1, 10]}
\item \inlinepy{[1, 4, 10]}
\item \inlinepy{[10]}
\item \inlinepy{[4, 10]}
\end{enumerate}

        \end{multicols}
    
        \naloga*

        \begin{multicols}{2}
        \noindent
        Kateri pogoj preverja spodnja funkcija?
        \begin{minted}[autogobble]{python}
        
            def f(stavek):
                for z in stavek:
                    if z in 'aeiouAEIOU':
                        return True
                return False
            
        \end{minted}

        \begin{enumerate}[(a)]
\item ali niz ne vsebuje nobenega samoglasnika
\item ali niz vsebuje samo samoglasnike
\item ali niz vsebuje kakšen samoglasnik
\item ali niz vsebuje znak, ki ni samoglasnik
\end{enumerate}

        \end{multicols}
    
        \naloga*
        \begin{multicols}{2}
        \noindent
        V vsak prostor vpišite \textbf{natanko en znak} tako, da bo dobljeni program v spremenljivko \inlinepy{vs} shranil vsoto števil \inlinepy{q} in \inlinepy{p}:
        
        \columnbreak
        \begin{minted}[baselinestretch=1.2,escapeinside=||]{python}
        vs = |\answerbox{0.5}|
        while q > |\answerbox{0.5}|:
            vs += |\answerbox{0.5}|
            |\answerbox{0.5}| -= 1
        \end{minted}
        \end{multicols}
    
        \clearpage
        \naloga
        
        Katera izmed spodnjih funkcij izračuna ostanek pri deljenju naravnega števila \inlinepy{m} z naravnim številom \inlinepy{n}?
    
        \begin{multicols}{2}
        \begin{enumerate}[(a)]
\item 
                \begin{minted}[autogobble]{python}
                def ostanek(m, n):
                    if m < n:
                        return 0
                    else:
                        return ostanek(m - n, n)
                \end{minted}
            
\item 
                \begin{minted}[autogobble]{python}
                def ostanek(m, n):
                    if n == 0:
                        return m
                    else:
                        return ostanek(n, m % n)
                \end{minted}
            
\item 
                \begin{minted}[autogobble]{python}
                def ostanek(m, n):
                    if m < n:
                        return m
                    else:
                        return ostanek(m % n)
                \end{minted}
            
\item 
            \begin{minted}[autogobble]{python}
            def ostanek(m, n):
                if m < n:
                    return m
                else:
                    return ostanek(m - n, n)
            \end{minted}
        
\end{enumerate}

        \end{multicols}
    
        \naloga*
        
        Katera izmed funkcij vrača drugačne rezultate kot ostale?
    
        \begin{multicols}{2}
        \begin{enumerate}[(a)]
\item 
                \begin{minted}[autogobble]{python}
                def f(b, a, c):
                    if b and a:
                        return c
                    else:
                        return False
                \end{minted}
            
\item 
                \begin{minted}[autogobble]{python}
                def f(b, a, c):
                    if not a:
                        return False
                    else:
                        return b and c
                \end{minted}
            
\item 
            \begin{minted}[autogobble]{python}
            def f(b, a, c):
                if not c:
                    return b and a
                else:
                    return False
            \end{minted}
        
\item 
                \begin{minted}[autogobble]{python}
                def f(b, a, c):
                    if b:
                        return a and c
                    else:
                        return False
                \end{minted}
            
\end{enumerate}

        \end{multicols}
    
        \naloga*
        
        Kateri izmed spodnjih programov ima drugačen izpis kot ostali?
    
        \begin{multicols}{2}
        \begin{enumerate}[(a)]
\item 
                \begin{minted}[autogobble]{python}
                    for n in range(1, 150, 3):
                        if n > 50:
                            break
                        print(n)
                \end{minted}
            
\item 
            \begin{minted}[autogobble]{python}
                for n in range(1, 50):
                    if n % 3 != 0:
                        continue
                    print(n)
            \end{minted}
        
\item 
                \begin{minted}[autogobble]{python}
                    for n in range(1, 150, 3):
                        if n < 50:
                            print(n)
                \end{minted}
            
\item 
                \begin{minted}[autogobble]{python}
                    for n in range(1, 50):
                        if n % 3 == 1:
                            print(n)
                        continue
                \end{minted}
            
\end{enumerate}

        \end{multicols}
    
        \naloga*
        \begin{multicols}{2}
        \noindent
        Napišite primer vrednosti spremenljivk \inlinepy{niz1} in \inlinepy{niz2}, za kateri klic \inlinepy{g(niz1, niz2)} vrne \inlinepy{True}.
        \begin{minted}[baselinestretch=1.2, escapeinside=||]{python}
        niz1 = |\answerbox{3}|
        niz2 = |\answerbox{3}|
        \end{minted}
        \vfil
        \columnbreak
        \begin{minted}[autogobble]{python}
        def g(niz1, niz2):
            if len(niz1) != len(niz2) or len(niz1) < 3:
                return False
            for i in range(len(niz1)):
                if i % 2 == 0 and niz1[i] != niz2[i]:
                    return False
            return True
        \end{minted}
        \end{multicols}
    
        \naloga*
        \begin{multicols}{2}
        \noindent 
        S številkami od $0$ do $4$ označite vrstni red, v katerem moramo izvesti ukaze na desni, da bo na koncu spremenljivka \inlinepy{sez1} kazala na seznam \inlinepy{[4, 7, 2]}?
    
        \columnbreak
        \noindent
        \begin{minted}[baselinestretch=1.2, escapeinside=||]{python}
|\answerbox{0.5}| sez1 = sez2
|\answerbox{0.5}| sez2 = [4]
|\answerbox{0.5}| sez2 = [7]
|\answerbox{0.5}| sez1.append(2)
|\answerbox{0.5}| sez1 = sez2 + sez1

        \end{minted}
        \end{multicols}
    
            \izpit[ucilnica=205, naloge=-1]{Uvod v programiranje: Kolokvij \#018}{27.\ marec 2019}{
                Pri vsaki nalogi obkrožite črko pred pravilnim odgovorom ali vpišite pravilno vrednost v ustrezen prostor. \\
                Čas reševanja je 30 minut. Veliko uspeha!
            }
            
        \naloga*
        
        Kateri izmed programov pri začetnem stanju
            \inlinepy{x = 7} in
            \inlinepy{y = 1}
        nastavi vrednosti
            \inlinepy{x = 1},
            \inlinepy{y = 7} in
            \inlinepy{z = 8}?
    
        \begin{multicols}{4}
        \begin{enumerate}[(a)]
\item 
                \begin{minted}[autogobble]{python}
                x = y
                y = z
                z = x
                z = x + y
                \end{minted}
            
\item 
                \begin{minted}[autogobble]{python}
                x = y
                z = y
                y = x
                z = x + y
                \end{minted}
            
\item 
            \begin{minted}[autogobble]{python}
            z = x
            x = y
            y = z
            z = x + y
            \end{minted}
        
\item 
                \begin{minted}[autogobble]{python}
                z = x
                y = z
                x = y
                z = x + y
                \end{minted}
            
\end{enumerate}

        \end{multicols}
    
        \naloga*
        \begin{multicols}{2}
        \noindent
        Kakšne vrste napak vsebuje program na desni?

        \begin{enumerate}[(a)]
\item sintaktične napake, zaradi katerih Python programa noče izvesti.
\item oblikovne napake, ki ne vplivajo na pravilnost rezultata
\item vsebinske napake, zaradi katerih Python izračuna napačen rezultat
\item napake, zaradi katerih Python prekine z izvajanjem programa
\end{enumerate}

        \columnbreak

        \begin{minted}[baselinestretch=1.2,escapeinside=||, autogobble]{python}
        
            def fibonacci(n):
                if n<=0:
                    return (0)
                elif n==1:
                    return (1)
                else:
                    a = fibonacci(n - 1)
                    b = fibonacci(n - 2)
                    return   a + b
            
        \end{minted}

        \end{multicols}

    
        \naloga*
        Katere vrstice izpiše klic \inlinepy{print(a(b(1)))}, če sta funkciji \inlinepy{a} in \inlinepy{b} definirani kot spodaj?

        \begin{multicols}{2}
        \begin{minted}[autogobble]{python}
        
            def a(x):
                return x + 5
                print(x)

            def b(y):
                print(y)
                return 9 * y
        
        \end{minted}

        \begin{enumerate}[(a)]
\item \inlinepy{[1, 14]}
\item \inlinepy{[9, 14]}
\item \inlinepy{[1, 9, 14]}
\item \inlinepy{[14]}
\end{enumerate}

        \end{multicols}
    
        \naloga*

        \begin{multicols}{2}
        \noindent
        Kateri pogoj preverja spodnja funkcija?
        \begin{minted}[autogobble]{python}
        
            def f(besedilo):
                for x in besedilo:
                    if x in 'aeiouAEIOU':
                        return False
                return True
            
        \end{minted}

        \begin{enumerate}[(a)]
\item ali niz vsebuje znak, ki ni samoglasnik
\item ali niz ne vsebuje nobenega samoglasnika
\item ali niz vsebuje kakšen samoglasnik
\item ali niz vsebuje samo samoglasnike
\end{enumerate}

        \end{multicols}
    
        \naloga*
        \begin{multicols}{2}
        \noindent
        V vsak prostor vpišite \textbf{natanko en znak} tako, da bo dobljeni program v spremenljivko \inlinepy{vs} shranil vsoto števil \inlinepy{p} in \inlinepy{q}:
        
        \columnbreak
        \begin{minted}[baselinestretch=1.2,escapeinside=||]{python}
        vs = |\answerbox{0.5}|
        while p > |\answerbox{0.5}|:
            vs += |\answerbox{0.5}|
            |\answerbox{0.5}| -= 1
        \end{minted}
        \end{multicols}
    
        \clearpage
        \naloga
        
        Katera izmed spodnjih funkcij izračuna ostanek pri deljenju naravnega števila \inlinepy{u} z naravnim številom \inlinepy{v}?
    
        \begin{multicols}{2}
        \begin{enumerate}[(a)]
\item 
                \begin{minted}[autogobble]{python}
                def ostanek(u, v):
                    if v == 0:
                        return u
                    else:
                        return ostanek(v, u % v)
                \end{minted}
            
\item 
                \begin{minted}[autogobble]{python}
                def ostanek(u, v):
                    if u < v:
                        return u
                    else:
                        return ostanek(u % v)
                \end{minted}
            
\item 
                \begin{minted}[autogobble]{python}
                def ostanek(u, v):
                    if u < v:
                        return 0
                    else:
                        return ostanek(u - v, v)
                \end{minted}
            
\item 
            \begin{minted}[autogobble]{python}
            def ostanek(u, v):
                if u < v:
                    return u
                else:
                    return ostanek(u - v, v)
            \end{minted}
        
\end{enumerate}

        \end{multicols}
    
        \naloga*
        
        Katera izmed funkcij vrača drugačne rezultate kot ostale?
    
        \begin{multicols}{2}
        \begin{enumerate}[(a)]
\item 
            \begin{minted}[autogobble]{python}
            def f(y, z, x):
                if not x:
                    return y and z
                else:
                    return False
            \end{minted}
        
\item 
                \begin{minted}[autogobble]{python}
                def f(y, z, x):
                    if y:
                        return z and x
                    else:
                        return False
                \end{minted}
            
\item 
                \begin{minted}[autogobble]{python}
                def f(y, z, x):
                    if not z:
                        return False
                    else:
                        return y and x
                \end{minted}
            
\item 
                \begin{minted}[autogobble]{python}
                def f(y, z, x):
                    if y and z:
                        return x
                    else:
                        return False
                \end{minted}
            
\end{enumerate}

        \end{multicols}
    
        \naloga*
        
        Kateri izmed spodnjih programov ima drugačen izpis kot ostali?
    
        \begin{multicols}{2}
        \begin{enumerate}[(a)]
\item 
                \begin{minted}[autogobble]{python}
                    for i in range(1, 60, 3):
                        if i > 20:
                            break
                        print(i)
                \end{minted}
            
\item 
                \begin{minted}[autogobble]{python}
                    for i in range(1, 60, 3):
                        if i < 20:
                            print(i)
                \end{minted}
            
\item 
            \begin{minted}[autogobble]{python}
                for i in range(1, 20):
                    if i % 3 != 0:
                        continue
                    print(i)
            \end{minted}
        
\item 
                \begin{minted}[autogobble]{python}
                    for i in range(1, 20):
                        if i % 3 == 1:
                            print(i)
                        continue
                \end{minted}
            
\end{enumerate}

        \end{multicols}
    
        \naloga*
        \begin{multicols}{2}
        \noindent
        Napišite primer vrednosti spremenljivk \inlinepy{lst1} in \inlinepy{lst2}, za kateri klic \inlinepy{h(lst1, lst2)} vrne \inlinepy{True}.
        \begin{minted}[baselinestretch=1.2, escapeinside=||]{python}
        lst1 = |\answerbox{3}|
        lst2 = |\answerbox{3}|
        \end{minted}
        \vfil
        \columnbreak
        \begin{minted}[autogobble]{python}
        def h(lst1, lst2):
            if len(lst1) != len(lst2) or len(lst1) < 3:
                return False
            for i in range(len(lst1)):
                if i % 2 == 1 and lst1[i] != lst2[i]:
                    return False
            return True
        \end{minted}
        \end{multicols}
    
        \naloga*
        \begin{multicols}{2}
        \noindent 
        S številkami od $0$ do $4$ označite vrstni red, v katerem moramo izvesti ukaze na desni, da bo na koncu spremenljivka \inlinepy{lst1} kazala na seznam \inlinepy{[5, 8, 7]}?
    
        \columnbreak
        \noindent
        \begin{minted}[baselinestretch=1.2, escapeinside=||]{python}
|\answerbox{0.5}| lst1 = lst2
|\answerbox{0.5}| lst2 = [8]
|\answerbox{0.5}| lst1.append(7)
|\answerbox{0.5}| lst1 = lst2 + lst1
|\answerbox{0.5}| lst2 = [5]

        \end{minted}
        \end{multicols}
    
            \izpit[ucilnica=205, naloge=-1]{Uvod v programiranje: Kolokvij \#019}{27.\ marec 2019}{
                Pri vsaki nalogi obkrožite črko pred pravilnim odgovorom ali vpišite pravilno vrednost v ustrezen prostor. \\
                Čas reševanja je 30 minut. Veliko uspeha!
            }
            
        \naloga*
        
        Kateri izmed programov pri začetnem stanju
            \inlinepy{u = 2} in
            \inlinepy{v = 7}
        nastavi vrednosti
            \inlinepy{u = 7},
            \inlinepy{v = 2} in
            \inlinepy{w = 9}?
    
        \begin{multicols}{4}
        \begin{enumerate}[(a)]
\item 
                \begin{minted}[autogobble]{python}
                u = v
                w = v
                v = u
                w = u + v
                \end{minted}
            
\item 
            \begin{minted}[autogobble]{python}
            w = u
            u = v
            v = w
            w = u + v
            \end{minted}
        
\item 
                \begin{minted}[autogobble]{python}
                w = u
                v = w
                u = v
                w = u + v
                \end{minted}
            
\item 
                \begin{minted}[autogobble]{python}
                u = v
                v = w
                w = u
                w = u + v
                \end{minted}
            
\end{enumerate}

        \end{multicols}
    
        \naloga*
        \begin{multicols}{2}
        \noindent
        Kakšne vrste napak vsebuje program na desni?

        \begin{enumerate}[(a)]
\item vsebinske napake, zaradi katerih Python izračuna napačen rezultat
\item napake, zaradi katerih Python prekine z izvajanjem programa
\item oblikovne napake, ki ne vplivajo na pravilnost rezultata
\item sintaktične napake, zaradi katerih Python programa noče izvesti.
\end{enumerate}

        \columnbreak

        \begin{minted}[baselinestretch=1.2,escapeinside=||, autogobble]{python}
        
            def fibonacci(n):
                if n <= 0:
                    return 0
                elif n == 1:
                    return 0
                else:
                    a = fibonacci(n - 1)
                    b = fibonacci(n - 3)
                    return a * b
            
        \end{minted}

        \end{multicols}

    
        \naloga*
        Katere vrstice izpiše klic \inlinepy{print(f(g(1)))}, če sta funkciji \inlinepy{f} in \inlinepy{g} definirani kot spodaj?

        \begin{multicols}{2}
        \begin{minted}[autogobble]{python}
        
            def f(x):
                return x + 9
                print(x)

            def g(y):
                return 6 * y
                print(y)
        
        \end{minted}

        \begin{enumerate}[(a)]
\item \inlinepy{[15]}
\item \inlinepy{[6, 15]}
\item \inlinepy{[1, 15]}
\item \inlinepy{[1, 6, 15]}
\end{enumerate}

        \end{multicols}
    
        \naloga*

        \begin{multicols}{2}
        \noindent
        Kateri pogoj preverja spodnja funkcija?
        \begin{minted}[autogobble]{python}
        
            def f(stavek):
                for znak in stavek:
                    if znak not in 'aeiouAEIOU':
                        return False
                return True
            
        \end{minted}

        \begin{enumerate}[(a)]
\item ali niz vsebuje znak, ki ni samoglasnik
\item ali niz ne vsebuje nobenega samoglasnika
\item ali niz vsebuje kakšen samoglasnik
\item ali niz vsebuje samo samoglasnike
\end{enumerate}

        \end{multicols}
    
        \naloga*
        \begin{multicols}{2}
        \noindent
        V vsak prostor vpišite \textbf{natanko en znak} tako, da bo dobljeni program v spremenljivko \inlinepy{vs} shranil vsoto števil \inlinepy{a} in \inlinepy{b}:
        
        \columnbreak
        \begin{minted}[baselinestretch=1.2,escapeinside=||]{python}
        vs = |\answerbox{0.5}|
        while a > |\answerbox{0.5}|:
            vs += |\answerbox{0.5}|
            |\answerbox{0.5}| -= 1
        \end{minted}
        \end{multicols}
    
        \clearpage
        \naloga
        
        Katera izmed spodnjih funkcij izračuna ostanek pri deljenju naravnega števila \inlinepy{a} z naravnim številom \inlinepy{b}?
    
        \begin{multicols}{2}
        \begin{enumerate}[(a)]
\item 
                \begin{minted}[autogobble]{python}
                def ostanek(a, b):
                    if a < b:
                        return 0
                    else:
                        return ostanek(a - b, b)
                \end{minted}
            
\item 
                \begin{minted}[autogobble]{python}
                def ostanek(a, b):
                    if a < b:
                        return a
                    else:
                        return ostanek(a % b)
                \end{minted}
            
\item 
                \begin{minted}[autogobble]{python}
                def ostanek(a, b):
                    if b == 0:
                        return a
                    else:
                        return ostanek(b, a % b)
                \end{minted}
            
\item 
            \begin{minted}[autogobble]{python}
            def ostanek(a, b):
                if a < b:
                    return a
                else:
                    return ostanek(a - b, b)
            \end{minted}
        
\end{enumerate}

        \end{multicols}
    
        \naloga*
        
        Katera izmed funkcij vrača drugačne rezultate kot ostale?
    
        \begin{multicols}{2}
        \begin{enumerate}[(a)]
\item 
                \begin{minted}[autogobble]{python}
                def h(q, p, r):
                    if q:
                        return p and r
                    else:
                        return False
                \end{minted}
            
\item 
                \begin{minted}[autogobble]{python}
                def h(q, p, r):
                    if q and p:
                        return r
                    else:
                        return False
                \end{minted}
            
\item 
            \begin{minted}[autogobble]{python}
            def h(q, p, r):
                if not r:
                    return q and p
                else:
                    return False
            \end{minted}
        
\item 
                \begin{minted}[autogobble]{python}
                def h(q, p, r):
                    if not p:
                        return False
                    else:
                        return q and r
                \end{minted}
            
\end{enumerate}

        \end{multicols}
    
        \naloga*
        
        Kateri izmed spodnjih programov ima drugačen izpis kot ostali?
    
        \begin{multicols}{2}
        \begin{enumerate}[(a)]
\item 
            \begin{minted}[autogobble]{python}
                for j in range(1, 20):
                    if j % 5 != 0:
                        continue
                    print(j)
            \end{minted}
        
\item 
                \begin{minted}[autogobble]{python}
                    for j in range(1, 20):
                        if j % 5 == 1:
                            print(j)
                        continue
                \end{minted}
            
\item 
                \begin{minted}[autogobble]{python}
                    for j in range(1, 100, 5):
                        if j < 20:
                            print(j)
                \end{minted}
            
\item 
                \begin{minted}[autogobble]{python}
                    for j in range(1, 100, 5):
                        if j > 20:
                            break
                        print(j)
                \end{minted}
            
\end{enumerate}

        \end{multicols}
    
        \naloga*
        \begin{multicols}{2}
        \noindent
        Napišite primer vrednosti spremenljivk \inlinepy{str1} in \inlinepy{str2}, za kateri klic \inlinepy{h(str1, str2)} vrne \inlinepy{True}.
        \begin{minted}[baselinestretch=1.2, escapeinside=||]{python}
        str1 = |\answerbox{3}|
        str2 = |\answerbox{3}|
        \end{minted}
        \vfil
        \columnbreak
        \begin{minted}[autogobble]{python}
        def h(str1, str2):
            if len(str1) != len(str2) or len(str1) < 3:
                return False
            for j in range(len(str1)):
                if j % 2 == 1 and str1[j] != str2[j]:
                    return False
            return True
        \end{minted}
        \end{multicols}
    
        \naloga*
        \begin{multicols}{2}
        \noindent 
        S številkami od $0$ do $4$ označite vrstni red, v katerem moramo izvesti ukaze na desni, da bo na koncu spremenljivka \inlinepy{sez1} kazala na seznam \inlinepy{[9, 4, 3]}?
    
        \columnbreak
        \noindent
        \begin{minted}[baselinestretch=1.2, escapeinside=||]{python}
|\answerbox{0.5}| sez1.append(3)
|\answerbox{0.5}| sez2 = [4]
|\answerbox{0.5}| sez1 = sez2
|\answerbox{0.5}| sez2 = [9]
|\answerbox{0.5}| sez1 = sez2 + sez1

        \end{minted}
        \end{multicols}
    
            \izpit[ucilnica=205, naloge=-1]{Uvod v programiranje: Kolokvij \#020}{27.\ marec 2019}{
                Pri vsaki nalogi obkrožite črko pred pravilnim odgovorom ali vpišite pravilno vrednost v ustrezen prostor. \\
                Čas reševanja je 30 minut. Veliko uspeha!
            }
            
        \naloga*
        
        Kateri izmed programov pri začetnem stanju
            \inlinepy{k = 1} in
            \inlinepy{m = 2}
        nastavi vrednosti
            \inlinepy{k = 2},
            \inlinepy{m = 1} in
            \inlinepy{n = 3}?
    
        \begin{multicols}{4}
        \begin{enumerate}[(a)]
\item 
                \begin{minted}[autogobble]{python}
                k = m
                n = m
                m = k
                n = k + m
                \end{minted}
            
\item 
                \begin{minted}[autogobble]{python}
                k = m
                m = n
                n = k
                n = k + m
                \end{minted}
            
\item 
            \begin{minted}[autogobble]{python}
            n = k
            k = m
            m = n
            n = k + m
            \end{minted}
        
\item 
                \begin{minted}[autogobble]{python}
                n = k
                m = n
                k = m
                n = k + m
                \end{minted}
            
\end{enumerate}

        \end{multicols}
    
        \naloga*
        \begin{multicols}{2}
        \noindent
        Kakšne vrste napak vsebuje program na desni?

        \begin{enumerate}[(a)]
\item vsebinske napake, zaradi katerih Python izračuna napačen rezultat
\item sintaktične napake, zaradi katerih Python programa noče izvesti.
\item oblikovne napake, ki ne vplivajo na pravilnost rezultata
\item napake, zaradi katerih Python prekine z izvajanjem programa
\end{enumerate}

        \columnbreak

        \begin{minted}[baselinestretch=1.2,escapeinside=||, autogobble]{python}
        
            def fibonacci(n):
                if n <= 0:
                    return 0
                elif n == 1:
                    return 0
                else:
                    a = fibonacci(n - 1)
                    b = fibonacci(n - 3)
                    return a * b
            
        \end{minted}

        \end{multicols}

    
        \naloga*
        Katere vrstice izpiše klic \inlinepy{print(a(b(5)))}, če sta funkciji \inlinepy{a} in \inlinepy{b} definirani kot spodaj?

        \begin{multicols}{2}
        \begin{minted}[autogobble]{python}
        
            def a(x):
                print(x)
                return x + 1

            def b(y):
                return 6 * y
                print(y)
        
        \end{minted}

        \begin{enumerate}[(a)]
\item \inlinepy{[5, 30, 31]}
\item \inlinepy{[31]}
\item \inlinepy{[30, 31]}
\item \inlinepy{[5, 31]}
\end{enumerate}

        \end{multicols}
    
        \naloga*

        \begin{multicols}{2}
        \noindent
        Kateri pogoj preverja spodnja funkcija?
        \begin{minted}[autogobble]{python}
        
            def f(stavek):
                for x in stavek:
                    if x not in 'aeiouAEIOU':
                        return True
                return False
            
        \end{minted}

        \begin{enumerate}[(a)]
\item ali niz vsebuje kakšen samoglasnik
\item ali niz ne vsebuje nobenega samoglasnika
\item ali niz vsebuje znak, ki ni samoglasnik
\item ali niz vsebuje samo samoglasnike
\end{enumerate}

        \end{multicols}
    
        \naloga*
        \begin{multicols}{2}
        \noindent
        V vsak prostor vpišite \textbf{natanko en znak} tako, da bo dobljeni program v spremenljivko \inlinepy{vs} shranil vsoto števil \inlinepy{y} in \inlinepy{x}:
        
        \columnbreak
        \begin{minted}[baselinestretch=1.2,escapeinside=||]{python}
        vs = |\answerbox{0.5}|
        while y > |\answerbox{0.5}|:
            vs += |\answerbox{0.5}|
            |\answerbox{0.5}| -= 1
        \end{minted}
        \end{multicols}
    
        \clearpage
        \naloga
        
        Katera izmed spodnjih funkcij izračuna ostanek pri deljenju naravnega števila \inlinepy{u} z naravnim številom \inlinepy{v}?
    
        \begin{multicols}{2}
        \begin{enumerate}[(a)]
\item 
                \begin{minted}[autogobble]{python}
                def ostanek(u, v):
                    if u < v:
                        return 0
                    else:
                        return ostanek(u - v, v)
                \end{minted}
            
\item 
                \begin{minted}[autogobble]{python}
                def ostanek(u, v):
                    if v == 0:
                        return u
                    else:
                        return ostanek(v, u % v)
                \end{minted}
            
\item 
                \begin{minted}[autogobble]{python}
                def ostanek(u, v):
                    if u < v:
                        return u
                    else:
                        return ostanek(u % v)
                \end{minted}
            
\item 
            \begin{minted}[autogobble]{python}
            def ostanek(u, v):
                if u < v:
                    return u
                else:
                    return ostanek(u - v, v)
            \end{minted}
        
\end{enumerate}

        \end{multicols}
    
        \naloga*
        
        Katera izmed funkcij vrača drugačne rezultate kot ostale?
    
        \begin{multicols}{2}
        \begin{enumerate}[(a)]
\item 
                \begin{minted}[autogobble]{python}
                def g(b, a, c):
                    if not a:
                        return False
                    else:
                        return b and c
                \end{minted}
            
\item 
                \begin{minted}[autogobble]{python}
                def g(b, a, c):
                    if b and a:
                        return c
                    else:
                        return False
                \end{minted}
            
\item 
            \begin{minted}[autogobble]{python}
            def g(b, a, c):
                if not c:
                    return b and a
                else:
                    return False
            \end{minted}
        
\item 
                \begin{minted}[autogobble]{python}
                def g(b, a, c):
                    if b:
                        return a and c
                    else:
                        return False
                \end{minted}
            
\end{enumerate}

        \end{multicols}
    
        \naloga*
        
        Kateri izmed spodnjih programov ima drugačen izpis kot ostali?
    
        \begin{multicols}{2}
        \begin{enumerate}[(a)]
\item 
                \begin{minted}[autogobble]{python}
                    for j in range(1, 40, 2):
                        if j > 20:
                            break
                        print(j)
                \end{minted}
            
\item 
                \begin{minted}[autogobble]{python}
                    for j in range(1, 20):
                        if j % 2 == 1:
                            print(j)
                        continue
                \end{minted}
            
\item 
            \begin{minted}[autogobble]{python}
                for j in range(1, 20):
                    if j % 2 != 0:
                        continue
                    print(j)
            \end{minted}
        
\item 
                \begin{minted}[autogobble]{python}
                    for j in range(1, 40, 2):
                        if j < 20:
                            print(j)
                \end{minted}
            
\end{enumerate}

        \end{multicols}
    
        \naloga*
        \begin{multicols}{2}
        \noindent
        Napišite primer vrednosti spremenljivk \inlinepy{niz1} in \inlinepy{niz2}, za kateri klic \inlinepy{g(niz1, niz2)} vrne \inlinepy{True}.
        \begin{minted}[baselinestretch=1.2, escapeinside=||]{python}
        niz1 = |\answerbox{3}|
        niz2 = |\answerbox{3}|
        \end{minted}
        \vfil
        \columnbreak
        \begin{minted}[autogobble]{python}
        def g(niz1, niz2):
            if len(niz1) != len(niz2) or len(niz1) < 3:
                return False
            for j in range(len(niz1)):
                if j % 2 == 1 and niz1[j] != niz2[j]:
                    return False
            return True
        \end{minted}
        \end{multicols}
    
        \naloga*
        \begin{multicols}{2}
        \noindent 
        S številkami od $0$ do $4$ označite vrstni red, v katerem moramo izvesti ukaze na desni, da bo na koncu spremenljivka \inlinepy{lst2} kazala na seznam \inlinepy{[2, 7, 9]}?
    
        \columnbreak
        \noindent
        \begin{minted}[baselinestretch=1.2, escapeinside=||]{python}
|\answerbox{0.5}| lst2.append(9)
|\answerbox{0.5}| lst2 = lst1 + lst2
|\answerbox{0.5}| lst1 = [2]
|\answerbox{0.5}| lst1 = [7]
|\answerbox{0.5}| lst2 = lst1

        \end{minted}
        \end{multicols}
    
            \izpit[ucilnica=205, naloge=-1]{Uvod v programiranje: Kolokvij \#021}{27.\ marec 2019}{
                Pri vsaki nalogi obkrožite črko pred pravilnim odgovorom ali vpišite pravilno vrednost v ustrezen prostor. \\
                Čas reševanja je 30 minut. Veliko uspeha!
            }
            
        \naloga*
        
        Kateri izmed programov pri začetnem stanju
            \inlinepy{k = 1} in
            \inlinepy{m = 3}
        nastavi vrednosti
            \inlinepy{k = 3},
            \inlinepy{m = 1} in
            \inlinepy{n = 4}?
    
        \begin{multicols}{4}
        \begin{enumerate}[(a)]
\item 
                \begin{minted}[autogobble]{python}
                k = m
                m = n
                n = k
                n = k + m
                \end{minted}
            
\item 
                \begin{minted}[autogobble]{python}
                k = m
                n = m
                m = k
                n = k + m
                \end{minted}
            
\item 
                \begin{minted}[autogobble]{python}
                n = k
                m = n
                k = m
                n = k + m
                \end{minted}
            
\item 
            \begin{minted}[autogobble]{python}
            n = k
            k = m
            m = n
            n = k + m
            \end{minted}
        
\end{enumerate}

        \end{multicols}
    
        \naloga*
        \begin{multicols}{2}
        \noindent
        Kakšne vrste napak vsebuje program na desni?

        \begin{enumerate}[(a)]
\item sintaktične napake, zaradi katerih Python programa noče izvesti.
\item vsebinske napake, zaradi katerih Python izračuna napačen rezultat
\item napake, zaradi katerih Python prekine z izvajanjem programa
\item oblikovne napake, ki ne vplivajo na pravilnost rezultata
\end{enumerate}

        \columnbreak

        \begin{minted}[baselinestretch=1.2,escapeinside=||, autogobble]{python}
        
            define fibonacci(n):
                if n <= 0:
                    return 0
                elif n == 1
                    return 1
                else:
                    a = fibonacci(n - 1)
                    b = fibonacci(n - 2)
                  return a + b
            
        \end{minted}

        \end{multicols}

    
        \naloga*
        Katere vrstice izpiše klic \inlinepy{print(p(q(2)))}, če sta funkciji \inlinepy{p} in \inlinepy{q} definirani kot spodaj?

        \begin{multicols}{2}
        \begin{minted}[autogobble]{python}
        
            def p(a):
                print(a)
                return a + 7

            def q(b):
                return 1 * b
                print(b)
        
        \end{minted}

        \begin{enumerate}[(a)]
\item \inlinepy{[2, 2, 9]}
\item \inlinepy{[2, 9]}
\item \inlinepy{[2, 9]}
\item \inlinepy{[9]}
\end{enumerate}

        \end{multicols}
    
        \naloga*

        \begin{multicols}{2}
        \noindent
        Kateri pogoj preverja spodnja funkcija?
        \begin{minted}[autogobble]{python}
        
            def f(besedilo):
                for x in besedilo:
                    if x not in 'aeiouAEIOU':
                        return False
                return True
            
        \end{minted}

        \begin{enumerate}[(a)]
\item ali niz vsebuje znak, ki ni samoglasnik
\item ali niz vsebuje kakšen samoglasnik
\item ali niz ne vsebuje nobenega samoglasnika
\item ali niz vsebuje samo samoglasnike
\end{enumerate}

        \end{multicols}
    
        \naloga*
        \begin{multicols}{2}
        \noindent
        V vsak prostor vpišite \textbf{natanko en znak} tako, da bo dobljeni program v spremenljivko \inlinepy{vs} shranil vsoto števil \inlinepy{x} in \inlinepy{y}:
        
        \columnbreak
        \begin{minted}[baselinestretch=1.2,escapeinside=||]{python}
        vs = |\answerbox{0.5}|
        while x > |\answerbox{0.5}|:
            vs += |\answerbox{0.5}|
            |\answerbox{0.5}| -= 1
        \end{minted}
        \end{multicols}
    
        \clearpage
        \naloga
        
        Katera izmed spodnjih funkcij izračuna ostanek pri deljenju naravnega števila \inlinepy{m} z naravnim številom \inlinepy{n}?
    
        \begin{multicols}{2}
        \begin{enumerate}[(a)]
\item 
            \begin{minted}[autogobble]{python}
            def ostanek(m, n):
                if m < n:
                    return m
                else:
                    return ostanek(m - n, n)
            \end{minted}
        
\item 
                \begin{minted}[autogobble]{python}
                def ostanek(m, n):
                    if m < n:
                        return m
                    else:
                        return ostanek(m % n)
                \end{minted}
            
\item 
                \begin{minted}[autogobble]{python}
                def ostanek(m, n):
                    if m < n:
                        return 0
                    else:
                        return ostanek(m - n, n)
                \end{minted}
            
\item 
                \begin{minted}[autogobble]{python}
                def ostanek(m, n):
                    if n == 0:
                        return m
                    else:
                        return ostanek(n, m % n)
                \end{minted}
            
\end{enumerate}

        \end{multicols}
    
        \naloga*
        
        Katera izmed funkcij vrača drugačne rezultate kot ostale?
    
        \begin{multicols}{2}
        \begin{enumerate}[(a)]
\item 
            \begin{minted}[autogobble]{python}
            def f(y, x, z):
                if not z:
                    return y and x
                else:
                    return False
            \end{minted}
        
\item 
                \begin{minted}[autogobble]{python}
                def f(y, x, z):
                    if not x:
                        return False
                    else:
                        return y and z
                \end{minted}
            
\item 
                \begin{minted}[autogobble]{python}
                def f(y, x, z):
                    if y and x:
                        return z
                    else:
                        return False
                \end{minted}
            
\item 
                \begin{minted}[autogobble]{python}
                def f(y, x, z):
                    if y:
                        return x and z
                    else:
                        return False
                \end{minted}
            
\end{enumerate}

        \end{multicols}
    
        \naloga*
        
        Kateri izmed spodnjih programov ima drugačen izpis kot ostali?
    
        \begin{multicols}{2}
        \begin{enumerate}[(a)]
\item 
            \begin{minted}[autogobble]{python}
                for x in range(1, 20):
                    if x % 5 != 0:
                        continue
                    print(x)
            \end{minted}
        
\item 
                \begin{minted}[autogobble]{python}
                    for x in range(1, 100, 5):
                        if x < 20:
                            print(x)
                \end{minted}
            
\item 
                \begin{minted}[autogobble]{python}
                    for x in range(1, 20):
                        if x % 5 == 1:
                            print(x)
                        continue
                \end{minted}
            
\item 
                \begin{minted}[autogobble]{python}
                    for x in range(1, 100, 5):
                        if x > 20:
                            break
                        print(x)
                \end{minted}
            
\end{enumerate}

        \end{multicols}
    
        \naloga*
        \begin{multicols}{2}
        \noindent
        Napišite primer vrednosti spremenljivk \inlinepy{niz1} in \inlinepy{niz2}, za kateri klic \inlinepy{h(niz1, niz2)} vrne \inlinepy{True}.
        \begin{minted}[baselinestretch=1.2, escapeinside=||]{python}
        niz1 = |\answerbox{3}|
        niz2 = |\answerbox{3}|
        \end{minted}
        \vfil
        \columnbreak
        \begin{minted}[autogobble]{python}
        def h(niz1, niz2):
            if len(niz1) != len(niz2) or len(niz1) < 3:
                return False
            for i in range(len(niz1)):
                if i % 2 == 1 and niz1[i] != niz2[i]:
                    return False
            return True
        \end{minted}
        \end{multicols}
    
        \naloga*
        \begin{multicols}{2}
        \noindent 
        S številkami od $0$ do $4$ označite vrstni red, v katerem moramo izvesti ukaze na desni, da bo na koncu spremenljivka \inlinepy{sez1} kazala na seznam \inlinepy{[2, 6, 3]}?
    
        \columnbreak
        \noindent
        \begin{minted}[baselinestretch=1.2, escapeinside=||]{python}
|\answerbox{0.5}| sez1.append(3)
|\answerbox{0.5}| sez2 = [2]
|\answerbox{0.5}| sez1 = sez2 + sez1
|\answerbox{0.5}| sez1 = sez2
|\answerbox{0.5}| sez2 = [6]

        \end{minted}
        \end{multicols}
    
            \izpit[ucilnica=205, naloge=-1]{Uvod v programiranje: Kolokvij \#022}{27.\ marec 2019}{
                Pri vsaki nalogi obkrožite črko pred pravilnim odgovorom ali vpišite pravilno vrednost v ustrezen prostor. \\
                Čas reševanja je 30 minut. Veliko uspeha!
            }
            
        \naloga*
        
        Kateri izmed programov pri začetnem stanju
            \inlinepy{u = 3} in
            \inlinepy{v = 9}
        nastavi vrednosti
            \inlinepy{u = 9},
            \inlinepy{v = 3} in
            \inlinepy{w = 12}?
    
        \begin{multicols}{4}
        \begin{enumerate}[(a)]
\item 
            \begin{minted}[autogobble]{python}
            w = u
            u = v
            v = w
            w = u + v
            \end{minted}
        
\item 
                \begin{minted}[autogobble]{python}
                w = u
                v = w
                u = v
                w = u + v
                \end{minted}
            
\item 
                \begin{minted}[autogobble]{python}
                u = v
                v = w
                w = u
                w = u + v
                \end{minted}
            
\item 
                \begin{minted}[autogobble]{python}
                u = v
                w = v
                v = u
                w = u + v
                \end{minted}
            
\end{enumerate}

        \end{multicols}
    
        \naloga*
        \begin{multicols}{2}
        \noindent
        Kakšne vrste napak vsebuje program na desni?

        \begin{enumerate}[(a)]
\item napake, zaradi katerih Python prekine z izvajanjem programa
\item oblikovne napake, ki ne vplivajo na pravilnost rezultata
\item sintaktične napake, zaradi katerih Python programa noče izvesti.
\item vsebinske napake, zaradi katerih Python izračuna napačen rezultat
\end{enumerate}

        \columnbreak

        \begin{minted}[baselinestretch=1.2,escapeinside=||, autogobble]{python}
        
            def fibonacci(n):
                if n <= 0:
                    return 0
                elif n == 1:
                    return 0
                else:
                    a = fibonacci(n - 1)
                    b = fibonacci(n - 3)
                    return a * b
            
        \end{minted}

        \end{multicols}

    
        \naloga*
        Katere vrstice izpiše klic \inlinepy{print(p(q(6)))}, če sta funkciji \inlinepy{p} in \inlinepy{q} definirani kot spodaj?

        \begin{multicols}{2}
        \begin{minted}[autogobble]{python}
        
            def p(a):
                print(a)
                return a + 7

            def q(b):
                print(b)
                return 5 * b
        
        \end{minted}

        \begin{enumerate}[(a)]
\item \inlinepy{[30, 37]}
\item \inlinepy{[6, 37]}
\item \inlinepy{[6, 30, 37]}
\item \inlinepy{[37]}
\end{enumerate}

        \end{multicols}
    
        \naloga*

        \begin{multicols}{2}
        \noindent
        Kateri pogoj preverja spodnja funkcija?
        \begin{minted}[autogobble]{python}
        
            def f(niz):
                for x in niz:
                    if x in 'aeiouAEIOU':
                        return False
                return True
            
        \end{minted}

        \begin{enumerate}[(a)]
\item ali niz vsebuje kakšen samoglasnik
\item ali niz vsebuje znak, ki ni samoglasnik
\item ali niz vsebuje samo samoglasnike
\item ali niz ne vsebuje nobenega samoglasnika
\end{enumerate}

        \end{multicols}
    
        \naloga*
        \begin{multicols}{2}
        \noindent
        V vsak prostor vpišite \textbf{natanko en znak} tako, da bo dobljeni program v spremenljivko \inlinepy{zm} shranil zmnožek števil \inlinepy{x} in \inlinepy{y}:
        
        \columnbreak
        \begin{minted}[baselinestretch=1.2,escapeinside=||]{python}
        zm = |\answerbox{0.5}|
        while x > |\answerbox{0.5}|:
            zm += |\answerbox{0.5}|
            |\answerbox{0.5}| -= 1
        \end{minted}
        \end{multicols}
    
        \clearpage
        \naloga
        
        Katera izmed spodnjih funkcij izračuna ostanek pri deljenju naravnega števila \inlinepy{a} z naravnim številom \inlinepy{b}?
    
        \begin{multicols}{2}
        \begin{enumerate}[(a)]
\item 
                \begin{minted}[autogobble]{python}
                def ostanek(a, b):
                    if b == 0:
                        return a
                    else:
                        return ostanek(b, a % b)
                \end{minted}
            
\item 
            \begin{minted}[autogobble]{python}
            def ostanek(a, b):
                if a < b:
                    return a
                else:
                    return ostanek(a - b, b)
            \end{minted}
        
\item 
                \begin{minted}[autogobble]{python}
                def ostanek(a, b):
                    if a < b:
                        return a
                    else:
                        return ostanek(a % b)
                \end{minted}
            
\item 
                \begin{minted}[autogobble]{python}
                def ostanek(a, b):
                    if a < b:
                        return 0
                    else:
                        return ostanek(a - b, b)
                \end{minted}
            
\end{enumerate}

        \end{multicols}
    
        \naloga*
        
        Katera izmed funkcij vrača drugačne rezultate kot ostale?
    
        \begin{multicols}{2}
        \begin{enumerate}[(a)]
\item 
                \begin{minted}[autogobble]{python}
                def f(z, y, x):
                    if z and y:
                        return x
                    else:
                        return False
                \end{minted}
            
\item 
                \begin{minted}[autogobble]{python}
                def f(z, y, x):
                    if z:
                        return y and x
                    else:
                        return False
                \end{minted}
            
\item 
            \begin{minted}[autogobble]{python}
            def f(z, y, x):
                if not x:
                    return z and y
                else:
                    return False
            \end{minted}
        
\item 
                \begin{minted}[autogobble]{python}
                def f(z, y, x):
                    if not y:
                        return False
                    else:
                        return z and x
                \end{minted}
            
\end{enumerate}

        \end{multicols}
    
        \naloga*
        
        Kateri izmed spodnjih programov ima drugačen izpis kot ostali?
    
        \begin{multicols}{2}
        \begin{enumerate}[(a)]
\item 
            \begin{minted}[autogobble]{python}
                for j in range(1, 10):
                    if j % 2 != 0:
                        continue
                    print(j)
            \end{minted}
        
\item 
                \begin{minted}[autogobble]{python}
                    for j in range(1, 20, 2):
                        if j > 10:
                            break
                        print(j)
                \end{minted}
            
\item 
                \begin{minted}[autogobble]{python}
                    for j in range(1, 20, 2):
                        if j < 10:
                            print(j)
                \end{minted}
            
\item 
                \begin{minted}[autogobble]{python}
                    for j in range(1, 10):
                        if j % 2 == 1:
                            print(j)
                        continue
                \end{minted}
            
\end{enumerate}

        \end{multicols}
    
        \naloga*
        \begin{multicols}{2}
        \noindent
        Napišite primer vrednosti spremenljivk \inlinepy{str1} in \inlinepy{str2}, za kateri klic \inlinepy{g(str1, str2)} vrne \inlinepy{True}.
        \begin{minted}[baselinestretch=1.2, escapeinside=||]{python}
        str1 = |\answerbox{3}|
        str2 = |\answerbox{3}|
        \end{minted}
        \vfil
        \columnbreak
        \begin{minted}[autogobble]{python}
        def g(str1, str2):
            if len(str1) != len(str2) or len(str1) < 3:
                return False
            for j in range(len(str1)):
                if j % 2 == 1 and str1[j] != str2[j]:
                    return False
            return True
        \end{minted}
        \end{multicols}
    
        \naloga*
        \begin{multicols}{2}
        \noindent 
        S številkami od $0$ do $4$ označite vrstni red, v katerem moramo izvesti ukaze na desni, da bo na koncu spremenljivka \inlinepy{sez2} kazala na seznam \inlinepy{[4, 6, 7]}?
    
        \columnbreak
        \noindent
        \begin{minted}[baselinestretch=1.2, escapeinside=||]{python}
|\answerbox{0.5}| sez1 = [4]
|\answerbox{0.5}| sez2 = sez1 + sez2
|\answerbox{0.5}| sez2 = sez1
|\answerbox{0.5}| sez2.append(7)
|\answerbox{0.5}| sez1 = [6]

        \end{minted}
        \end{multicols}
    
            \izpit[ucilnica=205, naloge=-1]{Uvod v programiranje: Kolokvij \#023}{27.\ marec 2019}{
                Pri vsaki nalogi obkrožite črko pred pravilnim odgovorom ali vpišite pravilno vrednost v ustrezen prostor. \\
                Čas reševanja je 30 minut. Veliko uspeha!
            }
            
        \naloga*
        
        Kateri izmed programov pri začetnem stanju
            \inlinepy{x = 4} in
            \inlinepy{y = 1}
        nastavi vrednosti
            \inlinepy{x = 1},
            \inlinepy{y = 4} in
            \inlinepy{z = 5}?
    
        \begin{multicols}{4}
        \begin{enumerate}[(a)]
\item 
            \begin{minted}[autogobble]{python}
            z = x
            x = y
            y = z
            z = x + y
            \end{minted}
        
\item 
                \begin{minted}[autogobble]{python}
                x = y
                z = y
                y = x
                z = x + y
                \end{minted}
            
\item 
                \begin{minted}[autogobble]{python}
                z = x
                y = z
                x = y
                z = x + y
                \end{minted}
            
\item 
                \begin{minted}[autogobble]{python}
                x = y
                y = z
                z = x
                z = x + y
                \end{minted}
            
\end{enumerate}

        \end{multicols}
    
        \naloga*
        \begin{multicols}{2}
        \noindent
        Kakšne vrste napak vsebuje program na desni?

        \begin{enumerate}[(a)]
\item napake, zaradi katerih Python prekine z izvajanjem programa
\item vsebinske napake, zaradi katerih Python izračuna napačen rezultat
\item sintaktične napake, zaradi katerih Python programa noče izvesti.
\item oblikovne napake, ki ne vplivajo na pravilnost rezultata
\end{enumerate}

        \columnbreak

        \begin{minted}[baselinestretch=1.2,escapeinside=||, autogobble]{python}
        
            def fibonacci(n):
                if n <= 0:
                    return 0
                elif n == 1:
                    return 1
                else:
                    a = fibonacci(n - '1')
                    b = fib(n - 2)
                    return a + b
            
        \end{minted}

        \end{multicols}

    
        \naloga*
        Katere vrstice izpiše klic \inlinepy{print(f(g(1)))}, če sta funkciji \inlinepy{f} in \inlinepy{g} definirani kot spodaj?

        \begin{multicols}{2}
        \begin{minted}[autogobble]{python}
        
            def f(x):
                print(x)
                return x + 7

            def g(y):
                return 6 * y
                print(y)
        
        \end{minted}

        \begin{enumerate}[(a)]
\item \inlinepy{[13]}
\item \inlinepy{[1, 6, 13]}
\item \inlinepy{[1, 13]}
\item \inlinepy{[6, 13]}
\end{enumerate}

        \end{multicols}
    
        \naloga*

        \begin{multicols}{2}
        \noindent
        Kateri pogoj preverja spodnja funkcija?
        \begin{minted}[autogobble]{python}
        
            def f(niz):
                for z in niz:
                    if z not in 'aeiouAEIOU':
                        return False
                return True
            
        \end{minted}

        \begin{enumerate}[(a)]
\item ali niz vsebuje samo samoglasnike
\item ali niz vsebuje znak, ki ni samoglasnik
\item ali niz ne vsebuje nobenega samoglasnika
\item ali niz vsebuje kakšen samoglasnik
\end{enumerate}

        \end{multicols}
    
        \naloga*
        \begin{multicols}{2}
        \noindent
        V vsak prostor vpišite \textbf{natanko en znak} tako, da bo dobljeni program v spremenljivko \inlinepy{vs} shranil vsoto števil \inlinepy{q} in \inlinepy{p}:
        
        \columnbreak
        \begin{minted}[baselinestretch=1.2,escapeinside=||]{python}
        vs = |\answerbox{0.5}|
        while q > |\answerbox{0.5}|:
            vs += |\answerbox{0.5}|
            |\answerbox{0.5}| -= 1
        \end{minted}
        \end{multicols}
    
        \clearpage
        \naloga
        
        Katera izmed spodnjih funkcij izračuna ostanek pri deljenju naravnega števila \inlinepy{m} z naravnim številom \inlinepy{n}?
    
        \begin{multicols}{2}
        \begin{enumerate}[(a)]
\item 
            \begin{minted}[autogobble]{python}
            def ostanek(m, n):
                if m < n:
                    return m
                else:
                    return ostanek(m - n, n)
            \end{minted}
        
\item 
                \begin{minted}[autogobble]{python}
                def ostanek(m, n):
                    if m < n:
                        return 0
                    else:
                        return ostanek(m - n, n)
                \end{minted}
            
\item 
                \begin{minted}[autogobble]{python}
                def ostanek(m, n):
                    if m < n:
                        return m
                    else:
                        return ostanek(m % n)
                \end{minted}
            
\item 
                \begin{minted}[autogobble]{python}
                def ostanek(m, n):
                    if n == 0:
                        return m
                    else:
                        return ostanek(n, m % n)
                \end{minted}
            
\end{enumerate}

        \end{multicols}
    
        \naloga*
        
        Katera izmed funkcij vrača drugačne rezultate kot ostale?
    
        \begin{multicols}{2}
        \begin{enumerate}[(a)]
\item 
                \begin{minted}[autogobble]{python}
                def h(c, a, b):
                    if c and a:
                        return b
                    else:
                        return False
                \end{minted}
            
\item 
                \begin{minted}[autogobble]{python}
                def h(c, a, b):
                    if not a:
                        return False
                    else:
                        return c and b
                \end{minted}
            
\item 
            \begin{minted}[autogobble]{python}
            def h(c, a, b):
                if not b:
                    return c and a
                else:
                    return False
            \end{minted}
        
\item 
                \begin{minted}[autogobble]{python}
                def h(c, a, b):
                    if c:
                        return a and b
                    else:
                        return False
                \end{minted}
            
\end{enumerate}

        \end{multicols}
    
        \naloga*
        
        Kateri izmed spodnjih programov ima drugačen izpis kot ostali?
    
        \begin{multicols}{2}
        \begin{enumerate}[(a)]
\item 
            \begin{minted}[autogobble]{python}
                for j in range(1, 10):
                    if j % 3 != 0:
                        continue
                    print(j)
            \end{minted}
        
\item 
                \begin{minted}[autogobble]{python}
                    for j in range(1, 30, 3):
                        if j < 10:
                            print(j)
                \end{minted}
            
\item 
                \begin{minted}[autogobble]{python}
                    for j in range(1, 30, 3):
                        if j > 10:
                            break
                        print(j)
                \end{minted}
            
\item 
                \begin{minted}[autogobble]{python}
                    for j in range(1, 10):
                        if j % 3 == 1:
                            print(j)
                        continue
                \end{minted}
            
\end{enumerate}

        \end{multicols}
    
        \naloga*
        \begin{multicols}{2}
        \noindent
        Napišite primer vrednosti spremenljivk \inlinepy{niz1} in \inlinepy{niz2}, za kateri klic \inlinepy{f(niz1, niz2)} vrne \inlinepy{True}.
        \begin{minted}[baselinestretch=1.2, escapeinside=||]{python}
        niz1 = |\answerbox{3}|
        niz2 = |\answerbox{3}|
        \end{minted}
        \vfil
        \columnbreak
        \begin{minted}[autogobble]{python}
        def f(niz1, niz2):
            if len(niz1) != len(niz2) or len(niz1) < 3:
                return False
            for i in range(len(niz1)):
                if i % 2 == 1 and niz1[i] != niz2[i]:
                    return False
            return True
        \end{minted}
        \end{multicols}
    
        \naloga*
        \begin{multicols}{2}
        \noindent 
        S številkami od $0$ do $4$ označite vrstni red, v katerem moramo izvesti ukaze na desni, da bo na koncu spremenljivka \inlinepy{lst1} kazala na seznam \inlinepy{[3, 4, 5]}?
    
        \columnbreak
        \noindent
        \begin{minted}[baselinestretch=1.2, escapeinside=||]{python}
|\answerbox{0.5}| lst2 = [3]
|\answerbox{0.5}| lst1 = lst2
|\answerbox{0.5}| lst1 = lst2 + lst1
|\answerbox{0.5}| lst1.append(5)
|\answerbox{0.5}| lst2 = [4]

        \end{minted}
        \end{multicols}
    
            \izpit[ucilnica=205, naloge=-1]{Uvod v programiranje: Kolokvij \#024}{27.\ marec 2019}{
                Pri vsaki nalogi obkrožite črko pred pravilnim odgovorom ali vpišite pravilno vrednost v ustrezen prostor. \\
                Čas reševanja je 30 minut. Veliko uspeha!
            }
            
        \naloga*
        
        Kateri izmed programov pri začetnem stanju
            \inlinepy{k = 5} in
            \inlinepy{m = 8}
        nastavi vrednosti
            \inlinepy{k = 8},
            \inlinepy{m = 5} in
            \inlinepy{n = 13}?
    
        \begin{multicols}{4}
        \begin{enumerate}[(a)]
\item 
                \begin{minted}[autogobble]{python}
                k = m
                m = n
                n = k
                n = k + m
                \end{minted}
            
\item 
                \begin{minted}[autogobble]{python}
                n = k
                m = n
                k = m
                n = k + m
                \end{minted}
            
\item 
                \begin{minted}[autogobble]{python}
                k = m
                n = m
                m = k
                n = k + m
                \end{minted}
            
\item 
            \begin{minted}[autogobble]{python}
            n = k
            k = m
            m = n
            n = k + m
            \end{minted}
        
\end{enumerate}

        \end{multicols}
    
        \naloga*
        \begin{multicols}{2}
        \noindent
        Kakšne vrste napak vsebuje program na desni?

        \begin{enumerate}[(a)]
\item napake, zaradi katerih Python prekine z izvajanjem programa
\item sintaktične napake, zaradi katerih Python programa noče izvesti.
\item oblikovne napake, ki ne vplivajo na pravilnost rezultata
\item vsebinske napake, zaradi katerih Python izračuna napačen rezultat
\end{enumerate}

        \columnbreak

        \begin{minted}[baselinestretch=1.2,escapeinside=||, autogobble]{python}
        
            def fibonacci(n):
                if n<=0:
                    return (0)
                elif n==1:
                    return (1)
                else:
                    a = fibonacci(n - 1)
                    b = fibonacci(n - 2)
                    return   a + b
            
        \end{minted}

        \end{multicols}

    
        \naloga*
        Katere vrstice izpiše klic \inlinepy{print(f(g(5)))}, če sta funkciji \inlinepy{f} in \inlinepy{g} definirani kot spodaj?

        \begin{multicols}{2}
        \begin{minted}[autogobble]{python}
        
            def f(a):
                print(a)
                return a + 4

            def g(b):
                print(b)
                return 1 * b
        
        \end{minted}

        \begin{enumerate}[(a)]
\item \inlinepy{[5, 5, 9]}
\item \inlinepy{[9]}
\item \inlinepy{[5, 9]}
\item \inlinepy{[5, 9]}
\end{enumerate}

        \end{multicols}
    
        \naloga*

        \begin{multicols}{2}
        \noindent
        Kateri pogoj preverja spodnja funkcija?
        \begin{minted}[autogobble]{python}
        
            def f(stavek):
                for z in stavek:
                    if z not in 'aeiouAEIOU':
                        return False
                return True
            
        \end{minted}

        \begin{enumerate}[(a)]
\item ali niz ne vsebuje nobenega samoglasnika
\item ali niz vsebuje kakšen samoglasnik
\item ali niz vsebuje samo samoglasnike
\item ali niz vsebuje znak, ki ni samoglasnik
\end{enumerate}

        \end{multicols}
    
        \naloga*
        \begin{multicols}{2}
        \noindent
        V vsak prostor vpišite \textbf{natanko en znak} tako, da bo dobljeni program v spremenljivko \inlinepy{zm} shranil zmnožek števil \inlinepy{b} in \inlinepy{a}:
        
        \columnbreak
        \begin{minted}[baselinestretch=1.2,escapeinside=||]{python}
        zm = |\answerbox{0.5}|
        while b > |\answerbox{0.5}|:
            zm += |\answerbox{0.5}|
            |\answerbox{0.5}| -= 1
        \end{minted}
        \end{multicols}
    
        \clearpage
        \naloga
        
        Katera izmed spodnjih funkcij izračuna ostanek pri deljenju naravnega števila \inlinepy{u} z naravnim številom \inlinepy{v}?
    
        \begin{multicols}{2}
        \begin{enumerate}[(a)]
\item 
                \begin{minted}[autogobble]{python}
                def ostanek(u, v):
                    if v == 0:
                        return u
                    else:
                        return ostanek(v, u % v)
                \end{minted}
            
\item 
                \begin{minted}[autogobble]{python}
                def ostanek(u, v):
                    if u < v:
                        return 0
                    else:
                        return ostanek(u - v, v)
                \end{minted}
            
\item 
                \begin{minted}[autogobble]{python}
                def ostanek(u, v):
                    if u < v:
                        return u
                    else:
                        return ostanek(u % v)
                \end{minted}
            
\item 
            \begin{minted}[autogobble]{python}
            def ostanek(u, v):
                if u < v:
                    return u
                else:
                    return ostanek(u - v, v)
            \end{minted}
        
\end{enumerate}

        \end{multicols}
    
        \naloga*
        
        Katera izmed funkcij vrača drugačne rezultate kot ostale?
    
        \begin{multicols}{2}
        \begin{enumerate}[(a)]
\item 
                \begin{minted}[autogobble]{python}
                def g(a, b, c):
                    if a and b:
                        return c
                    else:
                        return False
                \end{minted}
            
\item 
            \begin{minted}[autogobble]{python}
            def g(a, b, c):
                if not c:
                    return a and b
                else:
                    return False
            \end{minted}
        
\item 
                \begin{minted}[autogobble]{python}
                def g(a, b, c):
                    if a:
                        return b and c
                    else:
                        return False
                \end{minted}
            
\item 
                \begin{minted}[autogobble]{python}
                def g(a, b, c):
                    if not b:
                        return False
                    else:
                        return a and c
                \end{minted}
            
\end{enumerate}

        \end{multicols}
    
        \naloga*
        
        Kateri izmed spodnjih programov ima drugačen izpis kot ostali?
    
        \begin{multicols}{2}
        \begin{enumerate}[(a)]
\item 
                \begin{minted}[autogobble]{python}
                    for n in range(1, 30, 3):
                        if n < 10:
                            print(n)
                \end{minted}
            
\item 
                \begin{minted}[autogobble]{python}
                    for n in range(1, 30, 3):
                        if n > 10:
                            break
                        print(n)
                \end{minted}
            
\item 
                \begin{minted}[autogobble]{python}
                    for n in range(1, 10):
                        if n % 3 == 1:
                            print(n)
                        continue
                \end{minted}
            
\item 
            \begin{minted}[autogobble]{python}
                for n in range(1, 10):
                    if n % 3 != 0:
                        continue
                    print(n)
            \end{minted}
        
\end{enumerate}

        \end{multicols}
    
        \naloga*
        \begin{multicols}{2}
        \noindent
        Napišite primer vrednosti spremenljivk \inlinepy{sez1} in \inlinepy{sez2}, za kateri klic \inlinepy{h(sez1, sez2)} vrne \inlinepy{True}.
        \begin{minted}[baselinestretch=1.2, escapeinside=||]{python}
        sez1 = |\answerbox{3}|
        sez2 = |\answerbox{3}|
        \end{minted}
        \vfil
        \columnbreak
        \begin{minted}[autogobble]{python}
        def h(sez1, sez2):
            if len(sez1) != len(sez2) or len(sez1) < 3:
                return False
            for i in range(len(sez1)):
                if i % 2 == 0 and sez1[i] != sez2[i]:
                    return False
            return True
        \end{minted}
        \end{multicols}
    
        \naloga*
        \begin{multicols}{2}
        \noindent 
        S številkami od $0$ do $4$ označite vrstni red, v katerem moramo izvesti ukaze na desni, da bo na koncu spremenljivka \inlinepy{lst1} kazala na seznam \inlinepy{[6, 3, 8]}?
    
        \columnbreak
        \noindent
        \begin{minted}[baselinestretch=1.2, escapeinside=||]{python}
|\answerbox{0.5}| lst1 = lst2 + lst1
|\answerbox{0.5}| lst2 = [3]
|\answerbox{0.5}| lst1 = lst2
|\answerbox{0.5}| lst1.append(8)
|\answerbox{0.5}| lst2 = [6]

        \end{minted}
        \end{multicols}
    
            \izpit[ucilnica=205, naloge=-1]{Uvod v programiranje: Kolokvij \#025}{27.\ marec 2019}{
                Pri vsaki nalogi obkrožite črko pred pravilnim odgovorom ali vpišite pravilno vrednost v ustrezen prostor. \\
                Čas reševanja je 30 minut. Veliko uspeha!
            }
            
        \naloga*
        
        Kateri izmed programov pri začetnem stanju
            \inlinepy{u = 7} in
            \inlinepy{v = 8}
        nastavi vrednosti
            \inlinepy{u = 8},
            \inlinepy{v = 7} in
            \inlinepy{w = 15}?
    
        \begin{multicols}{4}
        \begin{enumerate}[(a)]
\item 
                \begin{minted}[autogobble]{python}
                w = u
                v = w
                u = v
                w = u + v
                \end{minted}
            
\item 
            \begin{minted}[autogobble]{python}
            w = u
            u = v
            v = w
            w = u + v
            \end{minted}
        
\item 
                \begin{minted}[autogobble]{python}
                u = v
                v = w
                w = u
                w = u + v
                \end{minted}
            
\item 
                \begin{minted}[autogobble]{python}
                u = v
                w = v
                v = u
                w = u + v
                \end{minted}
            
\end{enumerate}

        \end{multicols}
    
        \naloga*
        \begin{multicols}{2}
        \noindent
        Kakšne vrste napak vsebuje program na desni?

        \begin{enumerate}[(a)]
\item napake, zaradi katerih Python prekine z izvajanjem programa
\item sintaktične napake, zaradi katerih Python programa noče izvesti.
\item vsebinske napake, zaradi katerih Python izračuna napačen rezultat
\item oblikovne napake, ki ne vplivajo na pravilnost rezultata
\end{enumerate}

        \columnbreak

        \begin{minted}[baselinestretch=1.2,escapeinside=||, autogobble]{python}
        
            def fibonacci(n):
                if n <= 0:
                    return 0
                elif n == 1:
                    return 0
                else:
                    a = fibonacci(n - 1)
                    b = fibonacci(n - 3)
                    return a * b
            
        \end{minted}

        \end{multicols}

    
        \naloga*
        Katere vrstice izpiše klic \inlinepy{print(f(g(9)))}, če sta funkciji \inlinepy{f} in \inlinepy{g} definirani kot spodaj?

        \begin{multicols}{2}
        \begin{minted}[autogobble]{python}
        
            def f(a):
                return a + 5
                print(a)

            def g(b):
                print(b)
                return 4 * b
        
        \end{minted}

        \begin{enumerate}[(a)]
\item \inlinepy{[9, 41]}
\item \inlinepy{[41]}
\item \inlinepy{[36, 41]}
\item \inlinepy{[9, 36, 41]}
\end{enumerate}

        \end{multicols}
    
        \naloga*

        \begin{multicols}{2}
        \noindent
        Kateri pogoj preverja spodnja funkcija?
        \begin{minted}[autogobble]{python}
        
            def f(besedilo):
                for znak in besedilo:
                    if znak in 'aeiouAEIOU':
                        return True
                return False
            
        \end{minted}

        \begin{enumerate}[(a)]
\item ali niz vsebuje znak, ki ni samoglasnik
\item ali niz vsebuje kakšen samoglasnik
\item ali niz vsebuje samo samoglasnike
\item ali niz ne vsebuje nobenega samoglasnika
\end{enumerate}

        \end{multicols}
    
        \naloga*
        \begin{multicols}{2}
        \noindent
        V vsak prostor vpišite \textbf{natanko en znak} tako, da bo dobljeni program v spremenljivko \inlinepy{vs} shranil vsoto števil \inlinepy{y} in \inlinepy{x}:
        
        \columnbreak
        \begin{minted}[baselinestretch=1.2,escapeinside=||]{python}
        vs = |\answerbox{0.5}|
        while y > |\answerbox{0.5}|:
            vs += |\answerbox{0.5}|
            |\answerbox{0.5}| -= 1
        \end{minted}
        \end{multicols}
    
        \clearpage
        \naloga
        
        Katera izmed spodnjih funkcij izračuna ostanek pri deljenju naravnega števila \inlinepy{x} z naravnim številom \inlinepy{y}?
    
        \begin{multicols}{2}
        \begin{enumerate}[(a)]
\item 
                \begin{minted}[autogobble]{python}
                def ostanek(x, y):
                    if y == 0:
                        return x
                    else:
                        return ostanek(y, x % y)
                \end{minted}
            
\item 
                \begin{minted}[autogobble]{python}
                def ostanek(x, y):
                    if x < y:
                        return x
                    else:
                        return ostanek(x % y)
                \end{minted}
            
\item 
            \begin{minted}[autogobble]{python}
            def ostanek(x, y):
                if x < y:
                    return x
                else:
                    return ostanek(x - y, y)
            \end{minted}
        
\item 
                \begin{minted}[autogobble]{python}
                def ostanek(x, y):
                    if x < y:
                        return 0
                    else:
                        return ostanek(x - y, y)
                \end{minted}
            
\end{enumerate}

        \end{multicols}
    
        \naloga*
        
        Katera izmed funkcij vrača drugačne rezultate kot ostale?
    
        \begin{multicols}{2}
        \begin{enumerate}[(a)]
\item 
            \begin{minted}[autogobble]{python}
            def g(p, r, q):
                if not q:
                    return p and r
                else:
                    return False
            \end{minted}
        
\item 
                \begin{minted}[autogobble]{python}
                def g(p, r, q):
                    if not r:
                        return False
                    else:
                        return p and q
                \end{minted}
            
\item 
                \begin{minted}[autogobble]{python}
                def g(p, r, q):
                    if p:
                        return r and q
                    else:
                        return False
                \end{minted}
            
\item 
                \begin{minted}[autogobble]{python}
                def g(p, r, q):
                    if p and r:
                        return q
                    else:
                        return False
                \end{minted}
            
\end{enumerate}

        \end{multicols}
    
        \naloga*
        
        Kateri izmed spodnjih programov ima drugačen izpis kot ostali?
    
        \begin{multicols}{2}
        \begin{enumerate}[(a)]
\item 
                \begin{minted}[autogobble]{python}
                    for x in range(1, 20):
                        if x % 2 == 1:
                            print(x)
                        continue
                \end{minted}
            
\item 
                \begin{minted}[autogobble]{python}
                    for x in range(1, 40, 2):
                        if x < 20:
                            print(x)
                \end{minted}
            
\item 
                \begin{minted}[autogobble]{python}
                    for x in range(1, 40, 2):
                        if x > 20:
                            break
                        print(x)
                \end{minted}
            
\item 
            \begin{minted}[autogobble]{python}
                for x in range(1, 20):
                    if x % 2 != 0:
                        continue
                    print(x)
            \end{minted}
        
\end{enumerate}

        \end{multicols}
    
        \naloga*
        \begin{multicols}{2}
        \noindent
        Napišite primer vrednosti spremenljivk \inlinepy{str1} in \inlinepy{str2}, za kateri klic \inlinepy{h(str1, str2)} vrne \inlinepy{True}.
        \begin{minted}[baselinestretch=1.2, escapeinside=||]{python}
        str1 = |\answerbox{3}|
        str2 = |\answerbox{3}|
        \end{minted}
        \vfil
        \columnbreak
        \begin{minted}[autogobble]{python}
        def h(str1, str2):
            if len(str1) != len(str2) or len(str1) < 3:
                return False
            for j in range(len(str1)):
                if j % 2 == 0 and str1[j] != str2[j]:
                    return False
            return True
        \end{minted}
        \end{multicols}
    
        \naloga*
        \begin{multicols}{2}
        \noindent 
        S številkami od $0$ do $4$ označite vrstni red, v katerem moramo izvesti ukaze na desni, da bo na koncu spremenljivka \inlinepy{sez1} kazala na seznam \inlinepy{[8, 7, 4]}?
    
        \columnbreak
        \noindent
        \begin{minted}[baselinestretch=1.2, escapeinside=||]{python}
|\answerbox{0.5}| sez2 = [7]
|\answerbox{0.5}| sez1 = sez2
|\answerbox{0.5}| sez2 = [8]
|\answerbox{0.5}| sez1.append(4)
|\answerbox{0.5}| sez1 = sez2 + sez1

        \end{minted}
        \end{multicols}
    
            \izpit[ucilnica=205, naloge=-1]{Uvod v programiranje: Kolokvij \#026}{27.\ marec 2019}{
                Pri vsaki nalogi obkrožite črko pred pravilnim odgovorom ali vpišite pravilno vrednost v ustrezen prostor. \\
                Čas reševanja je 30 minut. Veliko uspeha!
            }
            
        \naloga*
        
        Kateri izmed programov pri začetnem stanju
            \inlinepy{a = 2} in
            \inlinepy{b = 7}
        nastavi vrednosti
            \inlinepy{a = 7},
            \inlinepy{b = 2} in
            \inlinepy{c = 9}?
    
        \begin{multicols}{4}
        \begin{enumerate}[(a)]
\item 
                \begin{minted}[autogobble]{python}
                a = b
                b = c
                c = a
                c = a + b
                \end{minted}
            
\item 
                \begin{minted}[autogobble]{python}
                c = a
                b = c
                a = b
                c = a + b
                \end{minted}
            
\item 
                \begin{minted}[autogobble]{python}
                a = b
                c = b
                b = a
                c = a + b
                \end{minted}
            
\item 
            \begin{minted}[autogobble]{python}
            c = a
            a = b
            b = c
            c = a + b
            \end{minted}
        
\end{enumerate}

        \end{multicols}
    
        \naloga*
        \begin{multicols}{2}
        \noindent
        Kakšne vrste napak vsebuje program na desni?

        \begin{enumerate}[(a)]
\item napake, zaradi katerih Python prekine z izvajanjem programa
\item oblikovne napake, ki ne vplivajo na pravilnost rezultata
\item vsebinske napake, zaradi katerih Python izračuna napačen rezultat
\item sintaktične napake, zaradi katerih Python programa noče izvesti.
\end{enumerate}

        \columnbreak

        \begin{minted}[baselinestretch=1.2,escapeinside=||, autogobble]{python}
        
            def fibonacci(n):
                if n <= 0:
                    return 0
                elif n == 1:
                    return 1
                else:
                    a = fibonacci(n - '1')
                    b = fib(n - 2)
                    return a + b
            
        \end{minted}

        \end{multicols}

    
        \naloga*
        Katere vrstice izpiše klic \inlinepy{print(p(q(7)))}, če sta funkciji \inlinepy{p} in \inlinepy{q} definirani kot spodaj?

        \begin{multicols}{2}
        \begin{minted}[autogobble]{python}
        
            def p(a):
                print(a)
                return a + 6

            def q(b):
                print(b)
                return 9 * b
        
        \end{minted}

        \begin{enumerate}[(a)]
\item \inlinepy{[7, 63, 69]}
\item \inlinepy{[63, 69]}
\item \inlinepy{[69]}
\item \inlinepy{[7, 69]}
\end{enumerate}

        \end{multicols}
    
        \naloga*

        \begin{multicols}{2}
        \noindent
        Kateri pogoj preverja spodnja funkcija?
        \begin{minted}[autogobble]{python}
        
            def f(niz):
                for z in niz:
                    if z not in 'aeiouAEIOU':
                        return True
                return False
            
        \end{minted}

        \begin{enumerate}[(a)]
\item ali niz vsebuje samo samoglasnike
\item ali niz ne vsebuje nobenega samoglasnika
\item ali niz vsebuje kakšen samoglasnik
\item ali niz vsebuje znak, ki ni samoglasnik
\end{enumerate}

        \end{multicols}
    
        \naloga*
        \begin{multicols}{2}
        \noindent
        V vsak prostor vpišite \textbf{natanko en znak} tako, da bo dobljeni program v spremenljivko \inlinepy{zm} shranil zmnožek števil \inlinepy{q} in \inlinepy{p}:
        
        \columnbreak
        \begin{minted}[baselinestretch=1.2,escapeinside=||]{python}
        zm = |\answerbox{0.5}|
        while q > |\answerbox{0.5}|:
            zm += |\answerbox{0.5}|
            |\answerbox{0.5}| -= 1
        \end{minted}
        \end{multicols}
    
        \clearpage
        \naloga
        
        Katera izmed spodnjih funkcij izračuna ostanek pri deljenju naravnega števila \inlinepy{x} z naravnim številom \inlinepy{y}?
    
        \begin{multicols}{2}
        \begin{enumerate}[(a)]
\item 
                \begin{minted}[autogobble]{python}
                def ostanek(x, y):
                    if x < y:
                        return x
                    else:
                        return ostanek(x % y)
                \end{minted}
            
\item 
            \begin{minted}[autogobble]{python}
            def ostanek(x, y):
                if x < y:
                    return x
                else:
                    return ostanek(x - y, y)
            \end{minted}
        
\item 
                \begin{minted}[autogobble]{python}
                def ostanek(x, y):
                    if y == 0:
                        return x
                    else:
                        return ostanek(y, x % y)
                \end{minted}
            
\item 
                \begin{minted}[autogobble]{python}
                def ostanek(x, y):
                    if x < y:
                        return 0
                    else:
                        return ostanek(x - y, y)
                \end{minted}
            
\end{enumerate}

        \end{multicols}
    
        \naloga*
        
        Katera izmed funkcij vrača drugačne rezultate kot ostale?
    
        \begin{multicols}{2}
        \begin{enumerate}[(a)]
\item 
                \begin{minted}[autogobble]{python}
                def g(b, c, a):
                    if not c:
                        return False
                    else:
                        return b and a
                \end{minted}
            
\item 
            \begin{minted}[autogobble]{python}
            def g(b, c, a):
                if not a:
                    return b and c
                else:
                    return False
            \end{minted}
        
\item 
                \begin{minted}[autogobble]{python}
                def g(b, c, a):
                    if b:
                        return c and a
                    else:
                        return False
                \end{minted}
            
\item 
                \begin{minted}[autogobble]{python}
                def g(b, c, a):
                    if b and c:
                        return a
                    else:
                        return False
                \end{minted}
            
\end{enumerate}

        \end{multicols}
    
        \naloga*
        
        Kateri izmed spodnjih programov ima drugačen izpis kot ostali?
    
        \begin{multicols}{2}
        \begin{enumerate}[(a)]
\item 
                \begin{minted}[autogobble]{python}
                    for i in range(1, 10):
                        if i % 3 == 1:
                            print(i)
                        continue
                \end{minted}
            
\item 
            \begin{minted}[autogobble]{python}
                for i in range(1, 10):
                    if i % 3 != 0:
                        continue
                    print(i)
            \end{minted}
        
\item 
                \begin{minted}[autogobble]{python}
                    for i in range(1, 30, 3):
                        if i > 10:
                            break
                        print(i)
                \end{minted}
            
\item 
                \begin{minted}[autogobble]{python}
                    for i in range(1, 30, 3):
                        if i < 10:
                            print(i)
                \end{minted}
            
\end{enumerate}

        \end{multicols}
    
        \naloga*
        \begin{multicols}{2}
        \noindent
        Napišite primer vrednosti spremenljivk \inlinepy{niz1} in \inlinepy{niz2}, za kateri klic \inlinepy{g(niz1, niz2)} vrne \inlinepy{True}.
        \begin{minted}[baselinestretch=1.2, escapeinside=||]{python}
        niz1 = |\answerbox{3}|
        niz2 = |\answerbox{3}|
        \end{minted}
        \vfil
        \columnbreak
        \begin{minted}[autogobble]{python}
        def g(niz1, niz2):
            if len(niz1) != len(niz2) or len(niz1) < 3:
                return False
            for i in range(len(niz1)):
                if i % 2 == 0 and niz1[i] != niz2[i]:
                    return False
            return True
        \end{minted}
        \end{multicols}
    
        \naloga*
        \begin{multicols}{2}
        \noindent 
        S številkami od $0$ do $4$ označite vrstni red, v katerem moramo izvesti ukaze na desni, da bo na koncu spremenljivka \inlinepy{lst1} kazala na seznam \inlinepy{[4, 6, 2]}?
    
        \columnbreak
        \noindent
        \begin{minted}[baselinestretch=1.2, escapeinside=||]{python}
|\answerbox{0.5}| lst2 = [4]
|\answerbox{0.5}| lst2 = [6]
|\answerbox{0.5}| lst1 = lst2 + lst1
|\answerbox{0.5}| lst1 = lst2
|\answerbox{0.5}| lst1.append(2)

        \end{minted}
        \end{multicols}
    
            \izpit[ucilnica=205, naloge=-1]{Uvod v programiranje: Kolokvij \#027}{27.\ marec 2019}{
                Pri vsaki nalogi obkrožite črko pred pravilnim odgovorom ali vpišite pravilno vrednost v ustrezen prostor. \\
                Čas reševanja je 30 minut. Veliko uspeha!
            }
            
        \naloga*
        
        Kateri izmed programov pri začetnem stanju
            \inlinepy{k = 8} in
            \inlinepy{m = 7}
        nastavi vrednosti
            \inlinepy{k = 7},
            \inlinepy{m = 8} in
            \inlinepy{n = 15}?
    
        \begin{multicols}{4}
        \begin{enumerate}[(a)]
\item 
                \begin{minted}[autogobble]{python}
                k = m
                m = n
                n = k
                n = k + m
                \end{minted}
            
\item 
                \begin{minted}[autogobble]{python}
                k = m
                n = m
                m = k
                n = k + m
                \end{minted}
            
\item 
                \begin{minted}[autogobble]{python}
                n = k
                m = n
                k = m
                n = k + m
                \end{minted}
            
\item 
            \begin{minted}[autogobble]{python}
            n = k
            k = m
            m = n
            n = k + m
            \end{minted}
        
\end{enumerate}

        \end{multicols}
    
        \naloga*
        \begin{multicols}{2}
        \noindent
        Kakšne vrste napak vsebuje program na desni?

        \begin{enumerate}[(a)]
\item sintaktične napake, zaradi katerih Python programa noče izvesti.
\item vsebinske napake, zaradi katerih Python izračuna napačen rezultat
\item napake, zaradi katerih Python prekine z izvajanjem programa
\item oblikovne napake, ki ne vplivajo na pravilnost rezultata
\end{enumerate}

        \columnbreak

        \begin{minted}[baselinestretch=1.2,escapeinside=||, autogobble]{python}
        
            def fibonacci(n):
                if n <= 0:
                    return 0
                elif n == 1:
                    return 1
                else:
                    a = fibonacci(n - '1')
                    b = fib(n - 2)
                    return a + b
            
        \end{minted}

        \end{multicols}

    
        \naloga*
        Katere vrstice izpiše klic \inlinepy{print(f(g(1)))}, če sta funkciji \inlinepy{f} in \inlinepy{g} definirani kot spodaj?

        \begin{multicols}{2}
        \begin{minted}[autogobble]{python}
        
            def f(a):
                print(a)
                return a + 9

            def g(b):
                print(b)
                return 2 * b
        
        \end{minted}

        \begin{enumerate}[(a)]
\item \inlinepy{[1, 2, 11]}
\item \inlinepy{[11]}
\item \inlinepy{[1, 11]}
\item \inlinepy{[2, 11]}
\end{enumerate}

        \end{multicols}
    
        \naloga*

        \begin{multicols}{2}
        \noindent
        Kateri pogoj preverja spodnja funkcija?
        \begin{minted}[autogobble]{python}
        
            def f(niz):
                for x in niz:
                    if x in 'aeiouAEIOU':
                        return True
                return False
            
        \end{minted}

        \begin{enumerate}[(a)]
\item ali niz vsebuje kakšen samoglasnik
\item ali niz vsebuje samo samoglasnike
\item ali niz vsebuje znak, ki ni samoglasnik
\item ali niz ne vsebuje nobenega samoglasnika
\end{enumerate}

        \end{multicols}
    
        \naloga*
        \begin{multicols}{2}
        \noindent
        V vsak prostor vpišite \textbf{natanko en znak} tako, da bo dobljeni program v spremenljivko \inlinepy{zm} shranil zmnožek števil \inlinepy{x} in \inlinepy{y}:
        
        \columnbreak
        \begin{minted}[baselinestretch=1.2,escapeinside=||]{python}
        zm = |\answerbox{0.5}|
        while x > |\answerbox{0.5}|:
            zm += |\answerbox{0.5}|
            |\answerbox{0.5}| -= 1
        \end{minted}
        \end{multicols}
    
        \clearpage
        \naloga
        
        Katera izmed spodnjih funkcij izračuna ostanek pri deljenju naravnega števila \inlinepy{a} z naravnim številom \inlinepy{b}?
    
        \begin{multicols}{2}
        \begin{enumerate}[(a)]
\item 
                \begin{minted}[autogobble]{python}
                def ostanek(a, b):
                    if a < b:
                        return 0
                    else:
                        return ostanek(a - b, b)
                \end{minted}
            
\item 
            \begin{minted}[autogobble]{python}
            def ostanek(a, b):
                if a < b:
                    return a
                else:
                    return ostanek(a - b, b)
            \end{minted}
        
\item 
                \begin{minted}[autogobble]{python}
                def ostanek(a, b):
                    if a < b:
                        return a
                    else:
                        return ostanek(a % b)
                \end{minted}
            
\item 
                \begin{minted}[autogobble]{python}
                def ostanek(a, b):
                    if b == 0:
                        return a
                    else:
                        return ostanek(b, a % b)
                \end{minted}
            
\end{enumerate}

        \end{multicols}
    
        \naloga*
        
        Katera izmed funkcij vrača drugačne rezultate kot ostale?
    
        \begin{multicols}{2}
        \begin{enumerate}[(a)]
\item 
                \begin{minted}[autogobble]{python}
                def h(z, y, x):
                    if not y:
                        return False
                    else:
                        return z and x
                \end{minted}
            
\item 
                \begin{minted}[autogobble]{python}
                def h(z, y, x):
                    if z and y:
                        return x
                    else:
                        return False
                \end{minted}
            
\item 
                \begin{minted}[autogobble]{python}
                def h(z, y, x):
                    if z:
                        return y and x
                    else:
                        return False
                \end{minted}
            
\item 
            \begin{minted}[autogobble]{python}
            def h(z, y, x):
                if not x:
                    return z and y
                else:
                    return False
            \end{minted}
        
\end{enumerate}

        \end{multicols}
    
        \naloga*
        
        Kateri izmed spodnjih programov ima drugačen izpis kot ostali?
    
        \begin{multicols}{2}
        \begin{enumerate}[(a)]
\item 
            \begin{minted}[autogobble]{python}
                for i in range(1, 20):
                    if i % 3 != 0:
                        continue
                    print(i)
            \end{minted}
        
\item 
                \begin{minted}[autogobble]{python}
                    for i in range(1, 60, 3):
                        if i > 20:
                            break
                        print(i)
                \end{minted}
            
\item 
                \begin{minted}[autogobble]{python}
                    for i in range(1, 60, 3):
                        if i < 20:
                            print(i)
                \end{minted}
            
\item 
                \begin{minted}[autogobble]{python}
                    for i in range(1, 20):
                        if i % 3 == 1:
                            print(i)
                        continue
                \end{minted}
            
\end{enumerate}

        \end{multicols}
    
        \naloga*
        \begin{multicols}{2}
        \noindent
        Napišite primer vrednosti spremenljivk \inlinepy{sez1} in \inlinepy{sez2}, za kateri klic \inlinepy{h(sez1, sez2)} vrne \inlinepy{True}.
        \begin{minted}[baselinestretch=1.2, escapeinside=||]{python}
        sez1 = |\answerbox{3}|
        sez2 = |\answerbox{3}|
        \end{minted}
        \vfil
        \columnbreak
        \begin{minted}[autogobble]{python}
        def h(sez1, sez2):
            if len(sez1) != len(sez2) or len(sez1) < 3:
                return False
            for j in range(len(sez1)):
                if j % 2 == 0 and sez1[j] != sez2[j]:
                    return False
            return True
        \end{minted}
        \end{multicols}
    
        \naloga*
        \begin{multicols}{2}
        \noindent 
        S številkami od $0$ do $4$ označite vrstni red, v katerem moramo izvesti ukaze na desni, da bo na koncu spremenljivka \inlinepy{lst1} kazala na seznam \inlinepy{[7, 2, 6]}?
    
        \columnbreak
        \noindent
        \begin{minted}[baselinestretch=1.2, escapeinside=||]{python}
|\answerbox{0.5}| lst2 = [2]
|\answerbox{0.5}| lst1 = lst2
|\answerbox{0.5}| lst1 = lst2 + lst1
|\answerbox{0.5}| lst2 = [7]
|\answerbox{0.5}| lst1.append(6)

        \end{minted}
        \end{multicols}
    
            \izpit[ucilnica=205, naloge=-1]{Uvod v programiranje: Kolokvij \#028}{27.\ marec 2019}{
                Pri vsaki nalogi obkrožite črko pred pravilnim odgovorom ali vpišite pravilno vrednost v ustrezen prostor. \\
                Čas reševanja je 30 minut. Veliko uspeha!
            }
            
        \naloga*
        
        Kateri izmed programov pri začetnem stanju
            \inlinepy{k = 8} in
            \inlinepy{m = 9}
        nastavi vrednosti
            \inlinepy{k = 9},
            \inlinepy{m = 8} in
            \inlinepy{n = 17}?
    
        \begin{multicols}{4}
        \begin{enumerate}[(a)]
\item 
                \begin{minted}[autogobble]{python}
                n = k
                m = n
                k = m
                n = k + m
                \end{minted}
            
\item 
            \begin{minted}[autogobble]{python}
            n = k
            k = m
            m = n
            n = k + m
            \end{minted}
        
\item 
                \begin{minted}[autogobble]{python}
                k = m
                n = m
                m = k
                n = k + m
                \end{minted}
            
\item 
                \begin{minted}[autogobble]{python}
                k = m
                m = n
                n = k
                n = k + m
                \end{minted}
            
\end{enumerate}

        \end{multicols}
    
        \naloga*
        \begin{multicols}{2}
        \noindent
        Kakšne vrste napak vsebuje program na desni?

        \begin{enumerate}[(a)]
\item sintaktične napake, zaradi katerih Python programa noče izvesti.
\item oblikovne napake, ki ne vplivajo na pravilnost rezultata
\item vsebinske napake, zaradi katerih Python izračuna napačen rezultat
\item napake, zaradi katerih Python prekine z izvajanjem programa
\end{enumerate}

        \columnbreak

        \begin{minted}[baselinestretch=1.2,escapeinside=||, autogobble]{python}
        
            def fibonacci(n):
                if n <= 0:
                    return 0
                elif n == 1:
                    return 1
                else:
                    a = fibonacci(n - '1')
                    b = fib(n - 2)
                    return a + b
            
        \end{minted}

        \end{multicols}

    
        \naloga*
        Katere vrstice izpiše klic \inlinepy{print(f(g(1)))}, če sta funkciji \inlinepy{f} in \inlinepy{g} definirani kot spodaj?

        \begin{multicols}{2}
        \begin{minted}[autogobble]{python}
        
            def f(x):
                print(x)
                return x + 7

            def g(y):
                return 9 * y
                print(y)
        
        \end{minted}

        \begin{enumerate}[(a)]
\item \inlinepy{[1, 9, 16]}
\item \inlinepy{[16]}
\item \inlinepy{[9, 16]}
\item \inlinepy{[1, 16]}
\end{enumerate}

        \end{multicols}
    
        \naloga*

        \begin{multicols}{2}
        \noindent
        Kateri pogoj preverja spodnja funkcija?
        \begin{minted}[autogobble]{python}
        
            def f(niz):
                for x in niz:
                    if x in 'aeiouAEIOU':
                        return False
                return True
            
        \end{minted}

        \begin{enumerate}[(a)]
\item ali niz vsebuje samo samoglasnike
\item ali niz vsebuje znak, ki ni samoglasnik
\item ali niz vsebuje kakšen samoglasnik
\item ali niz ne vsebuje nobenega samoglasnika
\end{enumerate}

        \end{multicols}
    
        \naloga*
        \begin{multicols}{2}
        \noindent
        V vsak prostor vpišite \textbf{natanko en znak} tako, da bo dobljeni program v spremenljivko \inlinepy{zm} shranil zmnožek števil \inlinepy{x} in \inlinepy{y}:
        
        \columnbreak
        \begin{minted}[baselinestretch=1.2,escapeinside=||]{python}
        zm = |\answerbox{0.5}|
        while x > |\answerbox{0.5}|:
            zm += |\answerbox{0.5}|
            |\answerbox{0.5}| -= 1
        \end{minted}
        \end{multicols}
    
        \clearpage
        \naloga
        
        Katera izmed spodnjih funkcij izračuna ostanek pri deljenju naravnega števila \inlinepy{u} z naravnim številom \inlinepy{v}?
    
        \begin{multicols}{2}
        \begin{enumerate}[(a)]
\item 
                \begin{minted}[autogobble]{python}
                def ostanek(u, v):
                    if u < v:
                        return u
                    else:
                        return ostanek(u % v)
                \end{minted}
            
\item 
                \begin{minted}[autogobble]{python}
                def ostanek(u, v):
                    if u < v:
                        return 0
                    else:
                        return ostanek(u - v, v)
                \end{minted}
            
\item 
            \begin{minted}[autogobble]{python}
            def ostanek(u, v):
                if u < v:
                    return u
                else:
                    return ostanek(u - v, v)
            \end{minted}
        
\item 
                \begin{minted}[autogobble]{python}
                def ostanek(u, v):
                    if v == 0:
                        return u
                    else:
                        return ostanek(v, u % v)
                \end{minted}
            
\end{enumerate}

        \end{multicols}
    
        \naloga*
        
        Katera izmed funkcij vrača drugačne rezultate kot ostale?
    
        \begin{multicols}{2}
        \begin{enumerate}[(a)]
\item 
                \begin{minted}[autogobble]{python}
                def f(q, p, r):
                    if q:
                        return p and r
                    else:
                        return False
                \end{minted}
            
\item 
                \begin{minted}[autogobble]{python}
                def f(q, p, r):
                    if q and p:
                        return r
                    else:
                        return False
                \end{minted}
            
\item 
                \begin{minted}[autogobble]{python}
                def f(q, p, r):
                    if not p:
                        return False
                    else:
                        return q and r
                \end{minted}
            
\item 
            \begin{minted}[autogobble]{python}
            def f(q, p, r):
                if not r:
                    return q and p
                else:
                    return False
            \end{minted}
        
\end{enumerate}

        \end{multicols}
    
        \naloga*
        
        Kateri izmed spodnjih programov ima drugačen izpis kot ostali?
    
        \begin{multicols}{2}
        \begin{enumerate}[(a)]
\item 
                \begin{minted}[autogobble]{python}
                    for n in range(1, 40, 2):
                        if n < 20:
                            print(n)
                \end{minted}
            
\item 
            \begin{minted}[autogobble]{python}
                for n in range(1, 20):
                    if n % 2 != 0:
                        continue
                    print(n)
            \end{minted}
        
\item 
                \begin{minted}[autogobble]{python}
                    for n in range(1, 20):
                        if n % 2 == 1:
                            print(n)
                        continue
                \end{minted}
            
\item 
                \begin{minted}[autogobble]{python}
                    for n in range(1, 40, 2):
                        if n > 20:
                            break
                        print(n)
                \end{minted}
            
\end{enumerate}

        \end{multicols}
    
        \naloga*
        \begin{multicols}{2}
        \noindent
        Napišite primer vrednosti spremenljivk \inlinepy{lst1} in \inlinepy{lst2}, za kateri klic \inlinepy{g(lst1, lst2)} vrne \inlinepy{True}.
        \begin{minted}[baselinestretch=1.2, escapeinside=||]{python}
        lst1 = |\answerbox{3}|
        lst2 = |\answerbox{3}|
        \end{minted}
        \vfil
        \columnbreak
        \begin{minted}[autogobble]{python}
        def g(lst1, lst2):
            if len(lst1) != len(lst2) or len(lst1) < 3:
                return False
            for j in range(len(lst1)):
                if j % 2 == 0 and lst1[j] != lst2[j]:
                    return False
            return True
        \end{minted}
        \end{multicols}
    
        \naloga*
        \begin{multicols}{2}
        \noindent 
        S številkami od $0$ do $4$ označite vrstni red, v katerem moramo izvesti ukaze na desni, da bo na koncu spremenljivka \inlinepy{sez2} kazala na seznam \inlinepy{[1, 3, 4]}?
    
        \columnbreak
        \noindent
        \begin{minted}[baselinestretch=1.2, escapeinside=||]{python}
|\answerbox{0.5}| sez1 = [1]
|\answerbox{0.5}| sez2 = sez1
|\answerbox{0.5}| sez2 = sez1 + sez2
|\answerbox{0.5}| sez1 = [3]
|\answerbox{0.5}| sez2.append(4)

        \end{minted}
        \end{multicols}
    
            \izpit[ucilnica=205, naloge=-1]{Uvod v programiranje: Kolokvij \#029}{27.\ marec 2019}{
                Pri vsaki nalogi obkrožite črko pred pravilnim odgovorom ali vpišite pravilno vrednost v ustrezen prostor. \\
                Čas reševanja je 30 minut. Veliko uspeha!
            }
            
        \naloga*
        
        Kateri izmed programov pri začetnem stanju
            \inlinepy{x = 8} in
            \inlinepy{y = 5}
        nastavi vrednosti
            \inlinepy{x = 5},
            \inlinepy{y = 8} in
            \inlinepy{z = 13}?
    
        \begin{multicols}{4}
        \begin{enumerate}[(a)]
\item 
            \begin{minted}[autogobble]{python}
            z = x
            x = y
            y = z
            z = x + y
            \end{minted}
        
\item 
                \begin{minted}[autogobble]{python}
                z = x
                y = z
                x = y
                z = x + y
                \end{minted}
            
\item 
                \begin{minted}[autogobble]{python}
                x = y
                z = y
                y = x
                z = x + y
                \end{minted}
            
\item 
                \begin{minted}[autogobble]{python}
                x = y
                y = z
                z = x
                z = x + y
                \end{minted}
            
\end{enumerate}

        \end{multicols}
    
        \naloga*
        \begin{multicols}{2}
        \noindent
        Kakšne vrste napak vsebuje program na desni?

        \begin{enumerate}[(a)]
\item vsebinske napake, zaradi katerih Python izračuna napačen rezultat
\item napake, zaradi katerih Python prekine z izvajanjem programa
\item oblikovne napake, ki ne vplivajo na pravilnost rezultata
\item sintaktične napake, zaradi katerih Python programa noče izvesti.
\end{enumerate}

        \columnbreak

        \begin{minted}[baselinestretch=1.2,escapeinside=||, autogobble]{python}
        
            def fibonacci(n):
                if n <= 0:
                    return 0
                elif n == 1:
                    return 0
                else:
                    a = fibonacci(n - 1)
                    b = fibonacci(n - 3)
                    return a * b
            
        \end{minted}

        \end{multicols}

    
        \naloga*
        Katere vrstice izpiše klic \inlinepy{print(p(q(1)))}, če sta funkciji \inlinepy{p} in \inlinepy{q} definirani kot spodaj?

        \begin{multicols}{2}
        \begin{minted}[autogobble]{python}
        
            def p(a):
                print(a)
                return a + 3

            def q(b):
                return 9 * b
                print(b)
        
        \end{minted}

        \begin{enumerate}[(a)]
\item \inlinepy{[1, 12]}
\item \inlinepy{[1, 9, 12]}
\item \inlinepy{[12]}
\item \inlinepy{[9, 12]}
\end{enumerate}

        \end{multicols}
    
        \naloga*

        \begin{multicols}{2}
        \noindent
        Kateri pogoj preverja spodnja funkcija?
        \begin{minted}[autogobble]{python}
        
            def f(besedilo):
                for z in besedilo:
                    if z not in 'aeiouAEIOU':
                        return True
                return False
            
        \end{minted}

        \begin{enumerate}[(a)]
\item ali niz ne vsebuje nobenega samoglasnika
\item ali niz vsebuje znak, ki ni samoglasnik
\item ali niz vsebuje samo samoglasnike
\item ali niz vsebuje kakšen samoglasnik
\end{enumerate}

        \end{multicols}
    
        \naloga*
        \begin{multicols}{2}
        \noindent
        V vsak prostor vpišite \textbf{natanko en znak} tako, da bo dobljeni program v spremenljivko \inlinepy{vs} shranil vsoto števil \inlinepy{y} in \inlinepy{x}:
        
        \columnbreak
        \begin{minted}[baselinestretch=1.2,escapeinside=||]{python}
        vs = |\answerbox{0.5}|
        while y > |\answerbox{0.5}|:
            vs += |\answerbox{0.5}|
            |\answerbox{0.5}| -= 1
        \end{minted}
        \end{multicols}
    
        \clearpage
        \naloga
        
        Katera izmed spodnjih funkcij izračuna ostanek pri deljenju naravnega števila \inlinepy{x} z naravnim številom \inlinepy{y}?
    
        \begin{multicols}{2}
        \begin{enumerate}[(a)]
\item 
                \begin{minted}[autogobble]{python}
                def ostanek(x, y):
                    if x < y:
                        return x
                    else:
                        return ostanek(x % y)
                \end{minted}
            
\item 
                \begin{minted}[autogobble]{python}
                def ostanek(x, y):
                    if x < y:
                        return 0
                    else:
                        return ostanek(x - y, y)
                \end{minted}
            
\item 
                \begin{minted}[autogobble]{python}
                def ostanek(x, y):
                    if y == 0:
                        return x
                    else:
                        return ostanek(y, x % y)
                \end{minted}
            
\item 
            \begin{minted}[autogobble]{python}
            def ostanek(x, y):
                if x < y:
                    return x
                else:
                    return ostanek(x - y, y)
            \end{minted}
        
\end{enumerate}

        \end{multicols}
    
        \naloga*
        
        Katera izmed funkcij vrača drugačne rezultate kot ostale?
    
        \begin{multicols}{2}
        \begin{enumerate}[(a)]
\item 
            \begin{minted}[autogobble]{python}
            def g(p, r, q):
                if not q:
                    return p and r
                else:
                    return False
            \end{minted}
        
\item 
                \begin{minted}[autogobble]{python}
                def g(p, r, q):
                    if not r:
                        return False
                    else:
                        return p and q
                \end{minted}
            
\item 
                \begin{minted}[autogobble]{python}
                def g(p, r, q):
                    if p:
                        return r and q
                    else:
                        return False
                \end{minted}
            
\item 
                \begin{minted}[autogobble]{python}
                def g(p, r, q):
                    if p and r:
                        return q
                    else:
                        return False
                \end{minted}
            
\end{enumerate}

        \end{multicols}
    
        \naloga*
        
        Kateri izmed spodnjih programov ima drugačen izpis kot ostali?
    
        \begin{multicols}{2}
        \begin{enumerate}[(a)]
\item 
                \begin{minted}[autogobble]{python}
                    for x in range(1, 30, 3):
                        if x > 10:
                            break
                        print(x)
                \end{minted}
            
\item 
            \begin{minted}[autogobble]{python}
                for x in range(1, 10):
                    if x % 3 != 0:
                        continue
                    print(x)
            \end{minted}
        
\item 
                \begin{minted}[autogobble]{python}
                    for x in range(1, 30, 3):
                        if x < 10:
                            print(x)
                \end{minted}
            
\item 
                \begin{minted}[autogobble]{python}
                    for x in range(1, 10):
                        if x % 3 == 1:
                            print(x)
                        continue
                \end{minted}
            
\end{enumerate}

        \end{multicols}
    
        \naloga*
        \begin{multicols}{2}
        \noindent
        Napišite primer vrednosti spremenljivk \inlinepy{sez1} in \inlinepy{sez2}, za kateri klic \inlinepy{h(sez1, sez2)} vrne \inlinepy{True}.
        \begin{minted}[baselinestretch=1.2, escapeinside=||]{python}
        sez1 = |\answerbox{3}|
        sez2 = |\answerbox{3}|
        \end{minted}
        \vfil
        \columnbreak
        \begin{minted}[autogobble]{python}
        def h(sez1, sez2):
            if len(sez1) != len(sez2) or len(sez1) < 3:
                return False
            for i in range(len(sez1)):
                if i % 2 == 0 and sez1[i] != sez2[i]:
                    return False
            return True
        \end{minted}
        \end{multicols}
    
        \naloga*
        \begin{multicols}{2}
        \noindent 
        S številkami od $0$ do $4$ označite vrstni red, v katerem moramo izvesti ukaze na desni, da bo na koncu spremenljivka \inlinepy{sez2} kazala na seznam \inlinepy{[4, 8, 2]}?
    
        \columnbreak
        \noindent
        \begin{minted}[baselinestretch=1.2, escapeinside=||]{python}
|\answerbox{0.5}| sez1 = [4]
|\answerbox{0.5}| sez2 = sez1 + sez2
|\answerbox{0.5}| sez2.append(2)
|\answerbox{0.5}| sez1 = [8]
|\answerbox{0.5}| sez2 = sez1

        \end{minted}
        \end{multicols}
    
            \izpit[ucilnica=205, naloge=-1]{Uvod v programiranje: Kolokvij \#030}{27.\ marec 2019}{
                Pri vsaki nalogi obkrožite črko pred pravilnim odgovorom ali vpišite pravilno vrednost v ustrezen prostor. \\
                Čas reševanja je 30 minut. Veliko uspeha!
            }
            
        \naloga*
        
        Kateri izmed programov pri začetnem stanju
            \inlinepy{a = 2} in
            \inlinepy{b = 1}
        nastavi vrednosti
            \inlinepy{a = 1},
            \inlinepy{b = 2} in
            \inlinepy{c = 3}?
    
        \begin{multicols}{4}
        \begin{enumerate}[(a)]
\item 
            \begin{minted}[autogobble]{python}
            c = a
            a = b
            b = c
            c = a + b
            \end{minted}
        
\item 
                \begin{minted}[autogobble]{python}
                a = b
                b = c
                c = a
                c = a + b
                \end{minted}
            
\item 
                \begin{minted}[autogobble]{python}
                a = b
                c = b
                b = a
                c = a + b
                \end{minted}
            
\item 
                \begin{minted}[autogobble]{python}
                c = a
                b = c
                a = b
                c = a + b
                \end{minted}
            
\end{enumerate}

        \end{multicols}
    
        \naloga*
        \begin{multicols}{2}
        \noindent
        Kakšne vrste napak vsebuje program na desni?

        \begin{enumerate}[(a)]
\item napake, zaradi katerih Python prekine z izvajanjem programa
\item vsebinske napake, zaradi katerih Python izračuna napačen rezultat
\item oblikovne napake, ki ne vplivajo na pravilnost rezultata
\item sintaktične napake, zaradi katerih Python programa noče izvesti.
\end{enumerate}

        \columnbreak

        \begin{minted}[baselinestretch=1.2,escapeinside=||, autogobble]{python}
        
            def fibonacci(n):
                if n<=0:
                    return (0)
                elif n==1:
                    return (1)
                else:
                    a = fibonacci(n - 1)
                    b = fibonacci(n - 2)
                    return   a + b
            
        \end{minted}

        \end{multicols}

    
        \naloga*
        Katere vrstice izpiše klic \inlinepy{print(p(q(7)))}, če sta funkciji \inlinepy{p} in \inlinepy{q} definirani kot spodaj?

        \begin{multicols}{2}
        \begin{minted}[autogobble]{python}
        
            def p(a):
                print(a)
                return a + 3

            def q(b):
                return 8 * b
                print(b)
        
        \end{minted}

        \begin{enumerate}[(a)]
\item \inlinepy{[7, 56, 59]}
\item \inlinepy{[56, 59]}
\item \inlinepy{[59]}
\item \inlinepy{[7, 59]}
\end{enumerate}

        \end{multicols}
    
        \naloga*

        \begin{multicols}{2}
        \noindent
        Kateri pogoj preverja spodnja funkcija?
        \begin{minted}[autogobble]{python}
        
            def f(besedilo):
                for z in besedilo:
                    if z in 'aeiouAEIOU':
                        return True
                return False
            
        \end{minted}

        \begin{enumerate}[(a)]
\item ali niz ne vsebuje nobenega samoglasnika
\item ali niz vsebuje znak, ki ni samoglasnik
\item ali niz vsebuje kakšen samoglasnik
\item ali niz vsebuje samo samoglasnike
\end{enumerate}

        \end{multicols}
    
        \naloga*
        \begin{multicols}{2}
        \noindent
        V vsak prostor vpišite \textbf{natanko en znak} tako, da bo dobljeni program v spremenljivko \inlinepy{zm} shranil zmnožek števil \inlinepy{a} in \inlinepy{b}:
        
        \columnbreak
        \begin{minted}[baselinestretch=1.2,escapeinside=||]{python}
        zm = |\answerbox{0.5}|
        while a > |\answerbox{0.5}|:
            zm += |\answerbox{0.5}|
            |\answerbox{0.5}| -= 1
        \end{minted}
        \end{multicols}
    
        \clearpage
        \naloga
        
        Katera izmed spodnjih funkcij izračuna ostanek pri deljenju naravnega števila \inlinepy{a} z naravnim številom \inlinepy{b}?
    
        \begin{multicols}{2}
        \begin{enumerate}[(a)]
\item 
            \begin{minted}[autogobble]{python}
            def ostanek(a, b):
                if a < b:
                    return a
                else:
                    return ostanek(a - b, b)
            \end{minted}
        
\item 
                \begin{minted}[autogobble]{python}
                def ostanek(a, b):
                    if a < b:
                        return a
                    else:
                        return ostanek(a % b)
                \end{minted}
            
\item 
                \begin{minted}[autogobble]{python}
                def ostanek(a, b):
                    if b == 0:
                        return a
                    else:
                        return ostanek(b, a % b)
                \end{minted}
            
\item 
                \begin{minted}[autogobble]{python}
                def ostanek(a, b):
                    if a < b:
                        return 0
                    else:
                        return ostanek(a - b, b)
                \end{minted}
            
\end{enumerate}

        \end{multicols}
    
        \naloga*
        
        Katera izmed funkcij vrača drugačne rezultate kot ostale?
    
        \begin{multicols}{2}
        \begin{enumerate}[(a)]
\item 
                \begin{minted}[autogobble]{python}
                def g(p, q, r):
                    if p and q:
                        return r
                    else:
                        return False
                \end{minted}
            
\item 
                \begin{minted}[autogobble]{python}
                def g(p, q, r):
                    if not q:
                        return False
                    else:
                        return p and r
                \end{minted}
            
\item 
            \begin{minted}[autogobble]{python}
            def g(p, q, r):
                if not r:
                    return p and q
                else:
                    return False
            \end{minted}
        
\item 
                \begin{minted}[autogobble]{python}
                def g(p, q, r):
                    if p:
                        return q and r
                    else:
                        return False
                \end{minted}
            
\end{enumerate}

        \end{multicols}
    
        \naloga*
        
        Kateri izmed spodnjih programov ima drugačen izpis kot ostali?
    
        \begin{multicols}{2}
        \begin{enumerate}[(a)]
\item 
                \begin{minted}[autogobble]{python}
                    for x in range(1, 60, 3):
                        if x > 20:
                            break
                        print(x)
                \end{minted}
            
\item 
                \begin{minted}[autogobble]{python}
                    for x in range(1, 60, 3):
                        if x < 20:
                            print(x)
                \end{minted}
            
\item 
            \begin{minted}[autogobble]{python}
                for x in range(1, 20):
                    if x % 3 != 0:
                        continue
                    print(x)
            \end{minted}
        
\item 
                \begin{minted}[autogobble]{python}
                    for x in range(1, 20):
                        if x % 3 == 1:
                            print(x)
                        continue
                \end{minted}
            
\end{enumerate}

        \end{multicols}
    
        \naloga*
        \begin{multicols}{2}
        \noindent
        Napišite primer vrednosti spremenljivk \inlinepy{lst1} in \inlinepy{lst2}, za kateri klic \inlinepy{g(lst1, lst2)} vrne \inlinepy{True}.
        \begin{minted}[baselinestretch=1.2, escapeinside=||]{python}
        lst1 = |\answerbox{3}|
        lst2 = |\answerbox{3}|
        \end{minted}
        \vfil
        \columnbreak
        \begin{minted}[autogobble]{python}
        def g(lst1, lst2):
            if len(lst1) != len(lst2) or len(lst1) < 3:
                return False
            for j in range(len(lst1)):
                if j % 2 == 1 and lst1[j] != lst2[j]:
                    return False
            return True
        \end{minted}
        \end{multicols}
    
        \naloga*
        \begin{multicols}{2}
        \noindent 
        S številkami od $0$ do $4$ označite vrstni red, v katerem moramo izvesti ukaze na desni, da bo na koncu spremenljivka \inlinepy{lst2} kazala na seznam \inlinepy{[1, 2, 3]}?
    
        \columnbreak
        \noindent
        \begin{minted}[baselinestretch=1.2, escapeinside=||]{python}
|\answerbox{0.5}| lst2 = lst1 + lst2
|\answerbox{0.5}| lst2 = lst1
|\answerbox{0.5}| lst1 = [2]
|\answerbox{0.5}| lst1 = [1]
|\answerbox{0.5}| lst2.append(3)

        \end{minted}
        \end{multicols}
    
            \izpit[ucilnica=205, naloge=-1]{Uvod v programiranje: Kolokvij \#031}{27.\ marec 2019}{
                Pri vsaki nalogi obkrožite črko pred pravilnim odgovorom ali vpišite pravilno vrednost v ustrezen prostor. \\
                Čas reševanja je 30 minut. Veliko uspeha!
            }
            
        \naloga*
        
        Kateri izmed programov pri začetnem stanju
            \inlinepy{a = 1} in
            \inlinepy{b = 6}
        nastavi vrednosti
            \inlinepy{a = 6},
            \inlinepy{b = 1} in
            \inlinepy{c = 7}?
    
        \begin{multicols}{4}
        \begin{enumerate}[(a)]
\item 
                \begin{minted}[autogobble]{python}
                a = b
                b = c
                c = a
                c = a + b
                \end{minted}
            
\item 
                \begin{minted}[autogobble]{python}
                a = b
                c = b
                b = a
                c = a + b
                \end{minted}
            
\item 
                \begin{minted}[autogobble]{python}
                c = a
                b = c
                a = b
                c = a + b
                \end{minted}
            
\item 
            \begin{minted}[autogobble]{python}
            c = a
            a = b
            b = c
            c = a + b
            \end{minted}
        
\end{enumerate}

        \end{multicols}
    
        \naloga*
        \begin{multicols}{2}
        \noindent
        Kakšne vrste napak vsebuje program na desni?

        \begin{enumerate}[(a)]
\item sintaktične napake, zaradi katerih Python programa noče izvesti.
\item oblikovne napake, ki ne vplivajo na pravilnost rezultata
\item vsebinske napake, zaradi katerih Python izračuna napačen rezultat
\item napake, zaradi katerih Python prekine z izvajanjem programa
\end{enumerate}

        \columnbreak

        \begin{minted}[baselinestretch=1.2,escapeinside=||, autogobble]{python}
        
            def fibonacci(n):
                if n <= 0:
                    return 0
                elif n == 1:
                    return 0
                else:
                    a = fibonacci(n - 1)
                    b = fibonacci(n - 3)
                    return a * b
            
        \end{minted}

        \end{multicols}

    
        \naloga*
        Katere vrstice izpiše klic \inlinepy{print(a(b(9)))}, če sta funkciji \inlinepy{a} in \inlinepy{b} definirani kot spodaj?

        \begin{multicols}{2}
        \begin{minted}[autogobble]{python}
        
            def a(x):
                return x + 7
                print(x)

            def b(y):
                print(y)
                return 5 * y
        
        \end{minted}

        \begin{enumerate}[(a)]
\item \inlinepy{[45, 52]}
\item \inlinepy{[9, 52]}
\item \inlinepy{[52]}
\item \inlinepy{[9, 45, 52]}
\end{enumerate}

        \end{multicols}
    
        \naloga*

        \begin{multicols}{2}
        \noindent
        Kateri pogoj preverja spodnja funkcija?
        \begin{minted}[autogobble]{python}
        
            def f(besedilo):
                for z in besedilo:
                    if z not in 'aeiouAEIOU':
                        return False
                return True
            
        \end{minted}

        \begin{enumerate}[(a)]
\item ali niz vsebuje samo samoglasnike
\item ali niz ne vsebuje nobenega samoglasnika
\item ali niz vsebuje znak, ki ni samoglasnik
\item ali niz vsebuje kakšen samoglasnik
\end{enumerate}

        \end{multicols}
    
        \naloga*
        \begin{multicols}{2}
        \noindent
        V vsak prostor vpišite \textbf{natanko en znak} tako, da bo dobljeni program v spremenljivko \inlinepy{vs} shranil vsoto števil \inlinepy{b} in \inlinepy{a}:
        
        \columnbreak
        \begin{minted}[baselinestretch=1.2,escapeinside=||]{python}
        vs = |\answerbox{0.5}|
        while b > |\answerbox{0.5}|:
            vs += |\answerbox{0.5}|
            |\answerbox{0.5}| -= 1
        \end{minted}
        \end{multicols}
    
        \clearpage
        \naloga
        
        Katera izmed spodnjih funkcij izračuna ostanek pri deljenju naravnega števila \inlinepy{m} z naravnim številom \inlinepy{n}?
    
        \begin{multicols}{2}
        \begin{enumerate}[(a)]
\item 
                \begin{minted}[autogobble]{python}
                def ostanek(m, n):
                    if m < n:
                        return m
                    else:
                        return ostanek(m % n)
                \end{minted}
            
\item 
                \begin{minted}[autogobble]{python}
                def ostanek(m, n):
                    if m < n:
                        return 0
                    else:
                        return ostanek(m - n, n)
                \end{minted}
            
\item 
            \begin{minted}[autogobble]{python}
            def ostanek(m, n):
                if m < n:
                    return m
                else:
                    return ostanek(m - n, n)
            \end{minted}
        
\item 
                \begin{minted}[autogobble]{python}
                def ostanek(m, n):
                    if n == 0:
                        return m
                    else:
                        return ostanek(n, m % n)
                \end{minted}
            
\end{enumerate}

        \end{multicols}
    
        \naloga*
        
        Katera izmed funkcij vrača drugačne rezultate kot ostale?
    
        \begin{multicols}{2}
        \begin{enumerate}[(a)]
\item 
                \begin{minted}[autogobble]{python}
                def h(r, q, p):
                    if not q:
                        return False
                    else:
                        return r and p
                \end{minted}
            
\item 
            \begin{minted}[autogobble]{python}
            def h(r, q, p):
                if not p:
                    return r and q
                else:
                    return False
            \end{minted}
        
\item 
                \begin{minted}[autogobble]{python}
                def h(r, q, p):
                    if r:
                        return q and p
                    else:
                        return False
                \end{minted}
            
\item 
                \begin{minted}[autogobble]{python}
                def h(r, q, p):
                    if r and q:
                        return p
                    else:
                        return False
                \end{minted}
            
\end{enumerate}

        \end{multicols}
    
        \naloga*
        
        Kateri izmed spodnjih programov ima drugačen izpis kot ostali?
    
        \begin{multicols}{2}
        \begin{enumerate}[(a)]
\item 
                \begin{minted}[autogobble]{python}
                    for i in range(1, 60, 3):
                        if i > 20:
                            break
                        print(i)
                \end{minted}
            
\item 
                \begin{minted}[autogobble]{python}
                    for i in range(1, 60, 3):
                        if i < 20:
                            print(i)
                \end{minted}
            
\item 
            \begin{minted}[autogobble]{python}
                for i in range(1, 20):
                    if i % 3 != 0:
                        continue
                    print(i)
            \end{minted}
        
\item 
                \begin{minted}[autogobble]{python}
                    for i in range(1, 20):
                        if i % 3 == 1:
                            print(i)
                        continue
                \end{minted}
            
\end{enumerate}

        \end{multicols}
    
        \naloga*
        \begin{multicols}{2}
        \noindent
        Napišite primer vrednosti spremenljivk \inlinepy{str1} in \inlinepy{str2}, za kateri klic \inlinepy{h(str1, str2)} vrne \inlinepy{True}.
        \begin{minted}[baselinestretch=1.2, escapeinside=||]{python}
        str1 = |\answerbox{3}|
        str2 = |\answerbox{3}|
        \end{minted}
        \vfil
        \columnbreak
        \begin{minted}[autogobble]{python}
        def h(str1, str2):
            if len(str1) != len(str2) or len(str1) < 3:
                return False
            for j in range(len(str1)):
                if j % 2 == 0 and str1[j] != str2[j]:
                    return False
            return True
        \end{minted}
        \end{multicols}
    
        \naloga*
        \begin{multicols}{2}
        \noindent 
        S številkami od $0$ do $4$ označite vrstni red, v katerem moramo izvesti ukaze na desni, da bo na koncu spremenljivka \inlinepy{lst1} kazala na seznam \inlinepy{[7, 3, 9]}?
    
        \columnbreak
        \noindent
        \begin{minted}[baselinestretch=1.2, escapeinside=||]{python}
|\answerbox{0.5}| lst2 = [7]
|\answerbox{0.5}| lst1 = lst2 + lst1
|\answerbox{0.5}| lst2 = [3]
|\answerbox{0.5}| lst1.append(9)
|\answerbox{0.5}| lst1 = lst2

        \end{minted}
        \end{multicols}
    
            \izpit[ucilnica=205, naloge=-1]{Uvod v programiranje: Kolokvij \#032}{27.\ marec 2019}{
                Pri vsaki nalogi obkrožite črko pred pravilnim odgovorom ali vpišite pravilno vrednost v ustrezen prostor. \\
                Čas reševanja je 30 minut. Veliko uspeha!
            }
            
        \naloga*
        
        Kateri izmed programov pri začetnem stanju
            \inlinepy{x = 1} in
            \inlinepy{y = 8}
        nastavi vrednosti
            \inlinepy{x = 8},
            \inlinepy{y = 1} in
            \inlinepy{z = 9}?
    
        \begin{multicols}{4}
        \begin{enumerate}[(a)]
\item 
                \begin{minted}[autogobble]{python}
                x = y
                z = y
                y = x
                z = x + y
                \end{minted}
            
\item 
            \begin{minted}[autogobble]{python}
            z = x
            x = y
            y = z
            z = x + y
            \end{minted}
        
\item 
                \begin{minted}[autogobble]{python}
                x = y
                y = z
                z = x
                z = x + y
                \end{minted}
            
\item 
                \begin{minted}[autogobble]{python}
                z = x
                y = z
                x = y
                z = x + y
                \end{minted}
            
\end{enumerate}

        \end{multicols}
    
        \naloga*
        \begin{multicols}{2}
        \noindent
        Kakšne vrste napak vsebuje program na desni?

        \begin{enumerate}[(a)]
\item sintaktične napake, zaradi katerih Python programa noče izvesti.
\item oblikovne napake, ki ne vplivajo na pravilnost rezultata
\item vsebinske napake, zaradi katerih Python izračuna napačen rezultat
\item napake, zaradi katerih Python prekine z izvajanjem programa
\end{enumerate}

        \columnbreak

        \begin{minted}[baselinestretch=1.2,escapeinside=||, autogobble]{python}
        
            define fibonacci(n):
                if n <= 0:
                    return 0
                elif n == 1
                    return 1
                else:
                    a = fibonacci(n - 1)
                    b = fibonacci(n - 2)
                  return a + b
            
        \end{minted}

        \end{multicols}

    
        \naloga*
        Katere vrstice izpiše klic \inlinepy{print(f(g(5)))}, če sta funkciji \inlinepy{f} in \inlinepy{g} definirani kot spodaj?

        \begin{multicols}{2}
        \begin{minted}[autogobble]{python}
        
            def f(a):
                print(a)
                return a + 9

            def g(b):
                print(b)
                return 2 * b
        
        \end{minted}

        \begin{enumerate}[(a)]
\item \inlinepy{[5, 10, 19]}
\item \inlinepy{[5, 19]}
\item \inlinepy{[10, 19]}
\item \inlinepy{[19]}
\end{enumerate}

        \end{multicols}
    
        \naloga*

        \begin{multicols}{2}
        \noindent
        Kateri pogoj preverja spodnja funkcija?
        \begin{minted}[autogobble]{python}
        
            def f(niz):
                for z in niz:
                    if z not in 'aeiouAEIOU':
                        return True
                return False
            
        \end{minted}

        \begin{enumerate}[(a)]
\item ali niz vsebuje kakšen samoglasnik
\item ali niz vsebuje samo samoglasnike
\item ali niz vsebuje znak, ki ni samoglasnik
\item ali niz ne vsebuje nobenega samoglasnika
\end{enumerate}

        \end{multicols}
    
        \naloga*
        \begin{multicols}{2}
        \noindent
        V vsak prostor vpišite \textbf{natanko en znak} tako, da bo dobljeni program v spremenljivko \inlinepy{vs} shranil vsoto števil \inlinepy{b} in \inlinepy{a}:
        
        \columnbreak
        \begin{minted}[baselinestretch=1.2,escapeinside=||]{python}
        vs = |\answerbox{0.5}|
        while b > |\answerbox{0.5}|:
            vs += |\answerbox{0.5}|
            |\answerbox{0.5}| -= 1
        \end{minted}
        \end{multicols}
    
        \clearpage
        \naloga
        
        Katera izmed spodnjih funkcij izračuna ostanek pri deljenju naravnega števila \inlinepy{m} z naravnim številom \inlinepy{n}?
    
        \begin{multicols}{2}
        \begin{enumerate}[(a)]
\item 
                \begin{minted}[autogobble]{python}
                def ostanek(m, n):
                    if n == 0:
                        return m
                    else:
                        return ostanek(n, m % n)
                \end{minted}
            
\item 
            \begin{minted}[autogobble]{python}
            def ostanek(m, n):
                if m < n:
                    return m
                else:
                    return ostanek(m - n, n)
            \end{minted}
        
\item 
                \begin{minted}[autogobble]{python}
                def ostanek(m, n):
                    if m < n:
                        return m
                    else:
                        return ostanek(m % n)
                \end{minted}
            
\item 
                \begin{minted}[autogobble]{python}
                def ostanek(m, n):
                    if m < n:
                        return 0
                    else:
                        return ostanek(m - n, n)
                \end{minted}
            
\end{enumerate}

        \end{multicols}
    
        \naloga*
        
        Katera izmed funkcij vrača drugačne rezultate kot ostale?
    
        \begin{multicols}{2}
        \begin{enumerate}[(a)]
\item 
                \begin{minted}[autogobble]{python}
                def g(a, b, c):
                    if a:
                        return b and c
                    else:
                        return False
                \end{minted}
            
\item 
                \begin{minted}[autogobble]{python}
                def g(a, b, c):
                    if not b:
                        return False
                    else:
                        return a and c
                \end{minted}
            
\item 
                \begin{minted}[autogobble]{python}
                def g(a, b, c):
                    if a and b:
                        return c
                    else:
                        return False
                \end{minted}
            
\item 
            \begin{minted}[autogobble]{python}
            def g(a, b, c):
                if not c:
                    return a and b
                else:
                    return False
            \end{minted}
        
\end{enumerate}

        \end{multicols}
    
        \naloga*
        
        Kateri izmed spodnjih programov ima drugačen izpis kot ostali?
    
        \begin{multicols}{2}
        \begin{enumerate}[(a)]
\item 
                \begin{minted}[autogobble]{python}
                    for i in range(1, 60, 3):
                        if i > 20:
                            break
                        print(i)
                \end{minted}
            
\item 
            \begin{minted}[autogobble]{python}
                for i in range(1, 20):
                    if i % 3 != 0:
                        continue
                    print(i)
            \end{minted}
        
\item 
                \begin{minted}[autogobble]{python}
                    for i in range(1, 20):
                        if i % 3 == 1:
                            print(i)
                        continue
                \end{minted}
            
\item 
                \begin{minted}[autogobble]{python}
                    for i in range(1, 60, 3):
                        if i < 20:
                            print(i)
                \end{minted}
            
\end{enumerate}

        \end{multicols}
    
        \naloga*
        \begin{multicols}{2}
        \noindent
        Napišite primer vrednosti spremenljivk \inlinepy{sez1} in \inlinepy{sez2}, za kateri klic \inlinepy{g(sez1, sez2)} vrne \inlinepy{True}.
        \begin{minted}[baselinestretch=1.2, escapeinside=||]{python}
        sez1 = |\answerbox{3}|
        sez2 = |\answerbox{3}|
        \end{minted}
        \vfil
        \columnbreak
        \begin{minted}[autogobble]{python}
        def g(sez1, sez2):
            if len(sez1) != len(sez2) or len(sez1) < 3:
                return False
            for j in range(len(sez1)):
                if j % 2 == 1 and sez1[j] != sez2[j]:
                    return False
            return True
        \end{minted}
        \end{multicols}
    
        \naloga*
        \begin{multicols}{2}
        \noindent 
        S številkami od $0$ do $4$ označite vrstni red, v katerem moramo izvesti ukaze na desni, da bo na koncu spremenljivka \inlinepy{sez2} kazala na seznam \inlinepy{[6, 7, 5]}?
    
        \columnbreak
        \noindent
        \begin{minted}[baselinestretch=1.2, escapeinside=||]{python}
|\answerbox{0.5}| sez1 = [7]
|\answerbox{0.5}| sez1 = [6]
|\answerbox{0.5}| sez2.append(5)
|\answerbox{0.5}| sez2 = sez1
|\answerbox{0.5}| sez2 = sez1 + sez2

        \end{minted}
        \end{multicols}
    
            \izpit[ucilnica=205, naloge=-1]{Uvod v programiranje: Kolokvij \#033}{27.\ marec 2019}{
                Pri vsaki nalogi obkrožite črko pred pravilnim odgovorom ali vpišite pravilno vrednost v ustrezen prostor. \\
                Čas reševanja je 30 minut. Veliko uspeha!
            }
            
        \naloga*
        
        Kateri izmed programov pri začetnem stanju
            \inlinepy{a = 8} in
            \inlinepy{b = 9}
        nastavi vrednosti
            \inlinepy{a = 9},
            \inlinepy{b = 8} in
            \inlinepy{c = 17}?
    
        \begin{multicols}{4}
        \begin{enumerate}[(a)]
\item 
                \begin{minted}[autogobble]{python}
                a = b
                b = c
                c = a
                c = a + b
                \end{minted}
            
\item 
                \begin{minted}[autogobble]{python}
                c = a
                b = c
                a = b
                c = a + b
                \end{minted}
            
\item 
                \begin{minted}[autogobble]{python}
                a = b
                c = b
                b = a
                c = a + b
                \end{minted}
            
\item 
            \begin{minted}[autogobble]{python}
            c = a
            a = b
            b = c
            c = a + b
            \end{minted}
        
\end{enumerate}

        \end{multicols}
    
        \naloga*
        \begin{multicols}{2}
        \noindent
        Kakšne vrste napak vsebuje program na desni?

        \begin{enumerate}[(a)]
\item sintaktične napake, zaradi katerih Python programa noče izvesti.
\item vsebinske napake, zaradi katerih Python izračuna napačen rezultat
\item napake, zaradi katerih Python prekine z izvajanjem programa
\item oblikovne napake, ki ne vplivajo na pravilnost rezultata
\end{enumerate}

        \columnbreak

        \begin{minted}[baselinestretch=1.2,escapeinside=||, autogobble]{python}
        
            define fibonacci(n):
                if n <= 0:
                    return 0
                elif n == 1
                    return 1
                else:
                    a = fibonacci(n - 1)
                    b = fibonacci(n - 2)
                  return a + b
            
        \end{minted}

        \end{multicols}

    
        \naloga*
        Katere vrstice izpiše klic \inlinepy{print(f(g(4)))}, če sta funkciji \inlinepy{f} in \inlinepy{g} definirani kot spodaj?

        \begin{multicols}{2}
        \begin{minted}[autogobble]{python}
        
            def f(x):
                print(x)
                return x + 3

            def g(y):
                print(y)
                return 6 * y
        
        \end{minted}

        \begin{enumerate}[(a)]
\item \inlinepy{[4, 27]}
\item \inlinepy{[27]}
\item \inlinepy{[4, 24, 27]}
\item \inlinepy{[24, 27]}
\end{enumerate}

        \end{multicols}
    
        \naloga*

        \begin{multicols}{2}
        \noindent
        Kateri pogoj preverja spodnja funkcija?
        \begin{minted}[autogobble]{python}
        
            def f(niz):
                for z in niz:
                    if z in 'aeiouAEIOU':
                        return False
                return True
            
        \end{minted}

        \begin{enumerate}[(a)]
\item ali niz ne vsebuje nobenega samoglasnika
\item ali niz vsebuje znak, ki ni samoglasnik
\item ali niz vsebuje samo samoglasnike
\item ali niz vsebuje kakšen samoglasnik
\end{enumerate}

        \end{multicols}
    
        \naloga*
        \begin{multicols}{2}
        \noindent
        V vsak prostor vpišite \textbf{natanko en znak} tako, da bo dobljeni program v spremenljivko \inlinepy{zm} shranil zmnožek števil \inlinepy{a} in \inlinepy{b}:
        
        \columnbreak
        \begin{minted}[baselinestretch=1.2,escapeinside=||]{python}
        zm = |\answerbox{0.5}|
        while a > |\answerbox{0.5}|:
            zm += |\answerbox{0.5}|
            |\answerbox{0.5}| -= 1
        \end{minted}
        \end{multicols}
    
        \clearpage
        \naloga
        
        Katera izmed spodnjih funkcij izračuna ostanek pri deljenju naravnega števila \inlinepy{a} z naravnim številom \inlinepy{b}?
    
        \begin{multicols}{2}
        \begin{enumerate}[(a)]
\item 
                \begin{minted}[autogobble]{python}
                def ostanek(a, b):
                    if b == 0:
                        return a
                    else:
                        return ostanek(b, a % b)
                \end{minted}
            
\item 
                \begin{minted}[autogobble]{python}
                def ostanek(a, b):
                    if a < b:
                        return a
                    else:
                        return ostanek(a % b)
                \end{minted}
            
\item 
            \begin{minted}[autogobble]{python}
            def ostanek(a, b):
                if a < b:
                    return a
                else:
                    return ostanek(a - b, b)
            \end{minted}
        
\item 
                \begin{minted}[autogobble]{python}
                def ostanek(a, b):
                    if a < b:
                        return 0
                    else:
                        return ostanek(a - b, b)
                \end{minted}
            
\end{enumerate}

        \end{multicols}
    
        \naloga*
        
        Katera izmed funkcij vrača drugačne rezultate kot ostale?
    
        \begin{multicols}{2}
        \begin{enumerate}[(a)]
\item 
                \begin{minted}[autogobble]{python}
                def f(r, p, q):
                    if r and p:
                        return q
                    else:
                        return False
                \end{minted}
            
\item 
                \begin{minted}[autogobble]{python}
                def f(r, p, q):
                    if r:
                        return p and q
                    else:
                        return False
                \end{minted}
            
\item 
                \begin{minted}[autogobble]{python}
                def f(r, p, q):
                    if not p:
                        return False
                    else:
                        return r and q
                \end{minted}
            
\item 
            \begin{minted}[autogobble]{python}
            def f(r, p, q):
                if not q:
                    return r and p
                else:
                    return False
            \end{minted}
        
\end{enumerate}

        \end{multicols}
    
        \naloga*
        
        Kateri izmed spodnjih programov ima drugačen izpis kot ostali?
    
        \begin{multicols}{2}
        \begin{enumerate}[(a)]
\item 
                \begin{minted}[autogobble]{python}
                    for n in range(1, 10):
                        if n % 3 == 1:
                            print(n)
                        continue
                \end{minted}
            
\item 
                \begin{minted}[autogobble]{python}
                    for n in range(1, 30, 3):
                        if n > 10:
                            break
                        print(n)
                \end{minted}
            
\item 
                \begin{minted}[autogobble]{python}
                    for n in range(1, 30, 3):
                        if n < 10:
                            print(n)
                \end{minted}
            
\item 
            \begin{minted}[autogobble]{python}
                for n in range(1, 10):
                    if n % 3 != 0:
                        continue
                    print(n)
            \end{minted}
        
\end{enumerate}

        \end{multicols}
    
        \naloga*
        \begin{multicols}{2}
        \noindent
        Napišite primer vrednosti spremenljivk \inlinepy{sez1} in \inlinepy{sez2}, za kateri klic \inlinepy{h(sez1, sez2)} vrne \inlinepy{True}.
        \begin{minted}[baselinestretch=1.2, escapeinside=||]{python}
        sez1 = |\answerbox{3}|
        sez2 = |\answerbox{3}|
        \end{minted}
        \vfil
        \columnbreak
        \begin{minted}[autogobble]{python}
        def h(sez1, sez2):
            if len(sez1) != len(sez2) or len(sez1) < 3:
                return False
            for j in range(len(sez1)):
                if j % 2 == 1 and sez1[j] != sez2[j]:
                    return False
            return True
        \end{minted}
        \end{multicols}
    
        \naloga*
        \begin{multicols}{2}
        \noindent 
        S številkami od $0$ do $4$ označite vrstni red, v katerem moramo izvesti ukaze na desni, da bo na koncu spremenljivka \inlinepy{sez1} kazala na seznam \inlinepy{[4, 8, 3]}?
    
        \columnbreak
        \noindent
        \begin{minted}[baselinestretch=1.2, escapeinside=||]{python}
|\answerbox{0.5}| sez1 = sez2 + sez1
|\answerbox{0.5}| sez2 = [8]
|\answerbox{0.5}| sez1 = sez2
|\answerbox{0.5}| sez1.append(3)
|\answerbox{0.5}| sez2 = [4]

        \end{minted}
        \end{multicols}
    
            \izpit[ucilnica=205, naloge=-1]{Uvod v programiranje: Kolokvij \#034}{27.\ marec 2019}{
                Pri vsaki nalogi obkrožite črko pred pravilnim odgovorom ali vpišite pravilno vrednost v ustrezen prostor. \\
                Čas reševanja je 30 minut. Veliko uspeha!
            }
            
        \naloga*
        
        Kateri izmed programov pri začetnem stanju
            \inlinepy{u = 5} in
            \inlinepy{v = 8}
        nastavi vrednosti
            \inlinepy{u = 8},
            \inlinepy{v = 5} in
            \inlinepy{w = 13}?
    
        \begin{multicols}{4}
        \begin{enumerate}[(a)]
\item 
                \begin{minted}[autogobble]{python}
                w = u
                v = w
                u = v
                w = u + v
                \end{minted}
            
\item 
                \begin{minted}[autogobble]{python}
                u = v
                w = v
                v = u
                w = u + v
                \end{minted}
            
\item 
            \begin{minted}[autogobble]{python}
            w = u
            u = v
            v = w
            w = u + v
            \end{minted}
        
\item 
                \begin{minted}[autogobble]{python}
                u = v
                v = w
                w = u
                w = u + v
                \end{minted}
            
\end{enumerate}

        \end{multicols}
    
        \naloga*
        \begin{multicols}{2}
        \noindent
        Kakšne vrste napak vsebuje program na desni?

        \begin{enumerate}[(a)]
\item sintaktične napake, zaradi katerih Python programa noče izvesti.
\item vsebinske napake, zaradi katerih Python izračuna napačen rezultat
\item oblikovne napake, ki ne vplivajo na pravilnost rezultata
\item napake, zaradi katerih Python prekine z izvajanjem programa
\end{enumerate}

        \columnbreak

        \begin{minted}[baselinestretch=1.2,escapeinside=||, autogobble]{python}
        
            def fibonacci(n):
                if n <= 0:
                    return 0
                elif n == 1:
                    return 1
                else:
                    a = fibonacci(n - '1')
                    b = fib(n - 2)
                    return a + b
            
        \end{minted}

        \end{multicols}

    
        \naloga*
        Katere vrstice izpiše klic \inlinepy{print(f(g(8)))}, če sta funkciji \inlinepy{f} in \inlinepy{g} definirani kot spodaj?

        \begin{multicols}{2}
        \begin{minted}[autogobble]{python}
        
            def f(a):
                print(a)
                return a + 2

            def g(b):
                print(b)
                return 5 * b
        
        \end{minted}

        \begin{enumerate}[(a)]
\item \inlinepy{[8, 40, 42]}
\item \inlinepy{[42]}
\item \inlinepy{[8, 42]}
\item \inlinepy{[40, 42]}
\end{enumerate}

        \end{multicols}
    
        \naloga*

        \begin{multicols}{2}
        \noindent
        Kateri pogoj preverja spodnja funkcija?
        \begin{minted}[autogobble]{python}
        
            def f(besedilo):
                for z in besedilo:
                    if z not in 'aeiouAEIOU':
                        return False
                return True
            
        \end{minted}

        \begin{enumerate}[(a)]
\item ali niz vsebuje kakšen samoglasnik
\item ali niz vsebuje samo samoglasnike
\item ali niz ne vsebuje nobenega samoglasnika
\item ali niz vsebuje znak, ki ni samoglasnik
\end{enumerate}

        \end{multicols}
    
        \naloga*
        \begin{multicols}{2}
        \noindent
        V vsak prostor vpišite \textbf{natanko en znak} tako, da bo dobljeni program v spremenljivko \inlinepy{vs} shranil vsoto števil \inlinepy{a} in \inlinepy{b}:
        
        \columnbreak
        \begin{minted}[baselinestretch=1.2,escapeinside=||]{python}
        vs = |\answerbox{0.5}|
        while a > |\answerbox{0.5}|:
            vs += |\answerbox{0.5}|
            |\answerbox{0.5}| -= 1
        \end{minted}
        \end{multicols}
    
        \clearpage
        \naloga
        
        Katera izmed spodnjih funkcij izračuna ostanek pri deljenju naravnega števila \inlinepy{a} z naravnim številom \inlinepy{b}?
    
        \begin{multicols}{2}
        \begin{enumerate}[(a)]
\item 
                \begin{minted}[autogobble]{python}
                def ostanek(a, b):
                    if a < b:
                        return a
                    else:
                        return ostanek(a % b)
                \end{minted}
            
\item 
            \begin{minted}[autogobble]{python}
            def ostanek(a, b):
                if a < b:
                    return a
                else:
                    return ostanek(a - b, b)
            \end{minted}
        
\item 
                \begin{minted}[autogobble]{python}
                def ostanek(a, b):
                    if a < b:
                        return 0
                    else:
                        return ostanek(a - b, b)
                \end{minted}
            
\item 
                \begin{minted}[autogobble]{python}
                def ostanek(a, b):
                    if b == 0:
                        return a
                    else:
                        return ostanek(b, a % b)
                \end{minted}
            
\end{enumerate}

        \end{multicols}
    
        \naloga*
        
        Katera izmed funkcij vrača drugačne rezultate kot ostale?
    
        \begin{multicols}{2}
        \begin{enumerate}[(a)]
\item 
                \begin{minted}[autogobble]{python}
                def f(r, p, q):
                    if r:
                        return p and q
                    else:
                        return False
                \end{minted}
            
\item 
                \begin{minted}[autogobble]{python}
                def f(r, p, q):
                    if not p:
                        return False
                    else:
                        return r and q
                \end{minted}
            
\item 
                \begin{minted}[autogobble]{python}
                def f(r, p, q):
                    if r and p:
                        return q
                    else:
                        return False
                \end{minted}
            
\item 
            \begin{minted}[autogobble]{python}
            def f(r, p, q):
                if not q:
                    return r and p
                else:
                    return False
            \end{minted}
        
\end{enumerate}

        \end{multicols}
    
        \naloga*
        
        Kateri izmed spodnjih programov ima drugačen izpis kot ostali?
    
        \begin{multicols}{2}
        \begin{enumerate}[(a)]
\item 
                \begin{minted}[autogobble]{python}
                    for n in range(1, 30, 3):
                        if n < 10:
                            print(n)
                \end{minted}
            
\item 
            \begin{minted}[autogobble]{python}
                for n in range(1, 10):
                    if n % 3 != 0:
                        continue
                    print(n)
            \end{minted}
        
\item 
                \begin{minted}[autogobble]{python}
                    for n in range(1, 10):
                        if n % 3 == 1:
                            print(n)
                        continue
                \end{minted}
            
\item 
                \begin{minted}[autogobble]{python}
                    for n in range(1, 30, 3):
                        if n > 10:
                            break
                        print(n)
                \end{minted}
            
\end{enumerate}

        \end{multicols}
    
        \naloga*
        \begin{multicols}{2}
        \noindent
        Napišite primer vrednosti spremenljivk \inlinepy{lst1} in \inlinepy{lst2}, za kateri klic \inlinepy{f(lst1, lst2)} vrne \inlinepy{True}.
        \begin{minted}[baselinestretch=1.2, escapeinside=||]{python}
        lst1 = |\answerbox{3}|
        lst2 = |\answerbox{3}|
        \end{minted}
        \vfil
        \columnbreak
        \begin{minted}[autogobble]{python}
        def f(lst1, lst2):
            if len(lst1) != len(lst2) or len(lst1) < 3:
                return False
            for j in range(len(lst1)):
                if j % 2 == 1 and lst1[j] != lst2[j]:
                    return False
            return True
        \end{minted}
        \end{multicols}
    
        \naloga*
        \begin{multicols}{2}
        \noindent 
        S številkami od $0$ do $4$ označite vrstni red, v katerem moramo izvesti ukaze na desni, da bo na koncu spremenljivka \inlinepy{sez1} kazala na seznam \inlinepy{[7, 4, 9]}?
    
        \columnbreak
        \noindent
        \begin{minted}[baselinestretch=1.2, escapeinside=||]{python}
|\answerbox{0.5}| sez1 = sez2
|\answerbox{0.5}| sez1.append(9)
|\answerbox{0.5}| sez1 = sez2 + sez1
|\answerbox{0.5}| sez2 = [7]
|\answerbox{0.5}| sez2 = [4]

        \end{minted}
        \end{multicols}
    
            \izpit[ucilnica=205, naloge=-1]{Uvod v programiranje: Kolokvij \#035}{27.\ marec 2019}{
                Pri vsaki nalogi obkrožite črko pred pravilnim odgovorom ali vpišite pravilno vrednost v ustrezen prostor. \\
                Čas reševanja je 30 minut. Veliko uspeha!
            }
            
        \naloga*
        
        Kateri izmed programov pri začetnem stanju
            \inlinepy{u = 7} in
            \inlinepy{v = 5}
        nastavi vrednosti
            \inlinepy{u = 5},
            \inlinepy{v = 7} in
            \inlinepy{w = 12}?
    
        \begin{multicols}{4}
        \begin{enumerate}[(a)]
\item 
                \begin{minted}[autogobble]{python}
                u = v
                v = w
                w = u
                w = u + v
                \end{minted}
            
\item 
                \begin{minted}[autogobble]{python}
                w = u
                v = w
                u = v
                w = u + v
                \end{minted}
            
\item 
                \begin{minted}[autogobble]{python}
                u = v
                w = v
                v = u
                w = u + v
                \end{minted}
            
\item 
            \begin{minted}[autogobble]{python}
            w = u
            u = v
            v = w
            w = u + v
            \end{minted}
        
\end{enumerate}

        \end{multicols}
    
        \naloga*
        \begin{multicols}{2}
        \noindent
        Kakšne vrste napak vsebuje program na desni?

        \begin{enumerate}[(a)]
\item sintaktične napake, zaradi katerih Python programa noče izvesti.
\item napake, zaradi katerih Python prekine z izvajanjem programa
\item vsebinske napake, zaradi katerih Python izračuna napačen rezultat
\item oblikovne napake, ki ne vplivajo na pravilnost rezultata
\end{enumerate}

        \columnbreak

        \begin{minted}[baselinestretch=1.2,escapeinside=||, autogobble]{python}
        
            def fibonacci(n):
                if n <= 0:
                    return 0
                elif n == 1:
                    return 0
                else:
                    a = fibonacci(n - 1)
                    b = fibonacci(n - 3)
                    return a * b
            
        \end{minted}

        \end{multicols}

    
        \naloga*
        Katere vrstice izpiše klic \inlinepy{print(f(g(3)))}, če sta funkciji \inlinepy{f} in \inlinepy{g} definirani kot spodaj?

        \begin{multicols}{2}
        \begin{minted}[autogobble]{python}
        
            def f(a):
                return a + 2
                print(a)

            def g(b):
                return 8 * b
                print(b)
        
        \end{minted}

        \begin{enumerate}[(a)]
\item \inlinepy{[3, 24, 26]}
\item \inlinepy{[26]}
\item \inlinepy{[24, 26]}
\item \inlinepy{[3, 26]}
\end{enumerate}

        \end{multicols}
    
        \naloga*

        \begin{multicols}{2}
        \noindent
        Kateri pogoj preverja spodnja funkcija?
        \begin{minted}[autogobble]{python}
        
            def f(niz):
                for znak in niz:
                    if znak in 'aeiouAEIOU':
                        return True
                return False
            
        \end{minted}

        \begin{enumerate}[(a)]
\item ali niz vsebuje samo samoglasnike
\item ali niz ne vsebuje nobenega samoglasnika
\item ali niz vsebuje znak, ki ni samoglasnik
\item ali niz vsebuje kakšen samoglasnik
\end{enumerate}

        \end{multicols}
    
        \naloga*
        \begin{multicols}{2}
        \noindent
        V vsak prostor vpišite \textbf{natanko en znak} tako, da bo dobljeni program v spremenljivko \inlinepy{vs} shranil vsoto števil \inlinepy{y} in \inlinepy{x}:
        
        \columnbreak
        \begin{minted}[baselinestretch=1.2,escapeinside=||]{python}
        vs = |\answerbox{0.5}|
        while y > |\answerbox{0.5}|:
            vs += |\answerbox{0.5}|
            |\answerbox{0.5}| -= 1
        \end{minted}
        \end{multicols}
    
        \clearpage
        \naloga
        
        Katera izmed spodnjih funkcij izračuna ostanek pri deljenju naravnega števila \inlinepy{a} z naravnim številom \inlinepy{b}?
    
        \begin{multicols}{2}
        \begin{enumerate}[(a)]
\item 
                \begin{minted}[autogobble]{python}
                def ostanek(a, b):
                    if a < b:
                        return a
                    else:
                        return ostanek(a % b)
                \end{minted}
            
\item 
            \begin{minted}[autogobble]{python}
            def ostanek(a, b):
                if a < b:
                    return a
                else:
                    return ostanek(a - b, b)
            \end{minted}
        
\item 
                \begin{minted}[autogobble]{python}
                def ostanek(a, b):
                    if b == 0:
                        return a
                    else:
                        return ostanek(b, a % b)
                \end{minted}
            
\item 
                \begin{minted}[autogobble]{python}
                def ostanek(a, b):
                    if a < b:
                        return 0
                    else:
                        return ostanek(a - b, b)
                \end{minted}
            
\end{enumerate}

        \end{multicols}
    
        \naloga*
        
        Katera izmed funkcij vrača drugačne rezultate kot ostale?
    
        \begin{multicols}{2}
        \begin{enumerate}[(a)]
\item 
                \begin{minted}[autogobble]{python}
                def h(a, c, b):
                    if a:
                        return c and b
                    else:
                        return False
                \end{minted}
            
\item 
                \begin{minted}[autogobble]{python}
                def h(a, c, b):
                    if not c:
                        return False
                    else:
                        return a and b
                \end{minted}
            
\item 
                \begin{minted}[autogobble]{python}
                def h(a, c, b):
                    if a and c:
                        return b
                    else:
                        return False
                \end{minted}
            
\item 
            \begin{minted}[autogobble]{python}
            def h(a, c, b):
                if not b:
                    return a and c
                else:
                    return False
            \end{minted}
        
\end{enumerate}

        \end{multicols}
    
        \naloga*
        
        Kateri izmed spodnjih programov ima drugačen izpis kot ostali?
    
        \begin{multicols}{2}
        \begin{enumerate}[(a)]
\item 
                \begin{minted}[autogobble]{python}
                    for i in range(1, 100, 2):
                        if i < 50:
                            print(i)
                \end{minted}
            
\item 
                \begin{minted}[autogobble]{python}
                    for i in range(1, 100, 2):
                        if i > 50:
                            break
                        print(i)
                \end{minted}
            
\item 
                \begin{minted}[autogobble]{python}
                    for i in range(1, 50):
                        if i % 2 == 1:
                            print(i)
                        continue
                \end{minted}
            
\item 
            \begin{minted}[autogobble]{python}
                for i in range(1, 50):
                    if i % 2 != 0:
                        continue
                    print(i)
            \end{minted}
        
\end{enumerate}

        \end{multicols}
    
        \naloga*
        \begin{multicols}{2}
        \noindent
        Napišite primer vrednosti spremenljivk \inlinepy{lst1} in \inlinepy{lst2}, za kateri klic \inlinepy{g(lst1, lst2)} vrne \inlinepy{True}.
        \begin{minted}[baselinestretch=1.2, escapeinside=||]{python}
        lst1 = |\answerbox{3}|
        lst2 = |\answerbox{3}|
        \end{minted}
        \vfil
        \columnbreak
        \begin{minted}[autogobble]{python}
        def g(lst1, lst2):
            if len(lst1) != len(lst2) or len(lst1) < 3:
                return False
            for j in range(len(lst1)):
                if j % 2 == 1 and lst1[j] != lst2[j]:
                    return False
            return True
        \end{minted}
        \end{multicols}
    
        \naloga*
        \begin{multicols}{2}
        \noindent 
        S številkami od $0$ do $4$ označite vrstni red, v katerem moramo izvesti ukaze na desni, da bo na koncu spremenljivka \inlinepy{sez1} kazala na seznam \inlinepy{[2, 6, 5]}?
    
        \columnbreak
        \noindent
        \begin{minted}[baselinestretch=1.2, escapeinside=||]{python}
|\answerbox{0.5}| sez1.append(5)
|\answerbox{0.5}| sez1 = sez2 + sez1
|\answerbox{0.5}| sez2 = [6]
|\answerbox{0.5}| sez2 = [2]
|\answerbox{0.5}| sez1 = sez2

        \end{minted}
        \end{multicols}
    
            \izpit[ucilnica=205, naloge=-1]{Uvod v programiranje: Kolokvij \#036}{27.\ marec 2019}{
                Pri vsaki nalogi obkrožite črko pred pravilnim odgovorom ali vpišite pravilno vrednost v ustrezen prostor. \\
                Čas reševanja je 30 minut. Veliko uspeha!
            }
            
        \naloga*
        
        Kateri izmed programov pri začetnem stanju
            \inlinepy{a = 8} in
            \inlinepy{b = 6}
        nastavi vrednosti
            \inlinepy{a = 6},
            \inlinepy{b = 8} in
            \inlinepy{c = 14}?
    
        \begin{multicols}{4}
        \begin{enumerate}[(a)]
\item 
                \begin{minted}[autogobble]{python}
                a = b
                c = b
                b = a
                c = a + b
                \end{minted}
            
\item 
            \begin{minted}[autogobble]{python}
            c = a
            a = b
            b = c
            c = a + b
            \end{minted}
        
\item 
                \begin{minted}[autogobble]{python}
                a = b
                b = c
                c = a
                c = a + b
                \end{minted}
            
\item 
                \begin{minted}[autogobble]{python}
                c = a
                b = c
                a = b
                c = a + b
                \end{minted}
            
\end{enumerate}

        \end{multicols}
    
        \naloga*
        \begin{multicols}{2}
        \noindent
        Kakšne vrste napak vsebuje program na desni?

        \begin{enumerate}[(a)]
\item oblikovne napake, ki ne vplivajo na pravilnost rezultata
\item napake, zaradi katerih Python prekine z izvajanjem programa
\item vsebinske napake, zaradi katerih Python izračuna napačen rezultat
\item sintaktične napake, zaradi katerih Python programa noče izvesti.
\end{enumerate}

        \columnbreak

        \begin{minted}[baselinestretch=1.2,escapeinside=||, autogobble]{python}
        
            def fibonacci(n):
                if n <= 0:
                    return 0
                elif n == 1:
                    return 1
                else:
                    a = fibonacci(n - '1')
                    b = fib(n - 2)
                    return a + b
            
        \end{minted}

        \end{multicols}

    
        \naloga*
        Katere vrstice izpiše klic \inlinepy{print(f(g(6)))}, če sta funkciji \inlinepy{f} in \inlinepy{g} definirani kot spodaj?

        \begin{multicols}{2}
        \begin{minted}[autogobble]{python}
        
            def f(x):
                return x + 7
                print(x)

            def g(y):
                print(y)
                return 9 * y
        
        \end{minted}

        \begin{enumerate}[(a)]
\item \inlinepy{[61]}
\item \inlinepy{[6, 54, 61]}
\item \inlinepy{[54, 61]}
\item \inlinepy{[6, 61]}
\end{enumerate}

        \end{multicols}
    
        \naloga*

        \begin{multicols}{2}
        \noindent
        Kateri pogoj preverja spodnja funkcija?
        \begin{minted}[autogobble]{python}
        
            def f(stavek):
                for znak in stavek:
                    if znak not in 'aeiouAEIOU':
                        return True
                return False
            
        \end{minted}

        \begin{enumerate}[(a)]
\item ali niz vsebuje samo samoglasnike
\item ali niz vsebuje kakšen samoglasnik
\item ali niz vsebuje znak, ki ni samoglasnik
\item ali niz ne vsebuje nobenega samoglasnika
\end{enumerate}

        \end{multicols}
    
        \naloga*
        \begin{multicols}{2}
        \noindent
        V vsak prostor vpišite \textbf{natanko en znak} tako, da bo dobljeni program v spremenljivko \inlinepy{vs} shranil vsoto števil \inlinepy{y} in \inlinepy{x}:
        
        \columnbreak
        \begin{minted}[baselinestretch=1.2,escapeinside=||]{python}
        vs = |\answerbox{0.5}|
        while y > |\answerbox{0.5}|:
            vs += |\answerbox{0.5}|
            |\answerbox{0.5}| -= 1
        \end{minted}
        \end{multicols}
    
        \clearpage
        \naloga
        
        Katera izmed spodnjih funkcij izračuna ostanek pri deljenju naravnega števila \inlinepy{a} z naravnim številom \inlinepy{b}?
    
        \begin{multicols}{2}
        \begin{enumerate}[(a)]
\item 
            \begin{minted}[autogobble]{python}
            def ostanek(a, b):
                if a < b:
                    return a
                else:
                    return ostanek(a - b, b)
            \end{minted}
        
\item 
                \begin{minted}[autogobble]{python}
                def ostanek(a, b):
                    if a < b:
                        return a
                    else:
                        return ostanek(a % b)
                \end{minted}
            
\item 
                \begin{minted}[autogobble]{python}
                def ostanek(a, b):
                    if a < b:
                        return 0
                    else:
                        return ostanek(a - b, b)
                \end{minted}
            
\item 
                \begin{minted}[autogobble]{python}
                def ostanek(a, b):
                    if b == 0:
                        return a
                    else:
                        return ostanek(b, a % b)
                \end{minted}
            
\end{enumerate}

        \end{multicols}
    
        \naloga*
        
        Katera izmed funkcij vrača drugačne rezultate kot ostale?
    
        \begin{multicols}{2}
        \begin{enumerate}[(a)]
\item 
            \begin{minted}[autogobble]{python}
            def h(b, a, c):
                if not c:
                    return b and a
                else:
                    return False
            \end{minted}
        
\item 
                \begin{minted}[autogobble]{python}
                def h(b, a, c):
                    if not a:
                        return False
                    else:
                        return b and c
                \end{minted}
            
\item 
                \begin{minted}[autogobble]{python}
                def h(b, a, c):
                    if b and a:
                        return c
                    else:
                        return False
                \end{minted}
            
\item 
                \begin{minted}[autogobble]{python}
                def h(b, a, c):
                    if b:
                        return a and c
                    else:
                        return False
                \end{minted}
            
\end{enumerate}

        \end{multicols}
    
        \naloga*
        
        Kateri izmed spodnjih programov ima drugačen izpis kot ostali?
    
        \begin{multicols}{2}
        \begin{enumerate}[(a)]
\item 
                \begin{minted}[autogobble]{python}
                    for n in range(1, 30, 3):
                        if n > 10:
                            break
                        print(n)
                \end{minted}
            
\item 
                \begin{minted}[autogobble]{python}
                    for n in range(1, 10):
                        if n % 3 == 1:
                            print(n)
                        continue
                \end{minted}
            
\item 
                \begin{minted}[autogobble]{python}
                    for n in range(1, 30, 3):
                        if n < 10:
                            print(n)
                \end{minted}
            
\item 
            \begin{minted}[autogobble]{python}
                for n in range(1, 10):
                    if n % 3 != 0:
                        continue
                    print(n)
            \end{minted}
        
\end{enumerate}

        \end{multicols}
    
        \naloga*
        \begin{multicols}{2}
        \noindent
        Napišite primer vrednosti spremenljivk \inlinepy{str1} in \inlinepy{str2}, za kateri klic \inlinepy{f(str1, str2)} vrne \inlinepy{True}.
        \begin{minted}[baselinestretch=1.2, escapeinside=||]{python}
        str1 = |\answerbox{3}|
        str2 = |\answerbox{3}|
        \end{minted}
        \vfil
        \columnbreak
        \begin{minted}[autogobble]{python}
        def f(str1, str2):
            if len(str1) != len(str2) or len(str1) < 3:
                return False
            for j in range(len(str1)):
                if j % 2 == 1 and str1[j] != str2[j]:
                    return False
            return True
        \end{minted}
        \end{multicols}
    
        \naloga*
        \begin{multicols}{2}
        \noindent 
        S številkami od $0$ do $4$ označite vrstni red, v katerem moramo izvesti ukaze na desni, da bo na koncu spremenljivka \inlinepy{lst1} kazala na seznam \inlinepy{[6, 5, 2]}?
    
        \columnbreak
        \noindent
        \begin{minted}[baselinestretch=1.2, escapeinside=||]{python}
|\answerbox{0.5}| lst1.append(2)
|\answerbox{0.5}| lst2 = [6]
|\answerbox{0.5}| lst1 = lst2 + lst1
|\answerbox{0.5}| lst1 = lst2
|\answerbox{0.5}| lst2 = [5]

        \end{minted}
        \end{multicols}
    
            \izpit[ucilnica=205, naloge=-1]{Uvod v programiranje: Kolokvij \#037}{27.\ marec 2019}{
                Pri vsaki nalogi obkrožite črko pred pravilnim odgovorom ali vpišite pravilno vrednost v ustrezen prostor. \\
                Čas reševanja je 30 minut. Veliko uspeha!
            }
            
        \naloga*
        
        Kateri izmed programov pri začetnem stanju
            \inlinepy{k = 5} in
            \inlinepy{m = 2}
        nastavi vrednosti
            \inlinepy{k = 2},
            \inlinepy{m = 5} in
            \inlinepy{n = 7}?
    
        \begin{multicols}{4}
        \begin{enumerate}[(a)]
\item 
            \begin{minted}[autogobble]{python}
            n = k
            k = m
            m = n
            n = k + m
            \end{minted}
        
\item 
                \begin{minted}[autogobble]{python}
                n = k
                m = n
                k = m
                n = k + m
                \end{minted}
            
\item 
                \begin{minted}[autogobble]{python}
                k = m
                n = m
                m = k
                n = k + m
                \end{minted}
            
\item 
                \begin{minted}[autogobble]{python}
                k = m
                m = n
                n = k
                n = k + m
                \end{minted}
            
\end{enumerate}

        \end{multicols}
    
        \naloga*
        \begin{multicols}{2}
        \noindent
        Kakšne vrste napak vsebuje program na desni?

        \begin{enumerate}[(a)]
\item sintaktične napake, zaradi katerih Python programa noče izvesti.
\item oblikovne napake, ki ne vplivajo na pravilnost rezultata
\item napake, zaradi katerih Python prekine z izvajanjem programa
\item vsebinske napake, zaradi katerih Python izračuna napačen rezultat
\end{enumerate}

        \columnbreak

        \begin{minted}[baselinestretch=1.2,escapeinside=||, autogobble]{python}
        
            def fibonacci(n):
                if n <= 0:
                    return 0
                elif n == 1:
                    return 1
                else:
                    a = fibonacci(n - '1')
                    b = fib(n - 2)
                    return a + b
            
        \end{minted}

        \end{multicols}

    
        \naloga*
        Katere vrstice izpiše klic \inlinepy{print(f(g(9)))}, če sta funkciji \inlinepy{f} in \inlinepy{g} definirani kot spodaj?

        \begin{multicols}{2}
        \begin{minted}[autogobble]{python}
        
            def f(x):
                return x + 6
                print(x)

            def g(y):
                print(y)
                return 7 * y
        
        \end{minted}

        \begin{enumerate}[(a)]
\item \inlinepy{[69]}
\item \inlinepy{[9, 63, 69]}
\item \inlinepy{[63, 69]}
\item \inlinepy{[9, 69]}
\end{enumerate}

        \end{multicols}
    
        \naloga*

        \begin{multicols}{2}
        \noindent
        Kateri pogoj preverja spodnja funkcija?
        \begin{minted}[autogobble]{python}
        
            def f(besedilo):
                for znak in besedilo:
                    if znak not in 'aeiouAEIOU':
                        return False
                return True
            
        \end{minted}

        \begin{enumerate}[(a)]
\item ali niz vsebuje znak, ki ni samoglasnik
\item ali niz vsebuje samo samoglasnike
\item ali niz vsebuje kakšen samoglasnik
\item ali niz ne vsebuje nobenega samoglasnika
\end{enumerate}

        \end{multicols}
    
        \naloga*
        \begin{multicols}{2}
        \noindent
        V vsak prostor vpišite \textbf{natanko en znak} tako, da bo dobljeni program v spremenljivko \inlinepy{zm} shranil zmnožek števil \inlinepy{b} in \inlinepy{a}:
        
        \columnbreak
        \begin{minted}[baselinestretch=1.2,escapeinside=||]{python}
        zm = |\answerbox{0.5}|
        while b > |\answerbox{0.5}|:
            zm += |\answerbox{0.5}|
            |\answerbox{0.5}| -= 1
        \end{minted}
        \end{multicols}
    
        \clearpage
        \naloga
        
        Katera izmed spodnjih funkcij izračuna ostanek pri deljenju naravnega števila \inlinepy{m} z naravnim številom \inlinepy{n}?
    
        \begin{multicols}{2}
        \begin{enumerate}[(a)]
\item 
                \begin{minted}[autogobble]{python}
                def ostanek(m, n):
                    if m < n:
                        return m
                    else:
                        return ostanek(m % n)
                \end{minted}
            
\item 
            \begin{minted}[autogobble]{python}
            def ostanek(m, n):
                if m < n:
                    return m
                else:
                    return ostanek(m - n, n)
            \end{minted}
        
\item 
                \begin{minted}[autogobble]{python}
                def ostanek(m, n):
                    if m < n:
                        return 0
                    else:
                        return ostanek(m - n, n)
                \end{minted}
            
\item 
                \begin{minted}[autogobble]{python}
                def ostanek(m, n):
                    if n == 0:
                        return m
                    else:
                        return ostanek(n, m % n)
                \end{minted}
            
\end{enumerate}

        \end{multicols}
    
        \naloga*
        
        Katera izmed funkcij vrača drugačne rezultate kot ostale?
    
        \begin{multicols}{2}
        \begin{enumerate}[(a)]
\item 
                \begin{minted}[autogobble]{python}
                def f(b, c, a):
                    if b:
                        return c and a
                    else:
                        return False
                \end{minted}
            
\item 
            \begin{minted}[autogobble]{python}
            def f(b, c, a):
                if not a:
                    return b and c
                else:
                    return False
            \end{minted}
        
\item 
                \begin{minted}[autogobble]{python}
                def f(b, c, a):
                    if b and c:
                        return a
                    else:
                        return False
                \end{minted}
            
\item 
                \begin{minted}[autogobble]{python}
                def f(b, c, a):
                    if not c:
                        return False
                    else:
                        return b and a
                \end{minted}
            
\end{enumerate}

        \end{multicols}
    
        \naloga*
        
        Kateri izmed spodnjih programov ima drugačen izpis kot ostali?
    
        \begin{multicols}{2}
        \begin{enumerate}[(a)]
\item 
                \begin{minted}[autogobble]{python}
                    for j in range(1, 50):
                        if j % 3 == 1:
                            print(j)
                        continue
                \end{minted}
            
\item 
                \begin{minted}[autogobble]{python}
                    for j in range(1, 150, 3):
                        if j < 50:
                            print(j)
                \end{minted}
            
\item 
                \begin{minted}[autogobble]{python}
                    for j in range(1, 150, 3):
                        if j > 50:
                            break
                        print(j)
                \end{minted}
            
\item 
            \begin{minted}[autogobble]{python}
                for j in range(1, 50):
                    if j % 3 != 0:
                        continue
                    print(j)
            \end{minted}
        
\end{enumerate}

        \end{multicols}
    
        \naloga*
        \begin{multicols}{2}
        \noindent
        Napišite primer vrednosti spremenljivk \inlinepy{str1} in \inlinepy{str2}, za kateri klic \inlinepy{g(str1, str2)} vrne \inlinepy{True}.
        \begin{minted}[baselinestretch=1.2, escapeinside=||]{python}
        str1 = |\answerbox{3}|
        str2 = |\answerbox{3}|
        \end{minted}
        \vfil
        \columnbreak
        \begin{minted}[autogobble]{python}
        def g(str1, str2):
            if len(str1) != len(str2) or len(str1) < 3:
                return False
            for j in range(len(str1)):
                if j % 2 == 0 and str1[j] != str2[j]:
                    return False
            return True
        \end{minted}
        \end{multicols}
    
        \naloga*
        \begin{multicols}{2}
        \noindent 
        S številkami od $0$ do $4$ označite vrstni red, v katerem moramo izvesti ukaze na desni, da bo na koncu spremenljivka \inlinepy{lst2} kazala na seznam \inlinepy{[6, 3, 8]}?
    
        \columnbreak
        \noindent
        \begin{minted}[baselinestretch=1.2, escapeinside=||]{python}
|\answerbox{0.5}| lst2.append(8)
|\answerbox{0.5}| lst2 = lst1
|\answerbox{0.5}| lst1 = [6]
|\answerbox{0.5}| lst1 = [3]
|\answerbox{0.5}| lst2 = lst1 + lst2

        \end{minted}
        \end{multicols}
    
            \izpit[ucilnica=205, naloge=-1]{Uvod v programiranje: Kolokvij \#038}{27.\ marec 2019}{
                Pri vsaki nalogi obkrožite črko pred pravilnim odgovorom ali vpišite pravilno vrednost v ustrezen prostor. \\
                Čas reševanja je 30 minut. Veliko uspeha!
            }
            
        \naloga*
        
        Kateri izmed programov pri začetnem stanju
            \inlinepy{a = 1} in
            \inlinepy{b = 7}
        nastavi vrednosti
            \inlinepy{a = 7},
            \inlinepy{b = 1} in
            \inlinepy{c = 8}?
    
        \begin{multicols}{4}
        \begin{enumerate}[(a)]
\item 
                \begin{minted}[autogobble]{python}
                c = a
                b = c
                a = b
                c = a + b
                \end{minted}
            
\item 
            \begin{minted}[autogobble]{python}
            c = a
            a = b
            b = c
            c = a + b
            \end{minted}
        
\item 
                \begin{minted}[autogobble]{python}
                a = b
                c = b
                b = a
                c = a + b
                \end{minted}
            
\item 
                \begin{minted}[autogobble]{python}
                a = b
                b = c
                c = a
                c = a + b
                \end{minted}
            
\end{enumerate}

        \end{multicols}
    
        \naloga*
        \begin{multicols}{2}
        \noindent
        Kakšne vrste napak vsebuje program na desni?

        \begin{enumerate}[(a)]
\item sintaktične napake, zaradi katerih Python programa noče izvesti.
\item napake, zaradi katerih Python prekine z izvajanjem programa
\item vsebinske napake, zaradi katerih Python izračuna napačen rezultat
\item oblikovne napake, ki ne vplivajo na pravilnost rezultata
\end{enumerate}

        \columnbreak

        \begin{minted}[baselinestretch=1.2,escapeinside=||, autogobble]{python}
        
            def fibonacci(n):
                if n <= 0:
                    return 0
                elif n == 1:
                    return 0
                else:
                    a = fibonacci(n - 1)
                    b = fibonacci(n - 3)
                    return a * b
            
        \end{minted}

        \end{multicols}

    
        \naloga*
        Katere vrstice izpiše klic \inlinepy{print(f(g(5)))}, če sta funkciji \inlinepy{f} in \inlinepy{g} definirani kot spodaj?

        \begin{multicols}{2}
        \begin{minted}[autogobble]{python}
        
            def f(x):
                return x + 8
                print(x)

            def g(y):
                print(y)
                return 9 * y
        
        \end{minted}

        \begin{enumerate}[(a)]
\item \inlinepy{[45, 53]}
\item \inlinepy{[5, 45, 53]}
\item \inlinepy{[53]}
\item \inlinepy{[5, 53]}
\end{enumerate}

        \end{multicols}
    
        \naloga*

        \begin{multicols}{2}
        \noindent
        Kateri pogoj preverja spodnja funkcija?
        \begin{minted}[autogobble]{python}
        
            def f(besedilo):
                for znak in besedilo:
                    if znak not in 'aeiouAEIOU':
                        return False
                return True
            
        \end{minted}

        \begin{enumerate}[(a)]
\item ali niz vsebuje znak, ki ni samoglasnik
\item ali niz ne vsebuje nobenega samoglasnika
\item ali niz vsebuje samo samoglasnike
\item ali niz vsebuje kakšen samoglasnik
\end{enumerate}

        \end{multicols}
    
        \naloga*
        \begin{multicols}{2}
        \noindent
        V vsak prostor vpišite \textbf{natanko en znak} tako, da bo dobljeni program v spremenljivko \inlinepy{zm} shranil zmnožek števil \inlinepy{p} in \inlinepy{q}:
        
        \columnbreak
        \begin{minted}[baselinestretch=1.2,escapeinside=||]{python}
        zm = |\answerbox{0.5}|
        while p > |\answerbox{0.5}|:
            zm += |\answerbox{0.5}|
            |\answerbox{0.5}| -= 1
        \end{minted}
        \end{multicols}
    
        \clearpage
        \naloga
        
        Katera izmed spodnjih funkcij izračuna ostanek pri deljenju naravnega števila \inlinepy{m} z naravnim številom \inlinepy{n}?
    
        \begin{multicols}{2}
        \begin{enumerate}[(a)]
\item 
            \begin{minted}[autogobble]{python}
            def ostanek(m, n):
                if m < n:
                    return m
                else:
                    return ostanek(m - n, n)
            \end{minted}
        
\item 
                \begin{minted}[autogobble]{python}
                def ostanek(m, n):
                    if m < n:
                        return m
                    else:
                        return ostanek(m % n)
                \end{minted}
            
\item 
                \begin{minted}[autogobble]{python}
                def ostanek(m, n):
                    if n == 0:
                        return m
                    else:
                        return ostanek(n, m % n)
                \end{minted}
            
\item 
                \begin{minted}[autogobble]{python}
                def ostanek(m, n):
                    if m < n:
                        return 0
                    else:
                        return ostanek(m - n, n)
                \end{minted}
            
\end{enumerate}

        \end{multicols}
    
        \naloga*
        
        Katera izmed funkcij vrača drugačne rezultate kot ostale?
    
        \begin{multicols}{2}
        \begin{enumerate}[(a)]
\item 
            \begin{minted}[autogobble]{python}
            def g(p, r, q):
                if not q:
                    return p and r
                else:
                    return False
            \end{minted}
        
\item 
                \begin{minted}[autogobble]{python}
                def g(p, r, q):
                    if not r:
                        return False
                    else:
                        return p and q
                \end{minted}
            
\item 
                \begin{minted}[autogobble]{python}
                def g(p, r, q):
                    if p:
                        return r and q
                    else:
                        return False
                \end{minted}
            
\item 
                \begin{minted}[autogobble]{python}
                def g(p, r, q):
                    if p and r:
                        return q
                    else:
                        return False
                \end{minted}
            
\end{enumerate}

        \end{multicols}
    
        \naloga*
        
        Kateri izmed spodnjih programov ima drugačen izpis kot ostali?
    
        \begin{multicols}{2}
        \begin{enumerate}[(a)]
\item 
            \begin{minted}[autogobble]{python}
                for i in range(1, 20):
                    if i % 5 != 0:
                        continue
                    print(i)
            \end{minted}
        
\item 
                \begin{minted}[autogobble]{python}
                    for i in range(1, 100, 5):
                        if i > 20:
                            break
                        print(i)
                \end{minted}
            
\item 
                \begin{minted}[autogobble]{python}
                    for i in range(1, 20):
                        if i % 5 == 1:
                            print(i)
                        continue
                \end{minted}
            
\item 
                \begin{minted}[autogobble]{python}
                    for i in range(1, 100, 5):
                        if i < 20:
                            print(i)
                \end{minted}
            
\end{enumerate}

        \end{multicols}
    
        \naloga*
        \begin{multicols}{2}
        \noindent
        Napišite primer vrednosti spremenljivk \inlinepy{str1} in \inlinepy{str2}, za kateri klic \inlinepy{h(str1, str2)} vrne \inlinepy{True}.
        \begin{minted}[baselinestretch=1.2, escapeinside=||]{python}
        str1 = |\answerbox{3}|
        str2 = |\answerbox{3}|
        \end{minted}
        \vfil
        \columnbreak
        \begin{minted}[autogobble]{python}
        def h(str1, str2):
            if len(str1) != len(str2) or len(str1) < 3:
                return False
            for i in range(len(str1)):
                if i % 2 == 1 and str1[i] != str2[i]:
                    return False
            return True
        \end{minted}
        \end{multicols}
    
        \naloga*
        \begin{multicols}{2}
        \noindent 
        S številkami od $0$ do $4$ označite vrstni red, v katerem moramo izvesti ukaze na desni, da bo na koncu spremenljivka \inlinepy{sez1} kazala na seznam \inlinepy{[6, 8, 9]}?
    
        \columnbreak
        \noindent
        \begin{minted}[baselinestretch=1.2, escapeinside=||]{python}
|\answerbox{0.5}| sez1 = sez2 + sez1
|\answerbox{0.5}| sez1 = sez2
|\answerbox{0.5}| sez2 = [8]
|\answerbox{0.5}| sez1.append(9)
|\answerbox{0.5}| sez2 = [6]

        \end{minted}
        \end{multicols}
    
            \izpit[ucilnica=205, naloge=-1]{Uvod v programiranje: Kolokvij \#039}{27.\ marec 2019}{
                Pri vsaki nalogi obkrožite črko pred pravilnim odgovorom ali vpišite pravilno vrednost v ustrezen prostor. \\
                Čas reševanja je 30 minut. Veliko uspeha!
            }
            
        \naloga*
        
        Kateri izmed programov pri začetnem stanju
            \inlinepy{k = 9} in
            \inlinepy{m = 2}
        nastavi vrednosti
            \inlinepy{k = 2},
            \inlinepy{m = 9} in
            \inlinepy{n = 11}?
    
        \begin{multicols}{4}
        \begin{enumerate}[(a)]
\item 
                \begin{minted}[autogobble]{python}
                k = m
                n = m
                m = k
                n = k + m
                \end{minted}
            
\item 
            \begin{minted}[autogobble]{python}
            n = k
            k = m
            m = n
            n = k + m
            \end{minted}
        
\item 
                \begin{minted}[autogobble]{python}
                n = k
                m = n
                k = m
                n = k + m
                \end{minted}
            
\item 
                \begin{minted}[autogobble]{python}
                k = m
                m = n
                n = k
                n = k + m
                \end{minted}
            
\end{enumerate}

        \end{multicols}
    
        \naloga*
        \begin{multicols}{2}
        \noindent
        Kakšne vrste napak vsebuje program na desni?

        \begin{enumerate}[(a)]
\item napake, zaradi katerih Python prekine z izvajanjem programa
\item vsebinske napake, zaradi katerih Python izračuna napačen rezultat
\item oblikovne napake, ki ne vplivajo na pravilnost rezultata
\item sintaktične napake, zaradi katerih Python programa noče izvesti.
\end{enumerate}

        \columnbreak

        \begin{minted}[baselinestretch=1.2,escapeinside=||, autogobble]{python}
        
            def fibonacci(n):
                if n <= 0:
                    return 0
                elif n == 1:
                    return 1
                else:
                    a = fibonacci(n - '1')
                    b = fib(n - 2)
                    return a + b
            
        \end{minted}

        \end{multicols}

    
        \naloga*
        Katere vrstice izpiše klic \inlinepy{print(f(g(6)))}, če sta funkciji \inlinepy{f} in \inlinepy{g} definirani kot spodaj?

        \begin{multicols}{2}
        \begin{minted}[autogobble]{python}
        
            def f(a):
                return a + 8
                print(a)

            def g(b):
                return 4 * b
                print(b)
        
        \end{minted}

        \begin{enumerate}[(a)]
\item \inlinepy{[24, 32]}
\item \inlinepy{[32]}
\item \inlinepy{[6, 24, 32]}
\item \inlinepy{[6, 32]}
\end{enumerate}

        \end{multicols}
    
        \naloga*

        \begin{multicols}{2}
        \noindent
        Kateri pogoj preverja spodnja funkcija?
        \begin{minted}[autogobble]{python}
        
            def f(stavek):
                for z in stavek:
                    if z not in 'aeiouAEIOU':
                        return True
                return False
            
        \end{minted}

        \begin{enumerate}[(a)]
\item ali niz ne vsebuje nobenega samoglasnika
\item ali niz vsebuje kakšen samoglasnik
\item ali niz vsebuje znak, ki ni samoglasnik
\item ali niz vsebuje samo samoglasnike
\end{enumerate}

        \end{multicols}
    
        \naloga*
        \begin{multicols}{2}
        \noindent
        V vsak prostor vpišite \textbf{natanko en znak} tako, da bo dobljeni program v spremenljivko \inlinepy{vs} shranil vsoto števil \inlinepy{a} in \inlinepy{b}:
        
        \columnbreak
        \begin{minted}[baselinestretch=1.2,escapeinside=||]{python}
        vs = |\answerbox{0.5}|
        while a > |\answerbox{0.5}|:
            vs += |\answerbox{0.5}|
            |\answerbox{0.5}| -= 1
        \end{minted}
        \end{multicols}
    
        \clearpage
        \naloga
        
        Katera izmed spodnjih funkcij izračuna ostanek pri deljenju naravnega števila \inlinepy{m} z naravnim številom \inlinepy{n}?
    
        \begin{multicols}{2}
        \begin{enumerate}[(a)]
\item 
            \begin{minted}[autogobble]{python}
            def ostanek(m, n):
                if m < n:
                    return m
                else:
                    return ostanek(m - n, n)
            \end{minted}
        
\item 
                \begin{minted}[autogobble]{python}
                def ostanek(m, n):
                    if m < n:
                        return 0
                    else:
                        return ostanek(m - n, n)
                \end{minted}
            
\item 
                \begin{minted}[autogobble]{python}
                def ostanek(m, n):
                    if m < n:
                        return m
                    else:
                        return ostanek(m % n)
                \end{minted}
            
\item 
                \begin{minted}[autogobble]{python}
                def ostanek(m, n):
                    if n == 0:
                        return m
                    else:
                        return ostanek(n, m % n)
                \end{minted}
            
\end{enumerate}

        \end{multicols}
    
        \naloga*
        
        Katera izmed funkcij vrača drugačne rezultate kot ostale?
    
        \begin{multicols}{2}
        \begin{enumerate}[(a)]
\item 
                \begin{minted}[autogobble]{python}
                def h(y, z, x):
                    if y and z:
                        return x
                    else:
                        return False
                \end{minted}
            
\item 
            \begin{minted}[autogobble]{python}
            def h(y, z, x):
                if not x:
                    return y and z
                else:
                    return False
            \end{minted}
        
\item 
                \begin{minted}[autogobble]{python}
                def h(y, z, x):
                    if not z:
                        return False
                    else:
                        return y and x
                \end{minted}
            
\item 
                \begin{minted}[autogobble]{python}
                def h(y, z, x):
                    if y:
                        return z and x
                    else:
                        return False
                \end{minted}
            
\end{enumerate}

        \end{multicols}
    
        \naloga*
        
        Kateri izmed spodnjih programov ima drugačen izpis kot ostali?
    
        \begin{multicols}{2}
        \begin{enumerate}[(a)]
\item 
            \begin{minted}[autogobble]{python}
                for x in range(1, 20):
                    if x % 5 != 0:
                        continue
                    print(x)
            \end{minted}
        
\item 
                \begin{minted}[autogobble]{python}
                    for x in range(1, 100, 5):
                        if x < 20:
                            print(x)
                \end{minted}
            
\item 
                \begin{minted}[autogobble]{python}
                    for x in range(1, 100, 5):
                        if x > 20:
                            break
                        print(x)
                \end{minted}
            
\item 
                \begin{minted}[autogobble]{python}
                    for x in range(1, 20):
                        if x % 5 == 1:
                            print(x)
                        continue
                \end{minted}
            
\end{enumerate}

        \end{multicols}
    
        \naloga*
        \begin{multicols}{2}
        \noindent
        Napišite primer vrednosti spremenljivk \inlinepy{lst1} in \inlinepy{lst2}, za kateri klic \inlinepy{h(lst1, lst2)} vrne \inlinepy{True}.
        \begin{minted}[baselinestretch=1.2, escapeinside=||]{python}
        lst1 = |\answerbox{3}|
        lst2 = |\answerbox{3}|
        \end{minted}
        \vfil
        \columnbreak
        \begin{minted}[autogobble]{python}
        def h(lst1, lst2):
            if len(lst1) != len(lst2) or len(lst1) < 3:
                return False
            for i in range(len(lst1)):
                if i % 2 == 0 and lst1[i] != lst2[i]:
                    return False
            return True
        \end{minted}
        \end{multicols}
    
        \naloga*
        \begin{multicols}{2}
        \noindent 
        S številkami od $0$ do $4$ označite vrstni red, v katerem moramo izvesti ukaze na desni, da bo na koncu spremenljivka \inlinepy{sez1} kazala na seznam \inlinepy{[7, 2, 4]}?
    
        \columnbreak
        \noindent
        \begin{minted}[baselinestretch=1.2, escapeinside=||]{python}
|\answerbox{0.5}| sez2 = [2]
|\answerbox{0.5}| sez2 = [7]
|\answerbox{0.5}| sez1 = sez2 + sez1
|\answerbox{0.5}| sez1.append(4)
|\answerbox{0.5}| sez1 = sez2

        \end{minted}
        \end{multicols}
    
            \izpit[ucilnica=205, naloge=-1]{Uvod v programiranje: Kolokvij \#040}{27.\ marec 2019}{
                Pri vsaki nalogi obkrožite črko pred pravilnim odgovorom ali vpišite pravilno vrednost v ustrezen prostor. \\
                Čas reševanja je 30 minut. Veliko uspeha!
            }
            
        \naloga*
        
        Kateri izmed programov pri začetnem stanju
            \inlinepy{u = 5} in
            \inlinepy{v = 3}
        nastavi vrednosti
            \inlinepy{u = 3},
            \inlinepy{v = 5} in
            \inlinepy{w = 8}?
    
        \begin{multicols}{4}
        \begin{enumerate}[(a)]
\item 
            \begin{minted}[autogobble]{python}
            w = u
            u = v
            v = w
            w = u + v
            \end{minted}
        
\item 
                \begin{minted}[autogobble]{python}
                w = u
                v = w
                u = v
                w = u + v
                \end{minted}
            
\item 
                \begin{minted}[autogobble]{python}
                u = v
                w = v
                v = u
                w = u + v
                \end{minted}
            
\item 
                \begin{minted}[autogobble]{python}
                u = v
                v = w
                w = u
                w = u + v
                \end{minted}
            
\end{enumerate}

        \end{multicols}
    
        \naloga*
        \begin{multicols}{2}
        \noindent
        Kakšne vrste napak vsebuje program na desni?

        \begin{enumerate}[(a)]
\item vsebinske napake, zaradi katerih Python izračuna napačen rezultat
\item napake, zaradi katerih Python prekine z izvajanjem programa
\item sintaktične napake, zaradi katerih Python programa noče izvesti.
\item oblikovne napake, ki ne vplivajo na pravilnost rezultata
\end{enumerate}

        \columnbreak

        \begin{minted}[baselinestretch=1.2,escapeinside=||, autogobble]{python}
        
            def fibonacci(n):
                if n<=0:
                    return (0)
                elif n==1:
                    return (1)
                else:
                    a = fibonacci(n - 1)
                    b = fibonacci(n - 2)
                    return   a + b
            
        \end{minted}

        \end{multicols}

    
        \naloga*
        Katere vrstice izpiše klic \inlinepy{print(p(q(2)))}, če sta funkciji \inlinepy{p} in \inlinepy{q} definirani kot spodaj?

        \begin{multicols}{2}
        \begin{minted}[autogobble]{python}
        
            def p(a):
                return a + 4
                print(a)

            def q(b):
                print(b)
                return 5 * b
        
        \end{minted}

        \begin{enumerate}[(a)]
\item \inlinepy{[14]}
\item \inlinepy{[2, 10, 14]}
\item \inlinepy{[10, 14]}
\item \inlinepy{[2, 14]}
\end{enumerate}

        \end{multicols}
    
        \naloga*

        \begin{multicols}{2}
        \noindent
        Kateri pogoj preverja spodnja funkcija?
        \begin{minted}[autogobble]{python}
        
            def f(niz):
                for znak in niz:
                    if znak not in 'aeiouAEIOU':
                        return True
                return False
            
        \end{minted}

        \begin{enumerate}[(a)]
\item ali niz ne vsebuje nobenega samoglasnika
\item ali niz vsebuje kakšen samoglasnik
\item ali niz vsebuje znak, ki ni samoglasnik
\item ali niz vsebuje samo samoglasnike
\end{enumerate}

        \end{multicols}
    
        \naloga*
        \begin{multicols}{2}
        \noindent
        V vsak prostor vpišite \textbf{natanko en znak} tako, da bo dobljeni program v spremenljivko \inlinepy{vs} shranil vsoto števil \inlinepy{a} in \inlinepy{b}:
        
        \columnbreak
        \begin{minted}[baselinestretch=1.2,escapeinside=||]{python}
        vs = |\answerbox{0.5}|
        while a > |\answerbox{0.5}|:
            vs += |\answerbox{0.5}|
            |\answerbox{0.5}| -= 1
        \end{minted}
        \end{multicols}
    
        \clearpage
        \naloga
        
        Katera izmed spodnjih funkcij izračuna ostanek pri deljenju naravnega števila \inlinepy{a} z naravnim številom \inlinepy{b}?
    
        \begin{multicols}{2}
        \begin{enumerate}[(a)]
\item 
            \begin{minted}[autogobble]{python}
            def ostanek(a, b):
                if a < b:
                    return a
                else:
                    return ostanek(a - b, b)
            \end{minted}
        
\item 
                \begin{minted}[autogobble]{python}
                def ostanek(a, b):
                    if a < b:
                        return 0
                    else:
                        return ostanek(a - b, b)
                \end{minted}
            
\item 
                \begin{minted}[autogobble]{python}
                def ostanek(a, b):
                    if a < b:
                        return a
                    else:
                        return ostanek(a % b)
                \end{minted}
            
\item 
                \begin{minted}[autogobble]{python}
                def ostanek(a, b):
                    if b == 0:
                        return a
                    else:
                        return ostanek(b, a % b)
                \end{minted}
            
\end{enumerate}

        \end{multicols}
    
        \naloga*
        
        Katera izmed funkcij vrača drugačne rezultate kot ostale?
    
        \begin{multicols}{2}
        \begin{enumerate}[(a)]
\item 
                \begin{minted}[autogobble]{python}
                def f(b, a, c):
                    if b:
                        return a and c
                    else:
                        return False
                \end{minted}
            
\item 
            \begin{minted}[autogobble]{python}
            def f(b, a, c):
                if not c:
                    return b and a
                else:
                    return False
            \end{minted}
        
\item 
                \begin{minted}[autogobble]{python}
                def f(b, a, c):
                    if b and a:
                        return c
                    else:
                        return False
                \end{minted}
            
\item 
                \begin{minted}[autogobble]{python}
                def f(b, a, c):
                    if not a:
                        return False
                    else:
                        return b and c
                \end{minted}
            
\end{enumerate}

        \end{multicols}
    
        \naloga*
        
        Kateri izmed spodnjih programov ima drugačen izpis kot ostali?
    
        \begin{multicols}{2}
        \begin{enumerate}[(a)]
\item 
                \begin{minted}[autogobble]{python}
                    for x in range(1, 40, 2):
                        if x > 20:
                            break
                        print(x)
                \end{minted}
            
\item 
            \begin{minted}[autogobble]{python}
                for x in range(1, 20):
                    if x % 2 != 0:
                        continue
                    print(x)
            \end{minted}
        
\item 
                \begin{minted}[autogobble]{python}
                    for x in range(1, 40, 2):
                        if x < 20:
                            print(x)
                \end{minted}
            
\item 
                \begin{minted}[autogobble]{python}
                    for x in range(1, 20):
                        if x % 2 == 1:
                            print(x)
                        continue
                \end{minted}
            
\end{enumerate}

        \end{multicols}
    
        \naloga*
        \begin{multicols}{2}
        \noindent
        Napišite primer vrednosti spremenljivk \inlinepy{sez1} in \inlinepy{sez2}, za kateri klic \inlinepy{f(sez1, sez2)} vrne \inlinepy{True}.
        \begin{minted}[baselinestretch=1.2, escapeinside=||]{python}
        sez1 = |\answerbox{3}|
        sez2 = |\answerbox{3}|
        \end{minted}
        \vfil
        \columnbreak
        \begin{minted}[autogobble]{python}
        def f(sez1, sez2):
            if len(sez1) != len(sez2) or len(sez1) < 3:
                return False
            for j in range(len(sez1)):
                if j % 2 == 1 and sez1[j] != sez2[j]:
                    return False
            return True
        \end{minted}
        \end{multicols}
    
        \naloga*
        \begin{multicols}{2}
        \noindent 
        S številkami od $0$ do $4$ označite vrstni red, v katerem moramo izvesti ukaze na desni, da bo na koncu spremenljivka \inlinepy{lst2} kazala na seznam \inlinepy{[4, 5, 7]}?
    
        \columnbreak
        \noindent
        \begin{minted}[baselinestretch=1.2, escapeinside=||]{python}
|\answerbox{0.5}| lst2 = lst1
|\answerbox{0.5}| lst2.append(7)
|\answerbox{0.5}| lst1 = [5]
|\answerbox{0.5}| lst1 = [4]
|\answerbox{0.5}| lst2 = lst1 + lst2

        \end{minted}
        \end{multicols}
    
            \izpit[ucilnica=205, naloge=-1]{Uvod v programiranje: Kolokvij \#041}{27.\ marec 2019}{
                Pri vsaki nalogi obkrožite črko pred pravilnim odgovorom ali vpišite pravilno vrednost v ustrezen prostor. \\
                Čas reševanja je 30 minut. Veliko uspeha!
            }
            
        \naloga*
        
        Kateri izmed programov pri začetnem stanju
            \inlinepy{x = 3} in
            \inlinepy{y = 8}
        nastavi vrednosti
            \inlinepy{x = 8},
            \inlinepy{y = 3} in
            \inlinepy{z = 11}?
    
        \begin{multicols}{4}
        \begin{enumerate}[(a)]
\item 
                \begin{minted}[autogobble]{python}
                x = y
                z = y
                y = x
                z = x + y
                \end{minted}
            
\item 
            \begin{minted}[autogobble]{python}
            z = x
            x = y
            y = z
            z = x + y
            \end{minted}
        
\item 
                \begin{minted}[autogobble]{python}
                x = y
                y = z
                z = x
                z = x + y
                \end{minted}
            
\item 
                \begin{minted}[autogobble]{python}
                z = x
                y = z
                x = y
                z = x + y
                \end{minted}
            
\end{enumerate}

        \end{multicols}
    
        \naloga*
        \begin{multicols}{2}
        \noindent
        Kakšne vrste napak vsebuje program na desni?

        \begin{enumerate}[(a)]
\item oblikovne napake, ki ne vplivajo na pravilnost rezultata
\item napake, zaradi katerih Python prekine z izvajanjem programa
\item sintaktične napake, zaradi katerih Python programa noče izvesti.
\item vsebinske napake, zaradi katerih Python izračuna napačen rezultat
\end{enumerate}

        \columnbreak

        \begin{minted}[baselinestretch=1.2,escapeinside=||, autogobble]{python}
        
            def fibonacci(n):
                if n <= 0:
                    return 0
                elif n == 1:
                    return 1
                else:
                    a = fibonacci(n - '1')
                    b = fib(n - 2)
                    return a + b
            
        \end{minted}

        \end{multicols}

    
        \naloga*
        Katere vrstice izpiše klic \inlinepy{print(f(g(4)))}, če sta funkciji \inlinepy{f} in \inlinepy{g} definirani kot spodaj?

        \begin{multicols}{2}
        \begin{minted}[autogobble]{python}
        
            def f(x):
                return x + 7
                print(x)

            def g(y):
                return 1 * y
                print(y)
        
        \end{minted}

        \begin{enumerate}[(a)]
\item \inlinepy{[4, 11]}
\item \inlinepy{[11]}
\item \inlinepy{[4, 11]}
\item \inlinepy{[4, 4, 11]}
\end{enumerate}

        \end{multicols}
    
        \naloga*

        \begin{multicols}{2}
        \noindent
        Kateri pogoj preverja spodnja funkcija?
        \begin{minted}[autogobble]{python}
        
            def f(niz):
                for x in niz:
                    if x in 'aeiouAEIOU':
                        return False
                return True
            
        \end{minted}

        \begin{enumerate}[(a)]
\item ali niz vsebuje kakšen samoglasnik
\item ali niz vsebuje samo samoglasnike
\item ali niz vsebuje znak, ki ni samoglasnik
\item ali niz ne vsebuje nobenega samoglasnika
\end{enumerate}

        \end{multicols}
    
        \naloga*
        \begin{multicols}{2}
        \noindent
        V vsak prostor vpišite \textbf{natanko en znak} tako, da bo dobljeni program v spremenljivko \inlinepy{zm} shranil zmnožek števil \inlinepy{b} in \inlinepy{a}:
        
        \columnbreak
        \begin{minted}[baselinestretch=1.2,escapeinside=||]{python}
        zm = |\answerbox{0.5}|
        while b > |\answerbox{0.5}|:
            zm += |\answerbox{0.5}|
            |\answerbox{0.5}| -= 1
        \end{minted}
        \end{multicols}
    
        \clearpage
        \naloga
        
        Katera izmed spodnjih funkcij izračuna ostanek pri deljenju naravnega števila \inlinepy{u} z naravnim številom \inlinepy{v}?
    
        \begin{multicols}{2}
        \begin{enumerate}[(a)]
\item 
                \begin{minted}[autogobble]{python}
                def ostanek(u, v):
                    if u < v:
                        return u
                    else:
                        return ostanek(u % v)
                \end{minted}
            
\item 
                \begin{minted}[autogobble]{python}
                def ostanek(u, v):
                    if v == 0:
                        return u
                    else:
                        return ostanek(v, u % v)
                \end{minted}
            
\item 
                \begin{minted}[autogobble]{python}
                def ostanek(u, v):
                    if u < v:
                        return 0
                    else:
                        return ostanek(u - v, v)
                \end{minted}
            
\item 
            \begin{minted}[autogobble]{python}
            def ostanek(u, v):
                if u < v:
                    return u
                else:
                    return ostanek(u - v, v)
            \end{minted}
        
\end{enumerate}

        \end{multicols}
    
        \naloga*
        
        Katera izmed funkcij vrača drugačne rezultate kot ostale?
    
        \begin{multicols}{2}
        \begin{enumerate}[(a)]
\item 
                \begin{minted}[autogobble]{python}
                def f(p, r, q):
                    if not r:
                        return False
                    else:
                        return p and q
                \end{minted}
            
\item 
                \begin{minted}[autogobble]{python}
                def f(p, r, q):
                    if p and r:
                        return q
                    else:
                        return False
                \end{minted}
            
\item 
                \begin{minted}[autogobble]{python}
                def f(p, r, q):
                    if p:
                        return r and q
                    else:
                        return False
                \end{minted}
            
\item 
            \begin{minted}[autogobble]{python}
            def f(p, r, q):
                if not q:
                    return p and r
                else:
                    return False
            \end{minted}
        
\end{enumerate}

        \end{multicols}
    
        \naloga*
        
        Kateri izmed spodnjih programov ima drugačen izpis kot ostali?
    
        \begin{multicols}{2}
        \begin{enumerate}[(a)]
\item 
            \begin{minted}[autogobble]{python}
                for j in range(1, 20):
                    if j % 5 != 0:
                        continue
                    print(j)
            \end{minted}
        
\item 
                \begin{minted}[autogobble]{python}
                    for j in range(1, 20):
                        if j % 5 == 1:
                            print(j)
                        continue
                \end{minted}
            
\item 
                \begin{minted}[autogobble]{python}
                    for j in range(1, 100, 5):
                        if j < 20:
                            print(j)
                \end{minted}
            
\item 
                \begin{minted}[autogobble]{python}
                    for j in range(1, 100, 5):
                        if j > 20:
                            break
                        print(j)
                \end{minted}
            
\end{enumerate}

        \end{multicols}
    
        \naloga*
        \begin{multicols}{2}
        \noindent
        Napišite primer vrednosti spremenljivk \inlinepy{str1} in \inlinepy{str2}, za kateri klic \inlinepy{g(str1, str2)} vrne \inlinepy{True}.
        \begin{minted}[baselinestretch=1.2, escapeinside=||]{python}
        str1 = |\answerbox{3}|
        str2 = |\answerbox{3}|
        \end{minted}
        \vfil
        \columnbreak
        \begin{minted}[autogobble]{python}
        def g(str1, str2):
            if len(str1) != len(str2) or len(str1) < 3:
                return False
            for i in range(len(str1)):
                if i % 2 == 1 and str1[i] != str2[i]:
                    return False
            return True
        \end{minted}
        \end{multicols}
    
        \naloga*
        \begin{multicols}{2}
        \noindent 
        S številkami od $0$ do $4$ označite vrstni red, v katerem moramo izvesti ukaze na desni, da bo na koncu spremenljivka \inlinepy{lst1} kazala na seznam \inlinepy{[8, 1, 9]}?
    
        \columnbreak
        \noindent
        \begin{minted}[baselinestretch=1.2, escapeinside=||]{python}
|\answerbox{0.5}| lst2 = [8]
|\answerbox{0.5}| lst2 = [1]
|\answerbox{0.5}| lst1 = lst2 + lst1
|\answerbox{0.5}| lst1 = lst2
|\answerbox{0.5}| lst1.append(9)

        \end{minted}
        \end{multicols}
    
            \izpit[ucilnica=205, naloge=-1]{Uvod v programiranje: Kolokvij \#042}{27.\ marec 2019}{
                Pri vsaki nalogi obkrožite črko pred pravilnim odgovorom ali vpišite pravilno vrednost v ustrezen prostor. \\
                Čas reševanja je 30 minut. Veliko uspeha!
            }
            
        \naloga*
        
        Kateri izmed programov pri začetnem stanju
            \inlinepy{a = 4} in
            \inlinepy{b = 3}
        nastavi vrednosti
            \inlinepy{a = 3},
            \inlinepy{b = 4} in
            \inlinepy{c = 7}?
    
        \begin{multicols}{4}
        \begin{enumerate}[(a)]
\item 
                \begin{minted}[autogobble]{python}
                a = b
                c = b
                b = a
                c = a + b
                \end{minted}
            
\item 
                \begin{minted}[autogobble]{python}
                c = a
                b = c
                a = b
                c = a + b
                \end{minted}
            
\item 
            \begin{minted}[autogobble]{python}
            c = a
            a = b
            b = c
            c = a + b
            \end{minted}
        
\item 
                \begin{minted}[autogobble]{python}
                a = b
                b = c
                c = a
                c = a + b
                \end{minted}
            
\end{enumerate}

        \end{multicols}
    
        \naloga*
        \begin{multicols}{2}
        \noindent
        Kakšne vrste napak vsebuje program na desni?

        \begin{enumerate}[(a)]
\item vsebinske napake, zaradi katerih Python izračuna napačen rezultat
\item napake, zaradi katerih Python prekine z izvajanjem programa
\item sintaktične napake, zaradi katerih Python programa noče izvesti.
\item oblikovne napake, ki ne vplivajo na pravilnost rezultata
\end{enumerate}

        \columnbreak

        \begin{minted}[baselinestretch=1.2,escapeinside=||, autogobble]{python}
        
            def fibonacci(n):
                if n <= 0:
                    return 0
                elif n == 1:
                    return 1
                else:
                    a = fibonacci(n - '1')
                    b = fib(n - 2)
                    return a + b
            
        \end{minted}

        \end{multicols}

    
        \naloga*
        Katere vrstice izpiše klic \inlinepy{print(f(g(4)))}, če sta funkciji \inlinepy{f} in \inlinepy{g} definirani kot spodaj?

        \begin{multicols}{2}
        \begin{minted}[autogobble]{python}
        
            def f(a):
                print(a)
                return a + 9

            def g(b):
                return 8 * b
                print(b)
        
        \end{minted}

        \begin{enumerate}[(a)]
\item \inlinepy{[4, 41]}
\item \inlinepy{[41]}
\item \inlinepy{[32, 41]}
\item \inlinepy{[4, 32, 41]}
\end{enumerate}

        \end{multicols}
    
        \naloga*

        \begin{multicols}{2}
        \noindent
        Kateri pogoj preverja spodnja funkcija?
        \begin{minted}[autogobble]{python}
        
            def f(stavek):
                for znak in stavek:
                    if znak in 'aeiouAEIOU':
                        return True
                return False
            
        \end{minted}

        \begin{enumerate}[(a)]
\item ali niz vsebuje znak, ki ni samoglasnik
\item ali niz vsebuje samo samoglasnike
\item ali niz vsebuje kakšen samoglasnik
\item ali niz ne vsebuje nobenega samoglasnika
\end{enumerate}

        \end{multicols}
    
        \naloga*
        \begin{multicols}{2}
        \noindent
        V vsak prostor vpišite \textbf{natanko en znak} tako, da bo dobljeni program v spremenljivko \inlinepy{zm} shranil zmnožek števil \inlinepy{p} in \inlinepy{q}:
        
        \columnbreak
        \begin{minted}[baselinestretch=1.2,escapeinside=||]{python}
        zm = |\answerbox{0.5}|
        while p > |\answerbox{0.5}|:
            zm += |\answerbox{0.5}|
            |\answerbox{0.5}| -= 1
        \end{minted}
        \end{multicols}
    
        \clearpage
        \naloga
        
        Katera izmed spodnjih funkcij izračuna ostanek pri deljenju naravnega števila \inlinepy{u} z naravnim številom \inlinepy{v}?
    
        \begin{multicols}{2}
        \begin{enumerate}[(a)]
\item 
                \begin{minted}[autogobble]{python}
                def ostanek(u, v):
                    if u < v:
                        return u
                    else:
                        return ostanek(u % v)
                \end{minted}
            
\item 
                \begin{minted}[autogobble]{python}
                def ostanek(u, v):
                    if u < v:
                        return 0
                    else:
                        return ostanek(u - v, v)
                \end{minted}
            
\item 
                \begin{minted}[autogobble]{python}
                def ostanek(u, v):
                    if v == 0:
                        return u
                    else:
                        return ostanek(v, u % v)
                \end{minted}
            
\item 
            \begin{minted}[autogobble]{python}
            def ostanek(u, v):
                if u < v:
                    return u
                else:
                    return ostanek(u - v, v)
            \end{minted}
        
\end{enumerate}

        \end{multicols}
    
        \naloga*
        
        Katera izmed funkcij vrača drugačne rezultate kot ostale?
    
        \begin{multicols}{2}
        \begin{enumerate}[(a)]
\item 
            \begin{minted}[autogobble]{python}
            def f(r, q, p):
                if not p:
                    return r and q
                else:
                    return False
            \end{minted}
        
\item 
                \begin{minted}[autogobble]{python}
                def f(r, q, p):
                    if r and q:
                        return p
                    else:
                        return False
                \end{minted}
            
\item 
                \begin{minted}[autogobble]{python}
                def f(r, q, p):
                    if r:
                        return q and p
                    else:
                        return False
                \end{minted}
            
\item 
                \begin{minted}[autogobble]{python}
                def f(r, q, p):
                    if not q:
                        return False
                    else:
                        return r and p
                \end{minted}
            
\end{enumerate}

        \end{multicols}
    
        \naloga*
        
        Kateri izmed spodnjih programov ima drugačen izpis kot ostali?
    
        \begin{multicols}{2}
        \begin{enumerate}[(a)]
\item 
                \begin{minted}[autogobble]{python}
                    for x in range(1, 10):
                        if x % 3 == 1:
                            print(x)
                        continue
                \end{minted}
            
\item 
            \begin{minted}[autogobble]{python}
                for x in range(1, 10):
                    if x % 3 != 0:
                        continue
                    print(x)
            \end{minted}
        
\item 
                \begin{minted}[autogobble]{python}
                    for x in range(1, 30, 3):
                        if x > 10:
                            break
                        print(x)
                \end{minted}
            
\item 
                \begin{minted}[autogobble]{python}
                    for x in range(1, 30, 3):
                        if x < 10:
                            print(x)
                \end{minted}
            
\end{enumerate}

        \end{multicols}
    
        \naloga*
        \begin{multicols}{2}
        \noindent
        Napišite primer vrednosti spremenljivk \inlinepy{str1} in \inlinepy{str2}, za kateri klic \inlinepy{g(str1, str2)} vrne \inlinepy{True}.
        \begin{minted}[baselinestretch=1.2, escapeinside=||]{python}
        str1 = |\answerbox{3}|
        str2 = |\answerbox{3}|
        \end{minted}
        \vfil
        \columnbreak
        \begin{minted}[autogobble]{python}
        def g(str1, str2):
            if len(str1) != len(str2) or len(str1) < 3:
                return False
            for i in range(len(str1)):
                if i % 2 == 0 and str1[i] != str2[i]:
                    return False
            return True
        \end{minted}
        \end{multicols}
    
        \naloga*
        \begin{multicols}{2}
        \noindent 
        S številkami od $0$ do $4$ označite vrstni red, v katerem moramo izvesti ukaze na desni, da bo na koncu spremenljivka \inlinepy{lst1} kazala na seznam \inlinepy{[7, 9, 4]}?
    
        \columnbreak
        \noindent
        \begin{minted}[baselinestretch=1.2, escapeinside=||]{python}
|\answerbox{0.5}| lst1 = lst2
|\answerbox{0.5}| lst1.append(4)
|\answerbox{0.5}| lst2 = [9]
|\answerbox{0.5}| lst2 = [7]
|\answerbox{0.5}| lst1 = lst2 + lst1

        \end{minted}
        \end{multicols}
    
            \izpit[ucilnica=205, naloge=-1]{Uvod v programiranje: Kolokvij \#043}{27.\ marec 2019}{
                Pri vsaki nalogi obkrožite črko pred pravilnim odgovorom ali vpišite pravilno vrednost v ustrezen prostor. \\
                Čas reševanja je 30 minut. Veliko uspeha!
            }
            
        \naloga*
        
        Kateri izmed programov pri začetnem stanju
            \inlinepy{a = 6} in
            \inlinepy{b = 9}
        nastavi vrednosti
            \inlinepy{a = 9},
            \inlinepy{b = 6} in
            \inlinepy{c = 15}?
    
        \begin{multicols}{4}
        \begin{enumerate}[(a)]
\item 
            \begin{minted}[autogobble]{python}
            c = a
            a = b
            b = c
            c = a + b
            \end{minted}
        
\item 
                \begin{minted}[autogobble]{python}
                c = a
                b = c
                a = b
                c = a + b
                \end{minted}
            
\item 
                \begin{minted}[autogobble]{python}
                a = b
                c = b
                b = a
                c = a + b
                \end{minted}
            
\item 
                \begin{minted}[autogobble]{python}
                a = b
                b = c
                c = a
                c = a + b
                \end{minted}
            
\end{enumerate}

        \end{multicols}
    
        \naloga*
        \begin{multicols}{2}
        \noindent
        Kakšne vrste napak vsebuje program na desni?

        \begin{enumerate}[(a)]
\item sintaktične napake, zaradi katerih Python programa noče izvesti.
\item vsebinske napake, zaradi katerih Python izračuna napačen rezultat
\item napake, zaradi katerih Python prekine z izvajanjem programa
\item oblikovne napake, ki ne vplivajo na pravilnost rezultata
\end{enumerate}

        \columnbreak

        \begin{minted}[baselinestretch=1.2,escapeinside=||, autogobble]{python}
        
            define fibonacci(n):
                if n <= 0:
                    return 0
                elif n == 1
                    return 1
                else:
                    a = fibonacci(n - 1)
                    b = fibonacci(n - 2)
                  return a + b
            
        \end{minted}

        \end{multicols}

    
        \naloga*
        Katere vrstice izpiše klic \inlinepy{print(f(g(2)))}, če sta funkciji \inlinepy{f} in \inlinepy{g} definirani kot spodaj?

        \begin{multicols}{2}
        \begin{minted}[autogobble]{python}
        
            def f(x):
                return x + 6
                print(x)

            def g(y):
                print(y)
                return 4 * y
        
        \end{minted}

        \begin{enumerate}[(a)]
\item \inlinepy{[2, 8, 14]}
\item \inlinepy{[8, 14]}
\item \inlinepy{[2, 14]}
\item \inlinepy{[14]}
\end{enumerate}

        \end{multicols}
    
        \naloga*

        \begin{multicols}{2}
        \noindent
        Kateri pogoj preverja spodnja funkcija?
        \begin{minted}[autogobble]{python}
        
            def f(besedilo):
                for znak in besedilo:
                    if znak not in 'aeiouAEIOU':
                        return True
                return False
            
        \end{minted}

        \begin{enumerate}[(a)]
\item ali niz vsebuje samo samoglasnike
\item ali niz ne vsebuje nobenega samoglasnika
\item ali niz vsebuje znak, ki ni samoglasnik
\item ali niz vsebuje kakšen samoglasnik
\end{enumerate}

        \end{multicols}
    
        \naloga*
        \begin{multicols}{2}
        \noindent
        V vsak prostor vpišite \textbf{natanko en znak} tako, da bo dobljeni program v spremenljivko \inlinepy{zm} shranil zmnožek števil \inlinepy{p} in \inlinepy{q}:
        
        \columnbreak
        \begin{minted}[baselinestretch=1.2,escapeinside=||]{python}
        zm = |\answerbox{0.5}|
        while p > |\answerbox{0.5}|:
            zm += |\answerbox{0.5}|
            |\answerbox{0.5}| -= 1
        \end{minted}
        \end{multicols}
    
        \clearpage
        \naloga
        
        Katera izmed spodnjih funkcij izračuna ostanek pri deljenju naravnega števila \inlinepy{m} z naravnim številom \inlinepy{n}?
    
        \begin{multicols}{2}
        \begin{enumerate}[(a)]
\item 
                \begin{minted}[autogobble]{python}
                def ostanek(m, n):
                    if n == 0:
                        return m
                    else:
                        return ostanek(n, m % n)
                \end{minted}
            
\item 
                \begin{minted}[autogobble]{python}
                def ostanek(m, n):
                    if m < n:
                        return m
                    else:
                        return ostanek(m % n)
                \end{minted}
            
\item 
            \begin{minted}[autogobble]{python}
            def ostanek(m, n):
                if m < n:
                    return m
                else:
                    return ostanek(m - n, n)
            \end{minted}
        
\item 
                \begin{minted}[autogobble]{python}
                def ostanek(m, n):
                    if m < n:
                        return 0
                    else:
                        return ostanek(m - n, n)
                \end{minted}
            
\end{enumerate}

        \end{multicols}
    
        \naloga*
        
        Katera izmed funkcij vrača drugačne rezultate kot ostale?
    
        \begin{multicols}{2}
        \begin{enumerate}[(a)]
\item 
                \begin{minted}[autogobble]{python}
                def h(r, p, q):
                    if r and p:
                        return q
                    else:
                        return False
                \end{minted}
            
\item 
                \begin{minted}[autogobble]{python}
                def h(r, p, q):
                    if not p:
                        return False
                    else:
                        return r and q
                \end{minted}
            
\item 
                \begin{minted}[autogobble]{python}
                def h(r, p, q):
                    if r:
                        return p and q
                    else:
                        return False
                \end{minted}
            
\item 
            \begin{minted}[autogobble]{python}
            def h(r, p, q):
                if not q:
                    return r and p
                else:
                    return False
            \end{minted}
        
\end{enumerate}

        \end{multicols}
    
        \naloga*
        
        Kateri izmed spodnjih programov ima drugačen izpis kot ostali?
    
        \begin{multicols}{2}
        \begin{enumerate}[(a)]
\item 
                \begin{minted}[autogobble]{python}
                    for x in range(1, 10):
                        if x % 3 == 1:
                            print(x)
                        continue
                \end{minted}
            
\item 
                \begin{minted}[autogobble]{python}
                    for x in range(1, 30, 3):
                        if x > 10:
                            break
                        print(x)
                \end{minted}
            
\item 
                \begin{minted}[autogobble]{python}
                    for x in range(1, 30, 3):
                        if x < 10:
                            print(x)
                \end{minted}
            
\item 
            \begin{minted}[autogobble]{python}
                for x in range(1, 10):
                    if x % 3 != 0:
                        continue
                    print(x)
            \end{minted}
        
\end{enumerate}

        \end{multicols}
    
        \naloga*
        \begin{multicols}{2}
        \noindent
        Napišite primer vrednosti spremenljivk \inlinepy{niz1} in \inlinepy{niz2}, za kateri klic \inlinepy{g(niz1, niz2)} vrne \inlinepy{True}.
        \begin{minted}[baselinestretch=1.2, escapeinside=||]{python}
        niz1 = |\answerbox{3}|
        niz2 = |\answerbox{3}|
        \end{minted}
        \vfil
        \columnbreak
        \begin{minted}[autogobble]{python}
        def g(niz1, niz2):
            if len(niz1) != len(niz2) or len(niz1) < 3:
                return False
            for j in range(len(niz1)):
                if j % 2 == 1 and niz1[j] != niz2[j]:
                    return False
            return True
        \end{minted}
        \end{multicols}
    
        \naloga*
        \begin{multicols}{2}
        \noindent 
        S številkami od $0$ do $4$ označite vrstni red, v katerem moramo izvesti ukaze na desni, da bo na koncu spremenljivka \inlinepy{sez1} kazala na seznam \inlinepy{[6, 4, 9]}?
    
        \columnbreak
        \noindent
        \begin{minted}[baselinestretch=1.2, escapeinside=||]{python}
|\answerbox{0.5}| sez1 = sez2
|\answerbox{0.5}| sez2 = [6]
|\answerbox{0.5}| sez1 = sez2 + sez1
|\answerbox{0.5}| sez1.append(9)
|\answerbox{0.5}| sez2 = [4]

        \end{minted}
        \end{multicols}
    
            \izpit[ucilnica=205, naloge=-1]{Uvod v programiranje: Kolokvij \#044}{27.\ marec 2019}{
                Pri vsaki nalogi obkrožite črko pred pravilnim odgovorom ali vpišite pravilno vrednost v ustrezen prostor. \\
                Čas reševanja je 30 minut. Veliko uspeha!
            }
            
        \naloga*
        
        Kateri izmed programov pri začetnem stanju
            \inlinepy{a = 9} in
            \inlinepy{b = 7}
        nastavi vrednosti
            \inlinepy{a = 7},
            \inlinepy{b = 9} in
            \inlinepy{c = 16}?
    
        \begin{multicols}{4}
        \begin{enumerate}[(a)]
\item 
            \begin{minted}[autogobble]{python}
            c = a
            a = b
            b = c
            c = a + b
            \end{minted}
        
\item 
                \begin{minted}[autogobble]{python}
                c = a
                b = c
                a = b
                c = a + b
                \end{minted}
            
\item 
                \begin{minted}[autogobble]{python}
                a = b
                b = c
                c = a
                c = a + b
                \end{minted}
            
\item 
                \begin{minted}[autogobble]{python}
                a = b
                c = b
                b = a
                c = a + b
                \end{minted}
            
\end{enumerate}

        \end{multicols}
    
        \naloga*
        \begin{multicols}{2}
        \noindent
        Kakšne vrste napak vsebuje program na desni?

        \begin{enumerate}[(a)]
\item vsebinske napake, zaradi katerih Python izračuna napačen rezultat
\item napake, zaradi katerih Python prekine z izvajanjem programa
\item sintaktične napake, zaradi katerih Python programa noče izvesti.
\item oblikovne napake, ki ne vplivajo na pravilnost rezultata
\end{enumerate}

        \columnbreak

        \begin{minted}[baselinestretch=1.2,escapeinside=||, autogobble]{python}
        
            define fibonacci(n):
                if n <= 0:
                    return 0
                elif n == 1
                    return 1
                else:
                    a = fibonacci(n - 1)
                    b = fibonacci(n - 2)
                  return a + b
            
        \end{minted}

        \end{multicols}

    
        \naloga*
        Katere vrstice izpiše klic \inlinepy{print(f(g(7)))}, če sta funkciji \inlinepy{f} in \inlinepy{g} definirani kot spodaj?

        \begin{multicols}{2}
        \begin{minted}[autogobble]{python}
        
            def f(x):
                return x + 3
                print(x)

            def g(y):
                return 9 * y
                print(y)
        
        \end{minted}

        \begin{enumerate}[(a)]
\item \inlinepy{[63, 66]}
\item \inlinepy{[7, 63, 66]}
\item \inlinepy{[66]}
\item \inlinepy{[7, 66]}
\end{enumerate}

        \end{multicols}
    
        \naloga*

        \begin{multicols}{2}
        \noindent
        Kateri pogoj preverja spodnja funkcija?
        \begin{minted}[autogobble]{python}
        
            def f(besedilo):
                for x in besedilo:
                    if x in 'aeiouAEIOU':
                        return False
                return True
            
        \end{minted}

        \begin{enumerate}[(a)]
\item ali niz ne vsebuje nobenega samoglasnika
\item ali niz vsebuje samo samoglasnike
\item ali niz vsebuje kakšen samoglasnik
\item ali niz vsebuje znak, ki ni samoglasnik
\end{enumerate}

        \end{multicols}
    
        \naloga*
        \begin{multicols}{2}
        \noindent
        V vsak prostor vpišite \textbf{natanko en znak} tako, da bo dobljeni program v spremenljivko \inlinepy{vs} shranil vsoto števil \inlinepy{b} in \inlinepy{a}:
        
        \columnbreak
        \begin{minted}[baselinestretch=1.2,escapeinside=||]{python}
        vs = |\answerbox{0.5}|
        while b > |\answerbox{0.5}|:
            vs += |\answerbox{0.5}|
            |\answerbox{0.5}| -= 1
        \end{minted}
        \end{multicols}
    
        \clearpage
        \naloga
        
        Katera izmed spodnjih funkcij izračuna ostanek pri deljenju naravnega števila \inlinepy{u} z naravnim številom \inlinepy{v}?
    
        \begin{multicols}{2}
        \begin{enumerate}[(a)]
\item 
                \begin{minted}[autogobble]{python}
                def ostanek(u, v):
                    if u < v:
                        return 0
                    else:
                        return ostanek(u - v, v)
                \end{minted}
            
\item 
            \begin{minted}[autogobble]{python}
            def ostanek(u, v):
                if u < v:
                    return u
                else:
                    return ostanek(u - v, v)
            \end{minted}
        
\item 
                \begin{minted}[autogobble]{python}
                def ostanek(u, v):
                    if u < v:
                        return u
                    else:
                        return ostanek(u % v)
                \end{minted}
            
\item 
                \begin{minted}[autogobble]{python}
                def ostanek(u, v):
                    if v == 0:
                        return u
                    else:
                        return ostanek(v, u % v)
                \end{minted}
            
\end{enumerate}

        \end{multicols}
    
        \naloga*
        
        Katera izmed funkcij vrača drugačne rezultate kot ostale?
    
        \begin{multicols}{2}
        \begin{enumerate}[(a)]
\item 
            \begin{minted}[autogobble]{python}
            def g(z, y, x):
                if not x:
                    return z and y
                else:
                    return False
            \end{minted}
        
\item 
                \begin{minted}[autogobble]{python}
                def g(z, y, x):
                    if z and y:
                        return x
                    else:
                        return False
                \end{minted}
            
\item 
                \begin{minted}[autogobble]{python}
                def g(z, y, x):
                    if not y:
                        return False
                    else:
                        return z and x
                \end{minted}
            
\item 
                \begin{minted}[autogobble]{python}
                def g(z, y, x):
                    if z:
                        return y and x
                    else:
                        return False
                \end{minted}
            
\end{enumerate}

        \end{multicols}
    
        \naloga*
        
        Kateri izmed spodnjih programov ima drugačen izpis kot ostali?
    
        \begin{multicols}{2}
        \begin{enumerate}[(a)]
\item 
                \begin{minted}[autogobble]{python}
                    for x in range(1, 10):
                        if x % 3 == 1:
                            print(x)
                        continue
                \end{minted}
            
\item 
                \begin{minted}[autogobble]{python}
                    for x in range(1, 30, 3):
                        if x < 10:
                            print(x)
                \end{minted}
            
\item 
            \begin{minted}[autogobble]{python}
                for x in range(1, 10):
                    if x % 3 != 0:
                        continue
                    print(x)
            \end{minted}
        
\item 
                \begin{minted}[autogobble]{python}
                    for x in range(1, 30, 3):
                        if x > 10:
                            break
                        print(x)
                \end{minted}
            
\end{enumerate}

        \end{multicols}
    
        \naloga*
        \begin{multicols}{2}
        \noindent
        Napišite primer vrednosti spremenljivk \inlinepy{sez1} in \inlinepy{sez2}, za kateri klic \inlinepy{f(sez1, sez2)} vrne \inlinepy{True}.
        \begin{minted}[baselinestretch=1.2, escapeinside=||]{python}
        sez1 = |\answerbox{3}|
        sez2 = |\answerbox{3}|
        \end{minted}
        \vfil
        \columnbreak
        \begin{minted}[autogobble]{python}
        def f(sez1, sez2):
            if len(sez1) != len(sez2) or len(sez1) < 3:
                return False
            for i in range(len(sez1)):
                if i % 2 == 0 and sez1[i] != sez2[i]:
                    return False
            return True
        \end{minted}
        \end{multicols}
    
        \naloga*
        \begin{multicols}{2}
        \noindent 
        S številkami od $0$ do $4$ označite vrstni red, v katerem moramo izvesti ukaze na desni, da bo na koncu spremenljivka \inlinepy{sez1} kazala na seznam \inlinepy{[8, 9, 7]}?
    
        \columnbreak
        \noindent
        \begin{minted}[baselinestretch=1.2, escapeinside=||]{python}
|\answerbox{0.5}| sez2 = [8]
|\answerbox{0.5}| sez1 = sez2 + sez1
|\answerbox{0.5}| sez1.append(7)
|\answerbox{0.5}| sez2 = [9]
|\answerbox{0.5}| sez1 = sez2

        \end{minted}
        \end{multicols}
    
            \izpit[ucilnica=205, naloge=-1]{Uvod v programiranje: Kolokvij \#045}{27.\ marec 2019}{
                Pri vsaki nalogi obkrožite črko pred pravilnim odgovorom ali vpišite pravilno vrednost v ustrezen prostor. \\
                Čas reševanja je 30 minut. Veliko uspeha!
            }
            
        \naloga*
        
        Kateri izmed programov pri začetnem stanju
            \inlinepy{k = 3} in
            \inlinepy{m = 6}
        nastavi vrednosti
            \inlinepy{k = 6},
            \inlinepy{m = 3} in
            \inlinepy{n = 9}?
    
        \begin{multicols}{4}
        \begin{enumerate}[(a)]
\item 
                \begin{minted}[autogobble]{python}
                k = m
                m = n
                n = k
                n = k + m
                \end{minted}
            
\item 
                \begin{minted}[autogobble]{python}
                n = k
                m = n
                k = m
                n = k + m
                \end{minted}
            
\item 
                \begin{minted}[autogobble]{python}
                k = m
                n = m
                m = k
                n = k + m
                \end{minted}
            
\item 
            \begin{minted}[autogobble]{python}
            n = k
            k = m
            m = n
            n = k + m
            \end{minted}
        
\end{enumerate}

        \end{multicols}
    
        \naloga*
        \begin{multicols}{2}
        \noindent
        Kakšne vrste napak vsebuje program na desni?

        \begin{enumerate}[(a)]
\item vsebinske napake, zaradi katerih Python izračuna napačen rezultat
\item sintaktične napake, zaradi katerih Python programa noče izvesti.
\item napake, zaradi katerih Python prekine z izvajanjem programa
\item oblikovne napake, ki ne vplivajo na pravilnost rezultata
\end{enumerate}

        \columnbreak

        \begin{minted}[baselinestretch=1.2,escapeinside=||, autogobble]{python}
        
            define fibonacci(n):
                if n <= 0:
                    return 0
                elif n == 1
                    return 1
                else:
                    a = fibonacci(n - 1)
                    b = fibonacci(n - 2)
                  return a + b
            
        \end{minted}

        \end{multicols}

    
        \naloga*
        Katere vrstice izpiše klic \inlinepy{print(p(q(7)))}, če sta funkciji \inlinepy{p} in \inlinepy{q} definirani kot spodaj?

        \begin{multicols}{2}
        \begin{minted}[autogobble]{python}
        
            def p(a):
                return a + 9
                print(a)

            def q(b):
                return 1 * b
                print(b)
        
        \end{minted}

        \begin{enumerate}[(a)]
\item \inlinepy{[7, 16]}
\item \inlinepy{[7, 16]}
\item \inlinepy{[7, 7, 16]}
\item \inlinepy{[16]}
\end{enumerate}

        \end{multicols}
    
        \naloga*

        \begin{multicols}{2}
        \noindent
        Kateri pogoj preverja spodnja funkcija?
        \begin{minted}[autogobble]{python}
        
            def f(stavek):
                for z in stavek:
                    if z in 'aeiouAEIOU':
                        return False
                return True
            
        \end{minted}

        \begin{enumerate}[(a)]
\item ali niz ne vsebuje nobenega samoglasnika
\item ali niz vsebuje znak, ki ni samoglasnik
\item ali niz vsebuje kakšen samoglasnik
\item ali niz vsebuje samo samoglasnike
\end{enumerate}

        \end{multicols}
    
        \naloga*
        \begin{multicols}{2}
        \noindent
        V vsak prostor vpišite \textbf{natanko en znak} tako, da bo dobljeni program v spremenljivko \inlinepy{zm} shranil zmnožek števil \inlinepy{x} in \inlinepy{y}:
        
        \columnbreak
        \begin{minted}[baselinestretch=1.2,escapeinside=||]{python}
        zm = |\answerbox{0.5}|
        while x > |\answerbox{0.5}|:
            zm += |\answerbox{0.5}|
            |\answerbox{0.5}| -= 1
        \end{minted}
        \end{multicols}
    
        \clearpage
        \naloga
        
        Katera izmed spodnjih funkcij izračuna ostanek pri deljenju naravnega števila \inlinepy{m} z naravnim številom \inlinepy{n}?
    
        \begin{multicols}{2}
        \begin{enumerate}[(a)]
\item 
            \begin{minted}[autogobble]{python}
            def ostanek(m, n):
                if m < n:
                    return m
                else:
                    return ostanek(m - n, n)
            \end{minted}
        
\item 
                \begin{minted}[autogobble]{python}
                def ostanek(m, n):
                    if m < n:
                        return 0
                    else:
                        return ostanek(m - n, n)
                \end{minted}
            
\item 
                \begin{minted}[autogobble]{python}
                def ostanek(m, n):
                    if m < n:
                        return m
                    else:
                        return ostanek(m % n)
                \end{minted}
            
\item 
                \begin{minted}[autogobble]{python}
                def ostanek(m, n):
                    if n == 0:
                        return m
                    else:
                        return ostanek(n, m % n)
                \end{minted}
            
\end{enumerate}

        \end{multicols}
    
        \naloga*
        
        Katera izmed funkcij vrača drugačne rezultate kot ostale?
    
        \begin{multicols}{2}
        \begin{enumerate}[(a)]
\item 
                \begin{minted}[autogobble]{python}
                def f(z, x, y):
                    if z and x:
                        return y
                    else:
                        return False
                \end{minted}
            
\item 
                \begin{minted}[autogobble]{python}
                def f(z, x, y):
                    if z:
                        return x and y
                    else:
                        return False
                \end{minted}
            
\item 
            \begin{minted}[autogobble]{python}
            def f(z, x, y):
                if not y:
                    return z and x
                else:
                    return False
            \end{minted}
        
\item 
                \begin{minted}[autogobble]{python}
                def f(z, x, y):
                    if not x:
                        return False
                    else:
                        return z and y
                \end{minted}
            
\end{enumerate}

        \end{multicols}
    
        \naloga*
        
        Kateri izmed spodnjih programov ima drugačen izpis kot ostali?
    
        \begin{multicols}{2}
        \begin{enumerate}[(a)]
\item 
            \begin{minted}[autogobble]{python}
                for x in range(1, 10):
                    if x % 3 != 0:
                        continue
                    print(x)
            \end{minted}
        
\item 
                \begin{minted}[autogobble]{python}
                    for x in range(1, 10):
                        if x % 3 == 1:
                            print(x)
                        continue
                \end{minted}
            
\item 
                \begin{minted}[autogobble]{python}
                    for x in range(1, 30, 3):
                        if x < 10:
                            print(x)
                \end{minted}
            
\item 
                \begin{minted}[autogobble]{python}
                    for x in range(1, 30, 3):
                        if x > 10:
                            break
                        print(x)
                \end{minted}
            
\end{enumerate}

        \end{multicols}
    
        \naloga*
        \begin{multicols}{2}
        \noindent
        Napišite primer vrednosti spremenljivk \inlinepy{lst1} in \inlinepy{lst2}, za kateri klic \inlinepy{g(lst1, lst2)} vrne \inlinepy{True}.
        \begin{minted}[baselinestretch=1.2, escapeinside=||]{python}
        lst1 = |\answerbox{3}|
        lst2 = |\answerbox{3}|
        \end{minted}
        \vfil
        \columnbreak
        \begin{minted}[autogobble]{python}
        def g(lst1, lst2):
            if len(lst1) != len(lst2) or len(lst1) < 3:
                return False
            for j in range(len(lst1)):
                if j % 2 == 1 and lst1[j] != lst2[j]:
                    return False
            return True
        \end{minted}
        \end{multicols}
    
        \naloga*
        \begin{multicols}{2}
        \noindent 
        S številkami od $0$ do $4$ označite vrstni red, v katerem moramo izvesti ukaze na desni, da bo na koncu spremenljivka \inlinepy{lst2} kazala na seznam \inlinepy{[9, 7, 6]}?
    
        \columnbreak
        \noindent
        \begin{minted}[baselinestretch=1.2, escapeinside=||]{python}
|\answerbox{0.5}| lst1 = [7]
|\answerbox{0.5}| lst2.append(6)
|\answerbox{0.5}| lst1 = [9]
|\answerbox{0.5}| lst2 = lst1 + lst2
|\answerbox{0.5}| lst2 = lst1

        \end{minted}
        \end{multicols}
    
            \izpit[ucilnica=205, naloge=-1]{Uvod v programiranje: Kolokvij \#046}{27.\ marec 2019}{
                Pri vsaki nalogi obkrožite črko pred pravilnim odgovorom ali vpišite pravilno vrednost v ustrezen prostor. \\
                Čas reševanja je 30 minut. Veliko uspeha!
            }
            
        \naloga*
        
        Kateri izmed programov pri začetnem stanju
            \inlinepy{u = 8} in
            \inlinepy{v = 4}
        nastavi vrednosti
            \inlinepy{u = 4},
            \inlinepy{v = 8} in
            \inlinepy{w = 12}?
    
        \begin{multicols}{4}
        \begin{enumerate}[(a)]
\item 
                \begin{minted}[autogobble]{python}
                u = v
                v = w
                w = u
                w = u + v
                \end{minted}
            
\item 
                \begin{minted}[autogobble]{python}
                w = u
                v = w
                u = v
                w = u + v
                \end{minted}
            
\item 
                \begin{minted}[autogobble]{python}
                u = v
                w = v
                v = u
                w = u + v
                \end{minted}
            
\item 
            \begin{minted}[autogobble]{python}
            w = u
            u = v
            v = w
            w = u + v
            \end{minted}
        
\end{enumerate}

        \end{multicols}
    
        \naloga*
        \begin{multicols}{2}
        \noindent
        Kakšne vrste napak vsebuje program na desni?

        \begin{enumerate}[(a)]
\item napake, zaradi katerih Python prekine z izvajanjem programa
\item sintaktične napake, zaradi katerih Python programa noče izvesti.
\item oblikovne napake, ki ne vplivajo na pravilnost rezultata
\item vsebinske napake, zaradi katerih Python izračuna napačen rezultat
\end{enumerate}

        \columnbreak

        \begin{minted}[baselinestretch=1.2,escapeinside=||, autogobble]{python}
        
            def fibonacci(n):
                if n <= 0:
                    return 0
                elif n == 1:
                    return 0
                else:
                    a = fibonacci(n - 1)
                    b = fibonacci(n - 3)
                    return a * b
            
        \end{minted}

        \end{multicols}

    
        \naloga*
        Katere vrstice izpiše klic \inlinepy{print(f(g(3)))}, če sta funkciji \inlinepy{f} in \inlinepy{g} definirani kot spodaj?

        \begin{multicols}{2}
        \begin{minted}[autogobble]{python}
        
            def f(x):
                print(x)
                return x + 4

            def g(y):
                return 6 * y
                print(y)
        
        \end{minted}

        \begin{enumerate}[(a)]
\item \inlinepy{[18, 22]}
\item \inlinepy{[3, 22]}
\item \inlinepy{[3, 18, 22]}
\item \inlinepy{[22]}
\end{enumerate}

        \end{multicols}
    
        \naloga*

        \begin{multicols}{2}
        \noindent
        Kateri pogoj preverja spodnja funkcija?
        \begin{minted}[autogobble]{python}
        
            def f(niz):
                for x in niz:
                    if x in 'aeiouAEIOU':
                        return True
                return False
            
        \end{minted}

        \begin{enumerate}[(a)]
\item ali niz vsebuje samo samoglasnike
\item ali niz vsebuje kakšen samoglasnik
\item ali niz ne vsebuje nobenega samoglasnika
\item ali niz vsebuje znak, ki ni samoglasnik
\end{enumerate}

        \end{multicols}
    
        \naloga*
        \begin{multicols}{2}
        \noindent
        V vsak prostor vpišite \textbf{natanko en znak} tako, da bo dobljeni program v spremenljivko \inlinepy{vs} shranil vsoto števil \inlinepy{b} in \inlinepy{a}:
        
        \columnbreak
        \begin{minted}[baselinestretch=1.2,escapeinside=||]{python}
        vs = |\answerbox{0.5}|
        while b > |\answerbox{0.5}|:
            vs += |\answerbox{0.5}|
            |\answerbox{0.5}| -= 1
        \end{minted}
        \end{multicols}
    
        \clearpage
        \naloga
        
        Katera izmed spodnjih funkcij izračuna ostanek pri deljenju naravnega števila \inlinepy{m} z naravnim številom \inlinepy{n}?
    
        \begin{multicols}{2}
        \begin{enumerate}[(a)]
\item 
                \begin{minted}[autogobble]{python}
                def ostanek(m, n):
                    if m < n:
                        return 0
                    else:
                        return ostanek(m - n, n)
                \end{minted}
            
\item 
                \begin{minted}[autogobble]{python}
                def ostanek(m, n):
                    if n == 0:
                        return m
                    else:
                        return ostanek(n, m % n)
                \end{minted}
            
\item 
            \begin{minted}[autogobble]{python}
            def ostanek(m, n):
                if m < n:
                    return m
                else:
                    return ostanek(m - n, n)
            \end{minted}
        
\item 
                \begin{minted}[autogobble]{python}
                def ostanek(m, n):
                    if m < n:
                        return m
                    else:
                        return ostanek(m % n)
                \end{minted}
            
\end{enumerate}

        \end{multicols}
    
        \naloga*
        
        Katera izmed funkcij vrača drugačne rezultate kot ostale?
    
        \begin{multicols}{2}
        \begin{enumerate}[(a)]
\item 
                \begin{minted}[autogobble]{python}
                def g(r, q, p):
                    if r:
                        return q and p
                    else:
                        return False
                \end{minted}
            
\item 
                \begin{minted}[autogobble]{python}
                def g(r, q, p):
                    if r and q:
                        return p
                    else:
                        return False
                \end{minted}
            
\item 
            \begin{minted}[autogobble]{python}
            def g(r, q, p):
                if not p:
                    return r and q
                else:
                    return False
            \end{minted}
        
\item 
                \begin{minted}[autogobble]{python}
                def g(r, q, p):
                    if not q:
                        return False
                    else:
                        return r and p
                \end{minted}
            
\end{enumerate}

        \end{multicols}
    
        \naloga*
        
        Kateri izmed spodnjih programov ima drugačen izpis kot ostali?
    
        \begin{multicols}{2}
        \begin{enumerate}[(a)]
\item 
                \begin{minted}[autogobble]{python}
                    for x in range(1, 20):
                        if x % 5 == 1:
                            print(x)
                        continue
                \end{minted}
            
\item 
                \begin{minted}[autogobble]{python}
                    for x in range(1, 100, 5):
                        if x > 20:
                            break
                        print(x)
                \end{minted}
            
\item 
            \begin{minted}[autogobble]{python}
                for x in range(1, 20):
                    if x % 5 != 0:
                        continue
                    print(x)
            \end{minted}
        
\item 
                \begin{minted}[autogobble]{python}
                    for x in range(1, 100, 5):
                        if x < 20:
                            print(x)
                \end{minted}
            
\end{enumerate}

        \end{multicols}
    
        \naloga*
        \begin{multicols}{2}
        \noindent
        Napišite primer vrednosti spremenljivk \inlinepy{sez1} in \inlinepy{sez2}, za kateri klic \inlinepy{g(sez1, sez2)} vrne \inlinepy{True}.
        \begin{minted}[baselinestretch=1.2, escapeinside=||]{python}
        sez1 = |\answerbox{3}|
        sez2 = |\answerbox{3}|
        \end{minted}
        \vfil
        \columnbreak
        \begin{minted}[autogobble]{python}
        def g(sez1, sez2):
            if len(sez1) != len(sez2) or len(sez1) < 3:
                return False
            for j in range(len(sez1)):
                if j % 2 == 1 and sez1[j] != sez2[j]:
                    return False
            return True
        \end{minted}
        \end{multicols}
    
        \naloga*
        \begin{multicols}{2}
        \noindent 
        S številkami od $0$ do $4$ označite vrstni red, v katerem moramo izvesti ukaze na desni, da bo na koncu spremenljivka \inlinepy{lst1} kazala na seznam \inlinepy{[1, 3, 7]}?
    
        \columnbreak
        \noindent
        \begin{minted}[baselinestretch=1.2, escapeinside=||]{python}
|\answerbox{0.5}| lst1 = lst2 + lst1
|\answerbox{0.5}| lst2 = [1]
|\answerbox{0.5}| lst1.append(7)
|\answerbox{0.5}| lst1 = lst2
|\answerbox{0.5}| lst2 = [3]

        \end{minted}
        \end{multicols}
    
            \izpit[ucilnica=205, naloge=-1]{Uvod v programiranje: Kolokvij \#047}{27.\ marec 2019}{
                Pri vsaki nalogi obkrožite črko pred pravilnim odgovorom ali vpišite pravilno vrednost v ustrezen prostor. \\
                Čas reševanja je 30 minut. Veliko uspeha!
            }
            
        \naloga*
        
        Kateri izmed programov pri začetnem stanju
            \inlinepy{x = 6} in
            \inlinepy{y = 1}
        nastavi vrednosti
            \inlinepy{x = 1},
            \inlinepy{y = 6} in
            \inlinepy{z = 7}?
    
        \begin{multicols}{4}
        \begin{enumerate}[(a)]
\item 
                \begin{minted}[autogobble]{python}
                x = y
                z = y
                y = x
                z = x + y
                \end{minted}
            
\item 
            \begin{minted}[autogobble]{python}
            z = x
            x = y
            y = z
            z = x + y
            \end{minted}
        
\item 
                \begin{minted}[autogobble]{python}
                z = x
                y = z
                x = y
                z = x + y
                \end{minted}
            
\item 
                \begin{minted}[autogobble]{python}
                x = y
                y = z
                z = x
                z = x + y
                \end{minted}
            
\end{enumerate}

        \end{multicols}
    
        \naloga*
        \begin{multicols}{2}
        \noindent
        Kakšne vrste napak vsebuje program na desni?

        \begin{enumerate}[(a)]
\item oblikovne napake, ki ne vplivajo na pravilnost rezultata
\item vsebinske napake, zaradi katerih Python izračuna napačen rezultat
\item napake, zaradi katerih Python prekine z izvajanjem programa
\item sintaktične napake, zaradi katerih Python programa noče izvesti.
\end{enumerate}

        \columnbreak

        \begin{minted}[baselinestretch=1.2,escapeinside=||, autogobble]{python}
        
            def fibonacci(n):
                if n<=0:
                    return (0)
                elif n==1:
                    return (1)
                else:
                    a = fibonacci(n - 1)
                    b = fibonacci(n - 2)
                    return   a + b
            
        \end{minted}

        \end{multicols}

    
        \naloga*
        Katere vrstice izpiše klic \inlinepy{print(p(q(1)))}, če sta funkciji \inlinepy{p} in \inlinepy{q} definirani kot spodaj?

        \begin{multicols}{2}
        \begin{minted}[autogobble]{python}
        
            def p(a):
                print(a)
                return a + 7

            def q(b):
                return 6 * b
                print(b)
        
        \end{minted}

        \begin{enumerate}[(a)]
\item \inlinepy{[1, 13]}
\item \inlinepy{[6, 13]}
\item \inlinepy{[1, 6, 13]}
\item \inlinepy{[13]}
\end{enumerate}

        \end{multicols}
    
        \naloga*

        \begin{multicols}{2}
        \noindent
        Kateri pogoj preverja spodnja funkcija?
        \begin{minted}[autogobble]{python}
        
            def f(stavek):
                for x in stavek:
                    if x in 'aeiouAEIOU':
                        return True
                return False
            
        \end{minted}

        \begin{enumerate}[(a)]
\item ali niz ne vsebuje nobenega samoglasnika
\item ali niz vsebuje samo samoglasnike
\item ali niz vsebuje znak, ki ni samoglasnik
\item ali niz vsebuje kakšen samoglasnik
\end{enumerate}

        \end{multicols}
    
        \naloga*
        \begin{multicols}{2}
        \noindent
        V vsak prostor vpišite \textbf{natanko en znak} tako, da bo dobljeni program v spremenljivko \inlinepy{vs} shranil vsoto števil \inlinepy{p} in \inlinepy{q}:
        
        \columnbreak
        \begin{minted}[baselinestretch=1.2,escapeinside=||]{python}
        vs = |\answerbox{0.5}|
        while p > |\answerbox{0.5}|:
            vs += |\answerbox{0.5}|
            |\answerbox{0.5}| -= 1
        \end{minted}
        \end{multicols}
    
        \clearpage
        \naloga
        
        Katera izmed spodnjih funkcij izračuna ostanek pri deljenju naravnega števila \inlinepy{u} z naravnim številom \inlinepy{v}?
    
        \begin{multicols}{2}
        \begin{enumerate}[(a)]
\item 
                \begin{minted}[autogobble]{python}
                def ostanek(u, v):
                    if u < v:
                        return 0
                    else:
                        return ostanek(u - v, v)
                \end{minted}
            
\item 
                \begin{minted}[autogobble]{python}
                def ostanek(u, v):
                    if u < v:
                        return u
                    else:
                        return ostanek(u % v)
                \end{minted}
            
\item 
            \begin{minted}[autogobble]{python}
            def ostanek(u, v):
                if u < v:
                    return u
                else:
                    return ostanek(u - v, v)
            \end{minted}
        
\item 
                \begin{minted}[autogobble]{python}
                def ostanek(u, v):
                    if v == 0:
                        return u
                    else:
                        return ostanek(v, u % v)
                \end{minted}
            
\end{enumerate}

        \end{multicols}
    
        \naloga*
        
        Katera izmed funkcij vrača drugačne rezultate kot ostale?
    
        \begin{multicols}{2}
        \begin{enumerate}[(a)]
\item 
                \begin{minted}[autogobble]{python}
                def g(z, y, x):
                    if not y:
                        return False
                    else:
                        return z and x
                \end{minted}
            
\item 
                \begin{minted}[autogobble]{python}
                def g(z, y, x):
                    if z and y:
                        return x
                    else:
                        return False
                \end{minted}
            
\item 
                \begin{minted}[autogobble]{python}
                def g(z, y, x):
                    if z:
                        return y and x
                    else:
                        return False
                \end{minted}
            
\item 
            \begin{minted}[autogobble]{python}
            def g(z, y, x):
                if not x:
                    return z and y
                else:
                    return False
            \end{minted}
        
\end{enumerate}

        \end{multicols}
    
        \naloga*
        
        Kateri izmed spodnjih programov ima drugačen izpis kot ostali?
    
        \begin{multicols}{2}
        \begin{enumerate}[(a)]
\item 
                \begin{minted}[autogobble]{python}
                    for x in range(1, 150, 3):
                        if x < 50:
                            print(x)
                \end{minted}
            
\item 
            \begin{minted}[autogobble]{python}
                for x in range(1, 50):
                    if x % 3 != 0:
                        continue
                    print(x)
            \end{minted}
        
\item 
                \begin{minted}[autogobble]{python}
                    for x in range(1, 150, 3):
                        if x > 50:
                            break
                        print(x)
                \end{minted}
            
\item 
                \begin{minted}[autogobble]{python}
                    for x in range(1, 50):
                        if x % 3 == 1:
                            print(x)
                        continue
                \end{minted}
            
\end{enumerate}

        \end{multicols}
    
        \naloga*
        \begin{multicols}{2}
        \noindent
        Napišite primer vrednosti spremenljivk \inlinepy{sez1} in \inlinepy{sez2}, za kateri klic \inlinepy{h(sez1, sez2)} vrne \inlinepy{True}.
        \begin{minted}[baselinestretch=1.2, escapeinside=||]{python}
        sez1 = |\answerbox{3}|
        sez2 = |\answerbox{3}|
        \end{minted}
        \vfil
        \columnbreak
        \begin{minted}[autogobble]{python}
        def h(sez1, sez2):
            if len(sez1) != len(sez2) or len(sez1) < 3:
                return False
            for i in range(len(sez1)):
                if i % 2 == 0 and sez1[i] != sez2[i]:
                    return False
            return True
        \end{minted}
        \end{multicols}
    
        \naloga*
        \begin{multicols}{2}
        \noindent 
        S številkami od $0$ do $4$ označite vrstni red, v katerem moramo izvesti ukaze na desni, da bo na koncu spremenljivka \inlinepy{lst1} kazala na seznam \inlinepy{[1, 8, 2]}?
    
        \columnbreak
        \noindent
        \begin{minted}[baselinestretch=1.2, escapeinside=||]{python}
|\answerbox{0.5}| lst1 = lst2
|\answerbox{0.5}| lst2 = [8]
|\answerbox{0.5}| lst1 = lst2 + lst1
|\answerbox{0.5}| lst1.append(2)
|\answerbox{0.5}| lst2 = [1]

        \end{minted}
        \end{multicols}
    
            \izpit[ucilnica=205, naloge=-1]{Uvod v programiranje: Kolokvij \#048}{27.\ marec 2019}{
                Pri vsaki nalogi obkrožite črko pred pravilnim odgovorom ali vpišite pravilno vrednost v ustrezen prostor. \\
                Čas reševanja je 30 minut. Veliko uspeha!
            }
            
        \naloga*
        
        Kateri izmed programov pri začetnem stanju
            \inlinepy{u = 2} in
            \inlinepy{v = 5}
        nastavi vrednosti
            \inlinepy{u = 5},
            \inlinepy{v = 2} in
            \inlinepy{w = 7}?
    
        \begin{multicols}{4}
        \begin{enumerate}[(a)]
\item 
                \begin{minted}[autogobble]{python}
                u = v
                v = w
                w = u
                w = u + v
                \end{minted}
            
\item 
                \begin{minted}[autogobble]{python}
                w = u
                v = w
                u = v
                w = u + v
                \end{minted}
            
\item 
                \begin{minted}[autogobble]{python}
                u = v
                w = v
                v = u
                w = u + v
                \end{minted}
            
\item 
            \begin{minted}[autogobble]{python}
            w = u
            u = v
            v = w
            w = u + v
            \end{minted}
        
\end{enumerate}

        \end{multicols}
    
        \naloga*
        \begin{multicols}{2}
        \noindent
        Kakšne vrste napak vsebuje program na desni?

        \begin{enumerate}[(a)]
\item vsebinske napake, zaradi katerih Python izračuna napačen rezultat
\item oblikovne napake, ki ne vplivajo na pravilnost rezultata
\item napake, zaradi katerih Python prekine z izvajanjem programa
\item sintaktične napake, zaradi katerih Python programa noče izvesti.
\end{enumerate}

        \columnbreak

        \begin{minted}[baselinestretch=1.2,escapeinside=||, autogobble]{python}
        
            def fibonacci(n):
                if n<=0:
                    return (0)
                elif n==1:
                    return (1)
                else:
                    a = fibonacci(n - 1)
                    b = fibonacci(n - 2)
                    return   a + b
            
        \end{minted}

        \end{multicols}

    
        \naloga*
        Katere vrstice izpiše klic \inlinepy{print(f(g(3)))}, če sta funkciji \inlinepy{f} in \inlinepy{g} definirani kot spodaj?

        \begin{multicols}{2}
        \begin{minted}[autogobble]{python}
        
            def f(x):
                print(x)
                return x + 8

            def g(y):
                return 9 * y
                print(y)
        
        \end{minted}

        \begin{enumerate}[(a)]
\item \inlinepy{[27, 35]}
\item \inlinepy{[35]}
\item \inlinepy{[3, 35]}
\item \inlinepy{[3, 27, 35]}
\end{enumerate}

        \end{multicols}
    
        \naloga*

        \begin{multicols}{2}
        \noindent
        Kateri pogoj preverja spodnja funkcija?
        \begin{minted}[autogobble]{python}
        
            def f(niz):
                for x in niz:
                    if x not in 'aeiouAEIOU':
                        return False
                return True
            
        \end{minted}

        \begin{enumerate}[(a)]
\item ali niz ne vsebuje nobenega samoglasnika
\item ali niz vsebuje samo samoglasnike
\item ali niz vsebuje znak, ki ni samoglasnik
\item ali niz vsebuje kakšen samoglasnik
\end{enumerate}

        \end{multicols}
    
        \naloga*
        \begin{multicols}{2}
        \noindent
        V vsak prostor vpišite \textbf{natanko en znak} tako, da bo dobljeni program v spremenljivko \inlinepy{vs} shranil vsoto števil \inlinepy{p} in \inlinepy{q}:
        
        \columnbreak
        \begin{minted}[baselinestretch=1.2,escapeinside=||]{python}
        vs = |\answerbox{0.5}|
        while p > |\answerbox{0.5}|:
            vs += |\answerbox{0.5}|
            |\answerbox{0.5}| -= 1
        \end{minted}
        \end{multicols}
    
        \clearpage
        \naloga
        
        Katera izmed spodnjih funkcij izračuna ostanek pri deljenju naravnega števila \inlinepy{a} z naravnim številom \inlinepy{b}?
    
        \begin{multicols}{2}
        \begin{enumerate}[(a)]
\item 
                \begin{minted}[autogobble]{python}
                def ostanek(a, b):
                    if b == 0:
                        return a
                    else:
                        return ostanek(b, a % b)
                \end{minted}
            
\item 
            \begin{minted}[autogobble]{python}
            def ostanek(a, b):
                if a < b:
                    return a
                else:
                    return ostanek(a - b, b)
            \end{minted}
        
\item 
                \begin{minted}[autogobble]{python}
                def ostanek(a, b):
                    if a < b:
                        return a
                    else:
                        return ostanek(a % b)
                \end{minted}
            
\item 
                \begin{minted}[autogobble]{python}
                def ostanek(a, b):
                    if a < b:
                        return 0
                    else:
                        return ostanek(a - b, b)
                \end{minted}
            
\end{enumerate}

        \end{multicols}
    
        \naloga*
        
        Katera izmed funkcij vrača drugačne rezultate kot ostale?
    
        \begin{multicols}{2}
        \begin{enumerate}[(a)]
\item 
            \begin{minted}[autogobble]{python}
            def f(x, z, y):
                if not y:
                    return x and z
                else:
                    return False
            \end{minted}
        
\item 
                \begin{minted}[autogobble]{python}
                def f(x, z, y):
                    if not z:
                        return False
                    else:
                        return x and y
                \end{minted}
            
\item 
                \begin{minted}[autogobble]{python}
                def f(x, z, y):
                    if x and z:
                        return y
                    else:
                        return False
                \end{minted}
            
\item 
                \begin{minted}[autogobble]{python}
                def f(x, z, y):
                    if x:
                        return z and y
                    else:
                        return False
                \end{minted}
            
\end{enumerate}

        \end{multicols}
    
        \naloga*
        
        Kateri izmed spodnjih programov ima drugačen izpis kot ostali?
    
        \begin{multicols}{2}
        \begin{enumerate}[(a)]
\item 
                \begin{minted}[autogobble]{python}
                    for j in range(1, 20, 2):
                        if j < 10:
                            print(j)
                \end{minted}
            
\item 
                \begin{minted}[autogobble]{python}
                    for j in range(1, 10):
                        if j % 2 == 1:
                            print(j)
                        continue
                \end{minted}
            
\item 
                \begin{minted}[autogobble]{python}
                    for j in range(1, 20, 2):
                        if j > 10:
                            break
                        print(j)
                \end{minted}
            
\item 
            \begin{minted}[autogobble]{python}
                for j in range(1, 10):
                    if j % 2 != 0:
                        continue
                    print(j)
            \end{minted}
        
\end{enumerate}

        \end{multicols}
    
        \naloga*
        \begin{multicols}{2}
        \noindent
        Napišite primer vrednosti spremenljivk \inlinepy{lst1} in \inlinepy{lst2}, za kateri klic \inlinepy{h(lst1, lst2)} vrne \inlinepy{True}.
        \begin{minted}[baselinestretch=1.2, escapeinside=||]{python}
        lst1 = |\answerbox{3}|
        lst2 = |\answerbox{3}|
        \end{minted}
        \vfil
        \columnbreak
        \begin{minted}[autogobble]{python}
        def h(lst1, lst2):
            if len(lst1) != len(lst2) or len(lst1) < 3:
                return False
            for j in range(len(lst1)):
                if j % 2 == 1 and lst1[j] != lst2[j]:
                    return False
            return True
        \end{minted}
        \end{multicols}
    
        \naloga*
        \begin{multicols}{2}
        \noindent 
        S številkami od $0$ do $4$ označite vrstni red, v katerem moramo izvesti ukaze na desni, da bo na koncu spremenljivka \inlinepy{lst2} kazala na seznam \inlinepy{[9, 7, 8]}?
    
        \columnbreak
        \noindent
        \begin{minted}[baselinestretch=1.2, escapeinside=||]{python}
|\answerbox{0.5}| lst2 = lst1 + lst2
|\answerbox{0.5}| lst2.append(8)
|\answerbox{0.5}| lst1 = [9]
|\answerbox{0.5}| lst2 = lst1
|\answerbox{0.5}| lst1 = [7]

        \end{minted}
        \end{multicols}
    
            \izpit[ucilnica=205, naloge=-1]{Uvod v programiranje: Kolokvij \#049}{27.\ marec 2019}{
                Pri vsaki nalogi obkrožite črko pred pravilnim odgovorom ali vpišite pravilno vrednost v ustrezen prostor. \\
                Čas reševanja je 30 minut. Veliko uspeha!
            }
            
        \naloga*
        
        Kateri izmed programov pri začetnem stanju
            \inlinepy{x = 7} in
            \inlinepy{y = 1}
        nastavi vrednosti
            \inlinepy{x = 1},
            \inlinepy{y = 7} in
            \inlinepy{z = 8}?
    
        \begin{multicols}{4}
        \begin{enumerate}[(a)]
\item 
                \begin{minted}[autogobble]{python}
                x = y
                y = z
                z = x
                z = x + y
                \end{minted}
            
\item 
                \begin{minted}[autogobble]{python}
                x = y
                z = y
                y = x
                z = x + y
                \end{minted}
            
\item 
            \begin{minted}[autogobble]{python}
            z = x
            x = y
            y = z
            z = x + y
            \end{minted}
        
\item 
                \begin{minted}[autogobble]{python}
                z = x
                y = z
                x = y
                z = x + y
                \end{minted}
            
\end{enumerate}

        \end{multicols}
    
        \naloga*
        \begin{multicols}{2}
        \noindent
        Kakšne vrste napak vsebuje program na desni?

        \begin{enumerate}[(a)]
\item napake, zaradi katerih Python prekine z izvajanjem programa
\item sintaktične napake, zaradi katerih Python programa noče izvesti.
\item vsebinske napake, zaradi katerih Python izračuna napačen rezultat
\item oblikovne napake, ki ne vplivajo na pravilnost rezultata
\end{enumerate}

        \columnbreak

        \begin{minted}[baselinestretch=1.2,escapeinside=||, autogobble]{python}
        
            define fibonacci(n):
                if n <= 0:
                    return 0
                elif n == 1
                    return 1
                else:
                    a = fibonacci(n - 1)
                    b = fibonacci(n - 2)
                  return a + b
            
        \end{minted}

        \end{multicols}

    
        \naloga*
        Katere vrstice izpiše klic \inlinepy{print(f(g(6)))}, če sta funkciji \inlinepy{f} in \inlinepy{g} definirani kot spodaj?

        \begin{multicols}{2}
        \begin{minted}[autogobble]{python}
        
            def f(x):
                return x + 3
                print(x)

            def g(y):
                return 2 * y
                print(y)
        
        \end{minted}

        \begin{enumerate}[(a)]
\item \inlinepy{[6, 12, 15]}
\item \inlinepy{[6, 15]}
\item \inlinepy{[12, 15]}
\item \inlinepy{[15]}
\end{enumerate}

        \end{multicols}
    
        \naloga*

        \begin{multicols}{2}
        \noindent
        Kateri pogoj preverja spodnja funkcija?
        \begin{minted}[autogobble]{python}
        
            def f(stavek):
                for x in stavek:
                    if x in 'aeiouAEIOU':
                        return True
                return False
            
        \end{minted}

        \begin{enumerate}[(a)]
\item ali niz vsebuje znak, ki ni samoglasnik
\item ali niz ne vsebuje nobenega samoglasnika
\item ali niz vsebuje kakšen samoglasnik
\item ali niz vsebuje samo samoglasnike
\end{enumerate}

        \end{multicols}
    
        \naloga*
        \begin{multicols}{2}
        \noindent
        V vsak prostor vpišite \textbf{natanko en znak} tako, da bo dobljeni program v spremenljivko \inlinepy{zm} shranil zmnožek števil \inlinepy{q} in \inlinepy{p}:
        
        \columnbreak
        \begin{minted}[baselinestretch=1.2,escapeinside=||]{python}
        zm = |\answerbox{0.5}|
        while q > |\answerbox{0.5}|:
            zm += |\answerbox{0.5}|
            |\answerbox{0.5}| -= 1
        \end{minted}
        \end{multicols}
    
        \clearpage
        \naloga
        
        Katera izmed spodnjih funkcij izračuna ostanek pri deljenju naravnega števila \inlinepy{a} z naravnim številom \inlinepy{b}?
    
        \begin{multicols}{2}
        \begin{enumerate}[(a)]
\item 
                \begin{minted}[autogobble]{python}
                def ostanek(a, b):
                    if a < b:
                        return a
                    else:
                        return ostanek(a % b)
                \end{minted}
            
\item 
                \begin{minted}[autogobble]{python}
                def ostanek(a, b):
                    if b == 0:
                        return a
                    else:
                        return ostanek(b, a % b)
                \end{minted}
            
\item 
            \begin{minted}[autogobble]{python}
            def ostanek(a, b):
                if a < b:
                    return a
                else:
                    return ostanek(a - b, b)
            \end{minted}
        
\item 
                \begin{minted}[autogobble]{python}
                def ostanek(a, b):
                    if a < b:
                        return 0
                    else:
                        return ostanek(a - b, b)
                \end{minted}
            
\end{enumerate}

        \end{multicols}
    
        \naloga*
        
        Katera izmed funkcij vrača drugačne rezultate kot ostale?
    
        \begin{multicols}{2}
        \begin{enumerate}[(a)]
\item 
            \begin{minted}[autogobble]{python}
            def h(p, r, q):
                if not q:
                    return p and r
                else:
                    return False
            \end{minted}
        
\item 
                \begin{minted}[autogobble]{python}
                def h(p, r, q):
                    if not r:
                        return False
                    else:
                        return p and q
                \end{minted}
            
\item 
                \begin{minted}[autogobble]{python}
                def h(p, r, q):
                    if p:
                        return r and q
                    else:
                        return False
                \end{minted}
            
\item 
                \begin{minted}[autogobble]{python}
                def h(p, r, q):
                    if p and r:
                        return q
                    else:
                        return False
                \end{minted}
            
\end{enumerate}

        \end{multicols}
    
        \naloga*
        
        Kateri izmed spodnjih programov ima drugačen izpis kot ostali?
    
        \begin{multicols}{2}
        \begin{enumerate}[(a)]
\item 
            \begin{minted}[autogobble]{python}
                for j in range(1, 20):
                    if j % 5 != 0:
                        continue
                    print(j)
            \end{minted}
        
\item 
                \begin{minted}[autogobble]{python}
                    for j in range(1, 100, 5):
                        if j > 20:
                            break
                        print(j)
                \end{minted}
            
\item 
                \begin{minted}[autogobble]{python}
                    for j in range(1, 100, 5):
                        if j < 20:
                            print(j)
                \end{minted}
            
\item 
                \begin{minted}[autogobble]{python}
                    for j in range(1, 20):
                        if j % 5 == 1:
                            print(j)
                        continue
                \end{minted}
            
\end{enumerate}

        \end{multicols}
    
        \naloga*
        \begin{multicols}{2}
        \noindent
        Napišite primer vrednosti spremenljivk \inlinepy{niz1} in \inlinepy{niz2}, za kateri klic \inlinepy{g(niz1, niz2)} vrne \inlinepy{True}.
        \begin{minted}[baselinestretch=1.2, escapeinside=||]{python}
        niz1 = |\answerbox{3}|
        niz2 = |\answerbox{3}|
        \end{minted}
        \vfil
        \columnbreak
        \begin{minted}[autogobble]{python}
        def g(niz1, niz2):
            if len(niz1) != len(niz2) or len(niz1) < 3:
                return False
            for j in range(len(niz1)):
                if j % 2 == 1 and niz1[j] != niz2[j]:
                    return False
            return True
        \end{minted}
        \end{multicols}
    
        \naloga*
        \begin{multicols}{2}
        \noindent 
        S številkami od $0$ do $4$ označite vrstni red, v katerem moramo izvesti ukaze na desni, da bo na koncu spremenljivka \inlinepy{sez2} kazala na seznam \inlinepy{[4, 3, 9]}?
    
        \columnbreak
        \noindent
        \begin{minted}[baselinestretch=1.2, escapeinside=||]{python}
|\answerbox{0.5}| sez2 = sez1 + sez2
|\answerbox{0.5}| sez2.append(9)
|\answerbox{0.5}| sez2 = sez1
|\answerbox{0.5}| sez1 = [3]
|\answerbox{0.5}| sez1 = [4]

        \end{minted}
        \end{multicols}
    
            \izpit[ucilnica=205, naloge=-1]{Uvod v programiranje: Kolokvij \#050}{27.\ marec 2019}{
                Pri vsaki nalogi obkrožite črko pred pravilnim odgovorom ali vpišite pravilno vrednost v ustrezen prostor. \\
                Čas reševanja je 30 minut. Veliko uspeha!
            }
            
        \naloga*
        
        Kateri izmed programov pri začetnem stanju
            \inlinepy{a = 3} in
            \inlinepy{b = 5}
        nastavi vrednosti
            \inlinepy{a = 5},
            \inlinepy{b = 3} in
            \inlinepy{c = 8}?
    
        \begin{multicols}{4}
        \begin{enumerate}[(a)]
\item 
                \begin{minted}[autogobble]{python}
                c = a
                b = c
                a = b
                c = a + b
                \end{minted}
            
\item 
                \begin{minted}[autogobble]{python}
                a = b
                c = b
                b = a
                c = a + b
                \end{minted}
            
\item 
            \begin{minted}[autogobble]{python}
            c = a
            a = b
            b = c
            c = a + b
            \end{minted}
        
\item 
                \begin{minted}[autogobble]{python}
                a = b
                b = c
                c = a
                c = a + b
                \end{minted}
            
\end{enumerate}

        \end{multicols}
    
        \naloga*
        \begin{multicols}{2}
        \noindent
        Kakšne vrste napak vsebuje program na desni?

        \begin{enumerate}[(a)]
\item napake, zaradi katerih Python prekine z izvajanjem programa
\item oblikovne napake, ki ne vplivajo na pravilnost rezultata
\item sintaktične napake, zaradi katerih Python programa noče izvesti.
\item vsebinske napake, zaradi katerih Python izračuna napačen rezultat
\end{enumerate}

        \columnbreak

        \begin{minted}[baselinestretch=1.2,escapeinside=||, autogobble]{python}
        
            def fibonacci(n):
                if n <= 0:
                    return 0
                elif n == 1:
                    return 0
                else:
                    a = fibonacci(n - 1)
                    b = fibonacci(n - 3)
                    return a * b
            
        \end{minted}

        \end{multicols}

    
        \naloga*
        Katere vrstice izpiše klic \inlinepy{print(f(g(2)))}, če sta funkciji \inlinepy{f} in \inlinepy{g} definirani kot spodaj?

        \begin{multicols}{2}
        \begin{minted}[autogobble]{python}
        
            def f(a):
                print(a)
                return a + 9

            def g(b):
                print(b)
                return 5 * b
        
        \end{minted}

        \begin{enumerate}[(a)]
\item \inlinepy{[2, 19]}
\item \inlinepy{[10, 19]}
\item \inlinepy{[19]}
\item \inlinepy{[2, 10, 19]}
\end{enumerate}

        \end{multicols}
    
        \naloga*

        \begin{multicols}{2}
        \noindent
        Kateri pogoj preverja spodnja funkcija?
        \begin{minted}[autogobble]{python}
        
            def f(besedilo):
                for z in besedilo:
                    if z not in 'aeiouAEIOU':
                        return True
                return False
            
        \end{minted}

        \begin{enumerate}[(a)]
\item ali niz vsebuje kakšen samoglasnik
\item ali niz vsebuje samo samoglasnike
\item ali niz vsebuje znak, ki ni samoglasnik
\item ali niz ne vsebuje nobenega samoglasnika
\end{enumerate}

        \end{multicols}
    
        \naloga*
        \begin{multicols}{2}
        \noindent
        V vsak prostor vpišite \textbf{natanko en znak} tako, da bo dobljeni program v spremenljivko \inlinepy{vs} shranil vsoto števil \inlinepy{q} in \inlinepy{p}:
        
        \columnbreak
        \begin{minted}[baselinestretch=1.2,escapeinside=||]{python}
        vs = |\answerbox{0.5}|
        while q > |\answerbox{0.5}|:
            vs += |\answerbox{0.5}|
            |\answerbox{0.5}| -= 1
        \end{minted}
        \end{multicols}
    
        \clearpage
        \naloga
        
        Katera izmed spodnjih funkcij izračuna ostanek pri deljenju naravnega števila \inlinepy{x} z naravnim številom \inlinepy{y}?
    
        \begin{multicols}{2}
        \begin{enumerate}[(a)]
\item 
                \begin{minted}[autogobble]{python}
                def ostanek(x, y):
                    if x < y:
                        return x
                    else:
                        return ostanek(x % y)
                \end{minted}
            
\item 
                \begin{minted}[autogobble]{python}
                def ostanek(x, y):
                    if y == 0:
                        return x
                    else:
                        return ostanek(y, x % y)
                \end{minted}
            
\item 
            \begin{minted}[autogobble]{python}
            def ostanek(x, y):
                if x < y:
                    return x
                else:
                    return ostanek(x - y, y)
            \end{minted}
        
\item 
                \begin{minted}[autogobble]{python}
                def ostanek(x, y):
                    if x < y:
                        return 0
                    else:
                        return ostanek(x - y, y)
                \end{minted}
            
\end{enumerate}

        \end{multicols}
    
        \naloga*
        
        Katera izmed funkcij vrača drugačne rezultate kot ostale?
    
        \begin{multicols}{2}
        \begin{enumerate}[(a)]
\item 
                \begin{minted}[autogobble]{python}
                def h(p, r, q):
                    if not r:
                        return False
                    else:
                        return p and q
                \end{minted}
            
\item 
            \begin{minted}[autogobble]{python}
            def h(p, r, q):
                if not q:
                    return p and r
                else:
                    return False
            \end{minted}
        
\item 
                \begin{minted}[autogobble]{python}
                def h(p, r, q):
                    if p and r:
                        return q
                    else:
                        return False
                \end{minted}
            
\item 
                \begin{minted}[autogobble]{python}
                def h(p, r, q):
                    if p:
                        return r and q
                    else:
                        return False
                \end{minted}
            
\end{enumerate}

        \end{multicols}
    
        \naloga*
        
        Kateri izmed spodnjih programov ima drugačen izpis kot ostali?
    
        \begin{multicols}{2}
        \begin{enumerate}[(a)]
\item 
                \begin{minted}[autogobble]{python}
                    for n in range(1, 10):
                        if n % 5 == 1:
                            print(n)
                        continue
                \end{minted}
            
\item 
            \begin{minted}[autogobble]{python}
                for n in range(1, 10):
                    if n % 5 != 0:
                        continue
                    print(n)
            \end{minted}
        
\item 
                \begin{minted}[autogobble]{python}
                    for n in range(1, 50, 5):
                        if n < 10:
                            print(n)
                \end{minted}
            
\item 
                \begin{minted}[autogobble]{python}
                    for n in range(1, 50, 5):
                        if n > 10:
                            break
                        print(n)
                \end{minted}
            
\end{enumerate}

        \end{multicols}
    
        \naloga*
        \begin{multicols}{2}
        \noindent
        Napišite primer vrednosti spremenljivk \inlinepy{str1} in \inlinepy{str2}, za kateri klic \inlinepy{h(str1, str2)} vrne \inlinepy{True}.
        \begin{minted}[baselinestretch=1.2, escapeinside=||]{python}
        str1 = |\answerbox{3}|
        str2 = |\answerbox{3}|
        \end{minted}
        \vfil
        \columnbreak
        \begin{minted}[autogobble]{python}
        def h(str1, str2):
            if len(str1) != len(str2) or len(str1) < 3:
                return False
            for j in range(len(str1)):
                if j % 2 == 0 and str1[j] != str2[j]:
                    return False
            return True
        \end{minted}
        \end{multicols}
    
        \naloga*
        \begin{multicols}{2}
        \noindent 
        S številkami od $0$ do $4$ označite vrstni red, v katerem moramo izvesti ukaze na desni, da bo na koncu spremenljivka \inlinepy{lst1} kazala na seznam \inlinepy{[4, 2, 7]}?
    
        \columnbreak
        \noindent
        \begin{minted}[baselinestretch=1.2, escapeinside=||]{python}
|\answerbox{0.5}| lst1.append(7)
|\answerbox{0.5}| lst1 = lst2 + lst1
|\answerbox{0.5}| lst2 = [4]
|\answerbox{0.5}| lst1 = lst2
|\answerbox{0.5}| lst2 = [2]

        \end{minted}
        \end{multicols}
    \end{document}